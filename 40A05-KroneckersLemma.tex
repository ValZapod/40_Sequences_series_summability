\documentclass[12pt]{article}
\usepackage{pmmeta}
\pmcanonicalname{KroneckersLemma}
\pmcreated{2013-03-22 18:33:54}
\pmmodified{2013-03-22 18:33:54}
\pmowner{gel}{22282}
\pmmodifier{gel}{22282}
\pmtitle{Kronecker's lemma}
\pmrecord{6}{41288}
\pmprivacy{1}
\pmauthor{gel}{22282}
\pmtype{Theorem}
\pmcomment{trigger rebuild}
\pmclassification{msc}{40A05}
\pmclassification{msc}{40-00}
%\pmkeywords{sequence}
%\pmkeywords{convergence}
%\pmkeywords{partial sum}
\pmrelated{StolzCesaroTheorem}

\endmetadata

% this is the default PlanetMath preamble.  as your knowledge
% of TeX increases, you will probably want to edit this, but
% it should be fine as is for beginners.

% almost certainly you want these
\usepackage{amssymb}
\usepackage{amsmath}
\usepackage{amsfonts}

% used for TeXing text within eps files
%\usepackage{psfrag}
% need this for including graphics (\includegraphics)
%\usepackage{graphicx}
% for neatly defining theorems and propositions
\usepackage{amsthm}
% making logically defined graphics
%%%\usepackage{xypic}

% there are many more packages, add them here as you need them

% define commands here
\newtheorem*{theorem*}{Theorem}
\newtheorem*{lemma*}{Lemma}
\newtheorem*{corollary*}{Corollary}
\newtheorem{theorem}{Theorem}
\newtheorem{lemma}{Lemma}
\newtheorem{corollary}{Corollary}


\begin{document}
Kronecker's lemma gives a condition for convergence of partial sums of real numbers, and for example can be used in the proof of Kolmogorov's strong law of large numbers.

\begin{lemma*}[Kronecker]
Let $x_1,x_2,\ldots$ and $0<b_1<b_2<\cdots$ be sequences of real numbers such that $b_n$ increases to infinity as $n\rightarrow\infty$. Suppose that the sum $\sum_{n=1}^\infty x_n/b_n$ converges to a finite limit. Then, $b_n^{-1}\sum_{k=1}^nx_k\rightarrow 0$ as $n\rightarrow\infty$.  
\end{lemma*}
\begin{proof}
Set $u_n=\sum_{k=1}^nx_k/b_k$, so that the limit $u_\infty=\lim_{n\rightarrow\infty}u_n$ exists.
Also set $a_n=\sum_{k=1}^{n-1}(b_{k+1}-b_k)u_k$ so that
\begin{equation*}
\frac{a_{n+1}-a_n}{b_{n+1}-b_n}=u_n\rightarrow u_\infty
\end{equation*}
as $n\rightarrow\infty$. Then, the Stolz-Cesaro theorem says that $a_n/b_n$ also converges to $u_\infty$, so
\begin{equation*}
b_n^{-1}\sum_{k=1}^nx_k=b_n^{-1}\sum_{k=1}^nb_k(u_k-u_{k-1})=u_n-b_n^{-1}a_n\rightarrow 0.
\end{equation*}
\end{proof}

%%%%%
%%%%%
\end{document}
