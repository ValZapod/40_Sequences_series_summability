\documentclass[12pt]{article}
\usepackage{pmmeta}
\pmcanonicalname{ProofOfInfiniteProductOfSums1aiResultWithoutExponentials}
\pmcreated{2013-03-22 18:40:38}
\pmmodified{2013-03-22 18:40:38}
\pmowner{rspuzio}{6075}
\pmmodifier{rspuzio}{6075}
\pmtitle{proof of infinite product of sums $1\!+\!a_i$ result without exponentials}
\pmrecord{11}{41425}
\pmprivacy{1}
\pmauthor{rspuzio}{6075}
\pmtype{Result}
\pmcomment{trigger rebuild}
\pmclassification{msc}{40A20}
\pmclassification{msc}{26E99}

\endmetadata

% this is the default PlanetMath preamble.  as your knowledge
% of TeX increases, you will probably want to edit this, but
% it should be fine as is for beginners.

% almost certainly you want these
\usepackage{amssymb}
\usepackage{amsmath}
\usepackage{amsfonts}

% used for TeXing text within eps files
%\usepackage{psfrag}
% need this for including graphics (\includegraphics)
%\usepackage{graphicx}
% for neatly defining theorems and propositions
%\usepackage{amsthm}
% making logically defined graphics
%%%\usepackage{xypic}

% there are many more packages, add them here as you need them

% define commands here

\begin{document}
In this entry, we show how the proof presented in the parent entry
may be modified so as to avoid use of the exponential function.
This modification makes it more elementary by not requiring that
one first develop the theory of the exponential function before
proving this basic result about infinite products.  Note that it
is only necessary to redo the part of the result which states that,
if the series converges, then the product also converges because
the proof of the opposite implication did not involve the exponential
function.

We begin with a simple inequality.  Suppose that $a$ and $b$ are
real numbers such that $0 \le a$ and $0 \le b < 1/2$.  Then we
have $2ab \le a$, hence
\begin{align*}
 (1 + a) (1 + 2 b) &= 1 + 2b + a + 2 a b \cr
                   &\le 1 + 2b + a + a \cr
                   &= 1 + 2 (a + b) .
\end{align*}

Now suppose that the series $a_1 + a_2 + a_3 + \cdots$ converges
to a value $S$. 
Since the convergence of an infinite series or product is not
affected by removing a finite number of terms we may, without
loss of generality, assume that $S < 1/2$.  Then,
since the terms $a_n$ are nonnegative for all $n$, for each partial
sum $s_n$ we will have $0 \le s_n < 1/2$.

Clearly, $t_1 \le 1 + 2 s_1$.  Suppose that, for some $n$, we have
$t_n \le 1 + 2 s_n$.  Then, using the definitions of $t_n$ and
$s_n$ along with the inequality demonstrated above, we conclude that
\begin{align*}
t_{n+1} &= t_n (1 + a_{n+1}) \cr
        &\le (1 + 2 s_n)(1 + a_{n+1}) \cr
        &\le 1 + 2 (s_n + a_{n+1}) \cr 
        &= 1 + 2 s_{n+1}
\end{align*}

Hence, if $t_n \le 1 + 2 s_n$, then
$t_{n+1} \le 1 + 2 s_{n+1}$ as well.  By induction, we conclude that
$t_n \le 1 + 2 s_n$ for all $n$.

Thus, for all $n$, we have $t_n \le 1 + 2 s_n \le 1 + 2S$.  Substituting
this inequality for the inequality $t_n \le e^{s_n} \le e^S$ in the
parent entry, the rest of the proof proceeds in exactly the same manner.
%%%%%
%%%%%
\end{document}
