\documentclass[12pt]{article}
\usepackage{pmmeta}
\pmcanonicalname{PrimeHarmonicSeries}
\pmcreated{2013-03-22 15:07:16}
\pmmodified{2013-03-22 15:07:16}
\pmowner{Cosmin}{8605}
\pmmodifier{Cosmin}{8605}
\pmtitle{prime harmonic series}
\pmrecord{17}{36860}
\pmprivacy{1}
\pmauthor{Cosmin}{8605}
\pmtype{Theorem}
\pmcomment{trigger rebuild}
\pmclassification{msc}{40A05}
\pmclassification{msc}{11A41}
\pmsynonym{series of reciprocals of primes}{PrimeHarmonicSeries}
%\pmkeywords{prime series}
%\pmkeywords{prime sum}
%\pmkeywords{reciprocal prime}
%\pmkeywords{prime harmonic}
\pmrelated{Prime}
\pmrelated{HarmonicSeries}
\pmrelated{HarmonicNumber}
\pmrelated{Series}
\pmrelated{EulersConstant}
\pmrelated{HarmonicSeriesOfPrimes}

\endmetadata

% This is Cosmin's PlanetMath preamble.

% Packages
\usepackage{amssymb}
\usepackage{amsmath}
\usepackage{amsfonts}
\usepackage{graphicx}
\usepackage{amsthm}
\usepackage{mathrsfs}
%%%\usepackage{xypic}

% Theorem Environments
\newtheorem*{thm}{Theorem}
\newtheorem*{lem}{Lemma}
\newtheorem*{cor}{Corollary}

% New Commands
  %Sets
    \newcommand{\bbP}{\mathbb{P}}
    \newcommand{\bbN}{\mathbb{N}}
    \newcommand{\bbZ}{\mathbb{Z}}
    \newcommand{\bbQ}{\mathbb{Q}}
    \newcommand{\bbR}{\mathbb{R}}
    \newcommand{\bbC}{\mathbb{C}}
    \newcommand{\bbK}{\mathbb{K}}
    \newcommand{\bbB}{\mathbb{B}}
    \newcommand{\bbS}{\mathbb{S}}
    \newcommand{\bbA}{\mathbb{A}}
    \newcommand{\bbT}{\mathbb{T}}
  %Script and Cal Letters
    \newcommand{\scP}{\mathscr{P}}
    \newcommand{\scF}{\mathscr{F}}
    \newcommand{\scC}{\mathscr{C}}
    \newcommand{\scL}{\mathscr{L}}
    \newcommand{\caA}{\mathcal{A}}
    \newcommand{\caB}{\mathcal{B}}
    \newcommand{\caC}{\mathcal{C}}
    \newcommand{\caD}{\mathcal{D}}
    \newcommand{\caE}{\mathcal{E}}
    \newcommand{\caF}{\mathcal{F}}
    \newcommand{\caR}{\mathcal{R}}
    \newcommand{\caP}{\mathcal{P}}
    \newcommand{\caM}{\mathcal{M}}
    \newcommand{\caS}{\mathcal{S}}
    \newcommand{\caH}{\mathcal{H}}
    \newcommand{\caT}{\mathcal{T}}
    \newcommand{\caU}{\mathcal{U}}
    \newcommand{\caX}{\mathcal{X}}
    \newcommand{\caY}{\mathcal{Y}}
    \newcommand{\caZ}{\mathcal{Z}}
  %Other Commands
    \newcommand{\vect}[1]{\boldsymbol{#1}}
    \renewcommand{\div}{\!\mid\!}
\begin{document}
\PMlinkescapeword{infinite}
\PMlinkescapeword{constant}
\PMlinkescapeword{bound}
\PMlinkescapephrase{square-free part}
\PMlinkescapephrase{set of primes}
\PMlinkescapephrase{harmonic series}

The prime\footnote{\(\bbP\) denotes the set of 
primes.} harmonic series (also known as series of reciprocals of primes) is the infinite sum \(\sum_{p\in\bbP}\frac{1}{p}\). The following result was originally proved by Euler (using the Euler product of the Riemann Zeta function) but the following extremely elegant proof is due to Paul Erd\H{o}s \cite{Er}.

\begin{thm}
The series \( \sum_{p\in\bbP}\frac{1}{p} \) diverges.
\end{thm}
\begin{proof}
Assume that this series is convergent. If so, then, for a certain \( 
k\in\bbN \), we have:
\[ \sum_{i > k}\frac{1}{p_i} < \frac{1}{2}, \]
where \(p_i\) is the \(i^{\mathrm{th}}\) prime. Now, we define \( 
\lambda_k(n) := \#\{ m\leq n : p_i \div m \Rightarrow i \leq k \} \), the 
number of integers less than \(n\) divisible only by the first \(k\) 
primes. Any of these numbers can be expressed as 
\(m = p_1^{\omega_1} \dotsm p_k^{\omega_k} \cdot s^2,\; 
\omega_i\in\{0,1\} \) (i.e. a square multiplied by a square-free number). There are \(2^k\) ways to chose the square-free part and clearly \(s\leq 
\sqrt{n}\), so \( \lambda_k(n) \leq 2^k \sqrt{n}\). Now, the number of integers divisible by \(p_i\) less than \(n\) is \(\bigl\lfloor\frac{n}{p_i}\bigr\rfloor\), so the number of integers less than \(n\) divisible by primes bigger than \(p_k\) (which we shall denote by \( \Lambda_k(n) \)) is bounded above as follows:
\[\Lambda_k(n) \leq \sum_{i>k}\left\lfloor\frac{n}{p_i}\right\rfloor \leq \sum_{i>k}\frac{n}{p_i} < \frac{n}{2}.\]
However, by their definitions, \(\lambda_k(n)+\Lambda_k(n) = n\) for all \(n\in\bbN\) and so it is sufficient to find an \(n\) such that \(\lambda_k(n)\leq\frac{n}{2}\) for a contradiction and, using the previous bound for \( \lambda_k(n) \), which is \( \lambda_k(n) \leq 2^k \sqrt{n}\), we see that \( n = 2^{2k+2} \) works.
\end{proof}

The series is in some ways similar to the \PMlinkname{Harmonic series}{HarmonicSeries} \(H_n := \sum_{k = 1}^n \frac{1}{n}\). In fact, it is well known that \(H_n = \log n + \gamma + o(1) \), where \(\gamma = 0.577215664905\dots\) is Euler's constant, and this series obeys the similar asymptotic relation \(\sum_{k=1}^n\frac{1}{p_k}= \log\log n + C + o(1)\), where \(C = 0.26149721\dots\) and is sometimes called the Mertens constant. Its divergence, however, is extremely slow: for example, taking \(n\) as the biggest currently known prime, the \( 42^{\mathrm{nd}} \) Mersenne prime \(M_{25964951}\), we get \(\log\log(2^{25964951} - 1) + C \approx 16.9672\) (while \(H_{M_{25964951}}\approx 1.79975\cdot 10^{17} \) which is enormous considering \(H_n\)'s also slow divergence).

\begin{thebibliography}{2}
\bibitem{AZ} \textsc{M.~Aigner \& G.~M.~Ziegler}: \emph{Proofs from THE BOOK}, 3\(^\mathrm{rd}\) edition (2004), Springer-Verlag, 5--6.
\bibitem{Er} \textsc{P.~Erd\H{o}s}: \emph{\"Uber die Reihe \(\sum\frac{1}{p}\)}, Mathematica, Zutphen B 7 (1938).
\end{thebibliography}
%%%%%
%%%%%
\end{document}
