\documentclass[12pt]{article}
\usepackage{pmmeta}
\pmcanonicalname{UpcrossingsAndDowncrossings}
\pmcreated{2013-03-22 18:49:33}
\pmmodified{2013-03-22 18:49:33}
\pmowner{gel}{22282}
\pmmodifier{gel}{22282}
\pmtitle{upcrossings and downcrossings}
\pmrecord{5}{41629}
\pmprivacy{1}
\pmauthor{gel}{22282}
\pmtype{Definition}
\pmcomment{trigger rebuild}
\pmclassification{msc}{40A05}
\pmclassification{msc}{60G17}
%\pmkeywords{stochastic process}
\pmrelated{MartingaleConvergenceTheorem}
\pmrelated{ConvergenceOfASequenceWithFiniteUpcrossings}
\pmdefines{upcrossing}
\pmdefines{downcrossing}

% almost certainly you want these
\usepackage{amssymb}
\usepackage{amsmath}
\usepackage{amsfonts}

% used for TeXing text within eps files
%\usepackage{psfrag}
% need this for including graphics (\includegraphics)
\usepackage{graphicx,float}
% for neatly defining theorems and propositions
\usepackage{amsthm}
% making logically defined graphics
%%%\usepackage{xypic}

% there are many more packages, add them here as you need them

% define commands here
\newtheorem*{theorem*}{Theorem}
\newtheorem*{lemma*}{Lemma}
\newtheorem*{corollary*}{Corollary}
\newtheorem*{definition*}{Definition}
\newtheorem{theorem}{Theorem}
\newtheorem{lemma}{Lemma}
\newtheorem{corollary}{Corollary}
\newtheorem{definition}{Definition}

\begin{document}
\PMlinkescapeword{inequalities}
\PMlinkescapeword{number}
\PMlinkescapeword{theory}
\PMlinkescapeword{sequence}
\PMlinkescapeword{real numbers}
\PMlinkescapeword{limit}
\PMlinkescapeword{finiteness}
\PMlinkescapeword{regularity}
\PMlinkescapeword{applications}
\PMlinkescapeword{index set}
\PMlinkescapeword{subset}
\PMlinkescapeword{definitions}
\PMlinkescapeword{integer}
\PMlinkescapeword{interval}
\PMlinkescapeword{finite}
\PMlinkescapeword{infinite}
\PMlinkescapeword{terms}

Inequalities involving the number of times at which a stochastic process passes upwards or downwards through a bounded interval play an important role in the theory of stochastic processes. Whether or not a sequence of real numbers converges to a limit can be expressed in terms of the finiteness of the number of \emph{upcrossings} or \emph{downcrossings}, leading to results such as the martingale convergence theorem and regularity of martingale sample paths.
As the main applications are to stochastic processes, we suppose that $(X_t)_{t\in\mathbb{T}}$ is a real-valued stochastic process with totally ordered time index set $\mathbb{T}$, usually a subset of the real numbers. However, for the definitions here, it is enough to consider $X_t$ to be a sequence of real numbers, and the dependence on any underlying probability space is suppressed.

For real numbers $a<b$, the number of upcrossings of $X$ across the interval $[a,b]$ is the maximum nonnegative integer $n$ such that there exists times $s_k,t_k\in\mathbb{T}$ satisfying
\begin{equation}
s_1<t_1<s_2<t_2<\cdots<s_n<t_n\label{eq:1}
\end{equation}
and for which $X_{s_k}<a<b<X_{t_k}$. The number of upcrossings is denoted by $U[a,b]$. Note that this is either a nonnegative integer or is infinite. Similarly, the number of downcrossings, denoted by $D[a,b]$, is the maximum nonnegative integer $n$ such that there are times $s_k,t_k\in\mathbb{T}$ satisfying (\ref{eq:1}) and such that $X_{s_k}>b>a>X_{t_k}$.

\begin{figure}[H]
\centering
\includegraphics[scale=1.2,bb=71 519 364 714]{upcrossingspic}
\caption{Process with 3 upcrossings of the interval $[a,b]$.}
\label{fig:1}
\end{figure}

Note that between any two upcrossings there is a downcrossing. Given $s_k,t_k$ satisfying (\ref{eq:1}) and $X_{s_k}<a<b<X_{t_k}$, we can set $s^\prime_k=t_k$ and $t^\prime_k=s_{k+1}$ for $k=1,\ldots,n-1$. Then, $X_{s^\prime_k}>b>a>X_{t^\prime_k}$ from which we conclude that $D[a,b]\ge U[a,b]-1$. Similarly, we have $U[a,b]\ge D[a,b]-1$. Hence, the number of upcrossings and the number of downcrossings of an interval differ by at most $1$.

For a finite index set $\mathbb{T}=\{1,2,\ldots,N\}$, the number of upcrossings can be computed as follows. Set $t_0=0$ and define $s_1,s_2,\ldots$ and $t_1,t_2,\ldots$ by
\begin{align*}
&t_k=\inf\left\{t\in\mathbb{T}\colon t\ge s_k, X_t>b\right\}\in\mathbb{T}\cup\{\infty\}.\\
&s_k=\inf\left\{t\in\mathbb{T}\colon t\ge t_{k-1}, X_t<a\right\}\in\mathbb{T}\cup\{\infty\}.
\end{align*}
The number of upcrossings of $[a,b]$ is then equal to the maximum $n$ such that $t_n<\infty$ (see Figure \ref{fig:1}).

\begin{thebibliography}{9}
\bibitem{williams}
David Williams, \emph{Probability with martingales},
Cambridge Mathematical Textbooks, Cambridge University Press, 1991.
\end{thebibliography}

%%%%%
%%%%%
\end{document}
