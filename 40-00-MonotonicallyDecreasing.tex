\documentclass[12pt]{article}
\usepackage{pmmeta}
\pmcanonicalname{MonotonicallyDecreasing}
\pmcreated{2013-03-22 12:22:30}
\pmmodified{2013-03-22 12:22:30}
\pmowner{akrowne}{2}
\pmmodifier{akrowne}{2}
\pmtitle{monotonically decreasing}
\pmrecord{8}{32130}
\pmprivacy{1}
\pmauthor{akrowne}{2}
\pmtype{Definition}
\pmcomment{trigger rebuild}
\pmclassification{msc}{40-00}
\pmsynonym{monotone decreasing}{MonotonicallyDecreasing}
\pmsynonym{strictly decreasing}{MonotonicallyDecreasing}
\pmrelated{MonotonicallyIncreasing}

\usepackage{amssymb}
\usepackage{amsmath}
\usepackage{amsfonts}

%\usepackage{psfrag}
%\usepackage{graphicx}
%%%\usepackage{xypic}
\begin{document}
A sequence $(s_n)$ is \emph{monotonically decreasing} if 

$$ s_m < s_n \;\forall\; m > n $$

Similarly, a real function $f(x)$ is monotonically decreasing if

$$ f(x) < f(y) \;\forall\; x > y$$

Compare this to monotonically nonincreasing.

\textbf{Conflict note.} In other context, such as \cite{NIST}, this is called \emph{strictly decreasing}.  When this is the case, ``monotonically nonincreasing'' is instead called ``monotonically decreasing.'' 

\begin{thebibliography}{3}
\bibitem{NIST}  ``\PMlinkexternal{strictly decreasing}{http://www.nist.gov/dads/HTML/strictlyDecreasing.html},'' from the NIST Dictionary of Algorithms and Data Structures, Paul E. Black, ed.
\end{thebibliography}
%%%%%
%%%%%
\end{document}
