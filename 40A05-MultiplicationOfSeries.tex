\documentclass[12pt]{article}
\usepackage{pmmeta}
\pmcanonicalname{MultiplicationOfSeries}
\pmcreated{2014-10-31 20:23:29}
\pmmodified{2014-10-31 20:23:29}
\pmowner{pahio}{2872}
\pmmodifier{pahio}{2872}
\pmtitle{multiplication of series}
\pmrecord{21}{37431}
\pmprivacy{1}
\pmauthor{pahio}{2872}
\pmtype{Theorem}
\pmcomment{trigger rebuild}
\pmclassification{msc}{40A05}
\pmsynonym{Cauchy multiplication rule}{MultiplicationOfSeries}
\pmrelated{ManipulatingConvergentSeries}
\pmrelated{AlternatingHarmonicSeries}
\pmrelated{ErnstLindelof}

\endmetadata

% this is the default PlanetMath preamble.  as your knowledge
% of TeX increases, you will probably want to edit this, but
% it should be fine as is for beginners.

% almost certainly you want these
\usepackage{amssymb}
\usepackage{amsmath}
\usepackage{amsfonts}

% used for TeXing text within eps files
%\usepackage{psfrag}
% need this for including graphics (\includegraphics)
%\usepackage{graphicx}
% for neatly defining theorems and propositions
 \usepackage{amsthm}
% making logically defined graphics
%%%\usepackage{xypic}

% there are many more packages, add them here as you need them

% define commands here

\theoremstyle{definition}
\newtheorem*{thmplain}{Theorem}
\begin{document}
\textbf{Theorem} (Franz Mertens).
If the series $\sum_{k=1}^\infty a_k$ and $\sum_{k=1}^\infty b_k$ with real or complex \PMlinkescapetext{terms} converge and have the \PMlinkname{sums}{SumOfSeries} $A$ and $B$, respectively, and at least one of them converges absolutely, then also the series
\begin{align}
a_1b_1\!+\!(a_1b_2\!+\!a_2b_1)\!+\!(a_1b_3\!+\!a_2b_2\!+\!a_3b_1)\!+\cdots
\end{align}
is convergent and its \PMlinkescapetext{sum} is equal to $AB$.

{\em Proof.}\, Denote the partial sums of the series\, $A_n := a_1+a_2+\cdots+a_n$,\, $B_n := b_1+b_2+\cdots+b_n$\, and\, 
$s_n := a_1b_1+(a_1b_2+a_2b_1)+(a_1b_3+a_2b_2+a_3b_1)+\cdots
+(a_1b_n+a_2b_{n-1}+\cdots+a_nb_1)$\, for each $n$.\, Then we have\, $\lim_{n\to\infty}A_n = A$\, and\, $\lim_{n\to\infty}B_n = B$.\, Suppose that e.g. the series $\sum a_n$ converges absolutely and that at least one $a_n$ is distinct from zero; so the \PMlinkescapetext{sum}\, $\sum_{n=1}^\infty|a_n|$ is a real positive number $M$.\, Let $\varepsilon$ be an arbitrary positive number.

Now we can write the identities

$AB = (A-A_n)B+a_1B+a_2B+\cdots+a_nB$,

$s_n = a_1(b_1+b_2+\cdots+b_n)+a_2(b_1+b_2+\cdots+b_{n-1})+\cdots+a_nb_1 =
  a_1B_n+a_2B_{n-1}+\cdots+a_nB_1$,

$AB\!-\!s_n = (A-A_n)B+a_1(B-B_n)+a_2(B-B_{n-1})+\cdots+a_n(B-B_1)$\\
$= (A-A_n)B+[a_1(B-B_n)+a_2(B-B_{n-1})+\cdots+a_k(B-B_{n-k+1})]\\
+a_{k+1}(B-B_{n-k})+\cdots+a_n(B-B_1)$.

There is a positive number $n_1(\varepsilon)$ such that\,  
$|A\!-\!A_n| < \frac{\varepsilon}{3(|B|+1)}$\, when\, $n > n_1(\varepsilon)$.\, Then
\begin{align}
|(A\!-\!A_n)B| = |A\!-\!A_n|\cdot|B| < \frac{\varepsilon}{3(|B|\!+\!1)}(|B|\!+\!1) = \frac{\varepsilon}{3}.
\end{align}
The convergence of $\sum b_n$ implies that there is a number $n_2(\varepsilon)$ such that\, $|B\!-\!B_n| < \frac{\varepsilon}{3M}$\, when\, 
$n > n_2(\varepsilon)$.\, Thus we have
\begin{align}
|[\ldots]| \leq 
|a_1|\!\cdot\!|B\!-\!B_n|\!+\cdots+\!|a_k|\!\cdot\!|B\!-\!B_{n-k+1}| < 
(|a_1|\!+\cdots+\!|a_k|)\frac{\varepsilon}{3M} \leq M\!\cdot\!\frac{\varepsilon}{3M} = \frac{\varepsilon}{3}
\end{align}
if\, $n\!-\!k\!+\!1 > n_2(\varepsilon)$.\, Because\, $\lim_{n\to\infty}B_n = B$,\, the numbers $|B_n|$ are bounded, i.e. there is a positive number $K$ such that for each $j$ we have\, $|B_j| < K$\, and consequently\, $|B| \leq K$.\, It follows that\, $|B\!-\!B_j| \leq |B|\!+\!|B_j| < K\!+\!K = 2K$\, for every $j$.\, We apply Cauchy criterion for convergence to the series $\sum_{n=1}^\infty|a_n|$ getting a number $n_3(\varepsilon)$ such that for each $m$, one has the inequality\, $|a_{k+1}|+\cdots+|a_m| < \frac{\varepsilon}{6K}$\, if\, 
$k > n_3(\varepsilon)$.\, Accordingly we obtain the estimation
\begin{align}
|a_{k+1}(B\!-\!B_{n-k})\!+\cdots+\!a_n(B\!-\!B_1)| \leq |a_{k+1}||B\!-\!B_{n-k}|\!+\cdots+\!|a_n||B\!-\!B_1| < 2K\!\cdot\!\frac{\varepsilon}{6K} = \frac{\varepsilon}{3}
\end{align}
which is valid when\, $k > n_3(\varepsilon)$.

If we choose\, $n > \max\{n_1(\varepsilon\},\, n_2(\varepsilon)\!+\!n_3(\varepsilon)\}$\, and $k$ such that\, $n \geq k > n_3(\varepsilon)\!+\!1$,\, then the inequalities (2), (3) and (4) are satisfied, ensuring that
$$|AB\!-\!s_n| < 
\frac{\varepsilon}{3}\!+\!\frac{\varepsilon}{3}\!+\!\frac{\varepsilon}{3} = \varepsilon.$$
This means that the assertion of the theorem has been proved.\\


\textbf{Remark.}\, The mere convergence of both series does not 
suffice for convergence of (1).\, This is seen in the following 
example by Cauchy where both series are 
  $$1\!-\!\frac{1}{\sqrt{2}}\!+\!\frac{1}{\sqrt{3}}\!-+\cdots$$
They converge by virtue of Leibniz test, but not absolutely (see the \PMlinkname{$p$-test}{PTest}).\, In their product series
 $$1\!-\!(\frac{1}{\sqrt{2}}\!+\!\frac{1}{\sqrt{2}})\!+\!(\frac{1}{\sqrt{3}}\!+\!\frac{1}{\sqrt{2}}\frac{1}{\sqrt{2}}\!+\!\frac{1}{\sqrt{3}})\!-+\cdots$$
the absolute value of the $n^\mathrm{th}$ \PMlinkescapetext{term} is\,  $1\!\cdot\!\frac{1}{\sqrt{n}}\!+\!\frac{1}{\sqrt{2}}\frac{1}{\sqrt{n-1}}\!+\cdots
+\frac{1}{\sqrt{n}}\!\cdot1\!$,\, having $n$ summands which all are greater than $\frac{2}{n+1}$ (this is seen when one looks at the half circle\, $y = \sqrt{x}\sqrt{n\!+\!1\!-\!x}$\, or\, $(x\!-\!\frac{n+1}{2})^2\!+\!y^2 = (\frac{n+1}{2})^2$,\, which shows that\, $y \leq \frac{n+1}{2}$\, and thus\, $\frac{1}{y} \geq \frac{2}{n+1}$).\, Because\, $\lim_{n\to\infty}n\cdot\frac{2}{n+1} = 2 \neq 0$, the product series does not satisfy the \PMlinkname{necessary condition of convergence}{ThenA_kto0IfSum_k1inftyA_kConverges} and therefore the series diverges.


\begin{thebibliography}{8}
\bibitem{lindelof}{\sc Ernst Lindel\"of:} {\it Johdatus funktioteoriaan}. \,Mercatorin Kirjapaino Osakeyhti\"o. Helsinki (1936).
\end{thebibliography}
%%%%%
%%%%%
\end{document}
