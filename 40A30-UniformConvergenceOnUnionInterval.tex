\documentclass[12pt]{article}
\usepackage{pmmeta}
\pmcanonicalname{UniformConvergenceOnUnionInterval}
\pmcreated{2013-03-22 17:27:09}
\pmmodified{2013-03-22 17:27:09}
\pmowner{pahio}{2872}
\pmmodifier{pahio}{2872}
\pmtitle{uniform convergence on union interval}
\pmrecord{6}{39834}
\pmprivacy{1}
\pmauthor{pahio}{2872}
\pmtype{Theorem}
\pmcomment{trigger rebuild}
\pmclassification{msc}{40A30}
\pmrelated{MinimalAndMaximalNumber}

\endmetadata

% this is the default PlanetMath preamble.  as your knowledge
% of TeX increases, you will probably want to edit this, but
% it should be fine as is for beginners.

% almost certainly you want these
\usepackage{amssymb}
\usepackage{amsmath}
\usepackage{amsfonts}

% used for TeXing text within eps files
%\usepackage{psfrag}
% need this for including graphics (\includegraphics)
%\usepackage{graphicx}
% for neatly defining theorems and propositions
 \usepackage{amsthm}
% making logically defined graphics
%%%\usepackage{xypic}

% there are many more packages, add them here as you need them

% define commands here

\theoremstyle{definition}
\newtheorem*{thmplain}{Theorem}

\begin{document}
\textbf{Theorem.}  If\, $a < b < c$\, and the sequence \,$f_1,\,f_2,\,f_3,\,\ldots$\, of real functions converges uniformly both on the interval \,$[a,\,b]$\, and on the interval\, $[b,\,c]$,\, then the function sequence converges uniformly also on the \PMlinkname{union}{Union} interval\, $[a,\,c]$.

{\em Proof.}  We have the limit functions $\displaystyle f_{ab} := \lim_{n\to\infty}f_n$\, on\, $[a,\,b]$\, and\, 
$\displaystyle f_{bc} := \lim_{n\to\infty}f_n$.  It follows that
$$f_{ab}(b) = \lim_{n\to\infty}f_n(b) = f_{bc}(b).$$
Define the new function
\begin{align*}
f(x) :=
\begin{cases}
f_{ab}(x) \quad \forall x\,\in [a,\,b],\\
f_{bc}(x) \quad \forall x\,\in [b,\,c].
\end{cases}
\end{align*}
Choose an arbitrary positive number $\varepsilon$.  The supposed uniform convergences on the intervals\, $[a,\,b]$\, and\, $[b,\,c]$\, imply the existence of the numbers $n_1(\varepsilon)$ and $n_2(\varepsilon)$ such that
$$|f_n(x)-f(x)| < \varepsilon\;\;\forall x\,\in [a,\,b],\quad\mbox{when}\; n > n_1(\varepsilon)$$
and 
$$|f_n(x)-f(x)| < \varepsilon\;\;\forall x\,\in [b,\,c],\quad\mbox{when}\; n > n_2(\varepsilon).$$
If one takes\, $n > \max\{n_1(\varepsilon),\,n_2(\varepsilon)\}$,\, then one has simultaneously on both intervals\, $[a,\,b]$\, and\, $[b,\,c]$,\, i.e. on the whole greater interval\, $[a,\,c]$,\, the condition
$$|f_n(x)-f(x)| < \varepsilon.$$




%%%%%
%%%%%
\end{document}
