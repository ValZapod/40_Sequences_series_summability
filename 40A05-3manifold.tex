\documentclass[12pt]{article}
\usepackage{pmmeta}
\pmcanonicalname{3manifold}
\pmcreated{2013-03-11 19:23:28}
\pmmodified{2013-03-11 19:23:28}
\pmowner{juanman}{12619}
\pmmodifier{}{0}
\pmtitle{3-manifold}
\pmrecord{1}{50063}
\pmprivacy{1}
\pmauthor{juanman}{0}
\pmtype{Definition}

\endmetadata

%none for now
\begin{document}
\documentclass{article}

\begin{document}
A 3 dimensional manifold is a topological space which is locally homeomorphic to the euclidean space ${\bf R}^3$.

One can see from simple constructions the great variety of objects that indicate that they are worth to study. 

First without boundary:
\begin{enumerate}
\item For example, with the cartesian product we can get:
\begin{itemize}
\item $S^2\times S^1$
\item ${\bf R}P^2\times S^1$
\item $T\times S^1$
\item ...
\end{itemize}

\item Also by the generalization of the cartesian product: \emph{fiber bundles}, one can build bundles $E$ of the type
$$F\subset E\to S^1$$
where $F$ is any closed surface. 

\item Or interchanging the roles, bundles as:
$$S^1\subset E\to F$$

For the second type it is known that for each \emph{isotopy class} $[\phi]$ of maps $F\to F$ correspond to an unique bundle $E_{\phi}$. Any homeomorphism $f:F\to F$ representing the isotopy class $[\phi]$ is called a \emph{monodromy} for $E_{\phi}$.

From the previuos paragraph we infer that the \emph{mapping class group} play a important role inthe understanding at least for this subclass of objets.

For the third class above one can use an \emph{orbifold} instead of a simple surface to get a class of 3-manifolds called \emph{Seifert fiber spaces} which are a large class of spaces needed to understand the modern classifications for 3-manifolds.


{\bf References}
\begin{itemize}
\item [[G]] J.C. G\'omez-Larra\~naga. {\it 3-manifolds which are unions of three solid tori},
Manuscripta Math. 59 (1987), 325-330.
\item [[GGH]] J.C. G\'omez-Larra\~naga, F.J. Gonz\'alez-Acu\~na, J. Hoste. 
{\it Minimal Atlases on 3-manifolds},
Math. Proc. Camb. Phil. Soc. 109 (1991), 105-115.
\item [[H]] J. Hempel. {\it 3-manifolds}, Princeton University Press 1976.
\item [[O]] P. Orlik. {\it Seifert Manifolds}, Lecture Notes in Math. 291, 
1972 Springer-Verlag.
\item [[S]] P. Scott. {\it The geometry of 3-manifolds}, Bull. London Math. Soc. 15 (1983), 401-487.
\end{itemize}
   
\end{enumerate}




\end{document}
%%%%%
\end{document}
