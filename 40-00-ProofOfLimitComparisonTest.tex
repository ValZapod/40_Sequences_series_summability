\documentclass[12pt]{article}
\usepackage{pmmeta}
\pmcanonicalname{ProofOfLimitComparisonTest}
\pmcreated{2013-03-22 15:35:54}
\pmmodified{2013-03-22 15:35:54}
\pmowner{cvalente}{11260}
\pmmodifier{cvalente}{11260}
\pmtitle{proof of limit comparison test}
\pmrecord{4}{37513}
\pmprivacy{1}
\pmauthor{cvalente}{11260}
\pmtype{Proof}
\pmcomment{trigger rebuild}
\pmclassification{msc}{40-00}
%\pmkeywords{series}
%\pmkeywords{comparisson test}
%\pmkeywords{limit}
%\pmkeywords{finite}
%\pmkeywords{convergence}

\endmetadata

% this is the default PlanetMath preamble.  as your knowledge
% of TeX increases, you will probably want to edit this, but
% it should be fine as is for beginners.

% almost certainly you want these
\usepackage{amssymb}
\usepackage{amsmath}
\usepackage{amsfonts}

% used for TeXing text within eps files
%\usepackage{psfrag}
% need this for including graphics (\includegraphics)
%\usepackage{graphicx}
% for neatly defining theorems and propositions
%\usepackage{amsthm}
% making logically defined graphics
%%%\usepackage{xypic}

% there are many more packages, add them here as you need them

% define commands here
\begin{document}
The main theorem we will use is the comparison test, which basically states that if $a_n>0$, $b_n>0$ and there is an $N$ such that for all $n>N$, $a_n < b_n$ , then if $\sum_{i=1}^\infty b_n$ converges so will $\sum_{i=1}^\infty a_n$.

Suppose $\lim_{n\to \infty} \frac{a_n}{b_n} = L$ where $L$ can be a non negative real number or $+\infty$.

By definition, for $L$ finite, this means that for every $\epsilon>0$ there is a natural number $n_\epsilon$ such that for all $n > n_\epsilon$, $\left\| \frac{a_n}{b_n} -L \right \| < \epsilon$

To make matters more concrete choose $\epsilon = \frac{L}{2}$ and assume $L\ne0$ and finite.

$0< a_n < \frac{3L}{2} b_n $, for all $n > n_{\frac{L}{2}}$.

If $\sum_{i=1}^\infty b_n$ converges, so will $\sum_{i=1}^\infty \frac{3L}{2} b_n$ and thus by the comparison test, $\sum_{i=1}^\infty a_n$ will also be convergent.

For the reverse result, consider $\lim_{n \to \infty} \frac{b_n}{a_n} = \frac{1}{L}$, since if $L$ is finite so will $\frac{1}{L}$, applying the previous result we can say that if $\sum_{i=1}^\infty a_n$ converges so will $\sum_{i=1}^\infty b_n$

Consider the case $L=0$, clearly $L=0^+$ since both $a_n$ and $b_n$ are positive, this means that for all $\epsilon > 0$ there exists $n_\epsilon$ such that for all $n>n_\epsilon$, $0<a_n<\epsilon b_n$.

Considering $\epsilon=1$ we get the exact formulation of the comparison test, so if $\sum_{i=1}^\infty b_n$ converges so will $\sum_{i=1}^\infty a_n$.

For the case $L=+\infty$ just apply the result to $\lim_{n \to \infty} \frac{b_n}{a_n} = 0$ to conclude that if $\sum_{i=1}^\infty a_n$ converges so will $\sum_{i=1}^\infty b_n$
%%%%%
%%%%%
\end{document}
