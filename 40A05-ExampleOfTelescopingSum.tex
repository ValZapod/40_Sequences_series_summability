\documentclass[12pt]{article}
\usepackage{pmmeta}
\pmcanonicalname{ExampleOfTelescopingSum}
\pmcreated{2013-03-22 17:27:21}
\pmmodified{2013-03-22 17:27:21}
\pmowner{pahio}{2872}
\pmmodifier{pahio}{2872}
\pmtitle{example of telescoping sum}
\pmrecord{11}{39838}
\pmprivacy{1}
\pmauthor{pahio}{2872}
\pmtype{Example}
\pmcomment{trigger rebuild}
\pmclassification{msc}{40A05}
\pmrelated{GoniometricFormulae}
\pmrelated{ExampleOfSummationByParts}
\pmrelated{DirchletKernel}

% this is the default PlanetMath preamble.  as your knowledge
% of TeX increases, you will probably want to edit this, but
% it should be fine as is for beginners.

% almost certainly you want these
\usepackage{amssymb}
\usepackage{amsmath}
\usepackage{amsfonts}

% used for TeXing text within eps files
%\usepackage{psfrag}
% need this for including graphics (\includegraphics)
%\usepackage{graphicx}
% for neatly defining theorems and propositions
 \usepackage{amsthm}
 \usepackage[T2A]{fontenc}
 \usepackage[russian, english]{babel}

% making logically defined graphics
%%%\usepackage{xypic}

% there are many more packages, add them here as you need them

% define commands here

\theoremstyle{definition}
\newtheorem*{thmplain}{Theorem}
\begin{document}
\PMlinkescapeword{formula}
Some trigonometric sums, as $\sum_{k=1}^n\cos{k\alpha}$ and $\sum_{k=1}^n\sin{k\alpha}$, may be telescoped if the terms are first edited by a suitable \PMlinkname{goniometric formula}{GoniometricFormulae} (``product formula'').  E.g. we may write:
$$\sum_{k=1}^n\cos{k\alpha} \;=\; 
\frac{1}{\sin\frac{\alpha}{2}}\sum_{k=1}^n\cos{k\alpha}\sin\frac{\alpha}{2}$$
The product formula \,$\cos{x}\sin{y} \,=\, \frac{1}{2}[\sin(x\!+\!y)-\sin(x\!-\!y)]$\, alters this to
$$\sum_{k=1}^n\cos{k\alpha} \;=\;
\frac{1}{2\sin\frac{\alpha}{2}}\sum_{k=1}^n\left(\sin\frac{(2k\!+\!1)\alpha}{2}-\sin\frac{(2k\!-\!1)\alpha}{2}\right),$$
or
$$\sum_{k=1}^n\cos{k\alpha} \;=\; \frac{1}{2\sin\frac{\alpha}{2}}\left(\sin\frac{3\alpha}{2}-\sin\frac{\alpha}{2}+\sin\frac{5\alpha}{2}
-\sin\frac{3\alpha}{2}+-\ldots+\sin\frac{(2n\!+\!1)\alpha}{2}-\sin\frac{(2n\!-\!1)\alpha}{2}\right).$$
After cancelling the opposite numbers we obtain the formula
\begin{align}
\sum_{k=1}^n\cos{k\alpha} \;=\; \frac{\sin\frac{(2n+1)\alpha}{2}-\sin\frac{\alpha}{2}}{2\sin\frac{\alpha}{2}}.
\end{align}
The corresponding formula
\begin{align}
\sum_{k=1}^n\sin{k\alpha} \;=\; \frac{-\cos\frac{(2n+1)\alpha}{2}+\cos\frac{\alpha}{2}}{2\sin\frac{\alpha}{2}}.
\end{align}
is derived analogously.\\

\textbf{Note.}\, The formulae (1) and (2) are gotten also by adding the left side of the former and $i$ times the left side of the latter and then applying de Moivre identity.

\begin{thebibliography}{7}
\bibitem{LDK} \CYRL. \CYRD. \CYRK\cyru\cyrd\cyrr\cyrya\cyrv\cyrc\cyre\cyrv: 
{\em \CYRM\cyra\cyrt\cyre\cyrm\cyra\cyrt\cyri\cyrch\cyre\cyrs\cyrk\cyri\cyrishrt \,\cyra\cyrn\cyra\cyrl\cyri\cyrz. II \cyrt\cyro\cyrm}. \,\CYRI\cyrz\cyrd\cyra\cyrt\cyre\cyrl\cyrsftsn\cyrs\cyrt\cyrv\cyro \,
``\CYRV\cyrery\cyrs\cyrsh\cyra\cyrya \,\cyrsh\cyrk\cyro\cyrl\cyra''. \CYRM\cyro\cyrs\cyrk\cyrv\cyra \,(1970).
\end{thebibliography}
%%%%%
%%%%%
\end{document}
