\documentclass[12pt]{article}
\usepackage{pmmeta}
\pmcanonicalname{MethodsOfEvaluatingImproperIntegrals}
\pmcreated{2014-11-07 19:12:16}
\pmmodified{2014-11-07 19:12:16}
\pmowner{pahio}{2872}
\pmmodifier{pahio}{2872}
\pmtitle{methods of evaluating improper integrals}
\pmrecord{34}{41520}
\pmprivacy{1}
\pmauthor{pahio}{2872}
\pmtype{Application}
\pmcomment{trigger rebuild}
\pmclassification{msc}{40A10}
\pmrelated{IntegrationOfLaplaceTransformWithRespectToParameter}
\pmrelated{ListOfImproperIntegrals}
\pmrelated{IntegralRelatedToArcSine}
\pmrelated{ExampleOfChangingVariable}

% this is the default PlanetMath preamble.  as your knowledge
% of TeX increases, you will probably want to edit this, but
% it should be fine as is for beginners.

% almost certainly you want these
\usepackage{amssymb}
\usepackage{amsmath}
\usepackage{amsfonts}

% used for TeXing text within eps files
%\usepackage{psfrag}
% need this for including graphics (\includegraphics)
%\usepackage{graphicx}
% for neatly defining theorems and propositions
 \usepackage{amsthm}
% making logically defined graphics
%%%\usepackage{xypic}

% there are many more packages, add them here as you need them

% define commands here

\theoremstyle{definition}
\newtheorem*{thmplain}{Theorem}

\begin{document}
\PMlinkescapeword{place}

There are some general methods of evaluating improper integrals in such cases when one cannot directly use the antiderivative of the integrand.\, Which method is suitable in a certain instance, is dependent on the kind of the \PMlinkname{integral}{DefiniteIntegral}.\\

\begin{itemize}
\item Differentiation under the integral sign with respect to a 
parametre in the integrand; one can add a new parametre to a 
suitable place.\, The differentiated form may then be 
integrated directly or from a differential equation.\, Examples: 
\PMlinkname{a}{twoimproperintegrals}, 
\PMlinkname{b}{generalisationofgaussianintegral}, 
\PMlinkname{c}{relativeofexponentialintegral}, 
\PMlinkname{d}{integralrelatedtoarcsine}, 
\PMlinkid{e}{11617}.
\item \PMlinkname{Laplace transform}{laplacetransform}.\, If the 
integrand has, as above, a parametre in a suitable place, the 
Laplace transform of the integrand with respect to this 
parametre is often simpler to integrate and the new improper 
integral to evaluate; thereafter one simply 
\PMlinkescapetext{transforms} inversely.\, Examples: 
\PMlinkname{f}{sineintegralatinfinity}, 
\PMlinkname{g}{usingconvolutiontofindlaplacetransform}, 
\PMlinkname{h}{relativeofcosineintegral}, 
\PMlinkname{i}{laplacetransformoftnft}, 
\PMlinkid{j}{10637}.
\item Cauchy residue theorem.\, The integral may be obtained as 
limit of a contour integral in the complex plane.\, Examples: 
\PMlinkname{k}{usingresiduetheoremnearbranchpoint}, 
\PMlinkname{l}{fresnelformulas}, 
\PMlinkname{m}{laplaceintegrals}, 
\PMlinkid{n}{11489}.
\item Expanding the integrand to series.\, Example: 
\PMlinkname{o}{applicationoflogarithmseries}.
\item 
\PMlinkname{Changing variable}{ChangeOfVariableInDefiniteIntegral} in an improper integral sometimes may recur it to a known improper integral.\, Examples: 
\PMlinkname{p}{areaundergaussiancurve}, 
\PMlinkname{q}{exampleofimproperintegral},
\PMlinkname{r}{exampleofchangingvariable}.
\end{itemize}







%%%%%
%%%%%
\end{document}
