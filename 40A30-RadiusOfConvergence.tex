\documentclass[12pt]{article}
\usepackage{pmmeta}
\pmcanonicalname{RadiusOfConvergence}
\pmcreated{2013-03-22 12:32:59}
\pmmodified{2013-03-22 12:32:59}
\pmowner{PrimeFan}{13766}
\pmmodifier{PrimeFan}{13766}
\pmtitle{radius of convergence}
\pmrecord{13}{32794}
\pmprivacy{1}
\pmauthor{PrimeFan}{13766}
\pmtype{Theorem}
\pmcomment{trigger rebuild}
\pmclassification{msc}{40A30}
\pmclassification{msc}{30B10}
\pmsynonym{Abel's theorem on power series}{RadiusOfConvergence}
\pmrelated{ExampleOfAnalyticContinuation}
\pmrelated{NielsHenrikAbel}

% this is the default PlanetMath preamble.  as your knowledge
% of TeX increases, you will probably want to edit this, but
% it should be fine as is for beginners.

% almost certainly you want these
\usepackage{amssymb}
\usepackage{amsmath}
\usepackage{amsfonts}

% used for TeXing text within eps files
%\usepackage{psfrag}
% need this for including graphics (\includegraphics)
%\usepackage{graphicx}
% for neatly defining theorems and propositions
%\usepackage{amsthm}
% making logically defined graphics
%%%\usepackage{xypic}

% there are many more packages, add them here as you need them

% define commands here
\begin{document}
To the power series
\begin{equation}
\sum_{k=0}^{\infty}a_k(x-x_0)^k
\end{equation}
there exists a number $r\in [0,\infty]$, its \emph{radius of convergence}, such that the series converges absolutely for all (real or complex) numbers $x$ with $|x-x_0|<r$ and diverges whenever $|x-x_0|>r$. This is known as {\em Abel's theorem on power series}. (For $|x-x_0|= r$ no general statements can be made.)

The radius of convergence is given by:
\begin{equation}
r=\liminf_{k\to\infty}\frac{1}{\sqrt[k]{|a_k|}}
\end{equation}
and can also be computed as
\begin{equation}
r=\lim_{k\to\infty}\left|\frac{a_k}{a_{k+1}}\right|,
\end{equation}
if this limit exists.

It follows from the \PMlinkname{Weierstrass $M$-test}{WeierstrassMTest} that for any radius $r'$ smaller than the radius of convergence, the power series converges uniformly within the closed disk of radius $r'$.
%%%%%
%%%%%
\end{document}
