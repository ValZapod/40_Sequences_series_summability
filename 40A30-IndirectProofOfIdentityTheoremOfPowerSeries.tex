\documentclass[12pt]{article}
\usepackage{pmmeta}
\pmcanonicalname{IndirectProofOfIdentityTheoremOfPowerSeries}
\pmcreated{2013-03-22 16:47:48}
\pmmodified{2013-03-22 16:47:48}
\pmowner{pahio}{2872}
\pmmodifier{pahio}{2872}
\pmtitle{indirect proof of identity theorem of power series}
\pmrecord{9}{39030}
\pmprivacy{1}
\pmauthor{pahio}{2872}
\pmtype{Proof}
\pmcomment{trigger rebuild}
\pmclassification{msc}{40A30}
\pmclassification{msc}{30B10}
%\pmkeywords{proof by contradiction}

% this is the default PlanetMath preamble.  as your knowledge
% of TeX increases, you will probably want to edit this, but
% it should be fine as is for beginners.

% almost certainly you want these
\usepackage{amssymb}
\usepackage{amsmath}
\usepackage{amsfonts}

% used for TeXing text within eps files
%\usepackage{psfrag}
% need this for including graphics (\includegraphics)
%\usepackage{graphicx}
% for neatly defining theorems and propositions
 \usepackage{amsthm}
% making logically defined graphics
%%%\usepackage{xypic}

% there are many more packages, add them here as you need them

% define commands here

\theoremstyle{definition}
\newtheorem*{thmplain}{Theorem}

\begin{document}
\begin{align}
\sum_{n=0}^\infty a_n(z-z_0)^n \;=\; \sum_{n=0}^\infty b_n(z-z_0)^n
\end{align}
is valid in the set of points $z$ presumed in the \PMlinkname{theorem}{IdentityTheoremOfPowerSeries} to be proved.

Antithesis:\, There are integers $n$ such that\, $a_n \neq b_n$;\, let $\nu$ ($\geqq 0$) be least of them.

We can choose from the point set an infinite sequence\, $z_1,\,z_2,\,z_3,\,\ldots$\, which converges to $z_0$ with\, $z_n \neq z_0$\, for every $n$.\, Let $z$ in the equation (1) belong to\, 
$\{z_1,\,z_2,\,z_3,\,\ldots\}$\, and let's divide both \PMlinkescapetext{sides} of (1) by $(z-z_0)^{\nu}$ which is distinct from zero; we then have
\begin{align}
\underbrace{a_{\nu}+a_{\nu+1}(z-z_0)+a_{\nu+2}(z-z_0)^2+\ldots}_{f(z)}\, 
=\, \underbrace{b_{\nu}+b_{\nu+1}(z-z_0)+b_{\nu+2}(z-z_0)^2+\ldots}_{g(z)}
\end{align}
Let here $z$ to tend $z_0$ along the points $z_1,\,z_2,\,z_3,\,\ldots$, i.e. we take the limits $\lim_{n\to\infty}f(z_n)$ and $\lim_{n\to\infty}g(z_n)$.\, Because the sum of power series is always a continuous function, we see that in (2),
$$\mathrm{left\,side\,}\longrightarrow f(z_0) = a_{\nu}\quad\mathrm{and}\quad
\mathrm{right\,side\,}\longrightarrow g(z_0) = b_{\nu}$$
But all the time, the left and \PMlinkescapetext{right side} of (2) are equal, and thus also the limits.\, So we must have\, $a_{\nu} = b_{\nu}$,\, contrary to the antithesis.\, We conclude that the antithesis is wrong.\, This settles the proof.

\textbf{Note.}\, I learned this proof from my venerable teacher, the number-theorist Kustaa Inkeri (1908--1997).

%%%%%
%%%%%
\end{document}
