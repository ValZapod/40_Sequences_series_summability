\documentclass[12pt]{article}
\usepackage{pmmeta}
\pmcanonicalname{TopicOnTheAlgebraicFoundationsOfQuantumAlgebraicTopology12}
\pmcreated{2013-03-11 19:51:44}
\pmmodified{2013-03-11 19:51:44}
\pmowner{bci1}{20947}
\pmmodifier{}{0}
\pmtitle{topic on the Algebraic Foundations of Quantum Algebraic Topology}
\pmrecord{1}{50150}
\pmprivacy{1}
\pmauthor{bci1}{0}
\pmtype{Definition}

\endmetadata

%none for now
\begin{document}
%% Title: Quantum Algebraic Topology Foundations

%%I. C. Baianu, J. F. Glazebrook and R. Brown
%% format:  LaTeX2e
%% Aquantiz25.tex


\documentclass[11pt]{amsart}
\usepackage{amsmath, amssymb, amsfonts, amsthm, amscd, latexsym, enumerate}
\usepackage{xypic}
\usepackage[mathscr]{eucal}

\setlength{\textwidth}{6.5in}
%\setlength{\textwidth}{16cm}
\setlength{\textheight}{9.0in}
%\setlength{\textheight}{24cm}

\hoffset=-.75in     %%ps format
%\hoffset=-1.0in     %%hp format
\voffset=-.4in


\theoremstyle{plain}
\newtheorem{lemma}{Lemma}[section]
\newtheorem{proposition}{Proposition}[section]
\newtheorem{theorem}{Theorem}[section]
\newtheorem{corollary}{Corollary}[section]

\theoremstyle{definition}
\newtheorem{definition}{Definition}[section]
\newtheorem{example}{Example}[section]
%\theoremstyle{remark}
\newtheorem{remark}{Remark}[section]
\newtheorem*{notation}{Notation}
\newtheorem*{claim}{Claim}

\renewcommand{\thefootnote}{\ensuremath{\fnsymbol{footnote}}}
\numberwithin{equation}{section}

\newcommand{\Ad}{{\rm Ad}}
\newcommand{\Aut}{{\rm Aut}}
\newcommand{\Cl}{{\rm Cl}}
\newcommand{\Co}{{\rm Co}}
\newcommand{\DES}{{\rm DES}}
\newcommand{\Diff}{{\rm Diff}}
\newcommand{\Dom}{{\rm Dom}}
\newcommand{\Hol}{{\rm Hol}}
\newcommand{\Mon}{{\rm Mon}}
\newcommand{\Hom}{{\rm Hom}}
\newcommand{\Ker}{{\rm Ker}}
\newcommand{\Ind}{{\rm Ind}}
\newcommand{\IM}{{\rm Im}}
\newcommand{\Is}{{\rm Is}}
\newcommand{\ID}{{\rm id}}
\newcommand{\GL}{{\rm GL}}
\newcommand{\Iso}{{\rm Iso}}
\newcommand{\rO}{{\rm O}}
\newcommand{\Sem}{{\rm Sem}}
\newcommand{\St}{{\rm St}}
\newcommand{\Sym}{{\rm Sym}}
\newcommand{\SU}{{\rm SU}}
\newcommand{\Tor}{{\rm Tor}}
\newcommand{\U}{{\rm U}}

\newcommand{\A}{\mathcal A}
\newcommand{\Ce}{\mathcal C}
\newcommand{\D}{\mathcal D}
\newcommand{\E}{\mathcal E}
\newcommand{\F}{\mathcal F}
\newcommand{\G}{\mathcal G}
\renewcommand{\H}{\mathcal H}
\renewcommand{\cL}{\mathcal L}
\newcommand{\Q}{\mathcal Q}
\newcommand{\R}{\mathcal R}
\newcommand{\cS}{\mathcal S}
\newcommand{\cU}{\mathcal U}
\newcommand{\W}{\mathcal W}

\newcommand{\bA}{\mathbb{A}}
\newcommand{\bB}{\mathbb{B}}
\newcommand{\bC}{\mathbb{C}}
\newcommand{\bD}{\mathbb{D}}
\newcommand{\bE}{\mathbb{E}}
\newcommand{\bF}{\mathbb{F}}
\newcommand{\bG}{\mathbb{G}}
\newcommand{\bK}{\mathbb{K}}
\newcommand{\bM}{\mathbb{M}}
\newcommand{\bN}{\mathbb{N}}
\newcommand{\bO}{\mathbb{O}}
\newcommand{\bP}{\mathbb{P}}
\newcommand{\bR}{\mathbb{R}}
\newcommand{\bV}{\mathbb{V}}
\newcommand{\bZ}{\mathbb{Z}}

\newcommand{\bfE}{\mathbf{E}}
\newcommand{\bfX}{\mathbf{X}}
\newcommand{\bfY}{\mathbf{Y}}
\newcommand{\bfZ}{\mathbf{Z}}

\renewcommand{\O}{\Omega}
\renewcommand{\o}{\omega}
\newcommand{\vp}{\varphi}
\newcommand{\vep}{\varepsilon}

\newcommand{\diag}{{\rm diag}}
\newcommand{\grp}{{\mathsf{G}}}
\newcommand{\dgrp}{{\mathsf{D}}}
\newcommand{\desp}{{\mathsf{D}^{\rm{es}}}}
\newcommand{\Geod}{{\rm Geod}}
\newcommand{\geod}{{\rm geod}}
\newcommand{\hgr}{{\mathsf{H}}}
\newcommand{\mgr}{{\mathsf{M}}}
\newcommand{\ob}{{\rm Ob}}
\newcommand{\obg}{{\rm Ob(\mathsf{G)}}}
\newcommand{\obgp}{{\rm Ob(\mathsf{G}')}}
\newcommand{\obh}{{\rm Ob(\mathsf{H})}}
\newcommand{\Osmooth}{{\Omega^{\infty}(X,*)}}
\newcommand{\ghomotop}{{\rho_2^{\square}}}
\newcommand{\gcalp}{{\mathsf{G}(\mathcal P)}}

\newcommand{\rf}{{R_{\mathcal F}}}
\newcommand{\glob}{{\rm glob}}
\newcommand{\loc}{{\rm loc}}
\newcommand{\TOP}{{\rm TOP}}

\newcommand{\wti}{\widetilde}
\newcommand{\what}{\widehat}

\renewcommand{\a}{\alpha}
\newcommand{\be}{\beta}
\newcommand{\ga}{\gamma}
\newcommand{\Ga}{\Gamma}
\newcommand{\de}{\delta}
\newcommand{\del}{\partial}
\newcommand{\ka}{\kappa}
\newcommand{\si}{\sigma}
\newcommand{\ta}{\tau}

\newcommand{\med}{\medbreak}
\newcommand{\medn}{\medbreak \noindent}
\newcommand{\bign}{\bigbreak \noindent}

\newcommand{\lra}{{\longrightarrow}}
\newcommand{\ra}{{\rightarrow}}
\newcommand{\rat}{{\rightarrowtail}}
\newcommand{\ovset}[1]{\overset {#1}{\ra}}
\newcommand{\ovsetl}[1]{\overset {#1}{\lra}}
\newcommand{\hr}{{\hookrightarrow}}
 

\pagestyle{myheadings}
%\usepackage{geometry, amsmath,amssymb,latexsym,enumerate}
%\usepackage{xypic}

\def\baselinestretch{1.1}


\hyphenation{prod-ucts}

%\geometry{textwidth= 16 cm, textheight=21 cm}

\newcommand{\sqdiagram}[9]{$$ \diagram  #1  \rto^{#2} \dto_{#4}&
#3  \dto^{#5} \\ #6    \rto_{#7}  &  #8   \enddiagram
\eqno{\mbox{#9}}$$ }

\def\C{C^{\ast}}

\newcommand{\labto}[1]{\stackrel{#1}{\longrightarrow}}

%\newenvironment{proof}{\noindent {\bf Proof} }{ \hfill $\Box$
%{\mbox{}}

\newcommand{\quadr}[4]{\begin{pmatrix} & #1& \\[-1.1ex] #2 & & #3\\[-1.1ex]& #4&
 \end{pmatrix}}
\def\D{\mathsf{D}}

\begin{document}


\title[Research Topics in Algebraic Foundations of Quantum Algebraic Topology]
{topic on the Algebraic Foundations of Quantum Algebraic Topology}

This is a contributed topic on the algebraic foundations of Quantum Algebraic Topology (QAT)

\textbf{(A.)} \emph{Quantum Algebraic Topology (QAT)} is defined as the mathematical and physical study of general theories of quantum algebraic structures from the standpoint of Algebraic Topology, Category Theory and their Non-Abelian extensions in Higher Dimensional Algebra and Supercategories
in relation to, or petinent to, Quantum theories, Quantum Field Theories,
General Relativity and its Quantum extensions, Quantum Gravity.

\textbf{(B). Several suggested new QAT topics are:}

\begin{enumerate}

\item Poisson algebras, Quantization methods and Hamiltonian algebroids

\item K-S Theorem and its Quantum algebraic consequences in QAT

\item Logic Lattice algebras or Many-Valued (MV)  Logic algebras 

\item Quantum MV-Logic algebras and $\L{}-M_n$-noncommutative algebras

\item Quantum Operator Algebras ( such as :   involution, *-algebras, or $*$-algebras, von Neumann algebras, JB- and JL- algebras,   $C^*$ - or C*- algebras, etc.

\item Quantum von Neumann algebra and subfactors

\item Kac-Moody and K-algebras

\item Hopf algebras, Quantum Groups and Quantum group algebras

\item Quantum Groupoids and weak Hopf $C^*$-algebras

\item Groupoid C*-Convolution algebras and *-Convolution Algebroids

\item Quantum Spacetimes and Quantum Fundamental Groupoids 

\item Quantum Double Algebras

\item Quantum Gravity, supersymmetries, supergravity, superalgebras and graded `Lie' algebras

\item Quantum Categorical algebra and Higher Dimensional, $\L{}-M_n$- Toposes

\item Quantum R-categories, R-supercategories and Symmetry Breaking

\item  Extended Quantum Symmetries in Higher Dimensional Algebras (HDA), such as:  \\
algebroids, double algebroids, categorical algebroids, double groupoids, \\
convolution algebroids, groupoid $C^*$ -convolution algebroids

\item Universal algebras in R-Supercategories

\item Supercategorical algebras (SA) as concrete interpretations of the Theory of Elementary Abstract Supercategories (ETAS). 

\item  Quantum Non-Abelian Algebraic Topology (QNAAT)

\item Noncommutative Geometry, Quantum Geometry, and Non-Abelian Quantum Algebraic Geometry

\item Other -- Miscellaneous \textbf{[please add here your additions, changes, editing, 
remarks, proofs, conjectures, and so on...]}

\end{enumerate}

\begin{thebibliography} {9}

\bibitem{AS}
Alfsen, E.M. and F. W. Schultz: \emph{Geometry of State Spaces of Operator Algebras}, Birk\"auser, Boston--Basel--Berlin (2003).

\bibitem{AMF56}
Atyiah, M.F. 1956. On the Krull-Schmidt theorem with applications to sheaves.
\emph{Bull. Soc. Math. France}, \textbf{84}: 307--317.

\bibitem{AMF56}
Auslander, M. 1965. Coherent Functors. \emph{Proc. Conf. Cat. Algebra, La Jolla},
189--231.
  
\bibitem{AS-BC2k}
Awodey, S. \& Butz, C., 2000, Topological Completeness for Higher Order Logic., Journal of Symbolic Logic, 65, 3, 1168--1182. 


\bibitem{AS96}
Awodey, S., 1996, "Structure in Mathematics and Logic: A Categorical Perspective", Philosophia Mathematica, 3, 209--237. 

\bibitem{AS2k4}
Awodey, S., 2004, "An Answer to Hellman's Question: Does Category Theory Provide a Framework for Mathematical Structuralism", Philosophia Mathematica, 12, 54--64. 

\bibitem{AS2k6}
Awodey, S., 2006, Category Theory, Oxford: Clarendon Press. 

\bibitem{BAJ-DJ98a}
Baez, J. \& Dolan, J., 1998a, "Higher-Dimensional Algebra III. n-Categories and the Algebra of Opetopes", Advances in Mathematics, 135, 145--206.  


\bibitem{BAJ-DJ2k1}
Baez, J. \& Dolan, J., 2001, "From Finite Sets to Feynman Diagrams", Mathematics Unlimited -- 2001 and Beyond, Berlin: Springer, 29--50.  

\bibitem{BAJ-DJ97}
Baez, J., 1997, "An Introduction to n-Categories", Category Theory and Computer Science, Lecture Notes in Computer Science, 1290, Berlin: Springer-Verlag, 1--33. 

\bibitem{ICB3}
Baianu, I.C.: 1970, Organismic Supercategories: II. On Multistable Systems. \emph{Bulletin of Mathematical Biophysics}, \textbf{32}: 539-561.
 
\bibitem{ICB4}
Baianu, I.C.: 1971b, Categories, Functors and Quantum Algebraic
Computations, in P. Suppes (ed.), \emph{Proceed. Fourth Intl. Congress Logic-Mathematics-Philosophy of Science}, September 1--4, 1971, Bucharest.

\bibitem{ICBs5}
Baianu, I.C. and D. Scripcariu: 1973, On Adjoint Dynamical Systems. \emph{Bulletin of Mathematical Biophysics}, \textbf{35}(4), 475--486.

\bibitem{ICB5}
Baianu, I.C.: 1973, Some Algebraic Properties of \emph{\textbf{(M,R)}} -- Systems. \emph{Bulletin of Mathematical Biophysics} \textbf{35}, 213-217.

\bibitem{ICB6}
Baianu, I.C.: 1977, A Logical Model of Genetic Activities in \L ukasiewicz Algebras: The Non-linear Theory. \emph{Bulletin of Mathematical Biology},
\textbf{39}: 249-258.

\bibitem{Bgg2}
Baianu, I. C., Glazebrook, J. F. and G. Georgescu: 2004, Categories of Quantum Automata and N-Valued \L ukasiewicz Algebras in Relation to Dynamic Bionetworks, \textbf{(M,R)}--Systems and Their Higher Dimensional Algebra, \emph{Abstract and Preprint of Report}: $\\http://www.ag.uiuc.edu/fs401/QAuto.pdf $ and $http://www.medicalupapers.com/quantum+automata+math+categories+baianu/$

\bibitem{Bbg3}
Baianu, I.C., R. Brown and J.F. Glazebrook. : 2007a, Categorical Ontology of Complex Spacetime Structures: The Emergence of Life and Human Consciousness, Axiomathes, 17: 35-168.

\bibitem{Bggb4}
Baianu, I.C.,  R. Brown and J. F. Glazebrook: 2007b, A Non-Abelian, Categorical Ontology of Spacetimes and Quantum Gravity, Axiomathes, 17: 169-225.


\bibitem{QNAAT}
Baianu, I. C. et al. 2008. Quantum Non-Abelian Algebraic Topology (QNAAT): PM Exposition lec $id=68$.

\bibitem{Ba-We85}
Barr, M. and Wells, C., 1985, Toposes, Triples and Theories, New York: Springer-Verlag.
 
\bibitem{BM-CW99}
Barr, M. and Wells, C., 1999, Category Theory for Computing Science, Montreal: CRM. 

\bibitem{BJL81}
Bell, J. L., 1981, "Category Theory and the Foundations of Mathematics", British Journal for the Philosophy of Science, 32, 349--358. 
 
\bibitem{BJL82}
Bell, J. L., 1982, "Categories, Toposes and Sets", Synthese, 51, 3, 293--337. 
 
\bibitem{BJL86}
Bell, J. L., 1986, "From Absolute to Local Mathematics", Synthese, 69, 3, 409--426. 

\bibitem{BJL88} 
Bell, J. L., 1988, Toposes and Local Set Theories: An Introduction, Oxford: Oxford University Press. 

\bibitem{BG-MCLS99}
Birkoff, G. \& Mac Lane, S., 1999, Algebra, 3rd ed., Providence: AMS.  

\bibitem{Borceux94}
Borceux, F.: 1994, \emph{Handbook of Categorical Algebra}, vols: 1--3, 
in {\em Encyclopedia of Mathematics and its Applications} \textbf{50} to \textbf{52}, Cambridge University Press.

\bibitem{Bourbaki1}
Bourbaki, N. 1961 and 1964: \emph{Alg\`{e}bre commutative.},
in \`{E}l\'{e}ments de Math\'{e}matique., Chs. 1--6., Hermann: Paris.

\bibitem{BJk4}
Brown, R. and G. Janelidze: 2004, Galois theory and a new homotopy
double groupoid of a map of spaces, \emph{Applied Categorical
Structures} \textbf{12}: 63-80.

\bibitem{BHR2}
Brown, R., Higgins, P. J. and R. Sivera,: 2007a, \emph{Non-Abelian
Algebraic Topology}, in preparation.\\
http://www.bangor.ac.uk/~mas010/nonab-a-t.html ; \\
http://www.bangor.ac.uk/~mas010/nonab-t/partI010604.pdf

\bibitem{BGB2k7b}
Brown, R., Glazebrook, J. F. and I.C. Baianu.: 2007b, A Conceptual, Categorical and Higher Dimensional Algebra Framework of Universal Ontology and the Theory of Levels for Highly Complex Structures and Dynamics., \emph{Axiomathes} (17): 321--379.

\bibitem{Br-Har-Ka-Po2k2}
Brown, R., Hardie, K., Kamps, H. and T. Porter: 2002, The homotopy double groupoid of a Hausdorff space., \emph{Theory and Applications of Categories} \textbf{10}, 71-93.

\bibitem{Br-Hardy76}
Brown, R., and Hardy, J.P.L.:1976, Topological groupoids I: universal constructions, \emph{Math. Nachr.}, 71: 273-286.

\bibitem{Br-Sp76}
Brown, R. and Spencer, C.B.: 1976, Double groupoids and crossed modules, \emph{Cah.  Top. G\'{e}om. Diff.} \textbf{17}, 343-362.

\bibitem{BR-SCB76}
Brown R and Razak Salleh A (1999) Free crossed resolutions of groups and presentations of modules of
identities among relations. {\em LMS J. Comput. Math.}, \textbf{2}: 25--61.

\bibitem{BDA55}
Buchsbaum, D. A.: 1955, Exact categories and duality., Trans. Amer. Math. Soc. \textbf{80}: 1-34.

\bibitem{BDA55}
Buchsbaum, D. A.: 1969, A note on homology in categories., Ann. of Math. \textbf{69}: 66-74.

\bibitem{BL2k3}
Bunge, M. and S. Lack: 2003, Van Kampen theorems for toposes, \emph{Adv. in Math.} \textbf{179}, 291-317.

\bibitem{BM84}
Bunge, M., 1984, "Toposes in Logic and Logic in Toposes", Topoi, 3, no. 1, 13-22. 

\bibitem{BM-LS2k3}
Bunge M, Lack S (2003) Van Kampen theorems for toposes. {\em Adv Math}, \textbf {179}: 291-317.

\bibitem{CH-ES56}
Cartan, H. and Eilenberg, S. 1956. {\em Homological Algebra}, Princeton Univ. Press: Pinceton. 

\bibitem{CPM65}
Cohen, P.M. 1965. {\em Universal Algebra}, Harper and Row: New York, London and Tokyo.

\bibitem{CA94}
Connes A 1994. \emph{Noncommutative geometry}. Academic Press: New York.

\bibitem{CR-LL63}
Croisot, R. and Lesieur, L. 1963. \emph{Alg\`ebre noeth\'erienne non-commutative.}, Gauthier-Villard: Paris.


\end{thebibliography}

\end{document}

%%%%%
\end{document}
