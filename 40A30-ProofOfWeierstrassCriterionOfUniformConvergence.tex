\documentclass[12pt]{article}
\usepackage{pmmeta}
\pmcanonicalname{ProofOfWeierstrassCriterionOfUniformConvergence}
\pmcreated{2013-03-22 16:26:28}
\pmmodified{2013-03-22 16:26:28}
\pmowner{argerami}{15454}
\pmmodifier{argerami}{15454}
\pmtitle{proof of Weierstrass' criterion of uniform convergence}
\pmrecord{4}{38596}
\pmprivacy{1}
\pmauthor{argerami}{15454}
\pmtype{Proof}
\pmcomment{trigger rebuild}
\pmclassification{msc}{40A30}
\pmclassification{msc}{26A15}

% this is the default PlanetMath preamble.  as your knowledge
% of TeX increases, you will probably want to edit this, but
% it should be fine as is for beginners.

% almost certainly you want these
\usepackage{amssymb}
\usepackage{amsmath}
\usepackage{amsfonts}

% used for TeXing text within eps files
%\usepackage{psfrag}
% need this for including graphics (\includegraphics)
%\usepackage{graphicx}
% for neatly defining theorems and propositions
%\usepackage{amsthm}
% making logically defined graphics
%%%\usepackage{xypic}

% there are many more packages, add them here as you need them

% define commands here

\begin{document}
The assumption that $|f_n(x)| \le M_n$ for every $x$ guarantees that each numerical series $\sum_n f_n(x)$ converges absolutely. We call the limit $f(x)$.

To see that the convergence is uniform: let $\epsilon>0$. Then there exists $K$ such that $n>K$ implies $\sum_{n>K} M_n < \epsilon$. Now, if $k>K$,

\[
|f(x)-\sum_{n=1}^k f_n(x)| = |\sum_{n>k} f_n(x) | \le
\sum_{n>k} |f_n(x)| \le \sum_{n>k} M_n < \epsilon
\]

The $\epsilon$ does not depend on $x$, so the convergence is uniform.


%%%%%
%%%%%
\end{document}
