\documentclass[12pt]{article}
\usepackage{pmmeta}
\pmcanonicalname{ProofOfTaubersConvergenceTheorem}
\pmcreated{2013-03-22 13:08:20}
\pmmodified{2013-03-22 13:08:20}
\pmowner{rmilson}{146}
\pmmodifier{rmilson}{146}
\pmtitle{proof of Tauber's convergence theorem}
\pmrecord{7}{33576}
\pmprivacy{1}
\pmauthor{rmilson}{146}
\pmtype{Proof}
\pmcomment{trigger rebuild}
\pmclassification{msc}{40G10}

\usepackage{amsmath}
\usepackage{amsfonts}
\usepackage{amssymb}
\newcommand{\reals}{\mathbb{R}}
\newcommand{\natnums}{\mathbb{N}}
\newcommand{\cnums}{\mathbb{C}}
\newcommand{\znums}{\mathbb{Z}}
\newcommand{\lp}{\left(}
\newcommand{\rp}{\right)}
\newcommand{\lb}{\left[}
\newcommand{\rb}{\right]}
\newcommand{\supth}{^{\text{th}}}
\newtheorem{proposition}{Proposition}
\newtheorem{definition}[proposition]{Definition}

\newtheorem{theorem}[proposition]{Theorem}
\begin{document}
Let
$$f(z) = \sum_{n=0}^\infty a_n z^n,$$
be a complex power series, convergent in the open disk $\vert
z\vert<1$.  We suppose that
\begin{enumerate}
\item $n a_n\rightarrow 0$ as
$n\rightarrow\infty$, and that
\item  $f(r)$ converges to some finite  $L$ as
$r\rightarrow 1^-$;
\end{enumerate}
and wish to show that
$\sum_n a_n$ converges to the same $L$ as well.

Let $s_n=a_0+\cdots+a_n$, where $n=0,1,\ldots$, denote the partial
sums of the series in question.  The enabling idea in Tauber's
convergence result (as well as other Tauberian theorems) is the
existence of a correspondence in the evolution of the $s_n$ as
$n\rightarrow\infty$, and the evolution of $f(r)$ as $r\rightarrow
1^-$.  Indeed we shall show that
\begin{equation}
  \label{eq:e1}
  \left\vert s_n - f\lp \frac{n-1}{n} \rp\right\vert \rightarrow 0 \quad\text{as}\quad
  n\rightarrow \infty.  
\end{equation}
The desired result then follows in an obvious fashion.

For every real $0<r<1$ we have
$$s_n = f(r) + \sum_{k=0}^n a_k (1-r^k) - \sum_{k=n+1}^\infty a_k\, r^k.$$
Setting
$$\epsilon_n = \sup_{k>n}  \vert k a_k \vert,$$
and noting that
$$ 1-r^k = (1-r)(1+r+\cdots+r^{k-1}) < k(1-r),$$
we have that
$$
\vert s_n - f(r) \vert \leq (1-r)\sum_{k=0}^n ka_k +
\frac{\epsilon_n}{n}
\sum_{k=n+1}^\infty r^k.$$
Setting $r=1-1/n$ in the above inequality we get
$$
\vert s_n - f(1-1/n) \vert \leq \mu_n + \epsilon_n (1-1/n)^{n+1},$$
where
$$\mu_n = \frac{1}{n}\sum_{k=0}^n \vert k a_k \vert$$
are the Ces\`aro
means of the sequence $\vert k a_k\vert,\; k=0,1,\ldots$ 
Since the
latter sequence converges to zero, so do the means $\mu_n$, and the
suprema $\epsilon_n$.  Finally,
Euler's formula for $e$ gives
$$\lim_{n\rightarrow\infty}(1-1/n)^n = e^{-1}.$$
The validity of \eqref{eq:e1} follows immediately. QED

%%%%%
%%%%%
\end{document}
