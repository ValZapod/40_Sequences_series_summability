\documentclass[12pt]{article}
\usepackage{pmmeta}
\pmcanonicalname{ProofOfIdentityTheoremOfPowerSeries1}
\pmcreated{2013-03-22 16:48:46}
\pmmodified{2013-03-22 16:48:46}
\pmowner{rspuzio}{6075}
\pmmodifier{rspuzio}{6075}
\pmtitle{proof of identity theorem of power series}
\pmrecord{11}{39047}
\pmprivacy{1}
\pmauthor{rspuzio}{6075}
\pmtype{Proof}
\pmcomment{trigger rebuild}
\pmclassification{msc}{40A30}
\pmclassification{msc}{30B10}

% this is the default PlanetMath preamble.  as your knowledge
% of TeX increases, you will probably want to edit this, but
% it should be fine as is for beginners.

% almost certainly you want these
\usepackage{amssymb}
\usepackage{amsmath}
\usepackage{amsfonts}

% used for TeXing text within eps files
%\usepackage{psfrag}
% need this for including graphics (\includegraphics)
%\usepackage{graphicx}
% for neatly defining theorems and propositions
%\usepackage{amsthm}
% making logically defined graphics
%%%\usepackage{xypic}

% there are many more packages, add them here as you need them

% define commands here

\begin{document}
We can prove the identity theorem for power series
using divided differences.  From amongst the points
at which the two series are equal, pick a sequence
$\{w_k\}_{k=0}^\infty$ which satisfies the following
three conditions:
\begin{enumerate}
\item $\lim_{k \to \infty} w_k = z_0$
\item $w_m = w_n$ if and only if $m = n$.
\item $w_k \neq z_0$ for all $k$.
\end{enumerate}
Let $f$ be the function determined by one power series 
and let $g$ be the function determined by the other
power series:
\begin{align*}
f(z) &= \sum_{n=0}^\infty a_n (z - z_0)^n \\
g(z) &= \sum_{n=0}^\infty b_n (z - z_0)^n
\end{align*}

Because formation of divided differences involves
finite sums and dividing by differences of $w_k$'s
(which all differ from zero by condition 2 above,
so it is legitimate to divide by them), we may
carry out the formation of finite diffferences
on a term-by-term basis.  Using the result about
divided differences of powers, we have
\begin{align*}
\Delta^m f [w_k, \ldots, w_{k+m}] &=
\sum_{n=m}^\infty a_n D_{mnk} \\
\Delta^m f [w_k, \ldots, w_{k+m}] &=
\sum_{n=m}^\infty b_n D_{mnk}
\end{align*}
where
\[
D_{mnk} = \sum_{j_0 + \ldots j_m = n - m}
(w_k - z_0)^{j_0} \cdots (w_{k+m} - z_0)^{j_m}.
\]

Note that $\lim_{k \to infty} D_{mnk} = 0$
when $m > n$, but $D_{mmk} = 1$.  Since
power series converge uniformly, we may
intechange limit and summation to conclude
\begin{align*}
\lim_{k \to \infty} \Delta^m f [w_k, \ldots, w_{k+m}] &=
\sum_{n=m}^\infty a_n \lim_{k \to \infty} D_{mnk} = a_m \\
\lim_{k \to \infty} \Delta^m g [w_k, \ldots, w_{k+m}] &=
\sum_{n=m}^\infty b_n \lim_{k \to \infty} D_{mnk} = b_m.
\end{align*}
Since, by design, $f(w_k) = g(w_k)$, we have
\[
\Delta^m f [w_k, \ldots, w_{k+m}] = 
\Delta^m g [w_k, \ldots, w_{k+m}],
\]
hence $a_m = b_m$ for all $m$.
%%%%%
%%%%%
\end{document}
