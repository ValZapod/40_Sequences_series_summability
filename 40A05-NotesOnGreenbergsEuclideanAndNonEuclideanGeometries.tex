\documentclass[12pt]{article}
\usepackage{pmmeta}
\pmcanonicalname{NotesOnGreenbergsEuclideanAndNonEuclideanGeometries}
\pmcreated{2013-03-11 19:33:19}
\pmmodified{2013-03-11 19:33:19}
\pmowner{Wkbj79}{1863}
\pmmodifier{}{0}
\pmtitle{notes on Greenberg's Euclidean and Non-Euclidean Geometries}
\pmrecord{1}{50115}
\pmprivacy{1}
\pmauthor{Wkbj79}{0}
\pmtype{Definition}

%none for now
\begin{document}
\documentclass{amsart}
\begin{document}

\section{Notes on the Preface}

The beginning provides some important historical information and motivation for exploring these geometries.

I really like the long paragraph on the fifth page, which discusses ``humanizing'' mathematics.

\section{Notes on the Introduction}

The introduction also provides some important historical information and motivation for exploring these geometries.

\section{Notes on Chapter 1: Euclid's Geometry}

\subsection{Notes on ``The Origins of Geometry''}

The origin of the word geometry is Greek.  It comes from the word \emph{geometrein}, whose components are \emph{geo-}, meaning ``earth'', and \emph{-metrein}, meaning ``to measure''.

Herodotus credits the ancient Egyptian culture with the discovery of geometry, but other ancient cultures---Babylonian, Hindu, and Chinese---also discovered findings in geometry.

The ancient Babylonian, Chinese, and Jewish cultures considered $\pi$ to be equal to 3.  Some interesting notes on the ancient Jewish culture are Rabbi Nehemiah's failed attempt to ``change'' the value of $\pi$ to $\frac{22}{7}$ and the verse I Kings 7:23, which reads (NIV):

\begin{quote}
He made the Sea of cast metal, circular in shape, measuring ten cubits from rim to rim and five cubits high.  It took a line of thirty cubits to measure around it.
\end{quote}

The practice of declaring $\pi$ equal to 3 may have been adopted in ancient Roman culture as well, as evidenced by the architect Vitruvius.

The Egyptian approximation for $\pi$ according to the Rhind papyrus is $\frac{256}{81}$.  (I have more details on this in a research paper that I wrote for my ``History of Math'' course.)

Egyptian geometry was not rigorous:  Guessing was the common practice.

The Babylonians were ahead of the Egyptians in arithmetic and algebra.  They were aware of the Pythagorean theorem.  Otto Neugerbauer exposed the fact that Babylonian algebra influenced Greek mathematics.

Thales can probably be considered as the founder of logical geometry, as he went through Babylonian and Egyptian results and deduced which results were correct by means of proving such results.

Pythagoras adopted Thales' system and virtually turned it into a religion.  The beliefs and practices of the Pythagoreans included:

\begin{itemize}
\item immortality of soul;
\item reincarnation;
\item vegetarianism;
\item communal property;
\item study of music and mathematics leads to lifting of soul and union with God.
\end{itemize}

The discovery that $\sqrt{2}$ was irrational caused the Pythagoreans much dismay.  This caused the Pythagoreans to use geometry for doing algebra so that irrational numbers could be represented by line segments.

The main results of plane geometry done by the Pythagoreans appeared around 400 BC in Hippocrates' \emph{The Elements}.  (This is not the same person as the physician.)  Although Hippocrates' original work is lost to us, it was most likely incorporated into the first four books of Euclid's \emph{The Elements}, which appeared around 300 BC.

Eudoxus discovered a theory of proportions that transferred to irrationals.  His theory was incorporated into the fifth book of Euclid's \emph{The Elements}.

Plato's Academy of science and philosophy was founded around 387 BC.  In \emph{The Republic}, Plato wrote:

\begin{quote}
The study of mathematics develops and sets into operation a mental organism more valuable than a thousand eyes, because through it alone can truth be comprehended.
\end{quote}

The Socratic method led to the discovery of indirect proof.  Plato incorporated indirect proof and often cited the irrationality of $\sqrt{2}$ as an example.

Euclid belonged to the Platonic school.  In his masterpiece \emph{The Elements}, Euclid drew from the Pythagoreans for Books I-IV, VII, and IX; Eudoxus for Books V, VI, and XII; Archytas for Book VIII; and Theaetetus for Books X and XIII.  Due to the success of Euclid's work, few traces remain of his predecessors from whom he drew.  Euclid is the most widely read author in the history of mankind.  Geometry was taught for two thousand years based on Euclid's presentation.  Euclid's \emph{The Elements} is the prototype for pure mathematics.

Despite the fact that geometry had applications even in the days of Euclid, this is not discussed in Euclid's work.  According to legend, a new student of Euclid's once asked him, ``What shall I get by learning these things?''  In reply, Euclid beckoned his slave and told him, ``Give him a coin, since he must make gain out of what he learns.''  Even today, some mathematicians still study mathematics solely for its intrinsic beauty and elegance.

Actually, results from pure mathematics often have applications unbeknownst to the mathematicians who initially worked on the results.  Thus, the study of pure mathematics can end up bettering society in some way.

\subsection{Notes on ``The Axiomatic Method''}

With respect to mathematics, proofs are better than trial and error for the following reasons:

\begin{itemize}
\item Proofs provide assurance that a result is correct;
\item Proven results can be more general than results from trial and error;
\item Proofs provide insight into deeper relationships between concepts.
\end{itemize}

Two requirements are discussed for proofs using the axiomatic method:

\begin{itemize}
\item axioms;
\item rules of reasoning.
\end{itemize}

In \emph{The Elements}, Euclid was able to deduce 465 propositions from 5 postulates (axioms).  Many of these propositions are complicated and not obvious.

\subsection{Notes on ``Undefined Terms''}

Immediately, another requirement is discussed for proofs using the axiomatic method: mutual understanding.  Circular reasoning and the need for undefined terms are discussed in consequence.

Euclid attempted to define every geometric term, but some of his definitions are not insightful unless one already has an image of the term that Euclid is trying to define.  The terms for which Euclid does not supply an adequate definition (and thus can be taken as undefined terms) are:

\begin{itemize}
\item point
\item line
\item lie on
\item between
\item congruent
\end{itemize}

For solid geometry, ``plane'' would have to be added to this list.

This list does not include synonyms.  For instance, instead of saying ``$P$ lies on $\ell$'', one can instead say ``$\ell$ passes through $P$'' or ``$P$ is incident with $\ell$''.

There are many other undefined terms, but they are not geometric in nature and thus have been excluded from the list above.  Euclid called these terms ``common notions''.

\end{document}
%%%%%
\end{document}
