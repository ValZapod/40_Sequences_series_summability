\documentclass[12pt]{article}
\usepackage{pmmeta}
\pmcanonicalname{AbsoluteConvergenceOfIntegralAndBoundednessOfDerivative}
\pmcreated{2013-03-22 19:01:28}
\pmmodified{2013-03-22 19:01:28}
\pmowner{pahio}{2872}
\pmmodifier{pahio}{2872}
\pmtitle{absolute convergence of integral and boundedness of derivative}
\pmrecord{7}{41896}
\pmprivacy{1}
\pmauthor{pahio}{2872}
\pmtype{Theorem}
\pmcomment{trigger rebuild}
\pmclassification{msc}{40A10}
\pmrelated{NecessaryConditionOfConvergence}

% this is the default PlanetMath preamble.  as your knowledge
% of TeX increases, you will probably want to edit this, but
% it should be fine as is for beginners.

% almost certainly you want these
\usepackage{amssymb}
\usepackage{amsmath}
\usepackage{amsfonts}

% used for TeXing text within eps files
%\usepackage{psfrag}
% need this for including graphics (\includegraphics)
%\usepackage{graphicx}
% for neatly defining theorems and propositions
%\usepackage{amsthm}
% making logically defined graphics
%%%\usepackage{xypic}

% there are many more packages, add them here as you need them

% define commands here
\newcommand{\sijoitus}[2]%
{\operatornamewithlimits{\Big/}_{\!\!\!#1}^{\,#2}}
\begin{document}
\textbf{Theorem.}\, Assume that we have an \PMlinkid{absolutely converging}{11865} integral
$$\int_a^\infty\!f(x)\,dx$$
where the real function $f$ and its derivative $f'$ are continuous and $f'$ additionally bounded on the interval 
\,$[a,\,\infty)$.\, Then
\begin{align}
\lim_{x\to\infty}f(x) \;=\; 0.
\end{align}



\emph{Proof.}\, If\, $c > a$,\, we obtain
$$\int_a^c\!f(x)f'(x)\,dx \;=\; \frac{1}{2}\sijoitus{a}{\quad c}\!(f(x))^2 
\;=\; \frac{(f(c))^2-(f(a))^2}{2},$$
from which
\begin{align}
(f(c))^2 \;=\; (f(a))^2+2\!\int_a^c\!f(x)f'(x)\,dx.
\end{align}
Using the boundedness of $f'$ and the absolute convergence, we can estimate upwards the integral
$$\int_a^c\!|f(x)f'(x)|\,dx \;=\; \int_a^c\!|f(x)||f'(x)|\,dx \;\leqq\; M\!\int_a^c\!|f(x)|\,dx 
\;\leqq\; M\!\int_a^\infty\!|f(x)|\,dx \quad \forall c \in [a,\,\infty)$$
whence $\int_a^\infty\!|f(x)f'(x)|\,dx$ is finite and thus $\int_a^\infty\!f(x)f'(x)\,dx$ converges absolutely.\, Hence (2) implies
$$\lim_{c\to\infty}(f(c))^2 \;=\; (f(a))^2+2\int_a^\infty\!f(x)f'(x)\,dx,$$
i.e. $\displaystyle\lim_{x\to\infty}(f(x))^2$ exists as finite, therefore also
$$\lim_{x\to\infty}|f(x)| \;:=\; A.$$
Antithesis:\, $A > 0$.\; It implies that there is an $x_0\;(\geqq a)$ such that 
$$|f(x)| \;\geqq\; \frac{A}{2} \quad \forall x \geqq x_0.$$
If now\, $b > x_0$, then we had
$$\int_{x_0}^b\!|f(x)|\,dx \;\geqq\; \frac{A}{2}(b\!-\!x_0) \;\; \longrightarrow \infty \quad \mbox{as}\;\; b \to \infty.$$
This means that $\int_{x_0}^\infty|f(x)|\,dx$ and consequently also $\int_a^\infty|f(x)|\,dx$ would be divergent.\, Since it is not true, we infer that\, $A = 0$,\, i.e. that the assertion (1) is true. 

%%%%%
%%%%%
\end{document}
