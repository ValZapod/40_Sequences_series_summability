\documentclass[12pt]{article}
\usepackage{pmmeta}
\pmcanonicalname{ApplicationOfCauchyCriterionForConvergence}
\pmcreated{2013-03-22 19:03:22}
\pmmodified{2013-03-22 19:03:22}
\pmowner{pahio}{2872}
\pmmodifier{pahio}{2872}
\pmtitle{application of Cauchy criterion for convergence}
\pmrecord{9}{41936}
\pmprivacy{1}
\pmauthor{pahio}{2872}
\pmtype{Example}
\pmcomment{trigger rebuild}
\pmclassification{msc}{40A05}
\pmrelated{RealNumber}
\pmrelated{GeometricSeries}
\pmrelated{LogarithmusBinaris}
\pmrelated{NapiersConstant}

% this is the default PlanetMath preamble.  as your knowledge
% of TeX increases, you will probably want to edit this, but
% it should be fine as is for beginners.

% almost certainly you want these
\usepackage{amssymb}
\usepackage{amsmath}
\usepackage{amsfonts}

% used for TeXing text within eps files
%\usepackage{psfrag}
% need this for including graphics (\includegraphics)
%\usepackage{graphicx}
% for neatly defining theorems and propositions
 \usepackage{amsthm}
% making logically defined graphics
%%%\usepackage{xypic}

% there are many more packages, add them here as you need them

% define commands here

\theoremstyle{definition}
\newtheorem*{thmplain}{Theorem}

\begin{document}
Without using the methods of the entry determining series convergence, we show that the real-term series
$$\sum_{n=0}^\infty\frac{1}{n!} \;=\; 1+\frac{1}{1!}+\frac{1}{2!}+\ldots$$
is convergent by using Cauchy criterion for convergence, being in \PMlinkescapetext{force} in $\mathbb{R}$ equipped with the usual absolute value $|.|$ as \PMlinkid{norm}{1604}.\\

Let $\varepsilon$ be an arbitrary positive number.\, For any positive integer $n$, we have
$$\frac{1}{n!} \;\leqq\; \frac{1}{1\cdot2\cdot2\cdots2} \;=\; \frac{1}{2^{n-1}},$$
whence we can \PMlinkescapetext{estimate} as follows.
\begin{align*}
\left|\frac{1}{(n\!+\!1)!}+\ldots+\frac{1}{(n\!+\!p)!}\right| &\;=\; \frac{1}{(n\!+\!1)!}+\ldots+\frac{1}{(n\!+\!p)!}\\
&\;\leqq\; \frac{1}{2^n}+\ldots+\frac{1}{2^{n+p-1}}\\
&\;=\; \frac{1}{2^n}\left(1+\frac{1}{2}+\ldots+\frac{1}{2^{p-1}}\right)\\
&\;=\; \frac{1}{2^n}\cdot\frac{1-(1/2)^p}{1-1/2}\\
&\;<\; \frac{1}{2^{n-1}} \;<\; \varepsilon
\end{align*}
The last inequality is true for all positive integers $p$, when\, $n \;>\; 1-\mbox{lb}\,{\varepsilon}$.\, Thus the Cauchy criterion implies that the series converges.
%%%%%
%%%%%
\end{document}
