\documentclass[12pt]{article}
\usepackage{pmmeta}
\pmcanonicalname{TelescopingSum}
\pmcreated{2013-03-22 14:25:18}
\pmmodified{2013-03-22 14:25:18}
\pmowner{cvalente}{11260}
\pmmodifier{cvalente}{11260}
\pmtitle{telescoping sum}
\pmrecord{8}{35929}
\pmprivacy{1}
\pmauthor{cvalente}{11260}
\pmtype{Definition}
\pmcomment{trigger rebuild}
\pmclassification{msc}{40A05}
\pmdefines{telescope}

\endmetadata

% this is the default PlanetMath preamble.  as your knowledge
% of TeX increases, you will probably want to edit this, but
% it should be fine as is for beginners.

% almost certainly you want these
\usepackage{amssymb}
\usepackage{amsmath}
\usepackage{amsfonts}

% used for TeXing text within eps files
%\usepackage{psfrag}
% need this for including graphics (\includegraphics)
%\usepackage{graphicx}
% for neatly defining theorems and propositions
%\usepackage{amsthm}
% making logically defined graphics
%%%\usepackage{xypic}

% there are many more packages, add them here as you need them

% define commands here
\begin{document}
A {\em telescoping sum} is a sum in which cancellation occurs between subsequent terms, allowing the sum to be expressed using only the initial and final terms.

Formally a telescoping sum is or can be rewritten in the form

$$ S= \sum_{n=\alpha}^{\beta}\left(a_n - a_{n+1}\right) = a_\alpha - a_{\beta+1}$$

where $a_n$ is a sequence.

{\bf Example:}\\

Define $S(N) = \sum_{n=1}^{N} \frac{1}{n(n+1)}$.  Note that by partial fractions of expressions:
\[ \frac{1}{n(n+1)}= \frac{1}{n} - \frac{1}{n+1} \] and thus $a_n = \frac{1}{n}$ in this example.

\[ S(N) = \sum_{n=1}^{N} \left( \frac{1}{n} - \frac{1}{n+1} \right) \]
\[  = \left( 1 - \frac{1}{2} \right)
      + \cdots + \left( \frac{1}{n} - \frac{1}{n+1} \right) +
      \left( \frac{1}{n+1} - \frac{1}{n+2} \right)
      + \cdots + \left( \frac{1}{N} - \frac{1}{N+1} \right) \]
\[  = 1 + \left( - \frac{1}{2} + \frac{1}{2} \right)
      + \cdots + \left( -\frac{1}{n+1} +\frac{1}{n+1} \right)
      + \cdots - \frac{1}{N+1} \]
\[  = 1 - \frac{1}{N+1} \]
%%%%%
%%%%%
\end{document}
