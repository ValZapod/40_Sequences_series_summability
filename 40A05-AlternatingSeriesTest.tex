\documentclass[12pt]{article}
\usepackage{pmmeta}
\pmcanonicalname{AlternatingSeriesTest}
\pmcreated{2013-03-22 12:27:09}
\pmmodified{2013-03-22 12:27:09}
\pmowner{Koro}{127}
\pmmodifier{Koro}{127}
\pmtitle{alternating series test}
\pmrecord{18}{32588}
\pmprivacy{1}
\pmauthor{Koro}{127}
\pmtype{Theorem}
\pmcomment{trigger rebuild}
\pmclassification{msc}{40A05}
\pmclassification{msc}{40-00}
\pmsynonym{Leibniz's theorem}{AlternatingSeriesTest}
\pmsynonym{Leibniz test}{AlternatingSeriesTest}
\pmrelated{AlternatingSeries}

\endmetadata

\usepackage{amssymb}
\usepackage{amsmath}
\usepackage{amsfonts}
\begin{document}
\PMlinkescapeword{states}
\PMlinkescapeword{simple}
The {\bf alternating series test}, or
the {\bf Leibniz's Theorem}, states the following:

{\bf Theorem} \cite{rudin, kreyszig93}
Let $(a_n)_{n=1}^\infty$ be a non-negative, non-increasing sequence 
or real numbers such that $\lim_{n \rightarrow \infty} a_n = 0$.  
Then the infinite series $\sum_{n=1}^\infty (-1)^{(n+1)} a_n$ converges.

This test provides a necessary and sufficient condition for the convergence of an alternating series, since if $\sum_{n=1}^\infty a_n$ converges then $a_n\to 0$. 

{\bf Example:}  The series
$\sum_{k = 1}^{\infty}\frac{1}{k}$
does not converge, but the alternating series
$\sum_{k = 1}^{\infty}(-1)^{k+1}\frac{1}{k}$
converges to $\ln(2)$.

\begin{thebibliography}{9}
\bibitem{rudin}
 W. Rudin, \emph{Principles of Mathematical Analysis}, McGraw-Hill Inc., 1976.
 \bibitem {kreyszig93} E. Kreyszig,
 \emph{Advanced Engineering Mathematics},
 John Wiley \& Sons, 1993, 7th ed.
 \end{thebibliography}
%%%%%
%%%%%
\end{document}
