\documentclass[12pt]{article}
\usepackage{pmmeta}
\pmcanonicalname{ProofThatAMetricSpaceIsCompactIfAndOnlyIfItIsCompleteAndTotallyBounded}
\pmcreated{2013-03-22 18:01:06}
\pmmodified{2013-03-22 18:01:06}
\pmowner{rm50}{10146}
\pmmodifier{rm50}{10146}
\pmtitle{proof that a metric space is compact if and only if it is complete and totally bounded}
\pmrecord{5}{40534}
\pmprivacy{1}
\pmauthor{rm50}{10146}
\pmtype{Theorem}
\pmcomment{trigger rebuild}
\pmclassification{msc}{40A05}
\pmclassification{msc}{54D30}

% this is the default PlanetMath preamble.  as your knowledge
% of TeX increases, you will probably want to edit this, but
% it should be fine as is for beginners.

% almost certainly you want these
\usepackage{amssymb}
\usepackage{amsmath}
\usepackage{amsfonts}

% used for TeXing text within eps files
%\usepackage{psfrag}
% need this for including graphics (\includegraphics)
%\usepackage{graphicx}
% for neatly defining theorems and propositions
%\usepackage{amsthm}
% making logically defined graphics
%%%\usepackage{xypic}

% there are many more packages, add them here as you need them

% define commands here

\begin{document}
\textbf{Theorem:}\ A metric space is compact if and only if it is complete and totally bounded.

\textbf{Proof. }\ Let $X$ be a metric space with metric $d$. If $X$ is compact, then it is sequentially compact and thus complete. Since $X$ is compact, the covering of $X$ by all $\epsilon$-balls must have a finite subcover, so that $X$ is totally bounded.

Now assume that $X$ is complete and totally bounded. For metric spaces, compact and sequentially compact are equivalent; we prove that $X$ is sequentially compact. Choose a sequence $p_n\in X$; we will find a Cauchy subsequence (and hence a convergent subsequence, since $X$ is complete).

Cover $X$ by finitely many balls of radius $1$ (since $X$ is totally bounded). At least one of those balls must contain an infinite number of the $p_i$. Call that ball $B_1$, and let $S_1$ be the set of integers $i$ for which $p_i\in B_1$.

Proceeding inductively, it is clear that we can define, for each positive integer $k>1$, a ball $B_k$ of radius $1/k$ containing an infinite number of the $p_i$ for which $i\in S_{k-1}$; define $S_k$ to be the set of such $i$.

Each of the $S_k$ is infinite, so we can choose a sequence $n_k\in S_k$ with $n_k < n_{k+1}$ for all $k$. Since the $S_k$ are nested, we have that whenever $i,j\geq k$, then $n_i,n_j\in S_k$. Thus for all $i,j\geq k$, $p_{n_i}$ and $p_{n_j}$ are both contained in a ball of radius $1/k$. Hence the sequence $p_{n_k}$ is Cauchy.

\begin{thebibliography}{9}
\bibitem{munkres} J. Munkres, \emph{Topology} , Prentice Hall, 1975.
\end{thebibliography}
%%%%%
%%%%%
\end{document}
