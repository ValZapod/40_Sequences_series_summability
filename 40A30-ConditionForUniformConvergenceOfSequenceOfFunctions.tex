\documentclass[12pt]{article}
\usepackage{pmmeta}
\pmcanonicalname{ConditionForUniformConvergenceOfSequenceOfFunctions}
\pmcreated{2013-03-22 17:07:49}
\pmmodified{2013-03-22 17:07:49}
\pmowner{fernsanz}{8869}
\pmmodifier{fernsanz}{8869}
\pmtitle{condition for uniform convergence of sequence of functions}
\pmrecord{6}{39435}
\pmprivacy{1}
\pmauthor{fernsanz}{8869}
\pmtype{Proof}
\pmcomment{trigger rebuild}
\pmclassification{msc}{40A30}
\pmclassification{msc}{26A15}

% this is the default PlanetMath preamble.  as your knowledge
% of TeX increases, you will probably want to edit this, but
% it should be fine as is for beginners.

% almost certainly you want these
\usepackage{amssymb}
\usepackage{amsmath}
\usepackage{amsfonts}
\usepackage{amsthm}


% define commands here
\newtheorem{thm}{Theorem}
\newcommand{\abs}[1]{\left\vert#1\right\vert}
\begin{document}
\title{Proof of limits of functions}%
\author{Fernando Sanz Gamiz}%

\begin{thm}
Let\, $f_1,\,f_2,\,\ldots$\, be a sequence of real or complex
functions defined on the interval\, $[a,\,b]$.\, The sequence
converges uniformly to the \emph{limit function} $f$ on the
interval\, $[a,\,b]$ if and only if
$$\lim_{n\to\infty}\sup\{|f_n(x)-f(x)|, \,\, a \leq x \leq b\} = 0.$$
\end{thm}

\bigskip

\begin{proof}

Suppose the sequence converges uniformly. By the very definition of
uniform convergence, we have that for any $\epsilon$ there exist $N$
such that $$\abs{f_n(x)-f(x)}<\frac{\epsilon}{2}, \,\,\, a \leq x
\leq b \hspace{13 pt} \mbox{ for } n>N$$

\noindent hence $$\sup\{\abs{f_n(x)-f(x)}, \,\,\, a \leq x \leq b\}
<\epsilon \hspace{13 pt} \mbox{ for } n>N $$

\bigskip

\noindent Conversely, suppose the sequence does not converge
uniformly. This means that there is an $\epsilon$ for which there is
a sequence of increasing integers $n_i, i=1,2,...$ and points
$x_{n_i}$ with the corresponding subsequence of functions $f_{n_i}$
such that
$$\abs{f(x_{n_i})-f_{n_i}(x_{n_i})}>\epsilon \hspace{13 pt} \mbox{for all } i=1,2,...$$
therefore $$\sup\{\abs{f_n(x)-f(x)}, \,\,\, a \leq x \leq b\}
>\epsilon \hspace{13 pt} \mbox{ for infinitely many } n.$$
Consequently, it is not the case that
$$\lim_{n\to\infty}\sup\{|f_n(x)-f(x)|, \,\, a \leq x \leq b\} =
0.$$

\end{proof}

\begin{thm}
The uniform limit of a sequence of continuous complex or real
functions $f_n$ in the interval $[a,b]$ is continuous in $[a,b]$
\end{thm}

The proof is
\PMlinkname{here}{LimitOfAUniformlyConvergentSequenceOfContinuousFunctionsIsContinuous}
%%%%%
%%%%%
\end{document}
