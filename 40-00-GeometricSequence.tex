\documentclass[12pt]{article}
\usepackage{pmmeta}
\pmcanonicalname{GeometricSequence}
\pmcreated{2013-03-22 14:38:52}
\pmmodified{2013-03-22 14:38:52}
\pmowner{pahio}{2872}
\pmmodifier{pahio}{2872}
\pmtitle{geometric sequence}
\pmrecord{14}{36238}
\pmprivacy{1}
\pmauthor{pahio}{2872}
\pmtype{Definition}
\pmcomment{trigger rebuild}
\pmclassification{msc}{40-00}
\pmrelated{GeometricSeries}
\pmrelated{LimitOfRealNumberSequence}
\pmdefines{common ratio}

\endmetadata

% this is the default PlanetMath preamble.  as your knowledge
% of TeX increases, you will probably want to edit this, but
% it should be fine as is for beginners.

% almost certainly you want these
\usepackage{amssymb}
\usepackage{amsmath}
\usepackage{amsfonts}

% used for TeXing text within eps files
%\usepackage{psfrag}
% need this for including graphics (\includegraphics)
%\usepackage{graphicx}
% for neatly defining theorems and propositions
%\usepackage{amsthm}
% making logically defined graphics
%%%\usepackage{xypic}

% there are many more packages, add them here as you need them

% define commands here
\begin{document}
A sequence of the form
                       $$a,\,ar,\,ar^2,\,ar^3,\,\ldots$$
of real or complex numbers is called {\em geometric sequence}.\, \PMlinkescapetext{Characteristic} of the geometric sequence is thus that every two consecutive members of the sequence have the constant ratio $r$, called usually the {\em common ratio} of the sequence (if\, $ar = 0$, \PMlinkescapetext{strictly} speaking the ratio of members does not exist). 

The $n^\mathrm{th}$ member of the geometric sequence has the \PMlinkescapetext{formula}
                          $$a_n = ar^{n-1}.$$
Let\, $a \neq 0$.\, The sequence is convergent for\, $|r| < 1$\, having the \PMlinkname{limit}{LimitOfRealNumberSequence} 0, and for\, $r = 1$\, having as constant sequence the limit $a$.

When the members of the sequence are positive numbers, each member is the geometric mean of the preceding and the following member; the name ``geometric sequence''(or ``geometric series'') is due to this fact (a \PMlinkescapetext{comparable} fact is true for the harmonic series and harmonic mean).
%%%%%
%%%%%
\end{document}
