\documentclass[12pt]{article}
\usepackage{pmmeta}
\pmcanonicalname{LimitLawsForUniformConvergence}
\pmcreated{2013-03-22 15:23:06}
\pmmodified{2013-03-22 15:23:06}
\pmowner{stevecheng}{10074}
\pmmodifier{stevecheng}{10074}
\pmtitle{limit laws for uniform convergence}
\pmrecord{4}{37215}
\pmprivacy{1}
\pmauthor{stevecheng}{10074}
\pmtype{Theorem}
\pmcomment{trigger rebuild}
\pmclassification{msc}{40A30}

% this is the default PlanetMath preamble.  as your knowledge
% of TeX increases, you will probably want to edit this, but
% it should be fine as is for beginners.

% almost certainly you want these
\usepackage{amssymb}
\usepackage{amsmath}
\usepackage{amsfonts}

% used for TeXing text within eps files
%\usepackage{psfrag}
% need this for including graphics (\includegraphics)
%\usepackage{graphicx}
% for neatly defining theorems and propositions
%\usepackage{amsthm}
% making logically defined graphics
%%%\usepackage{xypic}

% there are many more packages, add them here as you need them
\usepackage{enumerate}
\usepackage{amsthm}


% define commands here
\newcommand{\real}{\mathbb{R}}
\newcommand{\rat}{\mathbb{Q}}
\newcommand{\nat}{\mathbb{N}}
\newcommand{\complex}{\mathbb{C}}

\providecommand{\abs}[1]{\lvert#1\rvert}
\providecommand{\absW}[1]{\left\lvert#1\right\rvert}
\providecommand{\absB}[1]{\Bigl\lvert#1\Bigr\rvert}
\providecommand{\norm}[1]{\lVert#1\rVert}
\providecommand{\normW}[1]{\left\lVert#1\right\rVert}
\providecommand{\normB}[1]{\Bigl\lVert#1\Bigr\rVert}
\providecommand{\defnterm}[1]{\emph{#1}}

\newtheorem{thm}{Theorem}
\begin{document}
As might be expected, the usual (pointwise) limit laws --- for instance, that the sum of limits is the limit of sums --- have analogues for uniform limits.  The laws for uniform limits are usually not mentioned in elementary textbooks, but they are useful in situations where having uniform convergence is crucial,
such as when working with infinite sums or products of holomorphic or meromorphic functions.

The uniform laws may be derived simply by reinterpreting the pointwise proofs from any calculus text, with some extra conditions.  Some of these results are listed below.  

In the following, $X$ is a metric space, and $Y$ is another metric space (with the appropriate operations defined), while $f_n\colon X \to Y$ and $g_n\colon X \to Y$ (with $n \in \nat$)
are functions that converge uniformly on $X$ to $f\colon X \to Y$ and $g\colon X \to Y$, respectively.  (For example, $X$ and $Y$ may be $\complex$, the complex plane.)

\begin{thm}
If $Y$ is a normed vector space, 
then
$f_n + g_n$ uniformly converge to $f + g$.
\end{thm}

\begin{thm}
If $Y$ is a Banach algebra, and both $f(X)$ and $g(X)$ are bounded,
then $f_n \cdot g_n$ uniformly converge to $f \cdot g$.
\end{thm}

\begin{thm}
Let $Z$ be another metric space. 
Suppose that $X$ is compact, $Y$ is locally compact,
and $f\colon X \to Y$, and $h\colon Y \to Z$
are continuous functions.  Then $h \circ f_n$ converge to $h \circ f$
uniformly.
\end{thm}
%%%%%
%%%%%
\end{document}
