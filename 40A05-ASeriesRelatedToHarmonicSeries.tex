\documentclass[12pt]{article}
\usepackage{pmmeta}
\pmcanonicalname{ASeriesRelatedToHarmonicSeries}
\pmcreated{2013-03-22 17:56:40}
\pmmodified{2013-03-22 17:56:40}
\pmowner{pahio}{2872}
\pmmodifier{pahio}{2872}
\pmtitle{a series related to harmonic series}
\pmrecord{5}{40441}
\pmprivacy{1}
\pmauthor{pahio}{2872}
\pmtype{Example}
\pmcomment{trigger rebuild}
\pmclassification{msc}{40A05}
%\pmkeywords{series with positive terms}
\pmrelated{PTest}
\pmrelated{RaabesCriteria}

\endmetadata

% this is the default PlanetMath preamble.  as your knowledge
% of TeX increases, you will probably want to edit this, but
% it should be fine as is for beginners.

% almost certainly you want these
\usepackage{amssymb}
\usepackage{amsmath}
\usepackage{amsfonts}

% used for TeXing text within eps files
%\usepackage{psfrag}
% need this for including graphics (\includegraphics)
%\usepackage{graphicx}
% for neatly defining theorems and propositions
 \usepackage{amsthm}
% making logically defined graphics
%%%\usepackage{xypic}

% there are many more packages, add them here as you need them

% define commands here

\theoremstyle{definition}
\newtheorem*{thmplain}{Theorem}

\begin{document}
The series 
\begin{align}
\sum_{n=1}^\infty\frac{1}{n\sqrt[n]{n}} = \sum_{n=1}^\infty \frac{1}{n^{1+\frac{1}{n}}}
\end{align}
is divergent.\, In fact, since for every positive integer n, one has\, $2^n > n$,\, i.e.\, $\sqrt[n]{n} < 2$, any \PMlinkescapetext{term} of the series satisfies
$$\frac{1}{n\sqrt[n]{n}} > \frac{1}{2n}.$$
Because the harmonic series and therefore also $\sum_{1}^\infty\frac{1}{2n}$ diverges, the comparison test implies that the series (1) diverges.

%%%%%
%%%%%
\end{document}
