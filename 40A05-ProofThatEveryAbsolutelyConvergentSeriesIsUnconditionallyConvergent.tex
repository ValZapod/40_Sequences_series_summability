\documentclass[12pt]{article}
\usepackage{pmmeta}
\pmcanonicalname{ProofThatEveryAbsolutelyConvergentSeriesIsUnconditionallyConvergent}
\pmcreated{2013-03-11 19:17:17}
\pmmodified{2013-03-11 19:17:17}
\pmowner{Filipe}{28191}
\pmmodifier{}{0}
\pmtitle{Proof that every absolutely convergent series is unconditionally convergent}
\pmrecord{7}{42626}
\pmprivacy{1}
\pmauthor{Filipe}{0}
\pmtype{Proof}
\pmclassification{msc}{40A05}
\pmsynonym{}{ProofThatEveryAbsolutelyConvergentSeriesIsUnconditionallyConvergent}
%\pmkeywords{}
\pmrelated{}
\pmdefines{}

\endmetadata


\begin{document}
Suppose the series $\sum_{k=1}^\infty a_k$ converges to a $A$ and that it is also absolutely convergent. So, as $\sum_{k=1}^\infty |a_k|$ converges, we have $\forall \epsilon >0$:

$$\exists N>0:\sum_{k=n+1}^\infty |a_k|<\frac{\epsilon}{2}, \forall n>N$$

As the series converges to $A$, we may choose $n$ such that:
$$|A-\sum_{k=1}^n a_k|<\frac{\epsilon}{2}$$

Now suppose that $\sum_{k=1}^\infty b_k$ is a rearrangement of $\sum_{k=1}^\infty a_k$, that is, it exists a bijection $\sigma:\mathbb{N}\rightarrow \mathbb{N}$, such that $b_k=a_{\sigma(k)}$

Let $J=max\lbrace j:\sigma(j)\leq n \rbrace$. Now take $m\geq J$. Then:

$$\sum_{k=1}^m b_k=b_1+...+b_m=a_{\sigma(1)}+...+a_{\sigma(m)}$$

This sum must include the terms $a_1,...,a_n$, otherwise $J$ wouldn't be maximum. Thus, for $m\geq J$, we have that $\sum_{j=1}^m b_j - \sum_{k=1}^n a_k$ is a finite sum of terms $a_k$, with $k\geq n+1$. So:

$$|\sum_{j=1}^m b_j - \sum_{k=1}^n a_k|\leq \sum_{k=n+1}^\infty |a_k|<\frac{\epsilon}{2}$$

Finally, taking $m\geq J$, we have:

$$|A-\sum_{k=1}^m b_k|\leq |A-\sum_{k=1}^n a_k|+|\sum_{j=1}^m b_j - \sum_{k=1}^n a_k|<\frac{\epsilon}{2}+\frac{\epsilon}{2}=\epsilon$$

So $\sum_{k=1}^\infty b_k=A$



%%%%%
\end{document}
