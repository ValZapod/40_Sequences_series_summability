\documentclass[12pt]{article}
\usepackage{pmmeta}
\pmcanonicalname{SumOfPowers}
\pmcreated{2013-03-22 17:35:50}
\pmmodified{2013-03-22 17:35:50}
\pmowner{CWoo}{3771}
\pmmodifier{CWoo}{3771}
\pmtitle{sum of powers}
\pmrecord{21}{40011}
\pmprivacy{1}
\pmauthor{CWoo}{3771}
\pmtype{Definition}
\pmcomment{trigger rebuild}
\pmclassification{msc}{40A05}
\pmrelated{HarmonicSeries}
\pmrelated{SumOfKthPowersOfTheFirstNPositiveIntegers}
\pmdefines{Faulhaber's formula}

\endmetadata

\usepackage{amssymb,amscd}
\usepackage{amsmath}
\usepackage{amsfonts}
\usepackage{mathrsfs}
\usepackage{tabls}

% used for TeXing text within eps files
%\usepackage{psfrag}
% need this for including graphics (\includegraphics)
%\usepackage{graphicx}
% for neatly defining theorems and propositions
\usepackage{amsthm}
% making logically defined graphics
%%\usepackage{xypic}
\usepackage{pst-plot}
\usepackage{psfrag}

% define commands here
\newtheorem{prop}{Proposition}
\newtheorem{thm}{Theorem}
\newtheorem{ex}{Example}
\newcommand{\real}{\mathbb{R}}
\newcommand{\pdiff}[2]{\frac{\partial #1}{\partial #2}}
\newcommand{\mpdiff}[3]{\frac{\partial^#1 #2}{\partial #3^#1}}
\begin{document}
\subsection*{Introduction}

A \emph{sum of powers} is a number expressible in the form
$$1^m+2^m+\cdots + n^m$$
where $m$ and $n$ are positive integers.  Let us denote this sum by $S_m(n)$.  It is a well-known fact that when $m=1$, then we have the following equality
$$S_1(n):=1+2+\cdots +n = \frac{n(n+1)}{2}.$$
In fact, a famous story goes that, when the teacher asked the students to sum up all the integers from 1 to 100, Gauss, at the age of 10, solved the problem immediately.  Gauss realized that the sum can be written in two ways
$$\underbrace{
\xymatrix@R=0pt@C=0pt{
&1& +&2& +&\cdots& + &n &&\\
&n& + &(n-1)&+ &\cdots&+ &1 &&\\
\ar@{-}[rrrrrrrr]&&&&&&&& \\
&(n+1)& +&(n+1)& +&\cdots& + &(n+1) &&
}}_{\displaystyle{n}\mbox{ terms}}
$$
Therefore, $2S_1(n)=n(n+1)$ and hence the result: $1+2+\cdots+100=\frac{100(101)}{2}=50\cdot 100=5050$.

What Gauss did is actually a special case of a method that can be generalized to the case when $m>1$.  Thinking of a number $i$ as $1$ added to itself $i$ times, then using the following diagram:
$$
\xymatrix@R=0pt@C=0pt{
&1& && && & &&1&&\\
&1& + &1& &&& &&2&&\\
&\vdots&  &\vdots& &\ddots && &&\vdots &&\\
&1& +&1& +&\cdots& + &1 && n&&\\
\ar@{-}[rrrrrrrr]&&&&&&& &\ar@{-}[rr]&& \\
&n& +&(n-1)& +&\cdots& + &1 && S_1(n)&&
}
$$
Using the summation notation, it then follows that $$\sum_{i=1}^n (n-i+1)=n+(n-1)+\cdots +1=S_1(n)=1+2+\cdots +n=\sum_{i=1}^n i.$$  Then, $$S_1(n)=\sum_{i=1}^n (n-i+1) = \sum_{i=1}^n (n+1) - \sum_{i=1}^n i = (n+1)\sum_{i=1}^n 1 - S_1(n) = (n+1)n -S_1(n),$$ and we have the same result.

Let us now apply this method to the case when $m=2$.  This time, we use the following diagram
$$
\xymatrix@R=0pt@C=0pt{
&1& && && & &&S_1(1)&&\\
&1& + &2& &&& &&S_1(2)&&\\
&\vdots&  &\vdots& &\ddots && &&\vdots &&\\
&1& +&2& +&\cdots& + &n && S_1(n)&&\\
\ar@{-}[rrrrrrrr]&&&&&&& &\ar@{-}[rr]&& \\
&n\cdot 1& +&(n-1)\cdot 2& +&\cdots& + &1\cdot n && T&&
}
$$
Again, using the summation notation, we have
$$\sum_{i=1}^n (n-i+1)i = T = \sum_{i=1}^n S_1(i) = \sum_{i=1}^n \frac{i(i+1)}{2}.$$
Further algebraic manipulations give us the result:
$$S_2(n):=1^2+2^2+\cdots +n^2=\frac{n(n+1)(2n+1)}{6}.$$

\subsection*{The general case}

Using the method above, one can iteratively solve for $S_3(n),S_4(n),\ldots$.  This means that if we know the formulas for $S_k(n)$ for all $k<p$, then we can derive the formula for $S_m(n)$.  The formulas for $m=3,4,5$ and $6$ are 
\begin{center}
\begin{tabular}{|l|l|}
\hline
$m$ & $S_m(n)$ \\
\hline\hline
$3$ & $\displaystyle{\frac{n^2(n+1)^2}{4}}$ \\
\hline
$4$ & $\displaystyle{\frac{n(n+1)(2n+1)(3n^2+3n-1)}{30}}$ \\
\hline
$5$ & $\displaystyle{\frac{n^2(n+1)^2(2n^2+2n-1)}{12}}$ \\
\hline
$6$ & $\displaystyle{\frac{n(n+1)(2n+1)(3n^4+6n^3-3n +1)}{42}}$ \\
\hline
\end{tabular}
\end{center}

\textbf{Remarks}.
\begin{enumerate}
\item
Notice that the formula for each of the powers $m=1,2,\ldots, 6$ is always a polynomial of degree $m+1$.  This is in fact true in the general case, and can be proved by the method of telescoping sum, together with induction: for each positive integer $n$, define $$x_n:=n^{m+1}-(n-1)^{m+1}.$$  Then $x_n$ is expressible as a polynomial $y(n)$ of degree $m$ in $n$.  So $$n^{m+1}=x_n+x_{n-1}+\cdots +x_1 = \sum_{i=1}^n y(i)=a_m \sum_{i=1}^n i^m + a_{m-1}\sum_{i=1}^n i^{m-1} + \cdots + a_0 \sum_{i=1}^n 1.$$
The right hand side is $a_mS_m(n)+a_{m-1}S_{m-1}(n)+\cdots + a_0n$.  By induction, each $S_k(n)$ is a polynomial of degree $k+1$ for all $k<m$.  But then right hand side is equal to the left hand side, which is a polynomial of degree $m+1$. Therefore, $S_m(n)$ is a polynomial of degree $m+1$.  So we can safely write $$S_m(n)=c_0+c_1n+\cdots + c_{m+1}n^{m+1}.$$
\item
Another property of $S_m(n)$ is that $c_0+c_1+\cdots + c_{m+1}=1$ for any $m$.  This is clear, since, one the one hand, $S_m(1)=1$ as a sum of just one term $1^m$.  On the other hand, its polynomial representation evaluated at $n=1$ is just the sum of the coefficients.
\item
In addition, there are also two curious properties: 
\begin{itemize}
\item
for each $m>0$, both $n$ and $n+1$ divide $S_m(n)$, and 
\item
$2n+1$ divides $S_m(n)$ iff $m$ is even.
\end{itemize}
These can be proved using the method above.  For example, to show that $n|S_m(n)$, use the expression $$n^{m+1}= \sum_{i=1}^n y(i)=a_m \sum_{i=1}^n i^m + a_{m-1}\sum_{i=1}^n i^{m-1} + \cdots + a_0 \sum_{i=1}^n 1,$$ and notice that the left hand side is divisible by $n$, and therefore so is the right hand side.  Since all sums with powers $k<m$ are divisible by $n$ (induction), so is the sum with power $m$, and hence $n|S_m(n)$.  From this, one also sees that $$S_m(n)=\sum_{i=1}^n i^m=-(n+1)^m+\sum_{i=1}^{n+1} i^m $$ is divisible by $n+1$ as well.
\item
Since $(n+1)|S_m(n)$, we may write $S_m(n)$ as a polynomial in terms of $n+1$:  $$S_m(n)= d_1(n+1)+d_2(n+1)^2+\cdots + d_{m+1}(n+1)^{m+1}.$$  Notice that there is no coefficient for the constant term.  It turns out that, when $S_m(n)$ is written this way, the coefficients are much easier to express.  Jacques Bernoulli gave the formula for the coefficients: 
\begin{equation}d_i=\frac{B_{m+1-i}}{m+1}\binom{m+1}{i},\end{equation} where $B_k$ is the $k$th Bernoulli number.  The formula for $S_m(n)$ using the polynomial representation with the coefficients $d_i$ is called the \emph{Faulhaber's formula}, since Faulhaber first wrote down the polynomial in this form in 1631 (before Bernoulli).
\item
The Faulhaber's formula can actually be extended to all real numbers.  In other words, $S_m(x)$ can be viewed as a polynomial of degree $m+1$ in the real variable $x$ with coefficients $d_i$ described by $(1)$.  When $x$ is some positive integer $n$, we are back to the formula for the sum of $m$-th powers of integers $1$ through $n$.  In general, $S_m(x)$ is the indefinite sum of $x^m$: $$S_m(x)=\Delta^{-1} x^m.$$
\item
When $m<0$, we have what is known as the \PMlinkname{$m$-series}{PSeries} of order $n$ with the special case $m=-1$ the harmonic series (of order $n$).  There are no polynomial representations for $S_m(n)$ when $m<0$.
\end{enumerate}

%For further properties of the sum of powers, see the refreshing article \PMlinkexternal{Sum of Powers of Integers}{http://www.xula.edu/math/Research/Colloquium/Past/Colloquium20030320-Paper-SahooP.pdf}.

\begin{thebibliography}{9}
\bibitem{AFB} A. F. Beardon, \emph{Sums of Powers of Integers}, Amer. Math. Monthly 103 (1996), pp. 201-213.
\bibitem{RC} R. Cook, \emph{On Sums of Powers of Integers}, J. Number Theory 11 (1979), pp. 516-528.
\bibitem{RWO} R. W. Owens, \emph{Sums of Powers of Integers}, Math. Mag 65 (1992), pp. 38-40.
\end{thebibliography}
%%%%%
%%%%%
\end{document}
