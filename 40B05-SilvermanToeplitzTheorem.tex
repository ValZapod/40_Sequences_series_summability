\documentclass[12pt]{article}
\usepackage{pmmeta}
\pmcanonicalname{SilvermanToeplitzTheorem}
\pmcreated{2013-03-22 14:51:28}
\pmmodified{2013-03-22 14:51:28}
\pmowner{rspuzio}{6075}
\pmmodifier{rspuzio}{6075}
\pmtitle{Silverman-Toeplitz theorem}
\pmrecord{10}{36531}
\pmprivacy{1}
\pmauthor{rspuzio}{6075}
\pmtype{Theorem}
\pmcomment{trigger rebuild}
\pmclassification{msc}{40B05}

% this is the default PlanetMath preamble.  as your knowledge
% of TeX increases, you will probably want to edit this, but
% it should be fine as is for beginners.

% almost certainly you want these
\usepackage{amssymb}
\usepackage{amsmath}
\usepackage{amsfonts}

% used for TeXing text within eps files
%\usepackage{psfrag}
% need this for including graphics (\includegraphics)
%\usepackage{graphicx}
% for neatly defining theorems and propositions
%\usepackage{amsthm}
% making logically defined graphics
%%%\usepackage{xypic}

% there are many more packages, add them here as you need them

% define commands here
\begin{document}
Let $\{a_{mn}\}$ be a double sequence of complex numbers and let $B$ be a positive real number such that:
\begin{enumerate}
\item $\sum_{n=0}^\infty |a_{mn}| \le B$ for all $m = 0, 1, 2, \ldots$
\item $\lim_{m \to \infty} \sum_{n=0}^\infty a_{mn} = 1$
\item For every $n = 0,1,2,\ldots$, it is the case that $\lim_{m \to \infty} a_{mn} = 0$
\end{enumerate}
Then, if the sequence $\{z_n\}$ converges, the series $\sum_{n=0}^ \infty a_{mn} z_n$ converges and
 $$\lim_{n \to \infty} z_n = \lim_{m \to \infty} \sum_{n=0}^ \infty a_{mn} z_n$$
%%%%%
%%%%%
\end{document}
