\documentclass[12pt]{article}
\usepackage{pmmeta}
\pmcanonicalname{ManipulatingConvergentSeries}
\pmcreated{2013-03-22 14:50:49}
\pmmodified{2013-03-22 14:50:49}
\pmowner{pahio}{2872}
\pmmodifier{pahio}{2872}
\pmtitle{manipulating convergent series}
\pmrecord{12}{36517}
\pmprivacy{1}
\pmauthor{pahio}{2872}
\pmtype{Theorem}
\pmcomment{trigger rebuild}
\pmclassification{msc}{40A05}
\pmclassification{msc}{26A06}
\pmrelated{SumOfSeries}
\pmrelated{MultiplicationOfSeries}

% this is the default PlanetMath preamble.  as your knowledge
% of TeX increases, you will probably want to edit this, but
% it should be fine as is for beginners.

% almost certainly you want these
\usepackage{amssymb}
\usepackage{amsmath}
\usepackage{amsfonts}

% used for TeXing text within eps files
%\usepackage{psfrag}
% need this for including graphics (\includegraphics)
%\usepackage{graphicx}
% for neatly defining theorems and propositions
 \usepackage{amsthm}
% making logically defined graphics
%%%\usepackage{xypic}

% there are many more packages, add them here as you need them

% define commands here
\theoremstyle{definition}
\newtheorem{thmplain}{Theorem}
\begin{document}
The \PMlinkescapetext{terms} of the series in the following theorems are supposed to be either real or complex numbers.

\begin{thmplain}
 \, If the series\, $a_1+a_2+\cdots$\, and\, $b_1+b_2+\cdots$\, converge and have the sums $a$ and $b$, respectively, then also the series
\begin{align}
           (a_1+b_1)+(a_2+b_2)+\cdots
\end{align}
converges and has the sum $a\!+\!b$.
\end{thmplain}

{\em Proof.} \,The $n^\mathrm{th}$ partial sum of (1) has the limit
$$\lim_{n\to\infty}\sum_{j = 1}^n(a_j+b_j) = 
\lim_{n\to\infty}\sum_{j = 1}^na_j+\lim_{n\to\infty}\sum_{j = 1}^nb_j = a\!+\!b.$$

\begin{thmplain}
 \, If the series\, $a_1+a_2+\cdots$\, converges having the sum $a$ and if $c$ is any \PMlinkescapetext{constant}, then also the series
\begin{align}
              ca_1+ca_2+\cdots
\end{align}
converges and has the sum $ca$.
\end{thmplain}

{\em Proof.} \,The $n^\mathrm{th}$ partial sum of (2) has the limit
$$\lim_{n\to\infty}\sum_{j = 1}^nca_j = c\lim_{n\to\infty}\sum_{j = 1}^na_j = 
ca.$$

\begin{thmplain}
 \, If the \PMlinkescapetext{terms} of any converging series
\begin{align}
a_1+a_2+a_3+\cdots
\end{align}
are grouped arbitrarily without changing their \PMlinkescapetext{order}, then the resulting series
\begin{align}
(a_1+\cdots+a_{m_1})+(a_{m_1+1}+\cdots+a_{m_2})+(a_{m_2+1}+\cdots+a_{m_3})+\cdots
\end{align}
also converges and its sum equals to the sum of (3).
\end{thmplain}

{\em Proof.} \,Since all the partial sums of (4) are simultaneously partial sums of (3), they have as limit the sum of the series (3).
%%%%%
%%%%%
\end{document}
