\documentclass[12pt]{article}
\usepackage{pmmeta}
\pmcanonicalname{SlowerDivergentSeries}
\pmcreated{2013-03-22 15:08:27}
\pmmodified{2013-03-22 15:08:27}
\pmowner{pahio}{2872}
\pmmodifier{pahio}{2872}
\pmtitle{slower divergent series}
\pmrecord{13}{36885}
\pmprivacy{1}
\pmauthor{pahio}{2872}
\pmtype{Theorem}
\pmcomment{trigger rebuild}
\pmclassification{msc}{40A05}
\pmrelated{SlowerConvergentSeries}
\pmrelated{NonExistenceOfUniversalSeriesConvergenceCriterion}

% this is the default PlanetMath preamble.  as your knowledge
% of TeX increases, you will probably want to edit this, but
% it should be fine as is for beginners.

% almost certainly you want these
\usepackage{amssymb}
\usepackage{amsmath}
\usepackage{amsfonts}

% used for TeXing text within eps files
%\usepackage{psfrag}
% need this for including graphics (\includegraphics)
%\usepackage{graphicx}
% for neatly defining theorems and propositions
 \usepackage{amsthm}
% making logically defined graphics
%%%\usepackage{xypic}

% there are many more packages, add them here as you need them

% define commands here

\theoremstyle{definition}
\newtheorem*{thmplain}{Theorem}
\begin{document}
\begin{thmplain}
\, If 
\begin{align}
a_1\!+\!a_2\!+\!a_3\!+\cdots
\end{align}
is a diverging series with positive \PMlinkescapetext{terms}, then one can always form another diverging series 
$$s_1\!+\!s_2\!+\!s_3\!+\cdots$$
with positive \PMlinkescapetext{terms} such that 
\begin{align}
\lim_{n\to\infty}\frac{s_n}{a_n} = 0.
\end{align}
\end{thmplain}

{\em Proof.}\,\, Let\, $S_n = a_1\!+\!a_2\!+\cdots+\!a_n$\, be the $n^\mathrm{th}$ partial sum of (1).\, Then we have 
$$a_n = S_n\!-\!S_{n-1} = 
(\sqrt{S_n}\!+\!\sqrt{S_{n-1}})(\sqrt{S_n}\!-\!\sqrt{S_{n-1}}).$$
We set\, $s_1 := \sqrt{S_1}$\, and
$$s_n := \frac{a_n}{\sqrt{S_n}\!+\!\sqrt{S_{n-1}}} = 
\sqrt{S_n}\!-\!\sqrt{S_{n-1}}$$ 
for\, $n = 2,\,3,\,4,\,\ldots$\, Then the \PMlinkescapetext{terms} of the series
$$\sum_{n = 1}^{\infty}s_n = 
    \sqrt{S_1}\!+\!\sum_{n = 1}^{\infty}(\sqrt{S_{n+1}}\!-\!\sqrt{S_n})$$
apparently are positive.\, This series is however divergent, because the sum of its $n$ first \PMlinkescapetext{terms} is equal to $\sqrt{S_n}$ which grows without bound along with $n$ since (1) diverges.\, For this reason we also get the result (2).

\textbf{Remark.}\, Niels Henrik Abel has presented a simpler example on such series $s_1\!+\!s_2\!+\!s_3\!+\cdots$:
$$1\!+\!\frac{a_2}{a_1\!+\!a_2}\!+\!\frac{a_3}{a_1\!+\!a_2\!+\!a_3}\!
+\!\frac{a_4}{a_1\!+\!a_2\!+\!a_3\!+\!a_4}\!+\cdots$$
%%%%%
%%%%%
\end{document}
