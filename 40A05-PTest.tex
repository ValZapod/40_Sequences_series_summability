\documentclass[12pt]{article}
\usepackage{pmmeta}
\pmcanonicalname{PTest}
\pmcreated{2013-03-22 15:08:51}
\pmmodified{2013-03-22 15:08:51}
\pmowner{alozano}{2414}
\pmmodifier{alozano}{2414}
\pmtitle{$p$ test}
\pmrecord{6}{36894}
\pmprivacy{1}
\pmauthor{alozano}{2414}
\pmtype{Corollary}
\pmcomment{trigger rebuild}
\pmclassification{msc}{40A05}
\pmsynonym{p-test}{PTest}
\pmsynonym{$p$-test}{PTest}
\pmsynonym{p test}{PTest}
\pmsynonym{p series test}{PTest}
\pmsynonym{$p$-series test}{PTest}
\pmsynonym{$p$ series test}{PTest}
\pmrelated{ExamplesUsingComparisonTestWithoutLimit}
\pmrelated{ASeriesRelatedToHarmonicSeries}

% this is the default PlanetMath preamble.  as your knowledge
% of TeX increases, you will probably want to edit this, but
% it should be fine as is for beginners.

% almost certainly you want these
\usepackage{amssymb}
\usepackage{amsmath}
\usepackage{amsthm}
\usepackage{amsfonts}

% used for TeXing text within eps files
%\usepackage{psfrag}
% need this for including graphics (\includegraphics)
%\usepackage{graphicx}
% for neatly defining theorems and propositions
%\usepackage{amsthm}
% making logically defined graphics
%%%\usepackage{xypic}

% there are many more packages, add them here as you need them

% define commands here

\newtheorem{thm}{Theorem}
\newtheorem{defn}{Definition}
\newtheorem{prop}{Proposition}
\newtheorem{lemma}{Lemma}
\newtheorem*{cor}{Corollary}

\theoremstyle{definition}
\newtheorem{exa}{Example}

% Some sets
\newcommand{\Nats}{\mathbb{N}}
\newcommand{\Ints}{\mathbb{Z}}
\newcommand{\Reals}{\mathbb{R}}
\newcommand{\Complex}{\mathbb{C}}
\newcommand{\Rats}{\mathbb{Q}}
\newcommand{\Gal}{\operatorname{Gal}}
\newcommand{\Cl}{\operatorname{Cl}}
\newcommand{\lc}{\lim_{x\to c}}
\newcommand{\lzero}{\lim_{x\to 0}}
\newcommand{\lhzero}{\lim_{h\to 0}}
\newcommand{\linf}{\lim_{x\to \infty}}
\newcommand{\limn}{\lim_{n\to\infty}}
\newcommand{\sumi}{\sum_{i=1}^\infty }
\newcommand{\sumn}{\sum_{n=1}^\infty }
\newcommand{\sumno}{\sum_{n=0}^\infty }
\newcommand{\sumio}{\sum_{i=1}^\infty }
\begin{document}
The following is an immediate corollary of the integral test.

\begin{cor}[{\bf $p$-Test}]
A series of the form $\sumn \frac{1}{n^p}$ converges if $p>1$ and diverges if $p\leq 1$.
\end{cor}
\begin{proof}
The case $p=1$ is well-known, for $\sumn \frac{1}{n}$ is the harmonic series, which diverges (see \PMlinkname{this proof}{ProofOfDivergenceOfHarmonicSreies}). From now on, we assume $p\neq 1$ (notice that one could also use the integral test to prove the case $p=1$). In order to apply the integral test, we need to calculate the following improper integral:
$$\int_1^\infty \frac{1}{x^p} dx=\lim_{n\to \infty}\left[ \frac{x^{1-p}}{1-p} \right]_1^n= \limn \frac{n^{-p+1}}{1-p}-\frac{1}{1-p}.$$
Since $\limn n^t$ diverges when $t>0$ and converges for $t \leq 0$, the integral above converges for $1-p < 0$, i.e. for $p>1$ and diverges for $p<1$ (and also diverges for $p=1$). Therefore, the corollary follows by the integral test.
\end{proof}
%%%%%
%%%%%
\end{document}
