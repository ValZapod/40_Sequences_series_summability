\documentclass[12pt]{article}
\usepackage{pmmeta}
\pmcanonicalname{CauchysRootTest}
\pmcreated{2013-03-22 12:58:03}
\pmmodified{2013-03-22 12:58:03}
\pmowner{Mathprof}{13753}
\pmmodifier{Mathprof}{13753}
\pmtitle{Cauchy's root test}
\pmrecord{9}{33337}
\pmprivacy{1}
\pmauthor{Mathprof}{13753}
\pmtype{Theorem}
\pmcomment{trigger rebuild}
\pmclassification{msc}{40A05}
\pmsynonym{root test}{CauchysRootTest}
\pmrelated{LambertSeries}

\endmetadata

% this is the default PlanetMath preamble.  as your knowledge
% of TeX increases, you will probably want to edit this, but
% it should be fine as is for beginners.

% almost certainly you want these
\usepackage{amssymb}
\usepackage{amsmath}
\usepackage{amsfonts}

% used for TeXing text within eps files
%\usepackage{psfrag}
% need this for including graphics (\includegraphics)
%\usepackage{graphicx}
% for neatly defining theorems and propositions
%\usepackage{amsthm}
% making logically defined graphics
%%%\usepackage{xypic}

% there are many more packages, add them here as you need them

% define commands here
\begin{document}
If $\sum a_{n}$ is a series of positive real terms and
$$\sqrt[n]{a_{n}} < k < 1$$
for all $n > N$, then $\sum a_{n}$ is convergent.  If $\sqrt[n]{a_{n}} \geq 1$ for an infinite number of values of $n$, then $\sum a_{n}$ is divergent.

\subsubsection*{Limit form} Given a series $\sum a_{n}$ of complex terms, set
$$\rho = \limsup_{n \to \infty} \sqrt[n]{| a_{n} |}$$
The series $\sum a_{n}$ is absolutely convergent if $\rho < 1$ and is divergent if $\rho > 1$.  If $\rho = 1$, then the test is inconclusive.
%%%%%
%%%%%
\end{document}
