\documentclass[12pt]{article}
\usepackage{pmmeta}
\pmcanonicalname{SilverRectangle}
\pmcreated{2013-03-22 16:42:12}
\pmmodified{2013-03-22 16:42:12}
\pmowner{PrimeFan}{13766}
\pmmodifier{PrimeFan}{13766}
\pmtitle{silver rectangle}
\pmrecord{6}{38918}
\pmprivacy{1}
\pmauthor{PrimeFan}{13766}
\pmtype{Definition}
\pmcomment{trigger rebuild}
\pmclassification{msc}{40A05}

\endmetadata

% this is the default PlanetMath preamble.  as your knowledge
% of TeX increases, you will probably want to edit this, but
% it should be fine as is for beginners.

% almost certainly you want these
\usepackage{amssymb}
\usepackage{amsmath}
\usepackage{amsfonts}

% used for TeXing text within eps files
%\usepackage{psfrag}
% need this for including graphics (\includegraphics)
%\usepackage{graphicx}
% for neatly defining theorems and propositions
%\usepackage{amsthm}
% making logically defined graphics
%%%\usepackage{xypic}

% there are many more packages, add them here as you need them

% define commands here
\usepackage{pstricks}
\newcommand{\smallbox}{\psframe*(0,0)(0.9,0.9)}
\begin{document}
A {\em silver rectangle} is a rectangle with either the silver ratio for its width-to-length ratio or the ratio $1 : \sqrt{2}$. Like the golden rectangle, both silver rectangles have the property that if we remove from them a square, their ratio remains the same.

% A silver rectangle using psframe. Code by mps

\begin{figure}[A silver rectangle]
\begin{center}
\begin{pspicture*}(-1,-1)(15.142135623730950488,11)
\psset{unit=0.5cm}
\psframe(0,0)(14.142135623730950488,10)
\psline[linestyle=dotted](10,0)(10,10)
\end{pspicture*}
\caption{Removing the square on the left (demarcated by the dotted line) leaves a smaller silver rectangle on the right.}
\end{center}
\end{figure}

%%%%%
%%%%%
\end{document}
