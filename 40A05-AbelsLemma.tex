\documentclass[12pt]{article}
\usepackage{pmmeta}
\pmcanonicalname{AbelsLemma}
\pmcreated{2013-03-22 13:19:49}
\pmmodified{2013-03-22 13:19:49}
\pmowner{mathcam}{2727}
\pmmodifier{mathcam}{2727}
\pmtitle{Abel's lemma}
\pmrecord{14}{33843}
\pmprivacy{1}
\pmauthor{mathcam}{2727}
\pmtype{Theorem}
\pmcomment{trigger rebuild}
\pmclassification{msc}{40A05}
\pmsynonym{summation by parts}{AbelsLemma}
\pmsynonym{Abel's partial summation}{AbelsLemma}
\pmsynonym{Abel's identity}{AbelsLemma}
\pmsynonym{Abel's transformation}{AbelsLemma}
\pmrelated{PartialSummation}

% this is the default PlanetMath preamble.  as your knowledge
% of TeX increases, you will probably want to edit this, but
% it should be fine as is for beginners.

% almost certainly you want these
\usepackage{amssymb}
\usepackage{amsmath}
\usepackage{amsfonts}

% used for TeXing text within eps files
%\usepackage{psfrag}
% need this for including graphics (\includegraphics)
%\usepackage{graphicx}
% for neatly defining theorems and propositions
%\usepackage{amsthm}
% making logically defined graphics
%%%\usepackage{xypic}

% there are many more packages, add them here as you need them

% define commands here
\begin{document}
\PMlinkescapeword{limit}
{\bf Theorem 1 } Let 
$\{a_i\}_{i=0}^N$ and $\{b_i\}_{i=0}^N$ be sequences of 
real  (or complex) numbers with $N\ge 0$. 
For $n=0,\ldots, N$, let $A_n$ be the  partial sum
$A_n=\sum_{i=0}^na_i$.
Then 
$$\sum_{i=0}^N a_i b_i = \sum_{i=0}^{N-1}A_i(b_i-b_{i+1})+A_N b_N.$$

In the trivial case, when $N=0$, then sum on the right hand side
should be interpreted as identically zero. In other words, 
if the upper limit is below the lower limit, there is no summation. 

An inductive proof can be found \PMlinkname{here}{ProofOfAbelsLemmaByInduction}. 
The result can be found in \cite{guenther} (Exercise 3.3.5).

If the sequences are indexed from $M$ to $N$, we have the following 
variant:

{\bf Corollary}
Let $\{a_i\}_{i=M}^N$ and $\{b_i\}_{i=M}^N$ be sequences of 
real  (or complex) numbers with $0\le M \le N$. 
For $n=M,\ldots, N$, let $A_n$ be the  partial sum
$A_n=\sum_{i=M}^na_i$.
Then 
$$\sum_{i=M}^N a_i b_i = \sum_{i=M}^{N-1}A_i(b_i-b_{i+1})+A_N b_N.$$

\emph{Proof.} By defining
$a_0=\ldots =a_{M-1}=b_0=\ldots =b_{M-1} =0$, we can apply Theorem 1
to the sequences $\{a_i\}_{i=0}^N$ and $\{b_i\}_{i=0}^N$.
$\Box$
 
\begin{thebibliography}{9}
\bibitem{guenther}
R.B. Guenther, L.W. Lee,
\emph{Partial Differential Equations of Mathematical Physics and Integral Equations},
Dover Publications, 1988.
\end{thebibliography}
%%%%%
%%%%%
\end{document}
