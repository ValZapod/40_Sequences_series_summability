\documentclass[12pt]{article}
\usepackage{pmmeta}
\pmcanonicalname{UnderstandingTheZeroStateResponse}
\pmcreated{2013-03-11 19:27:00}
\pmmodified{2013-03-11 19:27:00}
\pmowner{swapnizzle}{13346}
\pmmodifier{}{0}
\pmtitle{Understanding the Zero-State Response}
\pmrecord{1}{50081}
\pmprivacy{1}
\pmauthor{swapnizzle}{0}
\pmtype{Definition}

\endmetadata

%none for now
\begin{document}
\documentclass[11pt]{article}
\usepackage{amssymb}
\usepackage{amsmath}
\usepackage{amsthm}
\usepackage{amsfonts}
\usepackage{array}
\usepackage[mathcal]{eucal}
\usepackage{xy}
\textheight 9in
\textwidth 7in
\oddsidemargin 0in
\evensidemargin 0in
\topmargin 0in
\headheight 0in
\headsep 0in
\title{Understanding the Zero-State Response}
\author{Swapnil Sunil Jain}
\date{15 January, 2007}
\begin{document}
\maketitle
By definition we know that the $\mbox{ZSR}(t)$ is the response of the system due to the input $x(t)$ when all the initial conditions are set to zero.

Now, let a LTI system $\mathbb{S}$ with input $x(t)$ and output $y(t)$ be described by the following differential equation
\begin{eqnarray}
&& a(D)y(t) = b(D)x(t)
\end{eqnarray}
where
\begin{eqnarray*}
&& a(D) = D^n + a_1 D^{n-1} + ... + a_n \\
&& b(D) = b_0 D^m + b_1 D^{m-1} + ... + b_n
\end{eqnarray*} 
and the initial conditions: $y^{(n-1)}(0^-), y^{(n-2)}(0^-), ..., y(0^-)$.
 
To find the zero-state response, we first set all the initial conditions to zero i.e. 
\begin{eqnarray*}
&& y^{(n-1)}(0^-)=0, \\
&& y^{(n-2)}(0^-)=0, \\
&& .\\
&& .\\
&& .\\
&& y(0^-)=0
\end{eqnarray*}

Then we note that x(t) can be written in the following way (due to the shifting property of $\delta(t)$):
\begin{eqnarray}
&& x(t) = \int_{-\infty}^{\infty} \delta(t-u)x(u) du
\end{eqnarray}

Now, we define a new function h(t) (known as the $\emph{impulse response}$ of the LTI system $\mathbb{S}$) the following way\footnote[1]{Simply put, the impulse response of a system $\mathbb{S}$ is the output we get when we drive the input by the impulse function $\delta(t)$} 
\begin{eqnarray}
&& \delta(t) \overset{\mathbb{S}}{\longrightarrow} h(t)
\end{eqnarray}

Then, due to the time-invariance property of the LTI system $\mathbb{S}$, it is true that, for some constant $t_0,$ 
\begin{eqnarray}
&& \delta(t-t_0) \overset{\mathbb{S}}{\longrightarrow} h(t-t_0)
\end{eqnarray}
and due to the linearity property of the LTI system $\mathbb{S}$, it is also follows that, for some constant $C_1$,
\begin{eqnarray}
&& C_1\delta(t)\overset{\mathbb{S}}{\longrightarrow} C_1 h(t)
\end{eqnarray}

Then, it follows from (2), (4) and (5) that\footnote[2]{This result makes sense if we think of the integral as an infinite sum and treat x(u) as some constant with respect to time t.}
\begin{eqnarray}
&& x(t) = \int_{-\infty}^{\infty} \delta(t-u)x(u) du \overset{\mathbb{S}}{\longrightarrow} y(t) = \int_{-\infty}^{\infty} h(t-u)x(u) du
\end{eqnarray}

Now, we define the output $y(t)$ in (6) as the zero-state response of x(t) i.e. 
\begin{eqnarray*}
&& \mbox{ZSR}(t) = \int_{-\infty}^{\infty} h(t-u)x(u) du
\end{eqnarray*}
and we also define a new binary operation '*', called the $\emph{convolution}$, as
\begin{eqnarray*}
&& h(t)*x(t) \equiv \int_{-\infty}^{\infty} h(t-u)x(u) du
\end{eqnarray*}
Using this notation, 
\begin{eqnarray}
&& \mbox{ZSR}(t) = h(t)*x(t) 
\end{eqnarray}

Thus, in order to find $\mbox{ZSR}_{(t)}$ for a given input $x(t)$, we need to find $h(t)$. In order to do this, we first define a function $z(t)$ that satisfies the following equality:
\begin{eqnarray}
&& a(D)z(t) = \delta(t)
\end{eqnarray}

Then the following also holds:
\begin{eqnarray*}
&& b_m[a(D)D^{0}(z(t))] = b_m[D^{0}(\delta(t))] \\
&& b_{m-1}[a(D)D^{1}(z(t))] = b_{m-1}[D^{1}(\delta(t))] \\
&& . \\
&& . \\
&& . \\
&& b_{m-k}[a(D)D^{k}(z(t))] = b_{m-k}[D^{k}(\delta(t))] \\
&& . \\
&& . \\
&& . \\
&& b_1[a(D)D^{m-1}(z(t))] = b_0[D^{m-1}(\delta(t))] \\
&& b_0[a(D)D^{m}(z(t))] = b_0[D^{m}(\delta(t))]
\end{eqnarray*}

Adding all the above equations together we get
\begin{eqnarray*}
&& a(D)[b_{0}D^{m}(z(t)) + b_{1}D^{m-1}(z(t)) + ... + b_{m}D^{0}(z(t))] \\
&& = b_{0}D^{m}(\delta(t)) + b_{1}D^{m-1}(\delta(t)) + ... + b_{m}D^{0}(\delta(t))
\end{eqnarray*}
or equivalently,\footnote[3]{One could have also easily come up with the following result by simply operating b(D) on both sides of (8)}
\begin{eqnarray*}
&& a(D)[b(D)z(t)] = b(D)\delta(t)
\end{eqnarray*}

Since the above equation has the same form as (1) with $x(t)=\delta(t)$, it follows that $h(t)=b(D)z(t)$ (since h(t), by definition, is the output when the input is the impulse function). Hence, 
\begin{eqnarray}
&& h(t) = b(D)z(t)
\end{eqnarray}
where $z(t)$ is given by 
\begin{eqnarray}
&& a(D)z(t)=\delta(t)
\end{eqnarray}
Thus, in order to find $h(t)$, we need to find $z(t)$ first. We will do this with the help of an example. Given,
\begin{eqnarray*}
&& a(D) = D^2 + 5D + 6 \\
&& \Rightarrow (D^2 + 5D + 6)z(t) = \delta(t) \\
&& \Rightarrow z''(t) + 5z'(t) + 6z(t) = \delta(t)
\end{eqnarray*}
with initial conditions $z'(0^-)=0, z(0^-)=0$. In order to get $\delta(t)$ on the R.H.S, $z''(t)$ must be $\delta(t)$ plus some ordinary function $m(t)$ i.e. 
\begin{eqnarray*}
&& z''(t) = \delta(t) + m(t) \\
&& \Rightarrow z'(t) = u(t) + \int_{0^-}^{t} m(a) da \\
&& \Rightarrow z(t) = r(t) +  \int_{0^-}^{t} db \int_{0^-}^{b} m(a) da 
\end{eqnarray*}

For $t \geq 0^+$, we have
\begin{eqnarray*}
&& z''(t) + 5z'(t) + 6z(t) = 0
\end{eqnarray*}
since $\delta(t)=0$ for $t \geq 0^+$. Furthermore, we now get new initial conditions at $t = 0^+$. Thus,
\begin{eqnarray*}
&& z'(t){\Big|}_{t=0^+} = 1 + 0 = 1 \\
&& z(t){\Big|}_{t=0^+} = 0 + 0 = 0 \\
\end{eqnarray*}

In general, for $t \geq 0^+$, the initial condition involving the n-1 derivative would be equal to 1 and all the other initial conditions would be 0. Thus, if
\begin{eqnarray*}
&& a(D)z(t) = \delta(t) \quad \mbox{with initial conditions } z^{(n-1)}(0^-)=0, z^{(n-2)}(0^-)=0, ..., z(0^-)=0 
\end{eqnarray*}
then for $t \geq 0^+$,
\begin{eqnarray*}
&& a(D)z(t) = 0 \quad \mbox{with initial conditions } z^{(n-1)}(0^+)=1, z^{(n-2)}(0^+)=0, ..., z(0^+)=0 
\end{eqnarray*}

In summary, we found out that the zero-state response is given by 
\begin{eqnarray}
&& \mbox{ZSR}(t) = h(t)*x(t)
\end{eqnarray}
where $h(t)$ is the impulse response of the system $\mathbb{S}$ and is given by
\begin{eqnarray}
&& h(t) = b(D)z(t)
\end{eqnarray}
where $z(t)$ is the solution of the differential equation
\begin{eqnarray}
&& a(D)z(t)=0
\end{eqnarray}
with initial conditions $z^{(n-1)}(0^+)=1, z^{(n-2)}(0^+)=0, ..., z'(0^+)=0, z(0^+)=0$. 
\end{document}
%%%%%
\end{document}
