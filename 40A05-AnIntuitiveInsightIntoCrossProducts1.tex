\documentclass[12pt]{article}
\usepackage{pmmeta}
\pmcanonicalname{AnIntuitiveInsightIntoCrossProducts1}
\pmcreated{2013-03-11 19:35:32}
\pmmodified{2013-03-11 19:35:32}
\pmowner{swapnizzle}{13346}
\pmmodifier{}{0}
\pmtitle{An Intuitive Insight Into Cross Products}
\pmrecord{1}{50135}
\pmprivacy{1}
\pmauthor{swapnizzle}{0}
\pmtype{Definition}

%none for now
\begin{document}
\documentclass[11pt]{article} \usepackage{amssymb} \usepackage{amsmath} \usepackage{amsthm} \usepackage{amsfonts} \usepackage{array} \usepackage[mathcal]{eucal} \usepackage{xy} \textheight 9in \textwidth 7in \oddsidemargin 0in \evensidemargin 0in \topmargin 0in \headheight 0in \headsep 0in \title{An Intuitive Insight Into Cross Products} \author{Swapnil Sunil Jain} \date{21 December, 2007} % irrelevant change to test editing - APK \begin{document} \maketitle When students first learn about cross products in high school or in college, they are told that cross-products are defined the following way: \begin{eqnarray*} \vec{a} \times \vec{b} &=& \begin{vmatrix} \hat{e}_1 & \hat{e}_2 & \hat{e}_3\ a_1 & a_2 & a_3 \ b_1 & b_2 & b_3 \ \end{vmatrix}\ \end{eqnarray*} Even though this expression is correct, it is very intimidating and confusing, especially for students who are not comfortable with matrices. It raises several question in their minds. How in the world can the elements of matrices be unit vectors? Why the cross product given by the determinant? Why is the determinant defined the way it is defined? In this article, I present a different approach to teaching cross products---an axiomatic approach. It turns out that all the essence of the cross product portrayed by the above matrix equation can be derived through these four basic axioms or definitions of the cross-product, \begin{eqnarray} && \vec{A}\times\vec{B} = -\vec{B}\times\vec{A}; \qquad \mbox{//anti-symmetry property} \ && \vec{A}\times(\vec{B}+\vec{C}) = \vec{A}\times\vec{B} + \vec{B}\times\vec{C} \qquad \mbox{//distributivity} \ && a(\vec{A}\times\vec{B}) = a\vec{A}\times\vec{B} = \vec{A}\times a\vec{B} \qquad \mbox{//scalar multiplication} \ && \vec{A}\times\vec{B} = \vec{C} \mbox{ with $\vec{C}$ $\perp$ to both $\vec{A}$ and $\vec{B}$ } \mbox{ //perpendicularity} \end{eqnarray} Then using these four properties we can derive the two basic properties of the cross product, \begin{eqnarray} && \vec{A}\times\vec{A} = -\vec{A}\times\vec{A} \nonumber \ && => \vec{A}\times\vec{A} = 0; \qquad \mbox{//self-product} \end{eqnarray} Now, given $\vec{A} = a_1\hat{e}_1 + a_2\hat{e}_2 + a_3\hat{e}_3$ and $\vec{B} = b_1\hat{e}_1 + b_2\hat{e}_2 + b_3\hat{e}_3$ we can say that, \begin{eqnarray*} && \vec{A}\times\vec{B} = (a_1\hat{e}_1 + a_2\hat{e}_2 + a_3\hat{e}_3)\times(b_1\hat{e}_1 + b_2\hat{e}_2 + b_3\hat{e}_3) \ && => a_1\hat{e}_1\times(b_1\hat{e}_1 + b_2\hat{e}_2 + b_3\hat{e}_3) + a_2\hat{e}_2\times(b_1\hat{e}_1 + b_2\hat{e}_2 + b_3\hat{e}_3) + a_3\hat{e}_3\times(b_1\hat{e}_1 + b_2\hat{e}_2 + b_3\hat{e}_3) \ && => a_1b_1(\hat{e}_1\times\hat{e}_1) + a_1b_2(\hat{e}_1\times\hat{e}_2) + a_1b_3(\hat{e}_1\times\hat{e}_3) + a_2b_1(\hat{e}_2\times\hat{e}_1) + \ && a_2b_2(\hat{e}_2\times\hat{e}_2) + a_2b_3(\hat{e}_2\times\hat{e}_3) + a_3b_1(\hat{e}_3\times\hat{e}_1) + a_3b_2(\hat{e}_3\times\hat{e}_2) + a_3b_3(\hat{e}_3\times\hat{e}_3) \ && => a_1b_2(\hat{e}_1\times\hat{e}_2) + a_1b_3(\hat{e}_1\times\hat{e}_3) + a_2b_1(\hat{e}_2\times\hat{e}_1) + a_2b_3(\hat{e}_2\times\hat{e}_3) + a_3b_1(\hat{e}_3\times\hat{e}_1) + a_3b_2(\hat{e}_3\times\hat{e}_2) \ && => a_1b_2(\hat{e}_1\times\hat{e}_2) + a_1b_3(\hat{e}_1\times\hat{e}_3) - a_2b_1(\hat{e}_1\times\hat{e}_2) + a_2b_3(\hat{e}_2\times\hat{e}_3) - a_3b_1(\hat{e}_1\times\hat{e}_3) - a_3b_2(\hat{e}_2\times\hat{e}_3) \ && => (a_1b_2 - a_2b_1)(\hat{e}_1\times\hat{e}_2) + (a_1b_3 - a_3b_1)(\hat{e}_1\times\hat{e}_3) + (a_2b_3 - a_3b_2)(\hat{e}_2\times\hat{e}_3) \ && => (a_1b_2 - a_2b_1)\hat{e}_3 - (a_1b_3 - a_3b_1)\hat{e}_2 + (a_2b_3 - a_3b_2)\hat{e}_1 \ && => (a_2b_3 - a_3b_2)\hat{e}_1 - (a_1b_3 - a_3b_1)\hat{e}_2 + (a_1b_2 - a_2b_1)\hat{e}_3 \end{eqnarray*} which is exactly how we "define" the cross product. \end{document} 
%%%%%
\end{document}
