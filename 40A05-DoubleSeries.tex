\documentclass[12pt]{article}
\usepackage{pmmeta}
\pmcanonicalname{DoubleSeries}
\pmcreated{2013-03-22 16:32:54}
\pmmodified{2013-03-22 16:32:54}
\pmowner{PrimeFan}{13766}
\pmmodifier{PrimeFan}{13766}
\pmtitle{double series}
\pmrecord{6}{38731}
\pmprivacy{1}
\pmauthor{PrimeFan}{13766}
\pmtype{Theorem}
\pmcomment{trigger rebuild}
\pmclassification{msc}{40A05}
\pmclassification{msc}{26A06}
\pmsynonym{double series theorem}{DoubleSeries}
%\pmkeywords{absolutely convergent}
\pmrelated{FourierSineAndCosineSeries}
\pmrelated{AbsoluteConvergenceOfDoubleSeries}
\pmrelated{PerfectPower}
\pmdefines{diagonal summing}

% this is the default PlanetMath preamble.  as your knowledge
% of TeX increases, you will probably want to edit this, but
% it should be fine as is for beginners.

% almost certainly you want these
\usepackage{amssymb}
\usepackage{amsmath}
\usepackage{amsfonts}

% used for TeXing text within eps files
%\usepackage{psfrag}
% need this for including graphics (\includegraphics)
%\usepackage{graphicx}
% for neatly defining theorems and propositions
 \usepackage{amsthm}
% making logically defined graphics
%%%\usepackage{xypic}

% there are many more packages, add them here as you need them

% define commands here

\theoremstyle{definition}
\newtheorem*{thmplain}{Theorem}

\begin{document}
\textbf{Theorem.}\, If the {\em double series}
\begin{align}
\sum_{m=1}^\infty\sum_{n=1}^\infty a_{mn} =
  \sum_{n=1}^\infty a_{1n}+\sum_{n=1}^\infty a_{2n}+\sum_{n=1}^\infty a_{3n}
+\ldots
\end{align}
converges and if it remains convergent when the \PMlinkescapetext{terms} of the partial series are replaced with their absolute values, i.e. if the series
\begin{align}
\sum_{n=1}^\infty|a_{1n}|+\sum_{n=1}^\infty|a_{2n}|+\sum_{n=1}^\infty|a_{3n}|
+\ldots
\end{align}
has a finite sum $M$, then the addition in (1) can be performed in reverse \PMlinkescapetext{order}, i.e.
$$\sum_{m=1}^\infty\sum_{n=1}^\infty a_{mn} = \sum_{n=1}^\infty\sum_{m=1}^\infty a_{mn} =
  \sum_{m=1}^\infty a_{m1}+\sum_{m=1}^\infty a_{m2}+\sum_{m=1}^\infty a_{m3}
+\ldots$$

{\em Proof.}\, The assumption on (2) implies that the sum of an arbitrary finite amount of the numbers $|a_{mn}|$ is always\ $\leqq M$.\, This means that (1) is absolutely convergent, and thus the order of summing is insignificant.

\textbf{Note.}\, The series satisfying the assumptions of the theorem is often denoted by
$$\sum_{m,n=1}^\infty a_{mn}$$
and this may by interpreted to \PMlinkescapetext{mean} an arbitrary summing \PMlinkescapetext{order}.\, One can use e.g. the {\em diagonal summing}:
$$a_{11}+a_{12}+a_{21}+a_{13}+a_{22}+a_{31}+\ldots$$
%%%%%
%%%%%
\end{document}
