\documentclass[12pt]{article}
\usepackage{pmmeta}
\pmcanonicalname{Ifsumk1inftyAkConvergesThenAkto0}
\pmcreated{2013-03-22 15:00:38}
\pmmodified{2013-03-22 15:00:38}
\pmowner{matte}{1858}
\pmmodifier{matte}{1858}
\pmtitle{if $\sum_{k=1}^\infty a_k$ converges then $a_k\to 0$}
\pmrecord{13}{36717}
\pmprivacy{1}
\pmauthor{matte}{1858}
\pmtype{Theorem}
\pmcomment{trigger rebuild}
\pmclassification{msc}{40-00}
\pmsynonym{necessary condition of convergence}{Ifsumk1inftyAkConvergesThenAkto0}
%\pmkeywords{divergence test}
\pmrelated{DeterminingSeriesConvergence}
\pmrelated{CompleteUltrametricField}
\pmrelated{ConvergenceConditionOfInfiniteProduct}
\pmrelated{LambertSeries}
\pmrelated{AbsoluteConvergenceOfIntegralAndBoundednessOfDerivative}
\pmrelated{ConvergentSeriesWhereNotOnlyA_nButAlsoNa_nTendsTo0}

% this is the default PlanetMath preamble.  as your knowledge
% of TeX increases, you will probably want to edit this, but
% it should be fine as is for beginners.

% almost certainly you want these
\usepackage{amssymb}
\usepackage{amsmath}
\usepackage{amsfonts}
\usepackage{amsthm}

\usepackage{mathrsfs}

% used for TeXing text within eps files
%\usepackage{psfrag}
% need this for including graphics (\includegraphics)
%\usepackage{graphicx}
% for neatly defining theorems and propositions
%
% making logically defined graphics
%%%\usepackage{xypic}

% there are many more packages, add them here as you need them

% define commands here

\newcommand{\sR}[0]{\mathbb{R}}
\newcommand{\sC}[0]{\mathbb{C}}
\newcommand{\sN}[0]{\mathbb{N}}
\newcommand{\sZ}[0]{\mathbb{Z}}

 \usepackage{bbm}
 \newcommand{\Z}{\mathbbmss{Z}}
 \newcommand{\C}{\mathbbmss{C}}
 \newcommand{\R}{\mathbbmss{R}}
 \newcommand{\Q}{\mathbbmss{Q}}



\newcommand*{\norm}[1]{\lVert #1 \rVert}
\newcommand*{\abs}[1]{| #1 |}



\newtheorem{thm}{Theorem}
\newtheorem{defn}{Definition}
\newtheorem{prop}{Proposition}
\newtheorem{lemma}{Lemma}
\newtheorem{cor}{Corollary}
\begin{document}
\begin{thm} Suppose $a_1,a_2, \ldots$ is a sequence of real or complex numbers.
If the series
$$
  \sum_{k=1}^\infty a_k
$$
converges, then $\lim_{k\to \infty} a_k = 0$.
\end{thm}

\subsubsection*{Remarks}
\begin{enumerate}
\item 
The harmonic series $\sum_{k=1}^\infty 1/k$ shows that the 
implication can not be reversed. 
\item This result can be used as a first test for convergence of a series
  $\sum_{k=1}^\infty a_k$. If $a_k$ does not converge to $0$, then 
  $\sum_{k=1}^\infty a_k$ does not converge either. 
\end{enumerate}

\begin{proof} Let $S\in \C$ be the value of the sum, and let $\varepsilon>0$
be arbitrary. Then there exists an $N\ge 1$ such that 
$$
  | \sum_{k=1}^M a_k -S | < \frac{\varepsilon}{2}
$$
for all $M\ge N$. For $j\ge N$ we then have
\begin{eqnarray*}
|a_{j+1}| &=& | \sum_{k=1}^{j+1} a_k -\sum_{k=1}^j a_k| \\
          &\le & | \sum_{k=1}^{j+1} a_k -S | + |S - \sum_{k=1}^j a_k| \\
          &<& \varepsilon,
\end{eqnarray*}
and the claim follows.
\end{proof}
%%%%%
%%%%%
\end{document}
