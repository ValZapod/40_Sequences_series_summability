\documentclass[12pt]{article}
\usepackage{pmmeta}
\pmcanonicalname{TheLimitOfAUniformlyConvergentSequenceOfContinuousFunctionsIsContinuous}
\pmcreated{2013-03-22 15:21:58}
\pmmodified{2013-03-22 15:21:58}
\pmowner{neapol1s}{9480}
\pmmodifier{neapol1s}{9480}
\pmtitle{the limit of a uniformly convergent sequence of continuous functions is continuous}
\pmrecord{13}{37191}
\pmprivacy{1}
\pmauthor{neapol1s}{9480}
\pmtype{Theorem}
\pmcomment{trigger rebuild}
\pmclassification{msc}{40A30}
\pmrelated{LimitFunctionOfSequence}

% this is the default PlanetMath preamble.  as your knowledge
% of TeX increases, you will probably want to edit this, but
% it should be fine as is for beginners.

% almost certainly you want these
\usepackage{amssymb}
\usepackage{amsmath}
\usepackage{amsfonts}

% used for TeXing text within eps files
%\usepackage{psfrag}
% need this for including graphics (\includegraphics)
%\usepackage{graphicx}
% for neatly defining theorems and propositions
%\usepackage{amsthm}
% making logically defined graphics
%%%\usepackage{xypic}

% there are many more packages, add them here as you need them

% define commands here
\begin{document}
{\bf Theorem.} The limit of a uniformly convergent sequence of continuous functions is continuous. 

\emph{Proof.} Let $f_n,f:X\rightarrow Y$, where $(X,\rho)$ and $(Y,d)$ are metric spaces. Suppose $f_n \rightarrow f$ uniformly and each $f_n$ is continuous. Then given any $\epsilon>0$, there exists $N$ such that $n>N$ implies $d(f(x),f_{n}(x)) < \frac{\epsilon}{3}$ for all $x$. Pick an arbitrary $n$ larger than $N$. Since $f_n$ is continuous, given any point $x_0$, there exists $\delta>0$ such that $0<\rho(x,x_0)<\delta$ implies $d(f_n(x), f_n(x_0))<\frac{\epsilon}{3}$. Therefore, given any $x_0$ and $\epsilon>0$, there exists $\delta>0$ such that 
\[0<\rho(x,x_0)<\delta\Rightarrow d(f(x),f(x_0)) \leq d(f(x),f_n(x)) + d(f_n(x),f_n(x_0)) + d(f_n(x_0),f(x_0)) < \epsilon.\]
Therefore, $f$ is continuous.

The theorem also generalizes to when $X$ is an arbitrary topological space. To generalize it to $X$ an arbitrary topological space, note that if $d(f_n(x),f(x))< \epsilon/3$ for all $x$, then
$x_0\in f_n^{-1}(B_{\epsilon/3}(f_n(x_0)))\subseteq f^{-1}(B_\epsilon(f(x_0)))$,
so $f^{-1}(B_\epsilon(f(x_0)))$ is a neighbourhood of $x_0$. Here $B_{\epsilon}(y)$ denote the open ball of radius $\epsilon$, centered at $y$. 
%%%%%
%%%%%
\end{document}
