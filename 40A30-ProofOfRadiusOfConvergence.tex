\documentclass[12pt]{article}
\usepackage{pmmeta}
\pmcanonicalname{ProofOfRadiusOfConvergence}
\pmcreated{2013-03-22 13:21:50}
\pmmodified{2013-03-22 13:21:50}
\pmowner{mathwizard}{128}
\pmmodifier{mathwizard}{128}
\pmtitle{proof of radius of convergence}
\pmrecord{6}{33890}
\pmprivacy{1}
\pmauthor{mathwizard}{128}
\pmtype{Proof}
\pmcomment{trigger rebuild}
\pmclassification{msc}{40A30}
\pmclassification{msc}{30B10}

% this is the default PlanetMath preamble.  as your knowledge
% of TeX increases, you will probably want to edit this, but
% it should be fine as is for beginners.

% almost certainly you want these
\usepackage{amssymb}
\usepackage{amsmath}
\usepackage{amsfonts}

% used for TeXing text within eps files
%\usepackage{psfrag}
% need this for including graphics (\includegraphics)
%\usepackage{graphicx}
% for neatly defining theorems and propositions
%\usepackage{amsthm}
% making logically defined graphics
%%%\usepackage{xypic}

% there are many more packages, add them here as you need them

% define commands here
\begin{document}
According to Cauchy's root test a power series is absolutely convergent if
$$\limsup_{k\to\infty}\sqrt[k]{|a_k(x-x_0)^k|}=|x-x_0|\limsup_{k\to\infty}\sqrt[k]{|a_k|}<1.$$
This is obviously true if
$$|x-x_0|<\frac{1}{\limsup_{k\to\infty}\sqrt[k]{|a_k|}}=\liminf_{k\to\infty}\frac{1}{\sqrt[k]{|a_k|}}=.$$
In the same way we see that the series is divergent if
$$|x-x_0|>\liminf_{k\to\infty}\frac{1}{\sqrt[k]{|a_k|}},$$
which means that the right hand side is the radius of convergence of the power series.
Now from the ratio test we see that the power series is absolutely convergent if
$$\lim_{k\to\infty}\left|\frac{a_{k+1}(x-x_0)^{k+1}}{a_k(x-x_0)^k}\right|=|x-x_0| \lim_{k\to\infty}\left|\frac{a_{k+1}}{a_k}\right|<1.$$
Again this is true if
$$|x-x_0|<\lim_{k\to\infty}\left|\frac{a_k}{a_{k+1}}\right|.$$
The series is divergent if
$$|x-x_0|>\lim_{k\to\infty}\left|\frac{a_k}{a_{k+1}}\right|,$$
as follows from the ratio test in the same way. So we see that in this way too we can \PMlinkescapetext{calculate} the radius of convergence.
%%%%%
%%%%%
\end{document}
