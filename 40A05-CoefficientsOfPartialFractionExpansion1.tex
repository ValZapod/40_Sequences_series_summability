\documentclass[12pt]{article}
\usepackage{pmmeta}
\pmcanonicalname{CoefficientsOfPartialFractionExpansion1}
\pmcreated{2013-03-11 19:26:06}
\pmmodified{2013-03-11 19:26:06}
\pmowner{swapnizzle}{13346}
\pmmodifier{}{0}
\pmtitle{Coefficients of Partial Fraction Expansion}
\pmrecord{1}{50078}
\pmprivacy{1}
\pmauthor{swapnizzle}{0}
\pmtype{Definition}

%none for now
\begin{document}
\documentclass[11pt]{article}
\usepackage{amssymb}
\usepackage{amsmath}
\usepackage{amsthm}
\usepackage{amsfonts}
\usepackage{array}
\usepackage[mathcal]{eucal}
\usepackage{xy}
\textheight 9in
\textwidth 7in
\oddsidemargin 0in
\evensidemargin 0in
\topmargin 0in
\headheight 0in
\headsep 0in
\title{Coefficients of Partial Fraction Expansion}
\author{Swapnil Sunil Jain}
\date{July 28 2006}
\begin{document}
\maketitle
Let us start with the assumption (or rather a Lemma) that any rational proper function $F(s)$ of the form
\begin{eqnarray}
&& F(s) = \frac{P(s)}{(s-q)^{r}(s-p_1)(s-p_2)...(s-p_i)...(s-p_n)}
\end{eqnarray}
has a partial fraction expansion given by
\begin{eqnarray}
&& F(s) = \frac{a_{0}}{(s-q)^{r}} +  \frac{a_{1}}{(s-q)^{r-1}} + ... +  \frac{a_{j}}{(s-q)^{r-j}} + ... +  \frac{a_{r-1}}{(s-q)} \nonumber \\ 
&& \qquad +  \frac{k_{1}}{(s-p_{1})} + \frac{k_{2}}{(s-p_{2})} + ... + \frac{k_{i}}{(s-p_{i})} + ... + \frac{k_{n}}{(s-p_{n})} 
\end{eqnarray}
where $ j = 0,1,2,...,r-1$ and $i = 1,2,3,...,n$ and $q\neq p_1 \neq p_2 \neq ... \neq p_n$. 

First, we determine the coefficient $k_i$. In order to do so, we multiply both sides of equation (2) by $(s-p_{i})$ which then gives us
\begin{eqnarray}
&& (s-p_{i})F(s) = \frac{a_{0}}{(s-q)^{r}}(s-p_{i}) +  \frac{a_{1}}{(s-q)^{r-1}}(s-p_{i}) + ... +  \frac{a_{j}}{(s-q)^{r-j}}(s-p_{i}) + ... +  \frac{a_{r-1}}{(s-q)}(s-p_{i}) \nonumber \\ 
&& \qquad +  \frac{k_{1}}{(s-p_{1})}(s-p_{i}) + \frac{k_{2}}{(s-p_{2})}(s-p_{i}) + ... + \frac{k_{i}}{(s-p_{i})}(s-p_{i}) + ... + \frac{k_{n}}{(s-p_{n})}(s-p_{i}) 
\end{eqnarray}
If we then let $s=p_{i}$, all the terms on the R.H.S drop out except the one containing the coefficient $k_i$ and we get
\begin{eqnarray}
&& \Big[(s-p_{i})F(s)\Big] {\Bigg|}_{s=p_{i}} = k_i
\end{eqnarray}

Now, in order to determine the coefficient $a_j$, we multiply both sides of (2) by $(s-q)^{r}$ which yields
\begin{eqnarray}
&& (s-q)^{r} F(s) = \frac{a_{0}}{(s-q)^{r}}(s-q)^{r} +  \frac{a_{1}}{(s-q)^{r-1}}(s-q)^{r} + ... +  \frac{a_{j}}{(s-q)^{r-j}}(s-q)^{r} + ... +  \frac{a_{r-1}}{(s-q)}(s-q)^{r} \nonumber \\ 
&& \qquad + (s-q)^{r}\Bigg[\frac{k_{1}}{(s-p_{1})} + \frac{k_{2}}{(s-p_{2})} + ... + \frac{k_{i}}{(s-p_{i})} + ... + \frac{k_{n}}{(s-p_{n})} \Bigg] \nonumber \\
&& \Rightarrow (s-q)^{r} F(s) = a_{0} +  a_{1}(s-q)^{1} + ... +  a_{j}(s-q)^{j} + ... +  a_{r-1}(s-q)^{r-1} +  (s-q)^{r}\frac{A(s)}{B(s)}
\end{eqnarray}
where we have defined
\begin{eqnarray*}
&& \frac{A(s)}{B(s)} \equiv \frac{k_{1}}{(s-p_{1})} + \frac{k_{2}}{(s-p_{2})} + ... + \frac{k_{i}}{(s-p_{i})} + ... + \frac{k_{n}}{(s-p_{n})} 
\end{eqnarray*}

Then if we take the derivative of the above equation with respect to $s$ and we obtain
\begin{eqnarray}
&& \frac{d}{ds}\Big[(s-q)^{r} F(s)\Big] = a_{1} + a_{2}(2)(s-q)+ ... +  a_{j}(j)(s-q)^{j-1} + ...  \nonumber \\
&& \qquad +  a_{r-1}(r-1)(s-q)^{r-2} +  \frac{d}{ds}\Big[ (s-q)^{r}\frac{A(s)}{B(s)}\Big]
\end{eqnarray}
If we again take the derivative of both sides of the above equation with respect to $s$ we get
\begin{eqnarray}
&& \frac{d^2}{ds^2}\Big[(s-q)^{r} F(s)\Big] = 2a_{2} + ... +  a_{j}(j)(j-1)(s-q)^{j-2} + ... \nonumber \\
&& \qquad + a_{r-1}(r-1)(r-2)(s-q)^{r-3} + \frac{d^2}{ds^2}\Big[ (s-q)^{r}\frac{A(s)}{B(s)}\Big]
\end{eqnarray}
If we keep taking derivatives this way until we have taken the derivative $j$ times, we arrive at
\begin{eqnarray}
&& \frac{d^j}{ds^j}\Big[(s-q)^{r} F(s)\Big] = a_{j}(j)(j-1)(j-2)...(2)(1)(s-q)^{j-j} + ...  \nonumber \\
&& \qquad + a_{r-1}(r-1)(r-2)...(r-j)(s-q)^{r-j-1} +  \frac{d^j}{ds^j}\Big[ (s-q)^{r}\frac{A(s)}{B(s)}\Big]
\end{eqnarray}
If we then let $s=q$, all the terms on the R.H.S drop out except the one containing the coefficient $a_j$ which yields 
\begin{eqnarray}
&& \Bigg(\frac{d^j}{ds^j}\Big[(s-q)F(s)\Big]\Bigg){\Bigg|}_{s=q} = a_{j}j!
\end{eqnarray}
or 
\begin{eqnarray}
&& a_{j} = \frac{1}{j!} \Bigg(\frac{d^j}{ds^j}\Big[(s-q)F(s)\Big]\Bigg){\Bigg|}_{s=q}
\end{eqnarray}
\end{document}
%%%%%
\end{document}
