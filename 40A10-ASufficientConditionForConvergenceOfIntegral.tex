\documentclass[12pt]{article}
\usepackage{pmmeta}
\pmcanonicalname{ASufficientConditionForConvergenceOfIntegral}
\pmcreated{2013-03-22 19:01:13}
\pmmodified{2013-03-22 19:01:13}
\pmowner{pahio}{2872}
\pmmodifier{pahio}{2872}
\pmtitle{a sufficient condition for convergence of integral}
\pmrecord{10}{41891}
\pmprivacy{1}
\pmauthor{pahio}{2872}
\pmtype{Theorem}
\pmcomment{trigger rebuild}
\pmclassification{msc}{40A10}
\pmrelated{RatioTest}

% this is the default PlanetMath preamble.  as your knowledge
% of TeX increases, you will probably want to edit this, but
% it should be fine as is for beginners.

% almost certainly you want these
\usepackage{amssymb}
\usepackage{amsmath}
\usepackage{amsfonts}

% used for TeXing text within eps files
%\usepackage{psfrag}
% need this for including graphics (\includegraphics)
%\usepackage{graphicx}
% for neatly defining theorems and propositions
 \usepackage{amsthm}
% making logically defined graphics
%%%\usepackage{xypic}

% there are many more packages, add them here as you need them

% define commands here

\theoremstyle{definition}
\newtheorem*{thmplain}{Theorem}

\begin{document}
Suppose that the real function $f$ is positive and continuous on the interval \,$[a,\,\infty)$.\, A sufficient condition for the \PMlinkname{convergence}{ConvergentIntegral} of the improper integral
\begin{align}
\int_a^\infty\!f(x)\,dx
\end{align}
is that
\begin{align}
\lim_{x\to\infty}\frac{f(x\!+\!1)}{f(x)} \;=\; q \;<\; 1.
\end{align}

\emph{Proof.}\, Assume that the condition (2) is in \PMlinkescapetext{force}.\, For an \PMlinkname{indirect proof}{ReductioAdAbsurdum}, make the antithesis that the \PMlinkname{integral}{RiemannIntegral} (1) \PMlinkname{diverges}{DivergentIntegral}.

Because of the positiveness, we have\, $\int_a^\infty\!f(x)\,dx = \infty$.\, We can use \PMlinkname{l'H\^opital's rule}{LHpitalsRule}:
$$\lim_{c\to\infty}\frac{\int_a^c\!f(x\!+\!1)\,dx}{\int_a^c\!f(x)\,dx} \;=\; \lim_{c\to\infty}\frac{f(c\!+\!1)}{f(c)}.$$
Using the \PMlinkid{substitution}{11373} \,$x\!+\!1 = t$\, we get
$$\int_a^c\!f(x\!+\!1)\,dx \;=\; \int_{a-1}^{c-1}f(t)\,dt 
\;=\; \int_{a-1}^a\!f(t)\,dt+\int_a^c\!f(t)\,dt-\int_{c-1}^c\!f(t)\,dt,$$
and dividing this equation by $\int_a^cf(t)\,dt$ and taking \PMlinkname{limits}{ImproperLimits} yield ($f$ is bounded!)
$$1 \;>\; q \;=\; \lim_{c\to\infty}\frac{\int_a^c\!f(x\!+\!1)\,dx}{\int_a^c\!f(x)\,dx} \;=\; 0+1-0 \;=\; 1.$$
This contradictory result shows that the antithesis is wrong; thus (1) must be \PMlinkname{convergent}{ConvergentIntegral}.\\

\textbf{Note.}\, The condition (2) is not necessary for the convergence of (1).\, This is seen e.g. in the case of the converging \PMlinkescapetext{integral $\int_1^\infty\!x^s\,dx$ with\, $s < -1$, where the limit} of (2) equals 1.
%%%%%
%%%%%
\end{document}
