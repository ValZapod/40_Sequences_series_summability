\documentclass[12pt]{article}
\usepackage{pmmeta}
\pmcanonicalname{ListOfImproperIntegrals}
\pmcreated{2014-11-07 19:08:22}
\pmmodified{2014-11-07 19:08:22}
\pmowner{pahio}{2872}
\pmmodifier{pahio}{2872}
\pmtitle{list of improper integrals}
\pmrecord{66}{41484}
\pmprivacy{1}
\pmauthor{pahio}{2872}
\pmtype{Topic}
\pmcomment{trigger rebuild}
\pmclassification{msc}{40A10}
%\pmkeywords{improper integral}
\pmrelated{ErrorFunction}
\pmrelated{SignumFunction}
\pmrelated{EulersConstant}
\pmrelated{MethodsOfEvaluatingImproperIntegrals}
\pmrelated{AreaFunctions}
\pmrelated{ConvergenceOfIntegrals}

\endmetadata

% this is the default PlanetMath preamble.  as your knowledge
% of TeX increases, you will probably want to edit this, but
% it should be fine as is for beginners.

% almost certainly you want these
\usepackage{amssymb}
\usepackage{amsmath}
\usepackage{amsfonts}

% used for TeXing text within eps files
%\usepackage{psfrag}
% need this for including graphics (\includegraphics)
%\usepackage{graphicx}
% for neatly defining theorems and propositions
 \usepackage{amsthm}
% making logically defined graphics
%%%\usepackage{xypic}

% there are many more packages, add them here as you need them

% define commands here

\theoremstyle{definition}
\newtheorem*{thmplain}{Theorem}

\begin{document}
Below, we list some \PMlinkescapetext{simple} 
\PMlinkname{convergent}{ConvergenceOfIntegrals} improper 
integrals.\\

\PMlinkname{\textbf{1.}}{areaundergaussiancurve} \; $\displaystyle\int_0^\infty e^{-x^2}\,dx \;=\; \frac{\sqrt{\pi}}{2}$\\

\PMlinkname{\textbf{2.}}{generalisationofgaussianintegral} \; $\displaystyle\int_0^\infty e^{-x^2}\cos{kx}\,dx\;=\;\frac{\sqrt{\pi}}{2}e^{-\frac{1}{4}k^2}$\\

\PMlinkname{\textbf{3.}}{usingconvolutiontofindlaplacetransform} \; $\displaystyle\int_0^\infty \frac{e^{-x^2}}{a^2\!+\!x^2}\,dx 
\;=\;\frac{\pi}{2a}e^{a^2}\,{\rm erfc}\,a$\\

\PMlinkname{\textbf{4.}}{fresnelformulas} \; $\displaystyle\int_0^\infty\sin{x^2}\,dx \;=\; \int_0^\infty\cos{x^2}\,dx \;=\; 
\frac{\sqrt{2\pi}}{4}$\\

\PMlinkname{\textbf{5.}}{sineintegralatinfinity} \; $\displaystyle\int_0^\infty\frac{\sin{ax}}{x}\,dx \;=\; (\mbox{sgn}\,a)\frac{\pi}{2}
\qquad (a \in \mathbb{R})$\\

\PMlinkname{\textbf{6.}}{twoimproperintegrals} \; $\displaystyle\int_0^\infty\left(\frac{\sin{x}}{x}\right)^2 dx \;=\; \frac{\pi}{2}$\\

\PMlinkname{\textbf{7.}}{twoimproperintegrals} \; $\displaystyle\int_0^\infty\frac{1-\cos{kx}}{x^2}\,dx \;=\; \frac{\pi k}{2}$\\

\PMlinkname{\textbf{8.}}{usingresiduetheoremnearbranchpoint} \; $\displaystyle\int_0^\infty\frac{x^{-k}}{x\!+\!1}\,dx \;=\; \frac{\pi}{\sin{\pi k}} 
\quad (0 < k < 1)$\\

\PMlinkname{\textbf{9.}}{exampleofchangingvariable} \; $\displaystyle\int_{-\infty}^\infty\frac{e^{kx}}{1\!+\!e^x}\,dx \;=\; \frac{\pi}{\sin{\pi k}} 
\quad (0 < k < 1)$\\

\PMlinkname{\textbf{10.}}{exampleofusingresiduetheorem} \; $\displaystyle\int_0^\infty\frac{\cos{kx}}{x^2\!+\!1}\,dx \;=\; \frac{\pi}{2e^k}$\\

\PMlinkname{\textbf{11.}}{laplaceintegrals} \; $\displaystyle\int_0^\infty\frac{a\cos{x}}{x^2\!+\!a^2}\,dx 
\;=\; \int_0^\infty\frac{x\sin{x}}{x^2\!+\!a^2}\,dx \;=\; \frac{\pi}{2e^a} \quad\; (a > 0)$\\

\PMlinkname{\textbf{12.}}{applicationofsineintegralatinfinity} \; $\displaystyle\int_0^\infty\frac{\sin{ax}}{x(x^2\!+\!1)}\,dx \;=\; \frac{\pi}{2}(1-e^{-a}) \quad\; (a > 0)$\\

\PMlinkid{13.}{9223} \; $\displaystyle\int_0^\infty e^{-x}x^{-\frac{3}{2}}\,dx \;=\; \sqrt{\pi}$\\

\PMlinkname{\textbf{14.}}{laplacetransformoftnft} \; $\displaystyle\int_0^\infty e^{-x}x^3\sin{x}\,dx \;=\; 0$\\

\PMlinkid{15.}{7891} \; $\displaystyle\int_0^\infty\!\left(\frac{1}{e^x\!-\!1}-\frac{1}{xe^x}\right) dx \;=\; \gamma$\\

\PMlinkname{\textbf{16.}}{relativeofcosineintegral} \; $\displaystyle\int_0^\infty\!\frac{\cos{ax^2}-\cos{ax}}{x} dx \;=\; \frac{\gamma+\ln{a}}{2} \quad (a > 0)$\\

\PMlinkname{\textbf{17.}}{relativeofexponentialintegral} \; $\displaystyle\int_0^\infty\frac{e^{-ax}\!-\!e^{-bx}}{x}\,dx \;=\; \ln\frac{b}{a} \quad (a > 0,\;\, b > 0)$\\

\PMlinkname{\textbf{18.}}{integralrelatedtoarcsine} \; $\displaystyle\int_1^\infty\left(\arcsin\frac{1}{x}-\frac{1}{x}\right)\,dx \;=\; 1+\ln{2}-\frac{\pi}{2}$\\

\PMlinkname{\textbf{19.}}{exampleofimproperintegral} \; $\displaystyle\int_0^1\frac{\arctan{x}}{x\sqrt{1\!-\!x^2}}\,dx 
\;=\; \frac{\pi}{2}\ln(1\!+\!\sqrt{2}) \;=\; \frac{\pi}{2}\,\mbox{arsinh}\,1$\\

\PMlinkname{\textbf{20.}}{applicationoflogarithmseries} \; 
$\displaystyle\int_0^1\frac{\ln(1\!+\!x)}{x}\,dx \;=\; 
\frac{\pi^2}{12}$\\

21. \; $\displaystyle\int_{\frac{1}{2}}^1\frac{\ln(1\!-\!x)}{x^2}\,dx \;=\; -2\ln{2}$\\
%%%%%
%%%%%
\end{document}
