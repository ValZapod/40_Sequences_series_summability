\documentclass[12pt]{article}
\usepackage{pmmeta}
\pmcanonicalname{RelationshipAmongDifferentKindsOfCompactness}
\pmcreated{2013-03-22 18:00:57}
\pmmodified{2013-03-22 18:00:57}
\pmowner{rm50}{10146}
\pmmodifier{rm50}{10146}
\pmtitle{relationship among different kinds of compactness}
\pmrecord{6}{40531}
\pmprivacy{1}
\pmauthor{rm50}{10146}
\pmtype{Theorem}
\pmcomment{trigger rebuild}
\pmclassification{msc}{40A05}
\pmclassification{msc}{54D30}

% this is the default PlanetMath preamble.  as your knowledge
% of TeX increases, you will probably want to edit this, but
% it should be fine as is for beginners.

% almost certainly you want these
\usepackage{amssymb}
\usepackage{amsmath}
\usepackage{amsfonts}

% used for TeXing text within eps files
%\usepackage{psfrag}
% need this for including graphics (\includegraphics)
%\usepackage{graphicx}
% for neatly defining theorems and propositions
\usepackage{amsthm}
% making logically defined graphics
%%\usepackage{xypic}

% there are many more packages, add them here as you need them

% define commands here
\newtheorem{thm}{Theorem}
\newtheorem{prop}{Proposition}

\begin{document}
The goal of this article is to prove
\begin{thm} If $X$ is second countable and $T_1$, or if $X$ is a metric space, then the following are equivalent:
\begin{enumerate}
\item $X$ is compact;
\item $X$ is limit point compact;
\item $X$ is sequentially compact.
\end{enumerate}
\end{thm}
We prove this using several subsidiary theorems, which prove the various implications in stronger settings.

\begin{thm} A compact topological space $T$ is limit point compact. (Here we make no assumptions about the topology on $T$).
\end{thm}
\textbf{Proof. } 
Choose an subset $A\subset T$, and suppose $A$ has no limit points. Then $A$ contains its (vacuous set of) limit points and is therefore closed. But closed subsets of compact spaces are compact, so $A$ is compact. Since $A$ has no limit points, we may choose a neighborhood $U_a$ of each $a\in A$ such that $U_a$ intersects $A$ only in $a$. But this cover clearly has a finite subcover only if $A$ is finite. So any set without limit points is finite, and thus any infinite set has a limit point. This concludes the proof.


\begin{thm} If $T$ is first countable, $T_1$, and limit point compact, then $T$ is sequentially compact.
\end{thm}

\textbf{Proof. }
Let $x_i$ be any sequence of points in $T$, and assume that $x_i$ takes infinitely many values (otherwise it obviously has a convergent subsequence). Choose a limit point $x$ for the sequence; we may assume wlog that $x_i$ is equal to $x$ for only finitely many $i$ (otherwise again the result holds trivially). So by ignoring a finite number of leading terms of the sequence, we may assume that $x_i\neq x$ for every $i$. Since $T$ is first countable, choose a countable basis $B_i$ at $x$; by replacing $B_n$ with $B_1\cap\ldots\cap B_n$, we may assume that $B_{i+1}\subset B_i$ for all $i$.

Now, choose $n_1$ such that $p_{n_1}\in B_1$. Inductively, assume we have chosen $n_1,\ldots,n_k$ with $p_{n_k}\in B_k$. Since $T$ is $T_1$, we may choose a neighborhood $U$ of $q$ that is disjoint from $p_{n_1},\ldots,p_{n_k}$; choose $p_{n_{k+1}}$ to be any point in $U\cap B_{k+1}$. Then inductively the $p_{n_i}$ form a subsequence with $p_{n_i}\in B_i$, and clearly the $p_{n_i}$ converge to $q$. This concludes the proof.

Note that every metric space and every second countable $T_1$ space is also first countable and $T_1$.

\begin{prop} Any sequentially compact metric space $M$ is second countable.
\end{prop}
\textbf{Proof. }
It clearly suffices to show that $M$ has a countable dense subset.

Claim first that for $\epsilon>0$, the set of $\epsilon$-balls in $M$ has a finite subcover. Suppose this is false for some particular $\epsilon$. Let $p_1\in M$ be any point, and construct inductively points $p_k$ with $p_k\notin B_{\epsilon}(p_1)\cup\ldots B_{\epsilon}(p_{k-1})$. Since $M$ is sequentially compact, we may replace the $p_i$ by a convergent subsequence, which we also call $p_i$, with $p_i\to p\in M$. But convergent sequences are Cauchy, so for $n$ large enough, we have $d(p_n,p_m)<\epsilon$, which contradicts the construction of the $p_i$. This proves the claim.

Then for each positive integer $n$, let $p_{n,n_1},\ldots,{p_{n,n_k}}\in M$ be a finite set of points such that the $\frac{1}{n}$-balls around those points cover $M$. This set of points is countable, and is obviously dense in $M$. This concludes the proof.

\begin{thm} If $T$ is second countable or is a metric space, and sequentially compact, then $T$ is compact.
\end{thm}

\textbf{Proof. } Assume first that $T$ is second countable. Choose any open cover of $T$; it has a countable subcover $U_i$. We use an argument very similar to that used in the above proposition. Suppose no finite subset of the $U_i$ covers $T$, and choose $p_k\in T\backslash(U_1\cup\ldots\cup U_k)$. Since $T$ is sequentially compact, the $p_k$ have a convergent subsequence $p_{n_k}$ converging to $p\in T$. But $p\in U_n$ for some $n$; since the $p_{n_k}$ converge to $p$, all $p_{n_k}\in U_n$ for $k$ large enough. But this is a contradiction to the construction of the $p_k$, so that a finite subset of the $U_i$ cover $T$ and $T$ is compact.

Since any sequentially compact metric space is second countable by the above proposition, we are done.


The main theorem follows trivially from the above. Note that we have in fact proven the following set of implications:
\begin{itemize}
\item Compact $\Rightarrow$ limit point compact for general topological spaces;
\item Limit point compact $\Rightarrow$ sequentially compact for first countable $T_1$ spaces;
\item Sequentially compact $\Rightarrow$ compact for second countable or metrizable spaces.
\end{itemize}

%%%%%
%%%%%
\end{document}
