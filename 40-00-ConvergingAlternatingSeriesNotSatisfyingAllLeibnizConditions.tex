\documentclass[12pt]{article}
\usepackage{pmmeta}
\pmcanonicalname{ConvergingAlternatingSeriesNotSatisfyingAllLeibnizConditions}
\pmcreated{2013-03-22 19:00:45}
\pmmodified{2013-03-22 19:00:45}
\pmowner{pahio}{2872}
\pmmodifier{pahio}{2872}
\pmtitle{converging alternating series not satisfying all Leibniz' conditions}
\pmrecord{7}{41882}
\pmprivacy{1}
\pmauthor{pahio}{2872}
\pmtype{Example}
\pmcomment{trigger rebuild}
\pmclassification{msc}{40-00}
\pmclassification{msc}{40A05}
\pmrelated{SumOfSeriesDependsOnOrder}
\pmrelated{LeibnizEstimateForAlternatingSeries}

\endmetadata

% this is the default PlanetMath preamble.  as your knowledge
% of TeX increases, you will probably want to edit this, but
% it should be fine as is for beginners.

% almost certainly you want these
\usepackage{amssymb}
\usepackage{amsmath}
\usepackage{amsfonts}

% used for TeXing text within eps files
%\usepackage{psfrag}
% need this for including graphics (\includegraphics)
%\usepackage{graphicx}
% for neatly defining theorems and propositions
 \usepackage{amsthm}
% making logically defined graphics
%%%\usepackage{xypic}

% there are many more packages, add them here as you need them

% define commands here

\theoremstyle{definition}
\newtheorem*{thmplain}{Theorem}

\begin{document}
The alternating series
\begin{align}
\sum_{n=1}^\infty\frac{(-1)^{n-1}}{n\!+\!(-1)^{n-1}} \;=\;
\frac{1}{2}-\frac{1}{1}+\frac{1}{4}-\frac{1}{3}+\frac{1}{6}-\frac{1}{5}+-\ldots
\end{align}
satisfies the other requirements of Leibniz test except the monotonicity of the absolute values of the terms.\, The convergence may however be shown by manipulating the terms as follows.

We first multiply the numerator and the denominator of the general term by the difference $n\!-\!(-1)^{n-1}$, getting from (1)
\begin{align}
\sum_{n=1}^\infty\frac{(-1)^{n-1}}{n\!+\!(-1)^{n-1}} 
\;=\; \frac{1}{2}+\sum_{n=2}^\infty\frac{n\!-\!(-1)^{n-1}}{n^2\!-\!1}(-1)^{n-1} 
\;=\; \frac{1}{2}+\sum_{n=2}^\infty\left(\frac{(-1)^{n-1}n}{n^2\!-\!1}-\frac{1}{n^2\!-\!1}\right).
\end{align}
One can \PMlinkescapetext{state} that the series
\begin{align}
\sum_{n=2}^\infty\frac{(-1)^{n-1}n}{n^2\!-\!1}
\end{align}
satisfies all requirements of Leibniz test and thus is convergent.\, Since 
$$
0 \;<\; \frac{1}{n^2\!-\!1} \;<\; \frac{1}{n^2\!-\!\frac{1}{2}n^2} \;=\; 2\cdot\frac{1}{n^2}\quad\mbox{for}\quad n \geqq 2,
$$
and the over-harmonic series $\sum_{n=2}^\infty\frac{1}{n^2}$ converges, the comparison test guarantees the convergence of the series
\begin{align}
\sum_{n=2}^\infty\frac{1}{n^2\!-\!1}.
\end{align}
Therefore the difference series of (3) and (4) and consequently, by (2), the given series (1) is convergent.


%%%%%
%%%%%
\end{document}
