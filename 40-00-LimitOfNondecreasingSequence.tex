\documentclass[12pt]{article}
\usepackage{pmmeta}
\pmcanonicalname{LimitOfNondecreasingSequence}
\pmcreated{2013-03-22 17:40:31}
\pmmodified{2013-03-22 17:40:31}
\pmowner{pahio}{2872}
\pmmodifier{pahio}{2872}
\pmtitle{limit of nondecreasing sequence}
\pmrecord{13}{40114}
\pmprivacy{1}
\pmauthor{pahio}{2872}
\pmtype{Theorem}
\pmcomment{trigger rebuild}
\pmclassification{msc}{40-00}
\pmsynonym{nondecreasing sequence with upper bound}{LimitOfNondecreasingSequence}
\pmsynonym{limit of increasing sequence}{LimitOfNondecreasingSequence}
%\pmkeywords{nondecreasing}
%\pmkeywords{bounded from above}
%\pmkeywords{nonincreasing}
%\pmkeywords{bounded from below}
\pmrelated{MonotonicallyIncreasing}
\pmrelated{MonotoneIncreasing}
\pmrelated{Supremum}
\pmrelated{Infimum}
\pmrelated{ConvergenceOfTheSequence11nn}

\endmetadata

% this is the default PlanetMath preamble.  as your knowledge
% of TeX increases, you will probably want to edit this, but
% it should be fine as is for beginners.

% almost certainly you want these
\usepackage{amssymb}
\usepackage{amsmath}
\usepackage{amsfonts}

% used for TeXing text within eps files
%\usepackage{psfrag}
% need this for including graphics (\includegraphics)
%\usepackage{graphicx}
% for neatly defining theorems and propositions
 \usepackage{amsthm}
% making logically defined graphics
%%%\usepackage{xypic}

% there are many more packages, add them here as you need them

% define commands here

\theoremstyle{definition}
\newtheorem*{thmplain}{Theorem}

\begin{document}
\PMlinkescapeword{nonincreasing}

\textbf{Theorem.}\, A monotonically nondecreasing sequence of real numbers with upper bound a number $M$ converges to a limit which does not exceed $M$.

{\em Proof.}\, Let\, $a_1 \leqq a_2 \leqq \ldots \leqq a_n \leqq \ldots \leqq M$.\, Therefore the set \,$\{a_1,\,a_2,\,\ldots\}$ has a finite supremum $s \leqq M$.\, We show that
\begin{align}                       
\lim_{n\to\infty}a_n \;=\; s.
\end{align}
Let $\varepsilon$ an arbitrary positive number.\, According to the definition of supremum we have\, $a_n \leqq s$\, for all $n$ and on the other hand, there exists a member $a_{n(\varepsilon)}$ of the sequence that is $> s-\varepsilon$.\, Then we have\, $s-\varepsilon < a_{n(\varepsilon)} \leqq s$,\, and since the sequence is nondecreasing,
      $$0 \;\leqq\; s-a_n \;\leqq\; s\!-\!a_{n(\varepsilon)} \;<\; \varepsilon \quad \mbox{for all}\;\, n \geqq n(\varepsilon).$$
Thus the equation (1) and the whole theorem has been proven.\\


For the nonincreasing sequences there is the corresponding

\textbf{Theorem.}\, A monotonically nonincreasing sequence of real numbers with lower bound a number $L$ converges to a limit which is not less than $L$.\\

\textbf{Note.}\, A good application of the latter theorem is in the proof that Euler's constant exists.

%%%%%
%%%%%
\end{document}
