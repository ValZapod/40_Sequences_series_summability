\documentclass[12pt]{article}
\usepackage{pmmeta}
\pmcanonicalname{PlanetMathAdministrationMainDocument}
\pmcreated{2013-03-11 19:32:23}
\pmmodified{2013-03-11 19:32:23}
\pmowner{jac}{4316}
\pmmodifier{}{0}
\pmtitle{PlanetMath Administration (Main Document)}
\pmrecord{1}{50109}
\pmprivacy{1}
\pmauthor{jac}{0}
\pmtype{Definition}

%none for now
\begin{document}
% -*- mode: LaTeX; -*-
\documentclass{article}

\usepackage{hyperref}
\usepackage{ulem}
\setcounter{tocdepth}{4}

\newcommand{\dlink}[2]{\texttt{\htmladdnormallink{#1}{#2}}}
\newcommand{\ulink}[1]{\texttt{\htmladdnormallink{#1}{#1}}}

\begin{document}

\title{Administration of PlanetMath}

\date{\today}

\maketitle

\tableofcontents

\section{About this document (jac)}

I wrote this document to address what I think is an overwhelming need
for improved communication among PlanetMath participants about
activities, goals, and resources.  It is early in its life cycle;
however, I hope it can serve a useful organizational role as it
develops.  Over the course of the next couple of months, I intend to
flesh it out with many additional details on goings-on around
PlanetMath, and supplement it with links to other documents of
administrative interest.  At the same time, I have made the document
world-editable, so that other PlanetMath users can add details or if
necessary make corrections directly.

\subsection{Formatting Convention (Wkbj79, jac)}

In order to make it clear who said what, the following convention is
used: In any section or subsection, the author(s) of that section will
put their username in parentheses.  Unsigned subsections of a signed
section should be assumed to have been created by the main section's
main author.  Contributing authors should feel free to modify existing
sections, although major thematic changes should certainly be signed
so that everyone knows roughly who said what.

\subsection{References to Other Administrative Documents (jac)}

This document will be a high-level map, and other documents
will go into detail on specific areas.  In general references
to these documents will appear in the text, with URLs in footnotes.
However, here is a short list of several documents that are
either especially important or that just haven't been integrated
into the text here in the manner I described.

\begin{enumerate}
\item
PM Community Guidelines \url{http://planetmath.org/?op=getobj&from=collab&id=112}
\item
PM Content Committee \url{http://planetmath.org/?op=getobj&from=collab&id=113}
\item
PM Member Rights \url{http://planetmath.org/?op=getobj&from=collab&id=116}
\item 
PM Point System \url{http://planetmath.org/?op=getobj&from=collab&id=114}
\item
PM Task List \url{http://planetmath.org/?op=getobj&from=collab&id=118}
\end{enumerate}

\section{Administration (jac)}

We should try to understand the terms better.  When I say
``administration'', really you could just think of a live updating map
of what is going on in the project, but it should be both as detailed
and as foldable as any individual wants, and capable of being viewed
from on at least as many different sides as there are viewers.

In other words, this is what I have been talking about with the
scholium system or ``Arxana'' all along\footnote{URL: 
\url{http://planetx.cc.vt.edu/AsteroidMeta/A\_scholium-based\_document\_model}}
-- except that it is not just some abstract construction -- the
administrative map should of course be filled with actual details such
as who is doing what and who is responsible for what, so that if
anyone has a question about the project, either they will find an
answer on the map, or they will find a good place to stick their
question to get it answered.

Now despite my optimism that some version of Arxana will be available
and consumable by the end of the summer, people have administered
projects for millennia without it, and I think it would be very apt of
us to look around for some older technology which we can use to build
our administrative map.

I would like to suggest a simple latex file saved as a PlanetMath
collaboration.  Sections can be cut and added, content can be moved
around or changed, questions can be inserted.  Additional
collaboration documents can supplement this one.

Thus, over the last few paragraphs I have done more than just make an
abstract suggestion -- I have kicked off an actual solution.  Let me
continue that briefly before I get on to my second question.

\subsection{An Administrative Matrix (jac)}

One of the key pieces of data about an organization is its {\sc
  tasklist}.  The tasklist is populated with tasks, which can be
defined broadly or finely, and which can be finite or ongoing.
Furthermore, each task can have subtasks, and this progession can
continue, down through the minut\ae\ of meticulous Fordism, or up into
the \ae therial realms of an organizational Mission.

But, we need not, at any given point in time, be so detailed.
However, the tasklist itself should be complemented by some basic
information about the tasks involved.  For example, a task might be
annotated with a description of its {\sc participants}, its {\sc
  consumers} (or ``{\sc stakeholders}''), its {\sc schedule}, and its
{\sc budget} (or ``{\sc requirements}'').  Other suchlike fields can
be added as interest and expediency indicate.

Having described what I mean by an \emph{administrative matrix}, I
will now construct the beginnings of such a matrix for PlanetMath, as
an example (see Table \ref{matrix}).  Since I do not know enough about
all the tasks or all the details of all the tasks going on at
PlanetMath, this will not be particularly detailed.  I will focus on
tasks that I know something about, i.e., things I have been working on
or following.

Please feel free to add or update with details on the tasks you have
been working on, if you like.  Mainly I am posting this for
discussion: however, I would like to see a more industrial-strength
version of this sort of thing put into play on PlanetMath soon.

\begin{table}
\begin{center}
\begin{tabular}{|l |p{2in}|p{2in} |p{2in}|}
\hline
{\sc \huge label} & {\sc \huge description} & {\sc \huge schedule} & {\sc \huge requirements}
\\
\hline
Arxana
& 
Make browser and interactive store, retrieve, display, edit features.
Try to get clusions and occlusions working.
&
Maybe add a few more tiny things now, but
it seems like a higher priority to get elephant
and noosphere stuff worked out first.
&
Internally, this should follow our Elephant storage
and query design; externally, it should follow the
noosphere (re)design.
\\
\hline
%%%%%%%%%%%%%%%%%%%%%%%%%%%%%%%%%%%%%%%%%%%%%%%%%%%%%%%%%%%%%%%%%%%%%%%%%%%%%%%%
SLIME 
&
Can SLIME connect to a MMTN server?
Can we once again use SLIME to connect to a regular server?
&
Low priority to connect to MMTN.
Get regular remote connection back up soon.
&
Long-term goal may depend on getting MMTN set up properly.
Short-term may depend on help from Nick Thomas?
\\
\hline
%%%%%%%%%%%%%%%%%%%%%%%%%%%%%%%%%%%%%%%%%%%%%%%%%%%%%%%%%%%%%%%%%%%%%%%%%%%%%%%%
MMTN
&
Implement proper daemon system.
Desocket for local connections (?).
Write a Lisp client. 
&
Cf. chat client ``erc'' soon to see how it connects to Emacs.
& 
Understand the current system.
Come up with trial application.
\\
\hline
%%%%%%%%%%%%%%%%%%%%%%%%%%%%%%%%%%%%%%%%%%%%%%%%%%%%%%%%%%%%%%%%%%%%%%%%%%%%%%%%
Elephant
&
Chalk out ontology and seach interface well-suited to our apps.
Locking and permissions.
&
Start right away listing desiderata.
&
Match our desiderata to the elephant docs.
Help from Ian Eslick and pals as needed.
\\
%%%%%%%%%%%%%%%%%%%%%%%%%%%%%%%%%%%%%%%%%%%%%%%%%%%%%%%%%%%%%%%%%%%%%%%%%%%%%%%%
aserve
&
Start with interactive web interface to elephant.
&
Perhaps get a REPL soon, as a demo?
&
Some time with the docs if we want to know what all this
thing can do.
\\
\hline
%%%%%%%%%%%%%%%%%%%%%%%%%%%%%%%%%%%%%%%%%%%%%%%%%%%%%%%%%%%%%%%%%%%%%%%%%%%%%%%%
noosphere rewrite
&
Survey to find featureset and reusable modules.
&
Our rewrite must be done by August 20 for GSoC review.
&
Would be good to talk more with APK as this develops.
Would be good to have a central place to do the
design and fill in pieces of code as they appear.
\\
%%%%%%%%%%%%%%%%%%%%%%%%%%%%%%%%%%%%%%%%%%%%%%%%%%%%%%%%%%%%%%%%%%%%%%%%%%%%%%%%
sys. admin.
&
Persist configs and work files.
Make laptops usably useful.
&
ASAP.
&
Internet connection.  Superuser passwords.  Physical
access to the computers in question.
\\
\hline
%%%%%%%%%%%%%%%%%%%%%%%%%%%%%%%%%%%%%%%%%%%%%%%%%%%%%%%%%%%%%%%%%%%%%%%%%%%%%%%%
math/tech talk
&
Prepare (and give) 15 minute talk on PM and HDM for Mathfest.
Use a holistic perspective and try to convey mathematical
connections/relevance.
&
Must be finished by the beginning of August.  May give
a variant talk around August 8?
&
Need a place to develop.  Feedback from Ray, Aaron, others.
Perhaps some background reading/references.
\\
\hline
%%%%%%%%%%%%%%%%%%%%%%%%%%%%%%%%%%%%%%%%%%%%%%%%%%%%%%%%%%%%%%%%%%%%%%%%%%%%%%%%
``main'' for PM
&
Keep maintaining this document.
Survey various tasks that other PM people are
working on, and provide annotated index here.
&
So far this is a ``whenever I want to do it'' 
thing.  But interestingly, or ironically, this
sort of matrix/map (especially in an improved format)
would be the sort of thing that would be good to
update everyday.
&
For this to work \emph{extremely} well,
it should be networked into people's methods for
participation, and have many participants.
\\
\hline
%%%%%%%%%%%%%%%%%%%%%%%%%%%%%%%%%%%%%%%%%%%%%%%%%%%%%%%%%%%%%%%%%%%%%%%%%%%%%%%%
content committee
&
This is Chi, Alvaro, Ray, et al's thing.
I'll want to watch and record what they
are doing, or involve them directly in a
suitable administrative approach to this
effort.
&
Would be nice to be updated of changes/progess,
and act on them to have them reflected right
away.
&
Some degree of cooperation from other the
content committee's participants will be necessary.
\\
\hline
%%%%%%%%%%%%%%%%%%%%%%%%%%%%%%%%%%%%%%%%%%%%%%%%%%%%%%%%%%%%%%%%%%%%%%%%%%%%%%%%
networking
&
Figure out possible places to get
support for our efforts.
&
Want to have a number of good options
lined up by the end of the summer.
Should build a more detailed schedule/plan
of action.
&
Phone (check); travel budget for West Coast trip;
some sort of sanction from PM to act as its 
representative.
\\
\hline
%%%%%%%%%%%%%%%%%%%%%%%%%%%%%%%%%%%%%%%%%%%%%%%%%%%%%%%%%%%%%%%%%%%%%%%%%%%%%%%%
fundraising
&
Figure out and share value propositions with possible
funders. (This is in some sense a projection of ``networking''.
Note that by building PM's budget, our efforts may be better
supported at PM; but this is not the only approach worth pursuing.)
&
Want to have a number of good options
lined up by the end of the summer.  (Hopefully
enough of a sample to see if our approaches are
going to pan out.)
&
Any individual working on this will need a PM sanction.
Other than that, collectively we will want a coordinated plan.
\\
\hline
%%%%%%%%%%%%%%%%%%%%%%%%%%%%%%%%%%%%%%%%%%%%%%%%%%%%%%%%%%%%%%%%%%%%%%%%%%%%%%%%
FEM
&
Make a print version of the PM encyclopedia.
Ideally it would be rebuilt frequently.
&
This project has been stalled for a long time:
it would be nice to revive it this summer, but
so far I don't know of a specific plan or timeline.
First goal would be to make one
&
A major requirement that we noticed, but were not
able to institute, was a system for unified notation
among articles included in the print collection.
There are various other technical steps that we would
want to make this work well.
\\
%%%%%%%%%%%%%%%%%%%%%%%%%%%%%%%%%%%%%%%%%%%%%%%%%%%%%%%%%%%%%%%%%%%%%%%%%%%%%%%%
\hline
\end{tabular}
\end{center}
\caption{An example of an administrative matrix \label{matrix}}
\end{table}

Note that the format used here is pretty far from ideal, certainly
from the point of data entry (since it would be nice to just be able
to zoom in on the tasks and fields that one is particularly interested
in updating), and also, at least in some senses, from the point of
view of display (since it would be nice to be able to fold certain
rows and columns out of the way).

One of the makes this effort somewhat difficult is that different
labels (e.g. ``noosphere'') may mean different things to different
people, so there is, at least at first, some degree of collision
between points of view on what a given task is about.  I think the
ambiguity can be resolved with enough creative relabeling.  

Another more complicated way to resolve this sort of conflict would be
to give each person their own matrix where they would be free to use
whatever labels they like.  But it would take a lot of work to
reconcile this approach with the properties of ``being an
administration'' as I have defined them above.


\section{Roles of PlanetMath participants (jac)}

A quick mental check indicates that we have at least these broad
categories of participants:

\begin{itemize}
\item President
\item Board Members
\item Site ``Admins''
\item Content Contributors
\item Interns
\item Consultants
\item Bounty Hunters
\end{itemize}

In the subsections that follow, I will discuss some of the roles taken
on by people in these categories -- I do not assert that this
discussion is in any way ``complete'' -- rather I'll just say, we can
come back and add details as appropriate, at any time.

\subsection{Role of the President}
\begin{itemize}
\item Almost every web-related issue gets routed through Aaron.

\item Almost every financial issue gets routed through Aaron.

\item Most practical ``beaurocratic'' decisions are made by Aaron.

\item Many key ``social networking'' issues are routed only through Aaron.
\end{itemize}

\subsection{Role of the Board of Directors}
\begin{itemize}
\item Less frequently, decisions are made by members of the board.

\item Given the fact that the board does not have public meetings very
  often, the ability of general members of the community to oversee or
  engage the board members vanishes to zero.

\item See the section below on Governance for more details.
\end{itemize}

\subsection{Role of Site Administrators}

\begin{itemize}
\item There are some arcane and non-obvious rules and responsibilities
  that devolve onto persons with so-called ``admin'' status on the
  website: I do not know the structure of this continent, so I will
  not say any more about it; it may be important for issues like those
  that Chi has been talking about, may not be so important for the
  actual business of ``running the non-profit'', but who knows?
\end{itemize}

\subsection{Generic Roles of Community Members}

\begin{itemize}
\item Obviously the most frequent role engaged in by community members
  is developing mathematical content for the site.  Currently most of
  this content is in the Ecyclopedia, but discussion of math topics
  in the forums also plays a significant role.

\item Media-wise, most ``important'' (at least, in the sense of being
  at all ``persistent'' and ``public'') conversations among the powers
  that be get mapped through the cumbersome forum software deployed on
  PlanetMath's home page.

\item Other than that, there are various unofficial and more (or less,
  in the case of the email list) transitory media

\item Various informal decisions and actions are taken by the
  quote-unquote ``powers that be'' around PlanetMath, which include
  myself and too many engaged PM contributors to list here -- but note
  that these powers are almost universally ``unofficial'' powers (all
  binding decisions must go through Aaron or, possibly, the board).
\end{itemize}

\subsection{Community Initiatives}

\begin{itemize}
\item content committee (otherwise known as ``community guidelines''; I don't
  know what the latest perspective is, but see footnote for reference to
  one writeup\footnote{URL: 
\url{http://planetx.cc.vt.edu/AsteroidMeta/Community\_Guidelines}})
\item maintaining this administrative document
\item fundraising and other outreach efforts
\item coordinating the PlanetMath membership situation (cf.  section on 
Governance below) \item ...
\end{itemize}

\subsection{Proposals for change (jac)}

\begin{itemize}

\item Well, I think that Aaron's current set of responsibilities made
  handleable by more people.  Not because he has in any way ``done a
  bad job'' -- far from it -- but because the rest of us have done
  essentially nothing, due to the existence of an administrative
  bottleneck that he just happens to mostly fill.

\item I would like to see somewhere an \emph{official} record of the
  things people are working on -- including me.  This could eventually
  lead into either ``job titles'' or ``volunteer roles'' (such as they
  have at Wikipedia), which I think would do quite a lot to give
  PlanetMath more than one ``official'' public face.

\item Extending this (and this is something we have talked a lot about
  but I think left sitting for a long time), I would like to see a
  clear description of ``members rights'' -- hopefully made with at
  least some serious participation by members.

\item I think that many of the problems around PlanetMath -- probably
  most of the serious ones in my view -- are properly administrative
  in nature (according to my definition above: we lack a good map!).
  But I think that this has repercussions for ``morale'' (among other
  things).  Participants who do not have any institutional authority
  are not likely to do much in the way of outreach, for example -- or
  to be able to do much.
\end{itemize}

\subsection{Comments on Administration (Wkbj79)}

Please add additional subsections if you wish to make further comments
on what has been said in or about the section on Administration.

With regards to the first point in the second bulleted list above.  It seems unfair that Aaron has had most of the burden placed on him.

As for jac's ``proposals for change'', why not add such an official
record to this document?  [Done or at least begun: see ``Community
  Initiatives''. --jac] On a related note, information about the soon to be formed content committee is available in other collaborations and documents which are similar to this one.

\section{Business (jac)}

\subsection{Basic Data on PlanetMath as a Business}

\begin{itemize}
\item The nature of the institution.  We describe PlanetMath as being
  ``a non-profit'', ``a commons-based peer production system'', and ``a
  community''.  As the name indicates, ``math'' is the main focus.
  (Although if we follow Marnita's advice, we may build several
  spin-off, partner, or child institutions.)  At present, there is no
  permanent paid staff retained by the non-profit.

\item Legal status.  The organization is (not without some technical
  snafus, at least potential ones) a tax-exempt charity \emph{based in
    the United States of America}; and the content and code is made
  available \emph{worldwide} under some ``free'' license or other.

\item Revenue streams.  Subject to limitations placed on charity
  organizations, PlanetMath can receive donations, have financial
  sponsors, get federal and other grants, and sell a few things (but
  not too many).

\item Work product.  The content of the PM site (encyclopedia, forms)
  and the code behind or adjacent to it (Noosphere,
  NNexus\footnote{URL:
    \url{http://planetx.cc.vt.edu/AsteroidMeta/NNexus}}, Arxana,
  etc.).
\end{itemize}

\subsection{Human Resources in a Mostly-Volunteer Organization}

(And, in particular, how to have them without upsetting any apple
carts.)

Experience indicates that paying someone for a service is not always a
good idea.  A volunteer might find the idea of being payed to work
distasteful: ``I'm doing this because I want to, man, not for
\emph{money}!''  

The person speaking could be an ideologically motivated hippie or an
ideologically motivated millionaire, but in either case, offering to
pay them may seem like an attempt to rob them of their chance to give.

At the same time, many people do like -- and most need -- to be paid
for some form of work.  Different payment schemes can be more or less
motivating for different people, in different contexts.  I've known a
number of people who have gotten paid a flat hourly rate and who
happened to learn that they could get away with very little or no work
on the job.  Certainly even for wage-laborers, money is not the sole
motivator.  There is also something to be said for such things as
pride in one's work, a sense of growth and achievement!

Which further highlights the fact that a ``payment scheme'' is not
going to answer all questions of personal motivation.

Speaking for myself, there are a number of different things I like to
do -- but I might not be as motivated if I was compelled for some
reason to focus exclusively on one of these things.  (At the same
time, complete dissipation wouldn't work so well either.)

\begin{itemize}
\item \emph{Payments for administrative work.}  Since I might be get
  into a payed arrangement with PlanetMath that includes
  administrative work, I want to think about how to do this ``right''
  -- for my own personal, subjective, reasons.  However, other people
  may get into similar relationships, so, it is also good to think
  about suitably general ways to make payment for services work in
  this community.  (Cf. the section on Work Plan, below.)
\item \emph{Payments for coding work.} In a setting in which most
  participants are not programmers, payments for coding may be the
  only way to get a sufficient flow of code into the organization.  At
  the same time, I think the software projects associated with
  PlanetMath should be very carefully administered, especially so that
  knowledge transfer can make it possible for more participants to
  become volunteer programmers.
\end{itemize}

\subsection{How to employ people in a non-profit organization (alozano)}

This section describes an ideal policy for hiring.

\begin{enumerate}
\item First, an active board should be elected, that closely oversees the 
operations of the organization (the current board at PlanetMath is sadly 
inactive in the community, except Aaron). 
\item Second, the board needs to explain (or somebody needs to explain to the 
board) the position to be created, the duties, responsibilities, etc attached 
to the position. It needs to be stated clearly how and why such position needs 
to be created, and determine what is the salary (if a salary will be awarded).

\item If a salary is to be awarded, the board (or whoever proposes the creation of a position) needs to guarantee that the salary will be sustainable within the current (and foreseeable) budget and finances of the organization.  
Most importantly, since the organization is built in volunteer and 
collaborative efforts, it needs to be stated and explained the need of paying 
for such position and why the duties cannot be fulfilled by the volunteer body. 
The board also needs to determine the minimum necessary qualifications for an 
applicant to be eligible for such position. 

\item The position needs to be approved by the board.

\item Once and if the position is approved, an application process shall occur. 
Since the company is a non-profit organization, the company needs to find the 
person most suited to fulfill that role. As in the case of ``code bounties'' 
the position should be made public, so everyone is informed about the job 
opportunity. It is of the outmost importance that the organization hires the 
person most qualified for the job. 
\item The position shall be advertised in and by all reasonable channels. In 
particular, the board should advertise the position in channels where the most 
desirable applicants may hear about the position. 
\item A reasonable period should be given for the position to be reasonably 
advertised. Once the period elapses, the board will review the applications and 
choose the best candidate available to fulfill the position. \end{enumerate}

\section{Economics (jac)}

Personally, I am hopeful, that in a time when ``Y-Combinator'' is
becoming a household word (see Newsweek Magazine, May 21 2007), there
\emph{is} money out there to support even some of our most outrageous
schemes.

But in this case, I think good things do not come to those who wait --
they come to those who ask.  And to ask a good question, you need a
good formulation of the question, hopefully together with an
illustrative example, model, or sketch.

\subsection{Free/Libre/Open \{Software, Content\} Business Models}

In case you are getting the name of this section is a bit awkward, I
agree!  I think people in the world of Free/Libre/Open Source Software
and content (acronym: FLOSS) have not found very clear ways to talk
about the economics side of what they do.

At the same time, there is at least one ``economicsy'' term that is in
use to talk about the way people \emph{produce} this stuff.  The term
is ``Commons-Based Peer Production'' (acronym: CBPP).  The connection
to economics is the suggestion that, for participants, their voluntary
contribution to the commons is a rational behavior.

Whatever else the analysis behind this term may get into, I don't
think it has much to say about the ways in which commons-based systems
relate economically to other systems.

I would like to simplify matters a great deal and point out that
frequently the things that these systems are generating have
\emph{public good-like} aspects from the point of view of agents
outside of the system.

This is important, since it says something about how money might be
expected to change hands to support the projects.  Specifically,
public goods tend to be paid for, and regulated, by governmental
bodies.

It would be great to get into a further analysis, but for now I am
just going to list a few words of wisdom on the topic of FLOSS
business.  

\begin{itemize}
\item \emph{``You can't sell something you're giving away for free.''}

\item \emph{``The best way to make money related to free content or
  software is to sell services.''}

\item \emph{``There are no restrictions on selling copies of free
  content or software.''} -- this is relevant to, e.g., the FEM, if we
  ever get that together.  (However, there may in fact be some
  restrictions coming from our legal status as a charity
  organization.)
\end{itemize}

\subsection{Semantics of Cash Flow}

(This might also be called the Cash Flow of Semantics.  I haven't
gotten all the details worked out yet.)

These ``brainstorms'' are a tentative sketch, based on my thinking
about Arxana.  For some further thoughts along these lines, see my
essay on political applications of the scholium system\footnote{URL: 
\url{http://planetx.cc.vt.edu/AsteroidMeta/HomePage/social\%2c\_political\%2c\_
and\_economic\_scholium\_systems}}.

\begin{itemize}
\item Every connection, every link, is an arbitrage opportunity. The
  moment I realized that, I was pretty excited -- since a lot of what
  I think about is links.  This is why Google Answers worked (for a
  while) -- but since they stopped working, we have to be careful.
  Not every arbitrage opportunity is ``viable'' and one can fail to
  make money for myriad reasons, whereas one must wait until the check
  clears to actually make any money (and, OK, even then, there are
  expenses to worry about!).

\item By the time a link has been made, you might say that the
  arbitrage opportunity is already gone -- but despite that fact, a
  new link may open up a whole ``plane'' of new arbitrage
  opportunities in the form of potential links, i.e. annotations,
  running perpendicular to it.
\end{itemize}

\subsection{Ways to Make Money}

\begin{itemize}
\item One idea is to get people to pay us to customize Noosphere; but
  an easy complaint is that we have not been able to pay to customize
  Noosphere in the directions we want. The synthesis is to seek
  synergistic relationships -- customizing Noosphere in directions we
  want, and having someone else pay us to do that!
\item There are a number of similar ideas related to ``other'' or
  ``next-generation'' software systems we have been discussing
  (Arxana,
  CodeMarket\footnote{URL: 
\url{http://planetx.cc.vt.edu/AsteroidMeta/Code_Market}}).
  Given that a number of our proposals fall outside of PlanetMath
  proper, I think that we would do well to put together an
  ``\"uber-administration'' that covers all of the things that we
  might want to share work on.
\end{itemize}

For any project or scheme for making money, I think it would be
worthwhile for us to come up with what's called a
``value-proposition'' (or, in some cases, an atlas of such
propositions).

\section{Governance (jac)}

\subsection{Bylaws (jac)}

The organization was incorporated with bylaws that
help it meet the legal requirements of incorporation\footnote{URL: \url{http://planetx.cc.vt.edu/AsteroidMeta/PMByLaws}}.
These bylaws were revised\footnote{URL: \url{http://planetx.cc.vt.edu/AsteroidMeta/PlanetMath\_ByLaws\_2.0}}.

(I'm not completely sure whether the second draft of the bylaws was
ever really ``put into effect''.)

\subsection{Board of Directors (jac)}

The role of the board of directors is described in the bylaws
mentioned above.  However, it is not completely clear that the board
has been following the protocol described in these bylaws.
This suggests that a review and possible a reconstitution of
the board and of the protocol is in order.

\subsection{Member Rights (Wkbj79)}

Members, especially new members, should be aware of what their rights are.  One aspect that instantly comes to mind is freedom of mathematical expression.  After all, PlanetMath is supposed to be ``Math for the people, by the people.''  A good example of freedom of mathematical expression is the Smarandache entries.  A poll was taken on PlanetMath regarding such entries, and although most active members here have doubts about the mathematical significance of the Smarandache entries, the decision to keep them here was the right one.  If this were a right
that were clearly expressed, then such a poll would not have had to be taken.

\section{Processing Results of Words on Fire Consultation (jac)}

The new info we got from Words on Fire\footnote{URL: 
\url{http://web.mac.com/marnita/iWeb/PMPlan/Tools.html}} will certainly need to 
be processed somewhat; and indeed, following through on this is certainly
an administrative function.  At the same time, some of this content
may be worked into the other sections of the current document.

\subsection{Mission (quoted form WOF)}

PlanetMath.Org's mission is to build and sustain an engaged
collaborative math community that is free and open for all individuals
with a serious interest in mathematics.

This is the jumping off point for your mission.  Before you can fully
articulate what you do, you have to agree on this basic element of the
organization.  Everything you do flows from and to your mission.

\subsection{Work Plan}

Some version of this is in the WOF documents, and Aaron is currently
putting together a working version this work plan.

\section{Technology (jac)}

\subsection{Current Technology}

\subsubsection{Noosphere}

PlanetMath runs Noosphere\footnote{URL:
  \url{http://aux.planetmath.org/noosphere/}}.  Understanding what
this means is in some sense an administrative problem -- and may also
require some data archeology.

\subsubsection{LaTeX}

Noosphere is integrated with \LaTeX.

\subsubsection{latex2html}

This is the current rendering engine for Noosphere.

\subsubsection{Wiki}

A goodly amount of project-coordinating information is available on
the AsteroidMeta wiki\footnote{URL: 
\url{http://planetx.cc.vt.edu/AsteroidMeta/HomePage}}, although
the wiki is not in very good shape and will hopefully be replaced soon.
(An effort to do this is underway thanks to effort thanks to the
Ghestalt team.)

\subsection{Current Technological Efforts}

\subsubsection{Coding}

Right now the primary coding efforts are related to work being done on
Google Summer of Code (see the wiki
page\footnote{URL: \url{http://planetx.cc.vt.edu/AsteroidMeta/2007\_PM\_Summer\
_of\_Code\_Coordination}}
for basic coordination information, and the project proposals
submitted by our interns for further details\footnote{URL: 
\url{http://groups.google.com/group/planetmath/browse_thread/thread/532e776555d
8884d?hl=en}}).

\begin{itemize}
\item NNexus is being extended
\item MUSN\footnote{URL: \url{http://planetx.cc.vt.edu/AsteroidMeta/MUSN}}
is being developed
\item a ratings/reputation system is being developed
\end{itemize}

It would be nice if more communication was taking place between
people working on code and other members of the community.

\subsubsection{System Administration}

As mentioned above, this is currently routed almost completely through
Aaron.

\subsection{Ideas for Technological Changes}

There are many ideas for features around, including
a recent compilation in another collaboration document
on PlanetMath\footnote{URL: 
\url{http://planetmath.org/?op=getobj\&from=collab\&id=82}}
and the huge ongoing feature request list from the wiki\footnote{URL:
\url{http://planetx.cc.vt.edu/AsteroidMeta/Feature\_Requests}}.

For the current document I am going to mention some of the ideas about
technology that I think are most related to ``administration'' in the
sense of this document.

\subsubsection{Issues}

This is a short sequence of points that develop a basic theme.

\begin{itemize}
\item \emph{Keeping track of things.}  On all of the different
  task-areas mentioned here, it would be great to know who is doing
  what.  Similarly with mathematics items in the encyclopedia -- which
  items have been modified recently?  Linked to recently?  Viewed
  recently?  What, in general, is going on?  -- The question seems too
  general, but in an administrative setting, we should be try to be
  most aware of the things people are most concerned about.  What are
  the biggest annoyances?  What are the best hopes we have for doing
  anything about them?  And so forth and so on.
\item \emph{The multidimensionality of information landscapes.}  We
  might be interested in integrating data from some field of
  information, and we could put this data into some sort of aggregator
  (see below).  But then, when we have a collection of aggregators,
  we'll want to see which ones people are using.  And if Alice
  realizes no one is addressing issues of concern to her, then she'll
  have to think about which factors are relevant here.  Why is does
  she care about something other people aren't interested in?  Or
  maybe other people are interested, and she is not using the right
  set of aggregators.  Or maybe the aggregators are not pointed at the
  right dataflow.  In either event, she will have to debug her own
  thought processes to find the answers -- which will require other
  introspective data aggregators, and probably some field research.
  As the cycle continues, there's more and more information to manage!
\item \emph{Network effects.} Is any given task ever completed
  perfectly?  Or will there always be problems that come up?  I
  suppose it depends on how your point of view; I suppose my point of
  view is that these actually aren't very good questions.  In the
  physical theories, systems flow towards a lower energy state.  They
  do not get to their destinations instantaneously, nor do they travel
  directly to the lowest energy state.  To call this either
  ``perfect'' or ``problem-ridden'' seems to stretch the language
  pretty thin.  In human affairs, the things we call ``problems'' are
  often opportunities for greater efficiency that we notice.  A quite
  general \emph{challenge} lies in the fact that many interpersonal or
  technological arrangements could be made more efficient in many
  different ways.  Any given event or class of events might have
  numerous criticisms associated with it -- indicating improvements
  that can be made along many different dimensions.  The first step in
  facing this challenge is to find a way to get these critical remarks
  ``networked in'', so that they can be processed further and not
  forgotten or neglected.
\end{itemize}

\subsubsection{Proposals}

\begin{itemize}
\item The basic interface is might be some collection of data
  ``aggregators''.  Just the other day I was looking at the PlanetMath
  homepage, and thinking about how it would be nice to have it work
  more customizable (like Mac desktops or Google homepages... or
  GNU/Linux systems!), allowing users to select or even create new
  widget-like tools to inspect the data flows they are most interested
  in.
\end{itemize}

\end{document}

%%%%%
\end{document}
