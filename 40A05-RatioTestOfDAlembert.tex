\documentclass[12pt]{article}
\usepackage{pmmeta}
\pmcanonicalname{RatioTestOfDAlembert}
\pmcreated{2013-03-22 19:12:28}
\pmmodified{2013-03-22 19:12:28}
\pmowner{pahio}{2872}
\pmmodifier{pahio}{2872}
\pmtitle{ratio test of d'Alembert}
\pmrecord{9}{42123}
\pmprivacy{1}
\pmauthor{pahio}{2872}
\pmtype{Theorem}
\pmcomment{trigger rebuild}
\pmclassification{msc}{40A05}
\pmrelated{FiniteChangesInConvergentSeries}

\endmetadata

% this is the default PlanetMath preamble.  as your knowledge
% of TeX increases, you will probably want to edit this, but
% it should be fine as is for beginners.

% almost certainly you want these
\usepackage{amssymb}
\usepackage{amsmath}
\usepackage{amsfonts}
\usepackage[T2A]{fontenc}
\usepackage[russian, english]{babel}


% used for TeXing text within eps files
%\usepackage{psfrag}
% need this for including graphics (\includegraphics)
%\usepackage{graphicx}
% for neatly defining theorems and propositions
%\usepackage{amsthm}
% making logically defined graphics
%%%\usepackage{xypic}

% there are many more packages, add them here as you need them

% define commands here
\newcommand{\sijoitus}[2]%
{\operatornamewithlimits{\Big/}_{\!\!\!#1}^{\,#2}}
\begin{document}
A lighter version of the ratio test is the

\textbf{Ratio test of d'Alembert.}\, Let\, $a_1\!+\!a_2\!+\ldots$\, be a series with positive terms.

$1^\circ$.\, If there exists a number $q$ such that\, $0 < q < 1$\, and
\begin{align}
\frac{a_{n+1}}{a_n} \;\le\; q \quad \mbox{for all}\;\; n \ge n_0,
\end{align}
then the series converges.

$2^\circ$.\, If there exists a number $n_0$ such that
\begin{align}
\frac{a_{n+1}}{a_n} \;\ge\; 1 \quad \mbox{for all}\;\; n \ge n_0,
\end{align}
then the series diverges.\\


\emph{Proof.}\, $1^\circ$. By the condition (1), we have\, $a_{n+1} \le a_nq$;\, thus we get the estimations
$$a_{n_0+1} \;\le\; a_{n_0}q,$$
$$a_{n_0+2} \;\le\; a_{n_0+1}q \;\le\; a_{n_0}q^2,$$
$$\cdots \quad \cdots \quad \cdots$$
$$a_{n_0+p} \;\le\; a_{n_0+p-1}q \;\le\; \ldots \;\le\; a_{n_0}q^p,$$
$$\cdots \quad \cdots \quad \cdots$$
Because\, $a_{n_0}q+a_{n_0}q^2+\ldots+a_{n_0}q^p+\ldots$\, is a convergent geometric series, those inequalities and the comparison test imply that the series
$$a_{n_0+1}\!+\!a_{n_0+2}\!+\ldots+\!a_{n_0+p}\!+\ldots$$
and as well the whole series\, $a_1\!+\!a_2\!+\ldots$\, is convergent.

$2^\circ$.\, The condition (2) yields
$$a_{n_0+1} \;\ge\; a_{n_0}, \quad a_{n_0+2} \;\ge\; a_{n_0+1} \;\ge\; a_{n_0}, \quad \ldots$$
and since\, $a_{n_0}$ is positive, the limit of $a_n$ as $n$ tends to infinity cannot be 0.\, Hence the given series does not fulfil the necessary condition of convergence.\\

\textbf{Example.}\, If the variable $x$ in the power series
$$\sum_{n=0}^\infty n!x^n$$
is distinct from zero, we have
$$\frac{|(n\!+\!1)!x^{n+1}|}{|n!x^n|} \;=\; (n\!+\!1)|x| \;\ge\; 1 \quad \mbox{for all}\;\; n \ge n_0.$$
Then the series does not \PMlinkname{converge absolutely}{AbsoluteConvergence}.\, The known theorem of Abel says that the series diverges for all\, 
$x \neq 0$.\, It means that the radius of convergence is 0.



\begin{thebibliography}{7}
\bibitem{LDK} \CYRL. \CYRD. \CYRK\cyru\cyrd\cyrr\cyrya\cyrv\cyrc\cyre\cyrv: 
{\em \CYRM\cyra\cyrt\cyre\cyrm\cyra\cyrt\cyri\cyrch\cyre\cyrs\cyrk\cyri\cyrishrt \,\cyra\cyrn\cyra\cyrl\cyri\cyrz. I \cyrt\cyro\cyrm}. \,\CYRI\cyrz\cyrd\cyra\cyrt\cyre\cyrl\cyrsftsn\cyrs\cyrt\cyrv\cyro \,
``\CYRV\cyrery\cyrs\cyrsh\cyra\cyrya \,\cyrsh\cyrk\cyro\cyrl\cyra''. \CYRM\cyro\cyrs\cyrk\cyrv\cyra \,(1970).
\end{thebibliography}








%%%%%
%%%%%
\end{document}
