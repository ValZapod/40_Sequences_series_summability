\documentclass[12pt]{article}
\usepackage{pmmeta}
\pmcanonicalname{LimitOfSequenceAsSumOfSeries}
\pmcreated{2013-03-22 17:28:21}
\pmmodified{2013-03-22 17:28:21}
\pmowner{pahio}{2872}
\pmmodifier{pahio}{2872}
\pmtitle{limit of sequence as sum of series}
\pmrecord{4}{39858}
\pmprivacy{1}
\pmauthor{pahio}{2872}
\pmtype{Theorem}
\pmcomment{trigger rebuild}
\pmclassification{msc}{40-00}
\pmrelated{SumOfSeries}

% this is the default PlanetMath preamble.  as your knowledge
% of TeX increases, you will probably want to edit this, but
% it should be fine as is for beginners.

% almost certainly you want these
\usepackage{amssymb}
\usepackage{amsmath}
\usepackage{amsfonts}

% used for TeXing text within eps files
%\usepackage{psfrag}
% need this for including graphics (\includegraphics)
%\usepackage{graphicx}
% for neatly defining theorems and propositions
 \usepackage{amsthm}
% making logically defined graphics
%%%\usepackage{xypic}

% there are many more packages, add them here as you need them

% define commands here

\theoremstyle{definition}
\newtheorem*{thmplain}{Theorem}

\begin{document}
If $U$ is the limit of a sequence
        $$u_1,\,u_2,\,u_3,\,\ldots$$
of real or complex numbers, then $U$ can be expressed as the series sum
     $$U = u_1+\sum_{i=1}^\infty(u_{i+1}-u_i).$$

{\em Proof.}  Let\, $\displaystyle s_n := u_1+\sum_{i=1}^{n-1}(u_{i+1}-u_i)$.\, We see that
$$s_n = u_1+\sum_{i=1}^{n-1}u_{i+1}-\sum_{i=1}^{n-1}u_i = u_1+\sum_{j=2}^nu_j-\sum_{i=1}^{n-1}u_i = u_n$$
for all\, $n = 1,\,2,\,3,\,\ldots$\, Thus
$$u_1+\sum_{i=1}^\infty(u_{i+1}-u_i) = \lim_{n\to\infty}s_n = \lim_{n\to\infty}u_n = U,$$
Q.E.D.
%%%%%
%%%%%
\end{document}
