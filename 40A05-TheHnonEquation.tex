\documentclass[12pt]{article}
\usepackage{pmmeta}
\pmcanonicalname{TheHnonEquation}
\pmcreated{2013-03-11 19:28:19}
\pmmodified{2013-03-11 19:28:19}
\pmowner{linor}{11198}
\pmmodifier{}{0}
\pmtitle{the Hénon equation}
\pmrecord{1}{50086}
\pmprivacy{1}
\pmauthor{linor}{0}
\pmtype{Definition}

%none for now
\begin{document}
\documentclass[12pt,leqno]{article}
\usepackage{amssymb}
\newcommand{\be}{\begin{equation}}
\newcommand{\ee}{\end{equation}}
\newcommand{\dk}{d\sigma_{\xi}}
\newcommand{\dx}{d\sigma_{x}}
\newcommand{\nd}{\frac{ \partial}{ \partial n}}
\newcommand{\ndk}{\disfrac{\textstyle \partial}{\textstyle \partial n_{ \xi}}}
\newcommand{\ndx}{\disfrac{\textstyle \partial}{\textstyle \partial n_{ x}}}
\newcommand{\ik}{\int_{ \Gamma}}
\newcommand{\ts}{\textstyle}


\begin{document}
The Henon equation and its associated energy functional can be written as
%\be  %\label{HN}
$$
\; \left\{\begin{array}{lll} 
-\Delta u=|x|^r|u|^{q-1}u,&\! x\in\Omega\\
u=0,&\!x\in\partial\Omega,
\end{array}\right.\;
J(u)=\int_{\Omega}\Big[\frac{1}{2}|\nabla u|^2-\frac{1}{q+1}|x|^r|u|^{q+1}\Big]dx
%\ee
$$
where $u\in H_0^1(\Omega)$, $\Omega\subset {\mathbb R}^N (N\ge 2)$ 
is an open bounded domain, $r\ge 0$ and $1<q\le \frac{N+2}{N-2}$. 

\end{document}
%%%%%
\end{document}
