\documentclass[12pt]{article}
\usepackage{pmmeta}
\pmcanonicalname{InfiniteDimensionalHamiltonianSystem1}
\pmcreated{2013-03-11 19:28:50}
\pmmodified{2013-03-11 19:28:50}
\pmowner{linor}{11198}
\pmmodifier{}{0}
\pmtitle{infinite dimensional hamiltonian system}
\pmrecord{1}{50091}
\pmprivacy{1}
\pmauthor{linor}{0}
\pmtype{Definition}

%none for now
\begin{document}
\documentclass[12pt,leqno]{article}
\usepackage{amssymb}
\usepackage{color}

\newcommand{\be}{\begin{equation}}
\newcommand{\ee}{\end{equation}}
\newcommand{\dk}{d\sigma_{\xi}}
\newcommand{\dx}{d\sigma_{x}}
\newcommand{\nd}{\frac{ \partial}{ \partial n}}
\newcommand{\ndk}{\disfrac{\textstyle \partial}{\textstyle \partial n_{ \xi}}}
\newcommand{\ndx}{\disfrac{\textstyle \partial}{\textstyle \partial n_{ x}}}
\newcommand{\ik}{\int_{ \Gamma}}
\newcommand{\ts}{\textstyle}


\begin{document}

An infinite dimensional Hamiltonian system takes the form
    $$ 
       (IHS) \;\; \left \{ \begin{array}{rll}
               \partial_t u -\Delta_x u &= H_v(t,x,u,v) & \\
               -\partial_t v -\Delta_x v &= H_u(t,x,u,v)  & 
\forall (t,x) \in \mathbb{R} \times \Omega
         \end{array}  \right.
   $$
where $\Omega\subset {\mathbb R}^N$ is a smoothly bounded domain,
$H \in \mathcal{C}^1 (\mathbb{R}\times \overline{\Omega} \times \mathbb{R}^2, \mathbb{R})$. 


\end{document}
%%%%%
\end{document}
