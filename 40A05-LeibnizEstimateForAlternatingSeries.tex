\documentclass[12pt]{article}
\usepackage{pmmeta}
\pmcanonicalname{LeibnizEstimateForAlternatingSeries}
\pmcreated{2014-07-22 15:34:38}
\pmmodified{2014-07-22 15:34:38}
\pmowner{pahio}{2872}
\pmmodifier{pahio}{2872}
\pmtitle{Leibniz' estimate for alternating series}
\pmrecord{35}{36523}
\pmprivacy{1}
\pmauthor{pahio}{2872}
\pmtype{Theorem}
\pmcomment{trigger rebuild}
\pmclassification{msc}{40A05}
\pmclassification{msc}{40-00}
\pmsynonym{Leibniz' estimate for remainder term}{LeibnizEstimateForAlternatingSeries}
\pmrelated{EIsIrrational2}
\pmrelated{ConvergingAlternatingSeriesNotSatisfyingAllLeibnizConditions}

% this is the default PlanetMath preamble.  as your knowledge
% of TeX increases, you will probably want to edit this, but
% it should be fine as is for beginners.

% almost certainly you want these
\usepackage{amssymb}
\usepackage{amsmath}
\usepackage{amsfonts}

% used for TeXing text within eps files
%\usepackage{psfrag}
% need this for including graphics (\includegraphics)
\usepackage{graphicx}
% for neatly defining theorems and propositions
 \usepackage{amsthm}
% making logically defined graphics
%%%\usepackage{xypic}

% there are many more packages, add them here as you need them

% define commands here
\theoremstyle{definition}
\newtheorem*{thmplain}{Theorem}

\providecommand{\abs}[1]{\lvert#1\rvert}
\begin{document}
\textbf{Theorem (Leibniz 1682).}
 \, If \,$p_1 > p_2 > p_3 > \cdots$\, and\, $\displaystyle\lim_{m\to\infty}p_m = 0$,\, then the alternating series 
\begin{align}
p_1-p_2+p_3-p_4+-\ldots
\end{align}
converges.\, Its remainder term has the same \PMlinkname{sign}{SignumFunction} as the first omitted \PMlinkescapetext{term} $\pm p_{m+1}$ and the absolute value less than $p_{m+1}$.\\

{\em Proof.}\, The convergence of (1) is proved 
\PMlinkname{here}{ProofOfAlternatingSeriesTest}.\, Now denote the sum of the series by $S$ and the partial sums of it by $S_1,\,S_2,\,S_3,\,\ldots$.\, Suppose that (1) is truncated after a negative \PMlinkescapetext{term} $-p_{2n}$.\, Then the remainder term
$$R_{2n} \;=\; S\!-\!S_{2n}$$
may be written in the form
$$R_{2n} \;=\; (p_{2n+1}-p_{2n+2})+(p_{2n+3}-p_{2n+4})+\ldots$$
or
$$R_{2n} \;=\; p_{2n+1}-(p_{2n+2}-p_{2n+3})-(p_{2n+4}-p_{2n+5})-\ldots$$
The former shows that $R_{2n}$ is positive as the first omitted \PMlinkescapetext{term} $p_{2n+1}$ and the latter that\, $|R_{2n}| < p_{2n+1}$.\, Similarly one can see the assertions true when the series (1) is truncated after a positive \PMlinkescapetext{term} $p_{2n-1}$.\\

\bigskip

\emph{A pictorial proof.}

\begin{center}
\includegraphics{leibniz-sequence.eps}
\end{center}

As seen in this diagram, whenever\, $m' > m$,\, we have\, 
$\abs{S_{m'}\!-\!S_m} \leq p_{m+1} \to 0$.\, Thus the partial sums form a Cauchy sequence, and hence converge.\, The limit lies in the \PMlinkescapetext{centre} of the spiral, strictly in \PMlinkescapetext{between} $S_m$ and $S_{m+1}$ for any $m$.\, So the remainder after the $m$th \PMlinkescapetext{term} must have the same direction as\, $\pm p_{m+1} = S_{m+1}\!-\!S_m$\, and lesser magnitude.\\

\textbf{Example 1.}\, The alternating series
$$\frac{1}{\sqrt{2}-1}-\frac{1}{\sqrt{2}+1}
+\frac{1}{\sqrt{3}-1}-\frac{1}{\sqrt{3}+1}
+\frac{1}{\sqrt{4}-1}-\frac{1}{\sqrt{4}+1}+-\ldots$$
does not fulfil the requirements of the theorem and is divergent.\\

\textbf{Example 2.}\, The alternating series
$$\frac{1}{\ln{2}}-\frac{1}{\ln{3}}+\frac{1}{\ln{4}}-\frac{1}{\ln{5}}+-\ldots$$
satisfies all conditions of the theorem and is convergent.
%%%%%
%%%%%
\end{document}
