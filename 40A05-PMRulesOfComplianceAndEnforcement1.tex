\documentclass[12pt]{article}
\usepackage{pmmeta}
\pmcanonicalname{PMRulesOfComplianceAndEnforcement1}
\pmcreated{2013-03-11 19:52:41}
\pmmodified{2013-03-11 19:52:41}
\pmowner{CWoo}{3771}
\pmmodifier{}{0}
\pmtitle{PM Rules of Compliance and Enforcement}
\pmrecord{1}{50156}
\pmprivacy{1}
\pmauthor{CWoo}{0}
\pmtype{Definition}

%none for now
\begin{document}
\documentclass{article}
\usepackage{hyperref}
\usepackage{ulem}
\begin{document}

\title{PlanetMath Content - Rules of Compliance and Enforcement - Draft Version}

\date{\today}

\maketitle

\tableofcontents

In this document, PM means PlanetMath, and PMCC (or CC for short), means the PlanetMath Content Committee.  Since the document deals specifically with PM Content, issues related to PM Forums are not covered here.

This document refers only to those objects that are under the supervision of the CC, which include: encyclopedia entries, papers, books, expositions and requests (on the Requests List).

\section{Offenses}

Offenses are usually brought to the attention of the CC via one of the following two channels: direct observation by CC, or complaint email from a PM user.\\

Offenses are classified by their severity:
\begin{enumerate}
\item A \emph{minor offense} is any one of the items below
\begin{enumerate}
\item failure to comply with a request by CC within the given timeframe, which includes 
\begin{enumerate}
\item request to revise/reclassify objects
\item request for a type 1 object deletion
\item request to halt productions of new objects
\item request to halt updates of existing objects
\item request to halt adoption of orphaned entries
\end{enumerate}

\item any actions performed by the CC on behalf of the offending user, including
\begin{enumerate}
\item object reclassification
\item type 1 object deletion
\item object confiscation
\end{enumerate}

\item three valid complaint notices (such as inappropriate encyclopedia entry, etc...) received from PM users within 60 days.  In order for a complaint notice to be valid, the offensive object has to be validated by CC, and then communicated to the owner of the entry. 
\end{enumerate}

\item A \emph{moderate offense} is any one of the items below
\begin{enumerate}
\item three minor offenses by a user within 4 months
\item type 2 object deletion
\end{enumerate}

\item A \emph{major offense} is any one of the items below
\begin{enumerate}
\item two moderate offenses within one year
\item type 3 object deletion
\end{enumerate}
\end{enumerate}

\subsection{A Note on Object Deletions}

There are three types of administrative object deletions, ordered in increasing severity:
\begin{enumerate}
\item type 1: an object deleted by CC due to the expiration of an object-deletion request made by CC to the owner of the offending object, for reason due to lack of clarity, lack of mathematical content, invalid request, etc...
\item type 2: an object deleted by CC due to the expiration of an object-deletion request made by CC to the owner of the offending object, for reason due to commercial spam, plagiarism, etc..
\item type 3: an object automatically (without notice to owner of the object) deleted by CC due to use of offensive language, personal threats, etc...
\end{enumerate}

\section{Counting Offenses}

\begin{enumerate}
\item Offenses are partially cumulative.  Each offense has an issue date and an expiration date.  For a given offense, the time between the issue date and the expiration date is the life span of the offense:
\begin{enumerate}
\item a minor offense has a 6-month life span
\item a moderate offense has a 18-month life span
\item a major offense has 36-month life span
\end{enumerate}
After the expiration date of an offense, the offense is erased.
\item Offenses are convertible: 
\begin{enumerate}
\item when three minor offenses are accumulated in a 4-month period, a moderate offense is generated with an issue date the issue date of the latest minor offense.  In the meantime, the three minor offenses are erased.
\item when two moderate offenses are accumulated in a 12-month period, a major offense is generated with an issue date the issue date of the latest moderate offense.  In the meantime, the two moderate offenses are erased.
\end{enumerate}
\item At most one offense per incident may be considered.  An incident is a sequence of events that are related by a single CC request.  For example, CC requests that a user deletes one of his entries, he fails to comply, and CC deletes the entry for him.  The whole sequence of events is considered one incident.  Although both the failure to comply with a request by CC (see 1.3.(a) above) and the subsequent deletion of the entry by CC (see 1.3.(b) above) are qualified minor offenses, only one minor offense may be considered.
\end{enumerate}

\section{Consequences}

Every offense is associated with a sequence of penalizing consequences.  Below are descriptions of these penalties and other administrative actions.

\subsection{Penalties}

To each severity type of offense, there corresponds a choice of a variety of penalties of like magnitude for the offending PM user:
\begin{enumerate}
\item minor offense: either a warning issued by CC, or up to 1000 points deduction
\item moderate offense: either a short term (30-day) suspension, or between 1000 to 5000 points deduction
\item major offense: either an indefinite suspension, or between 20000 to 50000 points deduction
\end{enumerate}
Note: the number of points being deducted as a result of a penalty imposed should be at most the number of points owned by the offending user.\\

During an offense conversion, the offending user will be penalized for the offense being converted to, instead of the offense being converted from.  For example, when the offending user receives the second moderate offense within a year, the two moderate offenses get converted into a major offense.  As a result, the user receives penalties associated with the major offense, and not the second moderate offense that has now just been erased.\\

\subsection{Other Actions}
In addition to receiving a penalty, there is usually an action (called a consequence) taken associated with the offending object (or objects).  These consequences include:
\begin{enumerate}
\item administrative reclassification of an object
\item administrative confiscation of an object
\item administrative deletion of an object
\end{enumerate}

\subsection{Reversibility of Consequences}
Certain penalties and administrative actions are reversible.  For example, points deducted by CC, or deleted by CC can be reinstated.  However, certain penalties are not.  User account suspension, for example, is an example of irreversible penalties.  CC has the discretion to reverse reversible actions.  Nevertheless, CC should try to apply reversible penalties when possible, and reserve irreversible penalties for extreme cases.  Also, offense conversion does not necessarily mean erasure of corresponding prior penalties, even if the penalties are reversible.  Again, CC has the complete discretion over this matter.

\section{CC Responsibilities}
If CC determines an offense has been committed by a PM user, it has the discretion to impose penalties on the offending user, and other actions on the offending objects.  The following guideline shows how this is done:

\begin{enumerate}
\item First, CC will weigh on the severity of the offense, and decide whether it is worth pursuing the matter further with the offending user.
\item If CC decides to continue with the disciplinary action, an email notification must then be sent to the offending user (from PMAdministration), spelling out the offense the user has committed.
\item If CC decides to impose penalties and other actions, the specific penalties and actions must also be clearly spelled in the same email notification.  These actions must be carried out in a timely manner.
\item All offenses and subsequent penalties and/or other actions must be documented in the Table of Disciplinary Actions.
\item If a penalty involves the suspension of a user account, CC must obtain approval from the PM membership enforcement committee first.
\item If an administrative action involves a type 1 entry deletion, the content of the offending entry must be saved (using get metadata) for documentation.
\end{enumerate}


\section*{Revisions}

\begin{enumerate}
\item Split from the ``original'' draft of the PM Community Guideline (11-11-2008)  --[[CWoo]]
\item Major revision (2-4-2009)  --[[CWoo]]
\item Added CC responsibilities Section (2-5-2009) --[[CWoo]]
\item Minor revision, changed ``entries'' by ``objects'' when appropriate (16-2-2009) --[[asteroid]]
\end{enumerate}

\end{document}
%%%%%
\end{document}
