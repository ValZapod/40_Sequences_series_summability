\documentclass[12pt]{article}
\usepackage{pmmeta}
\pmcanonicalname{ACATEGORYTHEORYANDHIGHERDIMENSIONALALGEBRAAPPROACHTOCOMPLEXSYSTEMSBIOLOGYMETASYSTEMSANDONTOLOGICALTHEORYOFLEVELS}
\pmcreated{2013-03-11 19:54:15}
\pmmodified{2013-03-11 19:54:15}
\pmowner{bci1}{20947}
\pmmodifier{}{0}
\pmtitle{A CATEGORY THEORY AND HIGHER DIMENSIONAL ALGEBRA APPROACH TO COMPLEX  SYSTEMS BIOLOGY, META-SYSTEMS AND  ONTOLOGICAL THEORY OF LEVELS:}
\pmrecord{1}{50170}
\pmprivacy{1}
\pmauthor{bci1}{0}
\pmtype{Definition}

%none for now
\begin{document}
% uiucs13New38rd.tex 08/10/10

\documentclass[11pt]{article}

\usepackage{amssymb}
\usepackage{amsmath}
\usepackage{amsfonts}
\usepackage{html}

\usepackage{amsmath, amssymb, amsfonts, amsthm, amscd, latexsym,enumerate}

%\usepackage{graphicx}

\usepackage{amsthm}

\pagestyle{empty}

\usepackage{xypic}

\DeclareMathOperator{\Tr}{Tr} % needed as example

% BibTeX and Xy-pic logo
\newcommand*{\BibTeX}{{\rm B\kern-.05em{\sc i\kern-.025em b}\kern-.08em
    T\kern-.1667em\lower.7ex\hbox{E}\kern-.125emX}}

\begin{htmlonly}
\renewcommand*{\BibTeX}{Bib\TeX}
\renewcommand*{\Xy}{Xy}
\end{htmlonly}


\usepackage{fancyhdr}
\usepackage{graphicx}

%\usepackage{xypic}

\usepackage[mathscr]{eucal}


%the next gives two direction arrows at the top of a 2 x 2 matrix

\newcommand{\directs}[2]{\def\objectstyle{\scriptstyle}  \objectmargin={0pt}
\xy
(0,4)*+{}="a",(0,-2)*+{\rule{0em}{1.5ex}#2}="b",(7,4)*+{\;#1}="c"
\ar@{->} "a";"b" \ar @{->}"a";"c" \endxy }

\theoremstyle{plain}
\newtheorem{lemma}{Lemma}[section]
\newtheorem{proposition}{Proposition}[section]
\newtheorem{theorem}{Theorem}[section]
\newtheorem{corollary}{Corollary}[section]
\newtheorem{conjecture}{Conjecture}[section]

\theoremstyle{definition}
\newtheorem{definition}{Definition}[section]
\newtheorem{example}{Example}[section]
%\theoremstyle{remark}
\newtheorem{remark}{Remark}[section]
\newtheorem*{notation}{Notation}
\newtheorem*{claim}{Claim}


\theoremstyle{plain}
\renewcommand{\thefootnote}{\ensuremath{\fnsymbol{footnote}}}
\numberwithin{equation}{section}
\newcommand{\Ad}{{\rm Ad}}
\newcommand{\Aut}{{\rm Aut}}
\newcommand{\Cl}{{\rm Cl}}
\newcommand{\Co}{{\rm Co}}
\newcommand{\DES}{{\rm DES}}
\newcommand{\Diff}{{\rm Diff}}
\newcommand{\Dom}{{\rm Dom}}
\newcommand{\Hol}{{\rm Hol}}
\newcommand{\Mon}{{\rm Mon}}
\newcommand{\Hom}{{\rm Hom}}
\newcommand{\Ker}{{\rm Ker}}
\newcommand{\Ind}{{\rm Ind}}
\newcommand{\IM}{{\rm Im}}
\newcommand{\Is}{{\rm Is}}
\newcommand{\ID}{{\rm id}}
\newcommand{\GL}{{\rm GL}}
\newcommand{\Iso}{{\rm Iso}}
\newcommand{\Sem}{{\rm Sem}}
\newcommand{\St}{{\rm St}}
\newcommand{\Sym}{{\rm Sym}}
\newcommand{\SU}{{\rm SU}}
\newcommand{\Tor}{{\rm Tor}}
\newcommand{\U}{{\rm U}}

\newcommand{\A}{\mathcal A}
\newcommand{\D}{\mathcal D}
\newcommand{\E}{\mathcal E}
\newcommand{\F}{\mathcal F}
\newcommand{\G}{\mathcal G}
\newcommand{\R}{\mathcal R}
\newcommand{\cS}{\mathcal S}
\newcommand{\cU}{\mathcal U}
\newcommand{\W}{\mathcal W}

\newcommand{\Ce}{\mathsf{C}}
\newcommand{\Q}{\mathsf{Q}}
\newcommand{\grp}{\mathsf{G}}
\newcommand{\dgrp}{\mathsf{D}}

\newcommand{\bA}{\mathbb{A}}
\newcommand{\bB}{\mathbb{B}}
\newcommand{\bC}{\mathbb{C}}
\newcommand{\bD}{\mathbb{D}}
\newcommand{\bE}{\mathbb{E}}
\newcommand{\bF}{\mathbb{F}}
\newcommand{\bG}{\mathbb{G}}
\newcommand{\bK}{\mathbb{K}}
\newcommand{\bM}{\mathbb{M}}
\newcommand{\bN}{\mathbb{N}}
\newcommand{\bO}{\mathbb{O}}
\newcommand{\bP}{\mathbb{P}}
\newcommand{\bR}{\mathbb{R}}
\newcommand{\bV}{\mathbb{V}}
\newcommand{\bZ}{\mathbb{Z}}

\newcommand{\bfE}{\mathbf{E}}
\newcommand{\bfX}{\mathbf{X}}
\newcommand{\bfY}{\mathbf{Y}}
\newcommand{\bfZ}{\mathbf{Z}}

\renewcommand{\O}{\Omega}
\renewcommand{\o}{\omega}
\newcommand{\vp}{\varphi}
\newcommand{\vep}{\varepsilon}

\newcommand{\diag}{{\rm diag}}
\newcommand{\desp}{{\mathbb D^{\rm{es}}}}
\newcommand{\Geod}{{\rm Geod}}
\newcommand{\geod}{{\rm geod}}
\newcommand{\hgr}{{\mathbb H}}
\newcommand{\mgr}{{\mathbb M}}
\newcommand{\ob}{\operatorname{Ob}}
\newcommand{\obg}{{\rm Ob(\mathbb G)}}
\newcommand{\obgp}{{\rm Ob(\mathbb G')}}
\newcommand{\obh}{{\rm Ob(\mathbb H)}}
\newcommand{\Osmooth}{{\Omega^{\infty}(X,*)}}
\newcommand{\ghomotop}{{\rho_2^{\square}}}
\newcommand{\gcalp}{{\mathbb G(\mathcal P)}}

\newcommand{\rf}{{R_{\mathcal F}}}
\newcommand{\glob}{{\rm glob}}
\newcommand{\loc}{{\rm loc}}
\newcommand{\TOP}{{\rm TOP}}

\newcommand{\wti}{\widetilde}
\newcommand{\what}{\widehat}

\renewcommand{\a}{\alpha}
\newcommand{\be}{\beta}
\newcommand{\ga}{\gamma}
\newcommand{\Ga}{\Gamma}
\newcommand{\de}{\delta}
\newcommand{\del}{\partial}
\newcommand{\ka}{\kappa}
\newcommand{\si}{\sigma}
\newcommand{\ta}{\tau}


\newcommand{\lra}{{\longrightarrow}}
\newcommand{\ra}{{\rightarrow}}
\newcommand{\rat}{{\rightarrowtail}}
\newcommand{\oset}[1]{\overset {#1}{\ra}}
\newcommand{\osetl}[1]{\overset {#1}{\lra}}
\newcommand{\hr}{{\hookrightarrow}}


\newcommand{\hdgb}{\boldsymbol{\rho}^\square}
\newcommand{\hdg}{\rho^\square_2}

\newcommand{\med}{\medbreak}
\newcommand{\medn}{\medbreak \noindent}
\newcommand{\bign}{\bigbreak \noindent}

\renewcommand{\leq}{{\leqslant}}
\renewcommand{\geq}{{\geqslant}}

\def\red{\textcolor{red}}
\def\magenta{\textcolor{magenta}}
\def\blue{\textcolor{blue}}
\def\<{\langle}
\def\>{\rangle}
\hyphenation{Psychol-ogy Psych-ology}


\pagestyle{fancy} \headheight 35pt \lhead{Journal: \emph{Acta Universitatis Apulensis}, Alba Iulia: \emph{ Conf. Proceedings: Understanding
Intelligent and Complex Systems}, Barna Iantovics et al, eds.} \cfoot{\thepage}
\setcounter{page}{1}



\addtolength{\oddsidemargin}{-0.5cm}
\addtolength{\evensidemargin}{2cm} \setlength{\topmargin}{-1.5cm}
\setlength{\textheight}{19cm} \setlength{\textwidth}{14cm}

\begin{document}

\vspace*{10mm}

\begin{center}
\uppercase{\textbf{A CATEGORY THEORY AND HIGHER DIMENSIONAL ALGEBRA APPROACH TO COMPLEX  SYSTEMS BIOLOGY, META-SYSTEMS AND  ONTOLOGICAL THEORY OF LEVELS:
\emph{Emergence of Life, Society,\\ Human Consciousness and Artificial Intelligence}}}
\end{center}


\bigskip



\begin{center}
\textsc{I. C. Baianu, James  F. Glazebrook and Ronald Brown}
\end{center}

\bigskip

\textsc{Abstract.}

 An attempt is made from the viewpoint of the recent theory of ontological levels [2],[40],[137],[206]-[209]
to understand the origins and emergence of life, the dynamics of the evolution of organisms and species, the
ascent of man and the co-emergence, as well as co-evolution of human consciousness within organised societies. 
The new concepts developed  for understanding the emergence and evolution of life, as well as human consciousness,
are  in terms of globalisation of multiple, underlying processes into the meta-levels of their existence. 
Such concepts are also useful in computer aided ontology and computer science [1],[194],[197]. 
In this monograph we present a novel approach to  the problems raised by higher complexity in both nature and the human society, by
considering the highest and most complex levels of objective existence as ontological meta-levels, such as those present in the
creative human minds and civilised, modern societies. 
Thus, a collection of sets may be a \emph{class}, instead of a set [59],[176]-[177]; it
may also be called a `super-set', or a \emph{meta-set}; a `theorem'
about theorems is a \emph{meta-theorem}, and a `theory' about
theories is a \emph{`meta-theory'}. In the same sense that a
statement about propositions is a higher-level $\<proposition\>$
rather than a simple proposition, a global process of subprocesses
is a \emph{meta-process}, and the emergence of higher levels of
reality \emph{via} such meta-processes results in the objective
existence of \emph{ontological meta-levels}. It is also attempted here to classify more precisely the levels of reality and
species of organisms than it has been thus far reported. The
selected approach for our broad-- but in-depth-- study of the
fundamental, relational structures and functions present in living,
higher organisms and of the extremely complex processes and
meta-processes of the  human mind combines new concepts from three
recently developed, related mathematical fields: Algebraic Topology,
Category Theory (CT) and Higher Dimensional Algebra (HDA).  Several
important relational structures present in organisms and the human
mind are naturally represented in terms of universal CT concepts,
variable topology, non-Abelian categories and HDA-based notions.
Such relatively new concepts are defined  in the appropriate
sequence beginning with the concept of groupoid which is fundamental
to all algebraic topology studies [63], [69], and that also turns
out to be essential to numerous applications in mathematical biology
[11]-[23],[34],[74], including those of higher dimensional
groupoids in theoretical neuroscience [38],[69]-[70]. 

 An unifying theme of local-to-global approaches to organismal development,
biological evolution and human consciousness leads to novel patterns
of relations that emerge in super- and ultra-  complex systems in
terms of global compositions of local procedures [33],[39].  This
novel algebraic topology concept of \emph{combination of local
procedures} is suggested to be relevant to both ontogenetic
development and organismal evolution, beginning with the origin of
species of higher organisms. Fundamentally inter--related, higher
homotopy and holonomy groupoid concepts may provide a formal
framework for an improved understanding of evolutionary biology and
the origin of species on multiple levels--from molecular to species
and biosphere levels. All key concepts pertaining to this context
are here defined for a self-contained presentation, notwithstanding
the difficulties associated with understanding the essence of life,
the human mind, consciousness and its origins. One can define
pragmatically the human brain in terms of  its neurophysiological
functions, anatomical and microscopic structure, but one cannot as
readily observe and define the much more elusive human mind which
depends both upon a fully functional human brain and its training or
education by the human society. Human minds that do not but weakly
interact with those of any other member of society are partially
disfunctional, and this creates increasing problems with the society
integration of large groups of people that only interact weakly with
all the other members of society.  Obviously, it does take a fully
functional mind to observe and understand the human mind.  It is
then claimed that human consciousness is an \emph{unique} phenomenon
which should be regarded as a composition, or combination of
ultra-complex, global processes of subprocesses, at a
\emph{meta-level} not sub--summed by, but compatible with, human
brain dynamics  [11]--[23],[33]. Thus, a defining characteristic of
such conscious processes involves a \emph{combination of global
procedures} or meta-processes-- such as the parallel processing of
both image and sound sensations, perceptions and emotions, decision
making and learned reflexes, etc.-- that ultimately leads to the
ontological meta-level of the ultra-complex, human mind.  In this
monograph we shall not attempt to debate if other species are
capable of consciousness, or to what extent, but focus instead on
the ultra-complex problems raised by human consciousness and its
emergence. Current thinking [87], [91],[182],[186],[188], [190], [195]-[196],[203],[247] considers the 
actual emergence of human consciousness [83],[91],[186],[190],[261]
--and also its ontic category-- to be critically dependent upon the
existence of both a human society level of \emph{minimal} (tribal)
organization [91],[186],[190], and that of an extremely complex structural
--functional unit --the human brain with an \emph{asymmetric}
network topology and a dynamic network connectivity of very
high-order [187],[218], [262]. Then, an extension of the concept of coevolution
of human consciousness and society leads one to the concept of \emph{social
consciousness} [190]. One arrives also at the conclusion that the human
mind and consciousness are the result not only of the
\emph{co-evolution} of man and his society [91],[186],[190], but that they are, in fact, the result of
the original \emph{co-emergence} of the meta-level of a minimally-organized human society with that of several,
ultra-complex human brains. Unlike the myth of only one Adam and one
Eve being the required generator of human society, our co-emergence
concept leads necessarily to the requirement of several such
`primitive' human couples co-existing in order to generate both a
minimally organized society and several, minimally self-conscious,
interacting \emph{H. sapiens} minds that shaped the first Rosetta
groupoids of \emph{H. sapiens} into human tribes.  The human
`spirit' and society are, thus, \emph{completely inseparable}--just
like the very rare Siamese twins. Therefore, the appearance of human
consciousness is considered to be critically dependent upon the
societal co-evolution, the emergence of an elaborate
language-symbolic communication system, as well as  the existence of
`virtual', higher dimensional, non--commutative processes that
involve separate space and time perceptions in the human mind.  Two
fundamental, logic adjointness theorems are considered that provide
a logical basis for categorical representations of functional genome
and organismal networks in variable categories and extended toposes,
or topoi, `classified' (or encoded)  by multi-valued logic algebras;
their subtly nuanced connections to the variable topology and
multiple geometric structures of developing organisms are also
pointed out. Theories of the mind are thus considered in the context
of a novel ontological  theory of levels.  Our ultra-complexity
viewpoint throws new light on previous semantic models in cognitive
science and on the theory of levels formulated within the framework
of Categorical Ontology [40],[69]. Our novel approach to
meta-systems and levels using Category Theory and HDA mathematical
representations is also applicable--albeit in a modified form--to
supercomputers, complex quantum computers, man--made neural networks
and novel designs of advanced artificial intelligence (AI) systems
(AAIS).  Anticipatory systems and complex causality at the top
levels of reality are also discussed in the context of Complex
Systems Biology (CSB),  psychology, sociology and ecology. A
paradigm shift towards \emph{non-commutative}, or more generally,
non-Abelian theories of highly complex dynamics [33],[40],[69] is
suggested to unfold now in physics, mathematics, life and cognitive
sciences, thus leading to the realizations of higher dimensional
algebras in neurosciences and psychology, as well as in human
genomics, bioinformatics and interactomics. The presence of strange
attractors in modern society dynamics, and especially the emergence
of new meta-levels of still-higher complexity in modern society,
gives rise to very serious concerns for the future of mankind and
the continued persistence of a multi-stable Biosphere if such
ultra-complexity, meta-level issues continue to be ignored.

\bigskip

\textsc{Keywords:} \textit{{\it Categorical Ontology of
Super-Complex and Ultra-Complex System Dynamics,Higher Dimensional
Algebra of Networks,Theoretical Biology and Variable Groupoids,
Non-Abelian Quantum Algebraic Topology and Quantum Double Groupoids,
Higher Homotopy-General van Kampen theorems; autistic children,
advanced artificial intelligence and biomimetics}}


\medskip



2000 \textit{Mathematics Subject Classification}: 16B50, 68Q15.




\lhead{} \chead{ I. C. Baianu, James  F. Glazebrook and Ronald Brown: Category Theory
\&  Emergence of Life, Society, Human Consciousness \& AI } \rhead{}

$$
\textsc{1. Introduction }
$$


Ontology has acquired over time several meanings, and it has also
been approached in many different ways, but all of  these are
connected to the concepts of an \emph{`objective existence'} and
categories of items.  A related, important function of Ontology is
to \emph{classify and/or categorize} items and essential aspects of
reality [2],[206]-[210]. We shall employ therefore the adjective
\emph{``ontological"} with the meaning of pertaining to objective,
real existence in its essential aspects.  We shall also consider
here the noun \emph{existence} as a basic, or primary concept which
cannot be defined in either simpler or atomic terms, with the latter
in the sense of Wittgenstein. Furthermore, generating
\emph{meaningful classifications of items} that belong to the
objective reality is also a related,  major task of ontology.
Mathematicians specialised in Group Theory are also familiar with
the classification problem into  various types of the  mathematical
objects called groups. Computer scientists that carry out
ontological classifications, or study AI and Cognitive Science [201], are
also interested in the logical foundations of computer science [1],[194],[197],[201].

For us the most interesting question  by far is how human
consciousness and civilisation emerged subsequent only to the
emergence of \emph{H. sapiens}.  This may have arisen through the
development of speech-syntactic language and an appropriately
organized `primitive' society [91],[186] (perhaps initially made of
hominins/hominides). No doubt, the details of this highly complex,
emergence process have been the subject of intense controversies
over the last several centuries, and  many differing opinions, even
among these authors, and they will continue to elude us since  much
of the essential data must remains either scarce or unattainable. It
is however known that the use of cooked food, and so of fire, was necessary for
the particular physiognomy of even \emph{H. erectus}, as against other
primates, and such use perhaps required a societal context several millenia even before
this hominin, partly in terms of the construction of hearths, which were a necessity
for the efficient cooking of food.


Other factors such as the better  use of purposefully designed
tools, simple weapons and the intense struggle for the survival of
the fittest have also contributed greatly to the selective
advantages of \emph{H. sapiens} in the fierce struggle for its
existence; nevertheless, there is an overwhelming consensus in the
specialised literature that the \emph{co-evolution} of the human
mind and society was the predominant, or key factor for the survival
of \emph{H. sapiens} over that of all other closely related species
in the genus \emph{Homo} that did not survive-- in spite of having
existed earlier, and some probably much longer than \emph{H.
sapiens}.

The authors aim at a concise presentation of novel methodologies for
studying such difficult, as well as controversial, ontological
problems of Space and Time at different levels of objective reality
defined here as Complex, Super--Complex and Ultra--Complex Dynamic
Systems, simply in order `to divide and conquer'. The latter two are
biological organisms, human (and perhaps also hominide) societies,
and more generally, variable `systems' and meta-systems that are not
recursively--computable. Rigorous definitions of the logical and
mathematical concepts employed here, as well as a step-by-step
construction of our conceptual framework, were provided in a recent
series of publications on categorical ontology of levels and complex
systems dynamics [33]-[34],[39]-[40]. The continuation of the very
existence of human society may now depend upon an improved
understanding of highly complex systems and the human mind, and also
upon how the global human society interacts with the rest of the
biosphere and its natural environment. It is most likely that such
tools that we shall suggest here might have value not only to the
sciences of complexity and Ontology but, more generally also, to all
philosophers seriously interested in keeping on the rigorous side of
the fence in their arguments. Following Kant's critique of `pure'
reason and Wittgenstein' s critique of language misuse in
philosophy, one needs also to critically examine the possibility of
using general and universal, mathematical language and tools in
formal approaches to a rigorous, formal Ontology.  Throughout this
monograph we shall use the attribute \emph{`categorial'} only for
philosophical and linguistic arguments. On the other hand, we shall
utilize the rigorous term \emph{`categorical'} only in conjunction
with applications of concepts and results from the more restrictive,
but still quite general, mathematical \emph{Theory of Categories,
Functors and Natural Transformations} (TC-FNT). According to SEP
(2006): ``Category theory ... is a general mathematical
\textbf{theory of structures and of systems of structures}.
\emph{Category theory is both an interesting object of philosophical
study, and a potentially powerful formal tool for philosophical
investigations of concepts such as space, system, and even truth...
It has come to occupy a central position in contemporary mathematics
and theoretical computer science, and is also applied to
mathematical physics.}" [248].  Traditional, modern philosophy--
considered as a search for improving knowledge and wisdom-- does
also aims at unity that might be obtained as suggested by Herbert
Spencer in 1862 through a \emph{`synthesis of syntheses'}; this
could be perhaps iterated many times because each treatment is based
upon a critical evaluation and provisional improvements of previous
treatments or stages. One notes however that this methodological
question is hotly debated by modern philosophers beginning, for
example, by Descartes before Kant and Spencer; Descartes championed
with a great deal of success the \emph{`analytical'} approach in
which \emph{all} available evidence is, in principle, examined
critically and skeptically first both by the proposer of novel
metaphysical claims and his, or her, readers. Descartes equated the
`synthetic' approach with the Euclidean `geometric' (axiomatic)
approach, and thus relegated synthesis to a secondary, perhaps less
significant, role than that of critical \emph{analysis} of
scientific `data' input, such as the laws, principles, axioms and
theories of all specific sciences. Spinoza's, Kant's and Spencer's
styles might be considered to be synthetic by Descartes and all
Cartesians, whereas Russell's approach might also be considered to
be analytical. Clearly and correctly, however, Descartes did not
regard analysis ($A$) and synthesis ($S$) as exactly inverse to each
other, such as $A \rightleftarrows S $, and also not merely as
`bottom--up' and `top--bottom' processes ($\downarrow \uparrow $).
Interestingly, unlike Descartes'  discourse of the philosophical
method, his treatise of philosophical principles comes closer to the
synthetic approach in having definitions and deductive attempts,
logical inferences, not unlike his `synthetic' predecessors, albeit
with completely different claims and perhaps a wider horizon. The
reader may immediately note that if one, as proposed by Descartes,
begins the presentation or method with an analysis $A$, followed by
a synthesis S, and then reversed the presentation in a follow-up
treatment by beginning with a synthesis $S*$ followed by an analysis
$A'$ of the predictions made by $S'$ consistent, or analogous, with
$A$, then obviously $AS\neq S'A'$  because we assumed that $A \simeq
A'$ and that $S \neq  S'$. Furthermore, if one did not make any
additional assumptions about analysis and synthesis, then $analysis
\rightarrow synthesis  \neq synthesis \rightarrow analysis $, or $AS
\neq SA$, that is analysis and synthesis obviously \emph{`do not
commute}'; such a theory when expressed mathematically would be then
called \emph{`non-Abelian}'. This is also a good example of the
meaning of the term non-Abelian in a philosophical, epistemological
context.


%Starting with the second page the header should contain the name
%of the author and a short title of the  paper.


$$
\textsc{2.The Theory of Levels in Categorial and Categorical Ontology}
$$

This section outlines our novel methodology and approach to the
ontological theory of levels, which is then applied in subsequent
sections in a manner consistent with our recently published
developments [33]-[34],[39]-[40]. Here, we are in harmony with the
theme and approach of Poli's ontological theory of levels of reality
[121], [206]--[211]) by considering both philosophical--categorial
aspects such as Kant's relational and modal categories, as well as
categorical--mathematical tools and models of complex systems in
terms of a dynamic, evolutionary viewpoint.

We are then presenting a Categorical Ontology of highly complex
systems, discussing the modalities and possible operational logics
of living organisms, in general. Then, we consider briefly those
integrated functions of the human brain that support the
ultra-complex human mind and its important roles in societies. Mores
specifically, we  propose to combine a critical analysis of language
with precisely defined, abstract categorical concepts from Algebraic
Topology reported by Brown et al, in  2007 [69], and the
general-mathematical Theory of Categories, Functors and Natural
Transformations:  [56], [80], [98]-[102],
[105]-[106],[113],[115-[119],[130], [133]-[135],[141]-[143],
[151],[154], [161]-[163],[165]-[168], [172], [175]-[177],[183],
[192]-[194],[198]-[199] [213]-[215],[225], [227],[246], [252], [256]
into a categorical framework which is suitable for further
ontological development, especially in the relational rather than
modal ontology of complex spacetime structures. Basic concepts of
Categorical Ontology are presented in this section, whereas formal
definitions are relegated to one of our recent, detailed reports
[69]. On the one hand, philosophical categories according to Kant
are: \emph{quantity, quality, relation} and \emph{modality}, and the
most complex and far-reaching questions concern the relational and
modality-related categories. On the other hand, mathematical
categories are considered at present as the most general and
universal structures in mathematics, consisting of related
\emph{abstract objects connected by arrows}. The abstract objects in
a category may, or may not, have a specified \emph{structure}, but
must all be of the same type or kind in any given category. The
arrows (also called \emph{`morphisms'}) can represent relations,
mappings/functions, operators, transformations, homeomorphisms, and
so on, thus allowing great flexibility in applications, including
those outside mathematics as in:  Logics [118]-[120], Computer
Science [1], [161]-[163]  [201],[248], [252], Life Sciences
[5],[11]-[17],[19],[23],[28]-[36],[39],[40],[42]-[44],[70],[74],[103]-[104],[230],[232],[234]-[238],[264],
Psychology,  Sociology [33],[34],[39],[40],[74], and Environmental
Sciences [169]. The mathematical category also has a form of
\emph{`internal' symmetry}, specified precisely as the
\emph{commutativity} of chains of morphism compositions that are
uni-directional only, or as \emph{naturality of diagrams} of
morphisms; finally, any object A of an abstract category has an
associated, unique, identity, $1_A$, and therefore, one can replace
all objects in abstract categories by the identity morphisms. When
all arrows are \emph{invertible}, the special category thus obtained
is called a \emph{`groupoid}', and plays a fundamental role in the
field of mathematics called Algebraic Topology.

The categorical viewpoint-- as emphasized by William Lawvere,
Charles Ehresmann and most mathematicians-- is that the key concept
and mathematical structure is that of \emph{morphisms} that can be
seen, for example, as abstract relations, mappings, functions,
connections, interactions, transformations, and so on. Thus, one
notes here how the philosophical category of \emph{`relation'} is
closely allied to the basic concept of morphism, or arrow, in an
abstract category; the implicit tenet is that \emph{arrows are what
counts}. One can therefore express all essential properties,
attributes, and structures by means of arrows that, in the most
general case, can represent either philosophical `relations' or
modalities, the question then remaining if philosophical--categorial
properties need be subjected to the categorical restriction of
\emph{commutativity}. As there is no \emph{a priori} reason in
either nature or `pure' reasoning, including any form of Kantian
`transcendental logic', that either relational or modal categories
should in general have any symmetry properties, one cannot impose
onto philosophy, and especially in ontology, all the strictures of
category theory, and especially commutativity. Interestingly, the
same comment applies to Logics: only the simplest forms of Logics,
the Boolean and intuitionistic, Heyting-Brouwer logic algebras are
commutative, whereas the algebras of many-valued (MV) logics, such
as \L{}ukasiewicz logic are \emph{non-commutative}  (or
\emph{non-Abelian}).



%Include as many sections as necessary

\begin{center}
\textsc{3. Basic Structure of Categorical Ontology.} \\
\textsc{The Theory of Levels: Emergence of Higher Levels, Meta--Levels and Their Sublevels}
\end{center}

With the provisos specified above, our proposed methodology and
approach employs concepts and mathematical techniques from Category
Theory which afford describing the characteristics and binding of
ontological levels besides their links with other theories. Whereas
Hartmann  in 1952 stratified levels in terms of the four frameworks:
physical, `organic'/biological, mental and spiritual [137], we
restrict here mainly to the first three. The categorical techniques
which we introduce provide a powerful means for describing levels in
both a linear and interwoven fashion, thus leading to the necessary
bill of fare: emergence, complexity and open non-equilibrium, or
irreversible systems. Furthermore, any effective approach to
Philosophical Ontology is concerned with \emph{universal items}
assembled in categories of objects and relations, involving, in
general, transformations and/or processes. Thus, Categorical
Ontology is fundamentally dependent upon both space and time
considerations. Therefore, one needs to  consider first a dynamic
classification of systems into different levels of reality,
beginning with the physical levels (including the fundamental
quantum level) and continuing in an increasing order of complexity
to the chemical--molecular levels, and then higher, towards the
biological, psychological, societal and environmental levels.
Indeed, it is the principal tenet in the theory of levels that :
\emph{``there is a two-way interaction between social and mental
systems that  impinges upon the material realm for which the latter
is the bearer of both"} [209]. Therefore, any effective Categorical
Ontology approach requires, or generates--in the constructive
sense--a \emph{`\textbf{structure}}' or \emph {pattern of linked
items} rather than a discrete set of items. The evolution in our
universe is thus seen to proceed from the level of `elementary'
quantum `wave--particles', their interactions \emph{via} quantized
fields (photons, bosons, gluons, etc.), also including the quantum
gravitation level, towards aggregates or categories of increasing
complexity. In this sense, the classical macroscopic systems are
defined as `simple' dynamical systems that are \emph{computable
recursively} as numerical solutions of mathematical systems of
either ordinary or partial differential equations. Underlying such
mathematical systems is always the Boolean, or chrysippian, logic,
namely, the logic of sets, Venn diagrams, digital computers and
perhaps automatic reflex movements/motor actions of animals. The
simple dynamical systems are always recursively computable (see for
example, Suppes, 1995--2006 [253]-[254], and also [23]), and in a
certain specific sense, both degenerate and \emph{non-generic}, and
consequently also they are \emph{structurally unstable} to small
perturbations; such systems are, in general, deterministic in the
classical sense, although there are arguments about the possibility
of chaos in quantum systems. The next higher order of systems is
then exemplified by `systems with chaotic dynamics' that are
conventionally called `complex' by physicists who study `chaotic'
dynamics/Chaos theories, computer scientists and modelers even
though such physical, dynamical systems are still completely
deterministic. It has been formally proven that such `systems with
chaos' are \emph{recursively non-computable} (see for example, refs.
[23] and [28]  for a 2-page, rigorous mathematical proof and
relevant references), and therefore they cannot be completely and
correctly simulated by digital computers, even though some are often
expressed mathematically in terms of iterated maps or
algorithmic-style formulas. Higher level systems above the chaotic
ones, that we shall call \emph{`Super--Complex, Biological
systems'}, are the living organisms, followed at still higher levels
by the \emph{ultra-complex `systems'} of the human mind and human
societies that will be discussed in the last sections. The evolution
to the highest order of complexity- the ultra-complex,
meta--`system' of processes of the human mind--may have become
possible, and indeed accelerated, only through human societal
interactions and effective, elaborate/rational and symbolic
communication through speech (rather than screech --as in the case
of chimpanzees, gorillas, baboons, etc).

\begin{center}
\textsc{4. Fundamental Concepts of Algebraic Topology with Potential Application to
Ontology Levels Theory and the Classification of SpaceTime Structures}
\end{center}

We shall consider in this section the potential impact of novel
Algebraic Topology concepts, methods and results on the problems of
defining and classifying rigorously Quantum Spacetimes (QSS)[3],
[36]-[38],[69], [78]-[79]. The 600-page project manuscript,
\emph{`Pursuing Stacks'} written by Alexander Grothendieck in 1983
was partly aimed at a \emph{non-Abelian homological algebra}; it did
not achieve this goal but has been very influential in the
development of weak $n$-categories and other \emph{higher
categorical structures} that are relevant to QSS structures. With
the advent of Quantum Groupoids--generalizing Quantum Groups,
Quantum Algebra and Quantum Algebraic Topology, several fundamental
concepts and new theorems of Algebraic Topology may also acquire an
increased importance through their potential applications to current
problems in theoretical and mathematical physics, such as those
described in an available preprint [38], and also in several other
recent publications [36]--[37], [69].
%%(Baianu, Brown and Glazebrook, 2010), (Baianu et al 2007a,b; Brown et al 2007). (Weinstein, 1996)
In such novel applications, both the internal and external groupoid
symmetries [265] may too acquire new physical significance. Thus, if
quantum theories were to reject the notion of a \emph{continuum}
model for spacetime, then it would also have to reject the notion of
the real line and the notion of a path. How then is one to construct
a homotopy theory? One possibility is to take the route signalled by
\v{C}ech [82], and which later developed in the hands of Borsuk into
`Shape Theory' [86]. Thus a quite general space is studied
indirectly by means of its approximation by open covers. Yet another
possible approach is briefly outlined in the next section.
%% (see, Cordier and Porter, 1989).

Several fundamental concepts of Algebraic Topology and Category
Theory that are needed throughout this monograph will be introduced
next so that we can reach an extremely wide range of applicability,
especially to the higher complexity levels of reality. Full
mathematical details are also available in a recent paper by Brown
et al. [69]  that focused on a mathematical--conceptual framework
for a formal approach to Categorical Ontology and the Theory of
Ontological  Levels [206], [40].

\bigbreak
\emph{Groupoids, Topological Groupoids, Groupoid Atlases and Locally Lie Groupoids}

Recall that a \emph{groupoid} $\grp$ is a small category in which every morphism is an isomorphism.
\bigbreak
\emph{Topological Groupoids}

An especially interesting concept is that of a \emph{topological
groupoid} which is a groupoid internal to the category
$\mathsf{Top}$; further mathematical details are presented in the
paper by Brown et al. in 2007 [69].
 \bigbreak

\emph{Groupoid Atlases}

Motivation for the notion of a groupoid atlas comes from considering
families of group actions, in the first instance on the same set. As
a notable instance, a subgroup $H$ of a group $G$ gives rise to a
group action of $H$ on $G$ whose orbits are the cosets of $H$ in
$G$. However a common situation is to have more than one subgroup of
$G$, and then the various actions of these subgroups on $G$ are
related to the actions of the intersections of the subgroups. This
situation is handled by the notion of {\it global action}, as
defined in [41]. A key point in this construction is that the orbits
of a group action then become the connected components of a
groupoid. Also this enables relations with other uses of groupoids.
The above account motivates the following. A \emph{groupoid atlas}
$\A$ on a set $X_{\A}$ consists of a family of `local groupoids'
$(\grp_{\A})$ defined with respective object sets $(X_\A)_{\a}$
taken to be subsets of $X_\A$. These local groupoids are indexed by
a set $\Psi_{\A}$, again called the \emph{coordinate system of $\A$}
which is equipped with a reflexive relation denoted by $\leq$~. This
data is to satisfy several conditions reported in [41]  by  Bak et
al. in 2006, and also discussed in [63] in the context of
Categorical Ontology.
%% Brown et al (2007).

\emph{The van Kampen Theorem and Its Generalisations to Groupoids and Higher Homotopy}

The van Kampen Theorem  has an important and also anomalous r\^{o}le
in algebraic topology. It allows computation of  an important
invariant for spaces built up out of simpler ones. It is anomalous
because it deals with a non-Abelian invariant, and has not been seen
as having higher dimensional analogues. However, Brown found in 1967
a generalisation of this theorem to groupoids [60], stated as
follows. In this, $\pi_1(X,X_0)$ is the \emph{fundamental groupoid}
of $X$ on a set $X_0$ of base points: so it consists of homotopy
classes rel end points of paths in $X$ joining points of $X_0 \cap
X$.  Such methods were extended successfully by R. Brown to
\emph{higher dimensions}.  The potential applications of the  Higher Homotopy
van Kampen Theorem [37]-38] were already discussed in a previous paper [69] published
by Brown, Glazebrook and Baianu in 2007.

\newpage

\begin{center}
\textsc{5. Local-to-Global Problems in Spacetime Structures. Symmetry Breaking,
Irreversibility and the Emergence of Highly Complex Dynamics}
\end{center}


\begin{center}
\textbf{Spacetime Local Inhomogeneity, Discreteness and Broken Symmetries: From Local to Global Structures.}
\end{center}

On summarizing in this section the evolution of the physical
concepts of space and time, we are pointing out first how the views
changed from homogeneity and continuity to \emph{inhomogeneity and
discreteness}. Then, we link this paradigm shift to a possible,
novel solution in terms of local-to-global approaches and procedures
to spacetime structures. These local-to-global procedures procedures
will therefore lead to a wide range of applications sketched in the
later sections, such as the \emph{emergence of higher dimensional
spacetime} structures through highly complex dynamics in organismic
development, adaptation, evolution, consciousness and society
interactions.

Classical physics, including GR involves a concept of both
\emph{continuous} and \emph{homogeneous} space and time with strict
causal (mechanistic) evolution of all physical processes
(``\emph{God does not play dice}", cf. Albert Einstein).
Furthermore, up to the introduction of \emph{quanta--discrete}
portions, or packets--of energy by Ernst Planck (which was further
elaborated by Einstein, Heisenberg, Dirac, Feynman, Weyl and other
eminent physicists of the last century), energy was also considered
to be a continuous function, though not homogeneously distributed in
space and time. Einstein's Relativity theories joined together space
and time into one `new' entity--the concept of \emph{spacetime}. In
the improved form of GR, inhomogeneities caused by the presence of
matter are also allowed to occur in spacetime. Causality, however,
remained \emph{strict}, but also more complicated than in the
Newtonian theories as discontinuities appear in spacetime in the
form of singularities, or `black holes. The standard GR theory, the
Maxwellian Theory of Electromagnetism and Newtonian mechanics can
all be considered \emph{Abelian}, even though GR not only allows,
but indeed, requires spacetime inhomogeneities to occur in the
presence of gravitational fields, unlike Newtonian mechanics
\emph{where space is both absolute and homogeneous. Recent efforts
to develop  non-Abelian} GR theories--especially with an intent to
develop Quantum Gravity theories-- seem to have considered both
possibilities of locally homogeneous or inhomogeneous, but still
globally continuous spacetimes. The successes of non-Abelian gauge
theories have become well known in physics since 1999, but they still await the
experimental discovery of their predicted Higgs
boson particles [267].


Although Einstein's Relativity theories incorporate the concept of
\emph{quantum of energy}, or photon, into their basic structures,
they also deny such discreteness to spacetime even though the
discreteness of energy is obviously accepted within Relativity
theories. The GR concept of spacetime being modified, or
\emph{distorted/`bent'}, by matter goes further back to Riemann, but
it was Einstein's GR theory that introduced the idea of representing
gravitation as the result of \emph{spacetime distortion by matter.}
Implicitly, such spacetime distortions remained continuous even
though the gravitational field energy --as all energy-- was allowed
to vary in \emph{discrete}, albeit very tiny portions--the
gravitational quanta. So far, however, the detection of gravitons
--the quanta of gravity--related to the spacetime distortions by
matter--has been unsuccessful. Mathematically elegant/precise and
physically `validated' through several crucial experiments and
astrophysical observations, Einstein's GR is obviously not
reconcilable with Quantum theories (QTs). GR was designed as the
\emph{large--scale} theory of the Universe, whereas Quantum
theories--at least in the beginning--were designed to address the
problems of \emph{microphysical} measurements at very tiny scales of
space and time involving extremely small quanta of energy. We see
therefore the QTs vs. GR as a local-to-global problem that has not
been yet resolved in the form of an universally valid Quantum
Gravity. Promising, partial solutions are suggested in three recent
papers [36],[38], [70]. Quantum theories (QTs)  were developed that
are just as elegant mathematically as GR, and they were also
physically `validated' through numerous, extremely sensitive and
carefully designed experiments.

%%(Baianu, Brown and Glazebrook, 2007b and Brown, Glazebrook and Baianu, 2007).

However, to date quantum theories have not yet been extended, or
generalized, to a form capable of recovering the results of
Einstein's GR as a quantum field theory over a GR-spacetime altered
by gravity. Furthermore, quantum symmetries occur not only on
microphysical scales, but also macroscopically in certain, `special'
cases, such as liquid $^3$He close to absolute zero and
superconductors where \emph{extended coherence} is possible for the
superfluid, long-range coupled Cooper electron-pairs. However,
explaining such interesting physical phenomena also requires the
consideration of \emph{symmetry breaking} resulting from the
Goldstone Boson Theorem as it was shown in [267].
%%(Weinberg, 2003).
Occasionally, symmetry breaking is also invoked in the recent
science literature as a `possible mechanism for human consciousness'
which also seems to be related to, or associated with some form of
`global coherence'--over most of the brain; however, the existence
of such a `\emph{quantum} coherence in the brain'--at least at
physiological temperatures--as it would be precisely
required/defined by QTs, is a most unlikely event. On the other
hand, a \emph{quantum symmetry breaking} in a neural network
considered metaphorically as a Hopfield (`spin-glass') network might
be conceivable close to physiological temperatures, except for the
lack of evidence of the existence of any requisite (electron) spin
lattice structure that is indeed an absolute requirement in such a
spin-glass metaphor.

Now comes the real, and very interesting part of the story: neuronal
networks do form functional patterns and structures that possess
partially `broken', or more general symmetries than those described
by quantum groups. Such \emph{extended symmetries} can be
mathematically determined, or specified, by certain
\emph{groupoids}--that were previously called
\emph{`neuro-groupoids' [33].} Even more generally, genetic networks also
exhibit extended symmetries that are present in biological species which are represented  by
a \emph{biogroupoid} structure, as previously defined and discussed
by Baianu, Brown, Georgescu and Glazebrook in [32]-[33]. Such
biogroupoid structures [33] can be experimentally validated, for example,
at least partially through Functional Genomics observations and
computer, bioinformatics processing [30]. We shall discuss further
several such interesting groupoid structures in the following
sections, and also how they have already been utilized in so-called
`local-to-global procedures' in order to construct `global'
solutions; such global solutions in quite complex (holonomy) cases
can still be \emph{unique} up to an isomorphism (\emph{the
Globalisation Theorem}, as it was  discussed in [69], and references
cited therein.  Last-but-not-least, \emph{holonomy} may provide a
global solution, or `explanation for memory storage by
`neuro-groupoids'. Uniqueness holonomy theorems might possibly
explain the existence of unique, persistent and resilient memories.


\begin{center}
\textbf{Towards Biological Postulates and Principles}
\end{center}

Whereas the hierarchical theory of levels provides a powerful,
systems approach through categorical ontology, the foundation of
science involves \emph{universal} models and theories pertaining to
different levels of reality. It would seem natural to expect that
theories aimed at different ontological levels of reality should
have different principles. We are advocating the need for developing
precise, but nevertheless `flexible', concepts and novel
mathematical representations suitable for understanding the
emergence of the higher complexity levels of reality. Such theories
are based on axioms, principles, postulates and laws operating on
distinct levels of reality with a specific degree of complexity.
Because of such distinctions, inter-level principles or laws are
rare and over-simplified principles abound. Alternative approaches
may be, however, possible based upon an improved ontological theory
of levels. Interestingly, the founder of Relational Biology, Nicolas
Rashevsky proposed in 1969 that physical laws and principles can be
expressed in terms of \emph{mathematical functions}, or mappings,
and are thus being predominantly expressed in a \emph{numerical}
form, whereas the laws and principles of biological organisms and
societies need take a more general form in terms of quite general,
or abstract--mathematical and logical relations which cannot always
be expressed numerically; the latter are often qualitative, whereas
the former are predominantly quantitative [224].

Rashevsky focused his Relational Biology/Society Organization papers
on a search for more general relations in Biology and Sociology that
are also compatible with the former. Furthermore, Rashevsky proposed
two biological principles that add to Darwin's natural selection and
the `survival of the fittest principle', \emph{the emergent
relational structure that are defining the adaptive organism}:

 \textbf{1. The Principle of Optimal Design}[233],\\
and

\textbf{2. The Principle of Relational Invariance} (initially phrased by \\ Rashevsky as \emph{``Biological Epimorphism"})[12]-[13],[15],[222].

In essence, the `Principle of Optimal Design' [233] defines the
organization and structure of the `fittest' organism which survives
in the natural selection process of competition between species, in
terms of an extremal criterion, similar to that of Maupertuis; the
optimally `designed' organism is that which acquires maximum
functionality essential to survival of the successful species at the
lowest `cost' possible [11]-[13]. The `design' in this case is commonly taken
in the sense of the result of a long evolutionary process that
occurred under various environmental and propagation constraints or
selection `pressures', such as that caused by sexual reproduction in
Darwin's model of the origin of species during biological evolution.
The `costs' are here defined in the context of the environmental
niche in terms of material, energy, genetic and organismic processes
required to produce/entail the pre-requisite biological function(s)
and their supporting anatomical structure(s) needed for competitive
survival in the selected niche. Further details were presented by
Robert Rosen in his short, but significant,  book on optimality
principles in theoretical biology [233], published in 1967.

The `Principle of Biological Epimorphism', on the other hand, states
that the highly specialized biological functions of higher organisms
can be mapped (through an epimorphism) onto those of the simpler
organisms, and ultimately onto those of a (hypothetical) primordial
organism (which is assumed to be unique up to an isomorphism or
\emph{selection-equivalence}). The latter proposition, as formulated
by Rashevsky, is more akin to a postulate than a principle. However,
it was then generalised and re-stated as the Postulate of Relational
Invariance [12]. Somewhat similarly, a dual principle and the
colimit construction were invoked for the ontogenetic development of
organisms [11], and more recently other quite similar colimit
constructions were considered in relation to `Memory Evolutive
Systems', or phylogeny [103]-[104].

An axiomatic system (ETAS) leading to higher dimensional algebras of
organisms in supercategories has also been formulated [18] which
specifies both the logical and the mathematical ($\pi-$ ) structures
required for complete self-reproduction and self-reference,
self-awareness, etc. of living organisms. To date,  there is no
higher dimensional algebra (HDA) axiomatics other than the ETAS
proposed for complete self-reproduction in super-complex systems, or
for self-reference in ultra-complex ones. On the other hand, the
preceding, simpler ETAC axiomatics introduced by Lawvere, was
proposed for the foundation of `all' mathematics, including
categories [166]-[167], but this seems to have occurred before the
actual emergence of HDA.

%%%%%%%%%%%%%%%%%%%%%%%%%%%%%%%%%%%%%%%%%%%%%%



\begin{center}
\textsc{6. Towards a Formal Theory of Levels in Ontology}
\end{center}


This subsection will introduce in a concise manner fundamental
concepts of the ontological theory of levels. Further details were
reported by Poli in [206]-[211], and by Baianu and Poli in this
volume [40].

\bigbreak
\emph{Fundamentals of Poli's Theory of Levels}
\bigbreak

The ontological theory of levels by Poli  [206]-[211] considers a
hierarchy of \emph{items} structured on different levels of reality,
or existence, with the higher levels \emph{emerging} from the lower,
but usually \emph{not} reducible to the latter, as claimed by
widespread reductionism.
%% (Poli, 2001, 2006a,b; 2008)
This approach modifies and expands considerably earlier work by
Hartmann [137]  both in its vision and the range of possibilities.
Thus, Poli  in [206]-[211] considers four realms or \emph{levels} of
reality: Material-inanimate/Physico-chemical,
Material-living/Biological, Psychological and Social. Poli in [211]
has stressed a need for understanding \emph{causal and
spatiotemporal} phenomena formulated within a \emph{descriptive
categorical context} for theoretical levels of reality. There is the
need in this context to develop a \emph{synthetic} methodology in
order to compensate for the critical ontic data analysis, although
one notes (cf. Rosen in 1987 [232]) that analysis and synthesis are
not the exact inverse of each other. At the same time, we address in
categorical form the \emph{internal dynamics}, the \emph{temporal
rhythm, or cycles}, and the subsequent unfolding of reality. The
genera of corresponding concepts such as `processes', `groups',
`essence', `stereotypes', and so on,  can be simply referred to as
\emph{`items}' which allow for the existence of many forms of causal
connection [210]-[211]. The implicit meaning is that the \emph{irreducible
multiplicity} of such connections converges, or it is ontologically
integrated within a \emph{unified synthesis}.


\emph{The Object-based Approach vs Process-based (Dynamic) Ontology}

In classifications, such as those developed over time in Biology for
organisms, or in Chemistry for chemical elements, the \emph{objects}
are the basic items being classified even if the `ultimate' goal may
be, for example, either evolutionary or mechanistic studies. An
ontology based strictly on object classification may have little to
offer from the point of view of its cognitive content. It is
interesting that D'Arcy W. Thompson arrived in 1941 at an ontologic
``\emph{principle of discontinuity}"  which ``is inherent in all our
classifications, whether mathematical, physical or biological... In
short, nature proceeds \emph{from one type to another} among organic
as well as inorganic forms... and to seek for stepping stones across
the gaps between is to seek in vain, for ever." (p.1094 of Thompson
in [259],  re-printed edition). Whereas the existence of different
ontological levels of reality is well-established, one cannot also
discard the study of emergence and co-emergence processes as a path
to improving our understanding of the relationships among the
ontological levels, and also as an important means of ontological
classification.  Furthermore, the emergence of ontological
meta-levels cannot be conceived in the absence of the simpler
levels, much the same way as the chemical properties of elements and
molecules cannot be properly understood without those of their
constituent electrons.

It is often thought that the \emph{object-oriented} approach can be
readily converted into a process-based one. It would seem, however,
that the answer to this question depends critically on the
ontological level selected. For example, at the quantum level,
\emph{object and process become inter-mingled}. Either comparing or
moving between levels-- for example through emergent processes--
requires ultimately a \emph{process-based} approach, especially in
Categorical Ontology where relations and inter-process connections
are essential to developing any valid theory. Ontologically, the
quantum level is a fundamentally important starting point which
needs to be taken into account by any theory of levels that aims at
completeness. Such completeness may not be attainable, however,
simply because an `extension' of G\"odel's theorem may hold here
also. The fundamental quantum level is generally accepted to be
dynamically, or intrinsically \emph{non-commutative}, in the sense
of the\emph{ non-commutative quantum logic} and also in the sense of
\emph{non-commuting quantum operators} for the essential quantum
observables such as position and momentum. Therefore, any
comprehensive theory of levels, in the sense of incorporating the
quantum level, is thus --\emph{mutatis mutandis}-- \emph
{non-Abelian}.  A paradigm shift towards a \emph{non-Abelian
Categorical Ontology} has already begun [33]-[34],[37]-[38],[40],[69].

%% (Brown et al, 2007: \emph{`Non-Abelian Algebraic Topology'}; Baianu, Brown and Glazebrook, 2006: NA-QAT; Baianu et al 2007a,b,c).

\begin{center}
\textbf{From Component Objects and Molecular/Anatomical Structure to Organismic Functions and
Relations: A Process--Based Approach to Ontology}
\end{center}


Wiener in 1950 made the important remark that implementation of
\emph{complex functionality} in a (complicated, but not necessarily
complex--in the sense defined above) machine requires also the
design and construction of a correspondingly \emph{complex
structure}, or structures [269].  A similar argument holds
\emph{mutatis mutandis}, or by induction, for \emph{variable}
machines, variable automata and variable dynamic systems [12]-[23];
%% (Baianu,1970 through 1986; Baianu and Marinescu, 1974)
therefore, if one represents organisms as variable dynamic systems,
one \emph{a fortiori} requires a \emph{super-complex structure} to
enable or entail \emph{super-complex dynamics}, and indeed this is
the case for organisms with their extremely intricate structures at
both the molecular and \emph{supra-molecular} levels. This seems to
be a key point which appears to have been missed in the early-stages
of Robert Rosen's theory of simple $(M,R)$-systems, prior to 1970,
that were deliberately designed to have ``no structure"  as it was
thought they would thus attain the highest degree of generality or
abstraction, but were then shown by Warner to be equivalent to a
special type of sequential machine or classical automaton [17],[264].

The essential properties that define the super-- and ultra-- complex
systems derive from the \emph{interactions, relations and dynamic
transformations} that are ubiquitous at such levels of reality--
which need to be distinguished from the levels of organization
internal to any biological organism or biosystem. Therefore, a
complete approach to Ontology should obviously include
\emph{relations and interconnections} between items, with the
emphasis on \emph{dynamic processes, complexity} and
\emph{functionality} of systems. This leads one to consider general
relations, such as \emph{morphisms} on different levels, and thus to
the \emph{categorical viewpoint} of Ontology.  The
\emph{process-based approach} to an Universal Ontology is therefore
essential to an understanding of the Ontology of Reality Levels,
hierarchies, complexity, anticipatory systems, Life, Consciousness
and the Universe(s). On the other hand, the opposite approach, based
on objects, is perhaps useful only at the initial cognitive stages
in experimental science, such as the simpler classification systems
used for efficiently organizing data and providing a simple data
structure. We note here also the distinct meaning of `object' in
psychology, which is much different from the one considered in this
subsection; for example, an external process can be `reflected' in
one's mind as an `object of study'. This duality, or complementarity
between `object' and `subject', `objective' and `subjective' seems
to be widely adopted in philosophy, beginning with Descartes and
continuing with Kant, Heidegger, and so on. A somewhat similar, but
not precisely analogous distinction is fundamental in standard
Quantum Theory-- the distinction between the observed/measured
system (which is the quantum, `subject' of the measurement ), and
the measuring instrument (which is a classical `object' that carries
out the measurement). \bigbreak

\textbf{Physicochemical Structure--Function Relationships}

It is generally accepted at present that structure-functionality
relationships are key to the understanding of super-complex systems
such as living cells and organisms. Integrating structure--function
relationships into a Categorical Ontology approach is undoubtedly a
viable alternative to any level reduction, and
philosophical/epistemologic reductionism in general. Such an
approach is also essential to the science of complex/super-complex
systems; it is also considerably more difficult than either
physicalist reductionism, entirely \emph{abstract relationalism} or
`rhetorical mathematics'. Moreover, because there are many
alternative ways in which the physico-chemical structures can be
combined within an organizational map or relational complex system,
there is a \emph{multiplicity of `solutions'} or mathematical models
that needs be investigated, and the latter are not computable with a
digital computer in the case of complex/super-complex systems such
as organisms [23],[232].
%% (Rosen, 1987).
The problem is further compounded by the presence of
\emph{structural disorder} (in the physical structure sense) which
leads to a very high \emph{multiplicity} of
dynamical-physicochemical structures (or `configurations') of a
biopolymer-- such as a protein, enzyme, or nucleic acid, of a
biomembrane, as well as of a living cell, that correspond to a
single function or a small number of physiological functions [20];
this complicates the assignment of a `fuzzy' physico-chemical
structure to a well-defined biological function unless extensive
experimental data are available, as for example, those derived
through computation from 2D-NMR spectroscopy data (as for example by
W\"utrich, in 1996 [271]), or neutron/X-ray scattering and related
multi-nuclear NMR spectroscopy/relaxation data  [20]
%%(see, for example, Chs. 2 to 9 of Baianu et al., 1995). %% (Baianu, 1980b);
Detailed considerations of the ubiquitous, or universal,  partial
disorder effects on the structure-functionality relationships were
reported for the first time by Baianu in 1980 [20]. Specific aspects
were also recently discussed by W\"utrich  in 1996  on the basis of
2D-NMR analysis of `small' protein configurations in solution [271].


As befitting the situation, there are devised \emph{universal}
categories of reality in its entirety, and also subcategories which
apply to the respective sub-domains of reality. We harmonize this
theme by considering categorical models of complex systems in terms
of an evolutionary dynamic viewpoint using the mathematical methods
of Category Theory which afford describing the characteristics,
classification and emergence of levels, besides the links with other
theories that are,  \emph{a priori}, essential requirements of any
ontological theory. We also underscore a significant component of
this essay that relates the ontology to geometry/topology;
specifically, if a level is defined via `iterates of local
procedures' (cf `items in iteration' cf. Brown and  $\dot{\rm
I}$\c{c}en in [71]), that will further expanded upon in the last
sections; then we will have a handle on describing its intrinsic
governing dynamics (with feedback). As we shall see in the next
subsection, categorical techniques-- which form an integral part of
our ontological considerations-- provide a means of describing a
hierarchy of levels in both a linear and interwoven, or
\emph{entangled}, fashion, thus leading to the necessary bill of
fare: emergence, higher complexity and open,
non-equilibrium/irreversible systems. We must emphasize that the
categorical methodology selected here is \emph{intrinsically `higher
dimensional'}, and can thus account for meta--levels,  such as
`processes between processes...' within, or between, the levels--and
sub-levels-- in question. Whereas a strictly Boolean classification
of levels allows only for the occurrence of \emph{discrete}
ontological levels, and also does not readily accommodate either
\emph{contingent} or \emph{stochastic sub-levels}, the LM-logic
algebra is readily extended to \emph{continuous}, \emph{contingent}
or even \emph{fuzzy} sub-levels, or levels of reality [11],[23],[32]-[34],[39]-[40],[120],[140].
 %%(cf. (Baianu and Marinescu, 1968;  Georgescu, 2006; Baianu, 1977, 1987; Baianu, Brown, Georgescu and Glazebrook, 2006).
Clearly, a Non-Abelian Ontology of Levels would require the
inclusion of either Q- or LM- logics algebraic categories (discussed
in the following section) because it begins at the fundamental
quantum level --where Q-logic reigns-- and `rises' to the emergent
ultra-complex level(s) with `all' of its possible sub-levels
represented by certain LM-logics. (Further considerations on the
meta--level question are presented by Baianu and Poli in this volume
[40]). On each level of the ontological hierarchy there is a
significant amount of connectivity through inter-dependence,
interactions or general relations often giving rise to complex
patterns that are not readily analyzed by partitioning or through
stochastic methods as they are neither simple, nor are they random
connections. This ontological situation gives rise to a wide variety
of networks, graphs, and/or mathematical categories, all with
different connectivity rules, different types of activities, and
also a hierarchy of super-networks of networks of subnetworks. Then,
the important question arises \emph{what types of basic symmetry or
patterns} such super-networks of items can have, and how do the
effects of their sub-networks percolate through the various levels.
From the categorical viewpoint, these are of two basic types: they
are either \emph{commutative} or \emph{non-commutative}, where, at
least at the quantum level, the latter takes precedence over the
former, as we shall further discuss and explain in the following
sections.

%%%%%%%%%%%%%%%%%%%%%%%%%%%%%%%%%%%%%%%%%%%
\end{document}



%%%%%
\end{document}
