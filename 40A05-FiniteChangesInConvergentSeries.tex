\documentclass[12pt]{article}
\usepackage{pmmeta}
\pmcanonicalname{FiniteChangesInConvergentSeries}
\pmcreated{2013-03-22 19:03:10}
\pmmodified{2013-03-22 19:03:10}
\pmowner{pahio}{2872}
\pmmodifier{pahio}{2872}
\pmtitle{finite changes in convergent series}
\pmrecord{9}{41931}
\pmprivacy{1}
\pmauthor{pahio}{2872}
\pmtype{Theorem}
\pmcomment{trigger rebuild}
\pmclassification{msc}{40A05}
\pmrelated{SumOfSeriesDependsOnOrder}
\pmrelated{RiemannSeriesTheorem}
\pmrelated{RatioTestOfDAlembert}

\endmetadata

% this is the default PlanetMath preamble.  as your knowledge
% of TeX increases, you will probably want to edit this, but
% it should be fine as is for beginners.

% almost certainly you want these
\usepackage{amssymb}
\usepackage{amsmath}
\usepackage{amsfonts}

% used for TeXing text within eps files
%\usepackage{psfrag}
% need this for including graphics (\includegraphics)
%\usepackage{graphicx}
% for neatly defining theorems and propositions
 \usepackage{amsthm}
% making logically defined graphics
%%%\usepackage{xypic}

% there are many more packages, add them here as you need them

% define commands here

\theoremstyle{definition}
\newtheorem*{thmplain}{Theorem}

\begin{document}
\PMlinkescapeword{terms}

The following theorem means that at the beginning of a convergent series, one can remove or attach a finite amount of terms without influencing on the convergence of the series -- the convergence is determined alone by the infinitely long ``tail'' of the series.\, Consequently, one can also freely change the \PMlinkescapetext{order} of a finite amount of terms.\\

\textbf{Theorem.}\, Let $k$ be a natural number.\, A series $\displaystyle\sum_{n=1}^\infty a_n$ converges iff
the series $\displaystyle\sum_{n=k+1}^\infty\!a_n$ converges.\, Then the sums of both series are \PMlinkescapetext{connected} with
\begin{align}
\sum_{n=k+1}^\infty\!a_n \;=\; \sum_{n=1}^\infty a_n-\sum_{n=1}^k a_n.
\end{align}


\emph{Proof.}\, Denote the $k$th partial sum of $\sum_{n=1}^\infty a_n$ by $S_k$ and the $n$th partial sum of 
$\sum_{n=k+1}^\infty a_n$ by $S_n'$.\, Then we have
\begin{align}
S_n' \;=\; \sum_{n=k+1}^{k+n}\!a_n \;=\; S_{k+n}\!-\!S_k.
\end{align}
$1^\circ$.\, If $\sum_{n=1}^\infty a_n$ converges, i.e.\, $\lim_{n\to\infty}S_n := S$\, exists as a finite number,  then (2) implies
$$\lim_{n\to\infty}S_n' \;=\; \lim_{n\to\infty}S_{k+n}-\lim_{n\to\infty}S_k \;=\; S\!-\!S_k.$$
Thus $\sum_{n=k+1}^\infty a_n$ converges and (1) is true.

$2^\circ$.\, If we suppose $\sum_{n=k+1}^\infty a_n$ to be convergent, i.e.\, $\lim_{n\to\infty}S_n' := S'$\, exists as finite, then (2) implies that
$$\lim_{n\to\infty}S_n \;=\; \lim_{n\to\infty}S_{k+n} \;=\; \lim_{n\to\infty}(S_k+S_n') \;=\; S_k\!+\!S'.$$
This means that $\sum_{n=1}^\infty a_n$ is convergent and\, $S = S_k\!+\!S'$,\, which is (1), is in \PMlinkescapetext{force}.


%%%%%
%%%%%
\end{document}
