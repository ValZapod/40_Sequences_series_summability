\documentclass[12pt]{article}
\usepackage{pmmeta}
\pmcanonicalname{CesaroSummability}
\pmcreated{2013-03-22 13:07:01}
\pmmodified{2013-03-22 13:07:01}
\pmowner{rmilson}{146}
\pmmodifier{rmilson}{146}
\pmtitle{Ces\`aro summability}
\pmrecord{6}{33547}
\pmprivacy{1}
\pmauthor{rmilson}{146}
\pmtype{Definition}
\pmcomment{trigger rebuild}
\pmclassification{msc}{40G05}
\pmrelated{CesaroMean}

\endmetadata

\usepackage{amsmath}
\usepackage{amsfonts}
\usepackage{amssymb}
\newcommand{\reals}{\mathbb{R}}
\newcommand{\natnums}{\mathbb{N}}
\newcommand{\cnums}{\mathbb{C}}
\newcommand{\znums}{\mathbb{Z}}
\newcommand{\lp}{\left(}
\newcommand{\rp}{\right)}
\newcommand{\lb}{\left[}
\newcommand{\rb}{\right]}
\newcommand{\supth}{^{\text{th}}}
\newtheorem{proposition}{Proposition}
\newtheorem{definition}[proposition]{Definition}

\newtheorem{theorem}[proposition]{Theorem}
\begin{document}
Ces\`aro summability is a generalized convergence criterion for infinite
series.  We say that a series $\sum_{n=0}^\infty a_n$ is
Ces\`aro summable if the Ces\`aro means of the partial sums converge to
some limit $L$.  To be more precise, letting $$s_N=\sum_{n=0}^N a_n$$
denote the $N\supth$ partial sum, we say that $\sum_{n=0}^\infty a_n$
Ces\`aro converges to a limit $L$, if
$$\frac{1}{N+1}(s_0+\ldots+s_N) \rightarrow L \quad\text{as}\quad N\rightarrow\infty.$$

Ces\`aro summability is a generalization of the usual definition of the
limit of an infinite series.  
\begin{proposition}
  Suppose that
  $$\sum_{n=0}^\infty a_n = L,$$
  in the usual sense that
  $s_N\rightarrow L$ as $N\rightarrow\infty$.  Then, the series in
  question Ces\`aro converges to the same
  limit.
\end{proposition}
The converse, however is false.  The standard example of a divergent
series, that is nonetheless Ces\`aro summable is
$$\sum_{n=0}^\infty (-1)^n.$$
The sequence of partial sums
$1,0,1,0,\ldots$ does not converge.  The Ces\`aro means, namely
$$\frac{1}{1},\frac{1}{2},\frac{2}{3},\frac{2}{4},\frac{3}{5},\frac{3}{6},
\ldots$$
do converge, with $1/2$ as the limit. Hence the series in
question is Ces\`aro summable.

There is also a relation between Ces\`aro summability and Abel
summability\footnote{This and similar results are often called Abelian
  theorems.}.
\begin{theorem}[Frobenius]
  A series that is Ces\`aro summable is also Abel summable.  To be more
  precise, suppose that
  $$\frac{1}{N+1}(s_0+\ldots+s_N) \rightarrow L \quad\text{as}\quad
  N\rightarrow\infty.$$
  Then,
  $$f(r) = \sum_{n=0}^\infty a_n r^n \rightarrow L \quad\text{as}\quad
  r\rightarrow 1^{-}$$ as well.
\end{theorem}
%%%%%
%%%%%
\end{document}
