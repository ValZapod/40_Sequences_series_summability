\documentclass[12pt]{article}
\usepackage{pmmeta}
\pmcanonicalname{ConvergenceOfIntegrals}
\pmcreated{2013-03-22 18:59:51}
\pmmodified{2013-03-22 18:59:51}
\pmowner{pahio}{2872}
\pmmodifier{pahio}{2872}
\pmtitle{convergence of integrals}
\pmrecord{9}{41865}
\pmprivacy{1}
\pmauthor{pahio}{2872}
\pmtype{Example}
\pmcomment{trigger rebuild}
\pmclassification{msc}{40A10}
%\pmkeywords{improper integral}
\pmrelated{UniformConvergenceOfIntegral}
\pmrelated{LogarithmicIntegral2}
\pmrelated{ListOfImproperIntegrals}
\pmrelated{SubstitutionNotation}
\pmrelated{ONotation}
\pmdefines{convergent integral}
\pmdefines{divergent integral}

% this is the default PlanetMath preamble.  as your knowledge
% of TeX increases, you will probably want to edit this, but
% it should be fine as is for beginners.

% almost certainly you want these
\usepackage{amssymb}
\usepackage{amsmath}
\usepackage{amsfonts}

% used for TeXing text within eps files
%\usepackage{psfrag}
% need this for including graphics (\includegraphics)
%\usepackage{graphicx}
% for neatly defining theorems and propositions
%\usepackage{amsthm}
% making logically defined graphics
%%%\usepackage{xypic}

% there are many more packages, add them here as you need them

% define commands here
\newcommand{\sijoitus}[2]%
{\operatornamewithlimits{\Big/}_{\!\!\!#1}^{\,#2}}
\begin{document}
\PMlinkescapeword{convergent} \PMlinkescapeword{divergent}

Similarly as one speaks of convergence of series, one can speak of \emph{convergence of integrals}, especially of Riemann integrals
$$\int_If(t)\,dt.$$
This integral is \emph{convergent}, if it exists, and otherwise \emph{divergent}.\, One can also speak of \emph{absolute convergence of integrals}.\\

\textbf{Example.}\, Study the convergence of the integral
\begin{align}
\int_1^2\frac{dx}{(\ln{x})^c}
\end{align}
where $c$ is a real constant.

According to the logarithm series, we may write for\, $1 < x < b$,\, where $b$ is sufficiently close to $1$, the estimations
\begin{align*}
\ln(x\!-\!1) \;=\; x-1+O((x\!-\!1)^2) \;=\; (x\!-\!1)[1+O(x\!-\!1)] \;
\begin{cases}
\leq 2(x\!-\!1), \\
\geq \frac{1}{2}(x\!-\!1). 
\end{cases}
\end{align*}
Let\, $1 < a < b$.\, \\

$1^\circ$.\, For\, $c > 1$:
\begin{align*}
\int_a^b\frac{dx}{(\ln{x})^c} &\geqq \int_a^b\frac{dx}{2^c(x\!-\!1)^c}
   \;=\; -\frac{1}{2^c}\!\sijoitus{x=a}{\quad b}\!\frac{1}{(c\!-\!1)(x\!-\!1)^{c-1}}\\
  &\;=\; \frac{1}{2^c(c\!-\!1)}\left[\frac{1}{(a\!-\!1)^{c-1}}-\frac{1}{(b\!-\!1)^{c-1}}\right]
\;\longrightarrow \infty \quad \mbox{as} \quad a \to 1+
\end{align*}

$2^\circ$.\, For\, $c = 1$:
\begin{align*}
\int_a^b\frac{dx}{\ln{x}} &\geqq \int_a^b\frac{dx}{2(x\!-\!1)}
   \;=\; \frac{1}{2}\!\sijoitus{a}{\quad b}\!\ln(x\!-\!1)\\
  &\;=\; \frac{1}{2}\left[\ln(b\!-\!1)-\ln(a\!-\!1)\right]
\;\longrightarrow \infty \quad \mbox{as} \quad a \to 1+
\end{align*}

$3^\circ$.\, For\, $c < 1$:
\begin{align*}
0 \;<\; \int_a^b\frac{dx}{(\ln{x})^c} &\leqq \int_a^b\frac{2^c\,dx}{(x\!-\!1)^c}
   \;=\; 2^c\!\sijoitus{x=a}{\quad b}\!\frac{x^{1-c}}{1\!-\!c}\\
  &\;=\; \frac{2^c}{1\!-\!c}\left[(b\!-\!1)^{1-c}-(a\!-\!1)^{1-c}\right]
\;\longrightarrow \frac{2^c}{1\!-\!c}(b\!-\!1)^{1-c} \quad \mbox{as} \quad a \to 1+
\end{align*}
Consequently, the integral $\int_a^b\frac{dx}{(\ln{x})^c}$, and thus also (1), converges if and only if\, $c < 1$.


%%%%%
%%%%%
\end{document}
