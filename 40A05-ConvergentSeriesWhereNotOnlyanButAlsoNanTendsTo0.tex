\documentclass[12pt]{article}
\usepackage{pmmeta}
\pmcanonicalname{ConvergentSeriesWhereNotOnlyanButAlsoNanTendsTo0}
\pmcreated{2013-03-22 19:03:29}
\pmmodified{2013-03-22 19:03:29}
\pmowner{pahio}{2872}
\pmmodifier{pahio}{2872}
\pmtitle{convergent series where not only$~a_n$ but also $na_n$ tends to 0}
\pmrecord{14}{41941}
\pmprivacy{1}
\pmauthor{pahio}{2872}
\pmtype{Theorem}
\pmcomment{trigger rebuild}
\pmclassification{msc}{40A05}
\pmsynonym{Olivier's theorem}{ConvergentSeriesWhereNotOnlyanButAlsoNanTendsTo0}
%\pmkeywords{decreasing}
\pmrelated{NecessaryConditionOfConvergence}
\pmrelated{AGeneralisationOfOlivierCriterion}

% this is the default PlanetMath preamble.  as your knowledge
% of TeX increases, you will probably want to edit this, but
% it should be fine as is for beginners.

% almost certainly you want these
\usepackage{amssymb}
\usepackage{amsmath}
\usepackage{amsfonts}

% used for TeXing text within eps files
%\usepackage{psfrag}
% need this for including graphics (\includegraphics)
%\usepackage{graphicx}
% for neatly defining theorems and propositions
 \usepackage{amsthm}
% making logically defined graphics
%%%\usepackage{xypic}

% there are many more packages, add them here as you need them

% define commands here

\theoremstyle{definition}
\newtheorem*{thmplain}{Theorem}

\begin{document}
\textbf{Proposition.}\, If the \PMlinkname{terms}{Series} $a_n$ of the convergent series
$$a_1+a_2+\ldots$$
are positive and form a monotonically decreasing sequence, then 
\begin{align}
\lim_{n\to\infty}na_n \;=\; 0.
\end{align}

\emph{Proof.}\, Let $\varepsilon$ be any positive number.\, By the Cauchy criterion for convergence and the positivity of the terms, there is a positive integer $m$ such that
$$0 \;<\; a_{m+1}+\ldots+a_{m+p} \;<\; \frac{\varepsilon}{2} \qquad (p \;=\; 1,\,2,\,\ldots).$$
Since the sequence \,$a_1,\,a_2,\,\ldots$\, is decreasing, this implies
\begin{align}
0 \;<\; pa_{m+p} \;<\; \frac{\varepsilon}{2} \qquad (p \;=\; 1,\,2,\,\ldots).
\end{align}
Choosing here especially\, $p := m$,\, we get
$$0 \;<\; ma_{m+m} \;<\; \frac{\varepsilon}{2},$$
whence again due to the decrease,
\begin{align}
0 \;<\; ma_{m+p} \;<\; \frac{\varepsilon}{2} \qquad (p \;=\; m,\,m\!+\!1,\,\ldots).
\end{align}
Adding the inequalities (2) and (3) with the common values\, $p = m,\,m\!+\!1,\,\ldots$\, then yields
$$0 \;<\; (m\!+\!p)a_{m+p} \;<\; \varepsilon \qquad \mbox{for}\quad p \;\geqq\; m.$$
This may be written also in the form
$$0 \;<\; na_n \;<\; \varepsilon \qquad \mbox{for} \quad n\;\geqq\; 2m$$
which means that\, $\lim_{n\to\infty}na_n \;=\; 0$.\\

\textbf{Remark.}\, The assumption of monotonicity in the Proposition is essential.\, I.e., without it, one cannot gererally get the limit result (1).\, A counterexample would be the series $a_1\!+\!a_2\!+\ldots$ where\, 
$a_n := \frac{1}{n}$\, for any perfect square $n$ but 0 for other values of $n$.\, Then this series is convergent (cf. the over-harmonic series), but\, $na_n = 1$\, for each perfect square $n$; so\, $na_n \not\to 0$\, as\, $n \to \infty$.

%%%%%
%%%%%
\end{document}
