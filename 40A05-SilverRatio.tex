\documentclass[12pt]{article}
\usepackage{pmmeta}
\pmcanonicalname{SilverRatio}
\pmcreated{2013-03-22 16:42:10}
\pmmodified{2013-03-22 16:42:10}
\pmowner{PrimeFan}{13766}
\pmmodifier{PrimeFan}{13766}
\pmtitle{silver ratio}
\pmrecord{5}{38917}
\pmprivacy{1}
\pmauthor{PrimeFan}{13766}
\pmtype{Definition}
\pmcomment{trigger rebuild}
\pmclassification{msc}{40A05}

\endmetadata

% this is the default PlanetMath preamble.  as your knowledge
% of TeX increases, you will probably want to edit this, but
% it should be fine as is for beginners.

% almost certainly you want these
\usepackage{amssymb}
\usepackage{amsmath}
\usepackage{amsfonts}

% used for TeXing text within eps files
%\usepackage{psfrag}
% need this for including graphics (\includegraphics)
%\usepackage{graphicx}
% for neatly defining theorems and propositions
%\usepackage{amsthm}
% making logically defined graphics
%%%\usepackage{xypic}

% there are many more packages, add them here as you need them

% define commands here

\begin{document}
The {\em silver ratio} is the sum of 1 and the square root of 2, represented by the Greek letter delta with a subscript S. That is, $\delta_S = 1 + \sqrt{2}$, with an approximate value of 2.4142135623730950488 (see A014176 in Sloane's OEIS). Its continued fraction is $$2 + \frac{1}{2 + \frac{1}{2 + \frac{1}{2 + \ddots}}},$$ which suggests that the Pell numbers $P_n$ can be used as convergents. Similarly, the $n$th power of the silver ratio for $n > 0$ is $P_n\delta_S + P_{n - 1}$.
%%%%%
%%%%%
\end{document}
