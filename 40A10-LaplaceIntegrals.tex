\documentclass[12pt]{article}
\usepackage{pmmeta}
\pmcanonicalname{LaplaceIntegrals}
\pmcreated{2013-03-22 18:43:17}
\pmmodified{2013-03-22 18:43:17}
\pmowner{pahio}{2872}
\pmmodifier{pahio}{2872}
\pmtitle{Laplace integrals}
\pmrecord{5}{41489}
\pmprivacy{1}
\pmauthor{pahio}{2872}
\pmtype{Definition}
\pmcomment{trigger rebuild}
\pmclassification{msc}{40A10}

% this is the default PlanetMath preamble.  as your knowledge
% of TeX increases, you will probably want to edit this, but
% it should be fine as is for beginners.

% almost certainly you want these
\usepackage{amssymb}
\usepackage{amsmath}
\usepackage{amsfonts}

% used for TeXing text within eps files
%\usepackage{psfrag}
% need this for including graphics (\includegraphics)
%\usepackage{graphicx}
% for neatly defining theorems and propositions
 \usepackage{amsthm}
% making logically defined graphics
%%%\usepackage{xypic}

% there are many more packages, add them here as you need them

% define commands here

\theoremstyle{definition}
\newtheorem*{thmplain}{Theorem}

\begin{document}
The improper integrals
$$\displaystyle\int_{-\infty}^\infty\frac{a\cos{x}}{x^2\!+\!a^2}\,dx \quad 
\mbox{and} \quad \int_{-\infty}^\infty\frac{x\sin{x}}{x^2\!+\!a^2}\,dx,$$
where $a$ is a positive \PMlinkescapetext{constant}, are called {\em Laplace integrals}.\, Both of them have the same value $\pi e^{-a}$.\\

The evaluation of the Laplace integrals can be performed by first determining the integrals
$$\int_{-\infty}^\infty\frac{e^{ix}}{x-ia}\,dx \quad 
\mbox{and} \quad \int_{-\infty}^\infty\frac{e^{ix}}{x+ia}\,dx$$
where one integrates along the real axis.\, Therefore one has to determine the integrals
$$\oint\frac{e^{iz}}{z-ia}\,dz \quad \mbox{and} \quad \oint\frac{e^{iz}}{z+ia}\,dz$$
around the perimeter of the half-disk with the arc in the upper half-plane, centered in the origin and with the diameter \,$(-R,\,+R)$.\, The residue theorem yields the values
$$\oint\frac{e^{iz}}{z-ia}\,dz \;=\; 2i\pi e^{-a}
\quad \mbox{and} \quad \oint\frac{e^{iz}}{z+ia}\,dz \;=\, 0.$$
As in the entry example of using residue theorem, the parts of these contour integrals along the half-circle tend to zero when\, $R \to \infty$.\, Consequently, 
$$\int_{-\infty}^\infty\frac{e^{ix}}{x-ia}\,dx \;=\; 2i\pi e^{-a}\quad 
\mbox{and} \quad \int_{-\infty}^\infty\frac{e^{ix}}{x+ia}\,dx \;=\; 0.$$
These equations imply by adding and subtracting and then taking the \PMlinkname{real}{RealPart} and the imaginary parts, the \PMlinkescapetext{formulas}
$$\displaystyle\int_{-\infty}^\infty\frac{a\cos{x}}{x^2\!+\!a^2}\,dx 
\;=\; \int_{-\infty}^\infty\frac{x\sin{x}}{x^2\!+\!a^2}\,dx \;=\; \pi e^{-a}.$$

\begin{thebibliography}{9}
\bibitem{NP}{\sc R. Nevanlinna \& V. Paatero}: {\em Funktioteoria}.\, Kustannusosakeyhti\"o Otava. Helsinki (1963).
\end{thebibliography}

%%%%%
%%%%%
\end{document}
