\documentclass[12pt]{article}
\usepackage{pmmeta}
\pmcanonicalname{ProofOfAbelsLemmabyInduction}
\pmcreated{2013-03-22 13:38:04}
\pmmodified{2013-03-22 13:38:04}
\pmowner{mathcam}{2727}
\pmmodifier{mathcam}{2727}
\pmtitle{proof of Abel's lemma (by induction)}
\pmrecord{9}{34283}
\pmprivacy{1}
\pmauthor{mathcam}{2727}
\pmtype{Proof}
\pmcomment{trigger rebuild}
\pmclassification{msc}{40A05}

\endmetadata

% this is the default PlanetMath preamble.  as your knowledge
% of TeX increases, you will probably want to edit this, but
% it should be fine as is for beginners.

% almost certainly you want these
\usepackage{amssymb}
\usepackage{amsmath}
\usepackage{amsfonts}

% used for TeXing text within eps files
%\usepackage{psfrag}
% need this for including graphics (\includegraphics)
%\usepackage{graphicx}
% for neatly defining theorems and propositions
%\usepackage{amsthm}
% making logically defined graphics
%%%\usepackage{xypic}

% there are many more packages, add them here as you need them

% define commands here
\begin{document}
\PMlinkescapeword{limit}
\PMlinkescapeword{normal}
\emph{Proof.} The proof is by induction. However, let us
first recall that sum on the right side is  a
piece-wise defined function of the upper limit $N-1$.
In other words, if the upper limit is below the lower 
limit $0$, the sum is identically set to zero. 
Otherwise, it is an ordinary sum.
We therefore need to manually check the first two cases.
For the trivial case $N=0$, both sides equal to $a_0 b_0$.
Also, for $N=1$ (when the sum is a normal sum), it is easy to verify that
both sides simplify to $a_0 b_0 + a_1 b_1$.
Then, for the induction step, suppose that the
claim holds for some $N\ge 1$. For $N+1$, we then have
\begin{eqnarray*}
\sum_{i=0}^{N+1} a_i b_i &=& \sum_{i=0}^{N} a_i b_i + a_{N+1} b_{N+1} \\
&=& \sum_{i=0}^{N-1}A_i(b_i-b_{i+1})+A_N b_N + a_{N+1} b_{N+1} \\
&=& \sum_{i=0}^{N}A_i(b_i-b_{i+1})-A_{N}(b_{N}-b_{N+1})+ A_N b_N + a_{N+1} b_{N+1}.
\end{eqnarray*}
Since $-A_{N}(b_{N}-b_{N+1})+ A_N b_N + a_{N+1} b_{N+1} = A_{N+1} b_{N+1}$,
the claim follows. $\Box$.
%%%%%
%%%%%
\end{document}
