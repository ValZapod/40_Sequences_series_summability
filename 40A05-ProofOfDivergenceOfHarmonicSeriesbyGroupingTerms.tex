\documentclass[12pt]{article}
\usepackage{pmmeta}
\pmcanonicalname{ProofOfDivergenceOfHarmonicSeriesbyGroupingTerms}
\pmcreated{2013-03-22 15:08:39}
\pmmodified{2013-03-22 15:08:39}
\pmowner{rspuzio}{6075}
\pmmodifier{rspuzio}{6075}
\pmtitle{proof of divergence of harmonic series (by grouping terms)}
\pmrecord{9}{36890}
\pmprivacy{1}
\pmauthor{rspuzio}{6075}
\pmtype{Proof}
\pmcomment{trigger rebuild}
\pmclassification{msc}{40A05}

\endmetadata

% this is the default PlanetMath preamble.  as your knowledge
% of TeX increases, you will probably want to edit this, but
% it should be fine as is for beginners.

% almost certainly you want these
\usepackage{amssymb}
\usepackage{amsmath}
\usepackage{amsfonts}

% used for TeXing text within eps files
%\usepackage{psfrag}
% need this for including graphics (\includegraphics)
%\usepackage{graphicx}
% for neatly defining theorems and propositions
%\usepackage{amsthm}
% making logically defined graphics
%%%\usepackage{xypic}

% there are many more packages, add them here as you need them

% define commands here
\begin{document}
The harmonic series can be shown to diverge by a simple argument involving grouping terms.  Write
 \[ \sum_{n=1}^{2^M} \frac{1}{n} = \sum_{m=1}^M \sum_{n=2^{m-1}+1}^{2^m} \frac{1}{n}. \]
Since $1/n \ge 1/N$ when $n \le N$, we have
 \[ \sum_{n=2^{m-1}+1}^{2^m} \frac{1}{n} \ge \sum_{n=2^{m-1}+1}^{2^m} 2^{-m} = (2^m - 2^{m-1}) 2^{-m} = \frac{1}{2} \]
Hence,
 \[ \sum_{n=1}^{2^M} \frac{1}{n} \ge \frac{M}{2} \]
so the series diverges in the limit $M \to \infty$.
%%%%%
%%%%%
\end{document}
