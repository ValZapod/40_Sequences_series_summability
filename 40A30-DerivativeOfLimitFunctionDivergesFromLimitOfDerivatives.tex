\documentclass[12pt]{article}
\usepackage{pmmeta}
\pmcanonicalname{DerivativeOfLimitFunctionDivergesFromLimitOfDerivatives}
\pmcreated{2013-03-22 19:00:29}
\pmmodified{2013-03-22 19:00:29}
\pmowner{pahio}{2872}
\pmmodifier{pahio}{2872}
\pmtitle{derivative of limit function diverges from limit of derivatives}
\pmrecord{7}{41877}
\pmprivacy{1}
\pmauthor{pahio}{2872}
\pmtype{Example}
\pmcomment{trigger rebuild}
\pmclassification{msc}{40A30}
\pmclassification{msc}{26A15}
\pmsynonym{limit of derivatives diverges from derivative of limit function}{DerivativeOfLimitFunctionDivergesFromLimitOfDerivatives}
%\pmkeywords{function sequence}
%\pmkeywords{uniform convergence}
\pmrelated{GrowthOfExponentialFunction}

\endmetadata

% this is the default PlanetMath preamble.  as your knowledge
% of TeX increases, you will probably want to edit this, but
% it should be fine as is for beginners.

% almost certainly you want these
\usepackage{amssymb}
\usepackage{amsmath}
\usepackage{amsfonts}

% used for TeXing text within eps files
%\usepackage{psfrag}
% need this for including graphics (\includegraphics)
%\usepackage{graphicx}
% for neatly defining theorems and propositions
 \usepackage{amsthm}
% making logically defined graphics
%%%\usepackage{xypic}

% there are many more packages, add them here as you need them

% define commands here

\theoremstyle{definition}
\newtheorem*{thmplain}{Theorem}

\begin{document}
\PMlinkescapeword{Theorem}
For a function sequence, one cannot always change the \PMlinkescapetext{order} of \PMlinkid{taking limit}{6209} and \PMlinkname{differentiating}{Differentiate}, i.e. it may well be
$$\lim_{n\to\infty}\frac{d}{dx}f_n(x) \;\neq\; \frac{d}{dx}\lim_{n\to\infty}f_n(x),$$
\PMlinkescapetext{even} in the case that a sequence of continuous (and differentiable) functions converges uniformly; cf. Theorem 2 of the \PMlinkname{parent entry}{LimitFunctionOfSequence}.\\

\textbf{Example.}\, The function sequence
\begin{align}
f_n(x) \;:=\; \sum_{j=1}^n\frac{x^3}{(1\!+\!x^2)^j} \;=\; x-\frac{x}{(1\!+\!x^2)^n} \qquad (n \;=\; 1,\,2,\,3,\,\ldots)
\end{align}
provides an instance; we consider it on the interval \,$[-1,\,1]$.\, It's a question of partial sum the converging geometric series
$$\frac{x^3}{1\!+\!x^2}+\frac{x^3}{(1\!+\!x^2)^2}+\frac{x^3}{(1\!+\!x^2)^2}+\ldots$$
(although one cannot use Weierstrass' criterion of uniform convergence).\, Since
the limit function is
$$f(x) \;:=\; \lim_{n\to\infty}\left(x-\frac{x}{(1\!+\!x^2)^n}\right) \;=\; x \quad \forall x \in [-1,\,1],$$
we have
$$\sup_{[-1,\,1]}|f_n(x)-f(x)| \;=\; \sup_{[-1,\,1]}\frac{|x|}{(1\!+\!x^2)^n} 
\,\longrightarrow 0 \quad \mbox{as}\;\; n \to \infty,$$
which means by Theorem 1 of the \PMlinkname{parent entry}{LimitFunction} that the sequence (1) converges uniformly on the interval to the identity function.\, Further, the members of the sequence are continuous and differentiable.\, Furthermore,
$$f_n'(x) \;=\; 1-\frac{1\!+(1\!-\!2n)x^2}{(1\!+\!x^2)^{n+1}},$$
whence 
$$\lim_{n\to\infty}f_n'(x) \;=\; 1 \quad (x \;\neq\; 0).$$
But in the point \,$x = 0$\, we have
$$\lim_{n\to\infty}f_n'(0) \;=\; \lim_{n\to\infty}0 \;=\; 0,$$ 
which says that the limit of derivative sequence of (1) is discontinuous in the origin.\, Because
$$f'(x) \;\equiv\; 1,$$
we may write
$$\lim_{n\to\infty}\frac{d}{dx}f_n \;\neq\; \frac{d}{dx}\lim_{n\to\infty}f_n.$$

%%%%%
%%%%%
\end{document}
