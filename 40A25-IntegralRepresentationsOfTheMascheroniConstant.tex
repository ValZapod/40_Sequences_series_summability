\documentclass[12pt]{article}
\usepackage{pmmeta}
\pmcanonicalname{IntegralRepresentationsOfTheMascheroniConstant}
\pmcreated{2013-03-22 15:53:24}
\pmmodified{2013-03-22 15:53:24}
\pmowner{rspuzio}{6075}
\pmmodifier{rspuzio}{6075}
\pmtitle{integral representations of the Mascheroni constant}
\pmrecord{8}{37891}
\pmprivacy{1}
\pmauthor{rspuzio}{6075}
\pmtype{Theorem}
\pmcomment{trigger rebuild}
\pmclassification{msc}{40A25}

\endmetadata

% this is the default PlanetMath preamble.  as your knowledge
% of TeX increases, you will probably want to edit this, but
% it should be fine as is for beginners.

% almost certainly you want these
\usepackage{amssymb}
\usepackage{amsmath}
\usepackage{amsfonts}

% used for TeXing text within eps files
%\usepackage{psfrag}
% need this for including graphics (\includegraphics)
%\usepackage{graphicx}
% for neatly defining theorems and propositions
%\usepackage{amsthm}
% making logically defined graphics
%%%\usepackage{xypic}

% there are many more packages, add them here as you need them

% define commands here

\begin{document}
Mascheroni's constant can be expressed by the following integrals:
\begin{eqnarray*}
\gamma &=& - \int_0^1 \log (- \log x) \, dx \\
\gamma &=& - \int_0^\infty e^{-x} \log x \, dx \\
\gamma &=& \int_0^\infty \left( {1 \over e^t - 1} - {1 \over t e^t} \right) \, dt \\
\gamma &=& \int_0^\infty \left( {1 \over t} - {1 \over 1 + t} - {1 \over t e^t} \right) \, dt \\
\end{eqnarray*}

% To derive this formula, we may first deral with the singularity as $t \to 0$ in % the integrand by cutting off the lower limit of integration, then expand in a 
% series:
% \begin{eqnarray*}
% \lim_{\lambda \to 0} \int_\lambda^\infty {e^{-t} \over 1 - e^{-t}} - {e^{-t} 
% \over t}
%%%%%
%%%%%
\end{document}
