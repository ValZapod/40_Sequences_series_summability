\documentclass[12pt]{article}
\usepackage{pmmeta}
\pmcanonicalname{LimitOfGeometricSequence}
\pmcreated{2013-03-22 18:32:43}
\pmmodified{2013-03-22 18:32:43}
\pmowner{pahio}{2872}
\pmmodifier{pahio}{2872}
\pmtitle{limit of geometric sequence}
\pmrecord{6}{41264}
\pmprivacy{1}
\pmauthor{pahio}{2872}
\pmtype{Proof}
\pmcomment{trigger rebuild}
\pmclassification{msc}{40-00}

\endmetadata

% this is the default PlanetMath preamble.  as your knowledge
% of TeX increases, you will probably want to edit this, but
% it should be fine as is for beginners.

% almost certainly you want these
\usepackage{amssymb}
\usepackage{amsmath}
\usepackage{amsfonts}

% used for TeXing text within eps files
%\usepackage{psfrag}
% need this for including graphics (\includegraphics)
%\usepackage{graphicx}
% for neatly defining theorems and propositions
 \usepackage{amsthm}
% making logically defined graphics
%%%\usepackage{xypic}

% there are many more packages, add them here as you need them

% define commands here

\theoremstyle{definition}
\newtheorem*{thmplain}{Theorem}

\begin{document}
As mentionned in the geometric sequence entry, 
\begin{align}
\lim_{n\to\infty}ar^n = 0
\end{align}
for\, $|r| < 1$.\, We will prove this for real or complex values of $r$.\\

We first remark, that for the values\, $s > 1$\, we have\; 
$\displaystyle\lim_{n\to\infty}s^n = \infty$ (cf. limit of real number sequence).\, In fact, if $M$ is an arbitrary positive number, the binomial theorem (or Bernoulli's inequality) implies that
$$s^n = (1+s-1)^n > 1^n+\binom{n}{1}(s-1) = 1+n(s-1) > n(s-1) > M$$
as soon as\, $\displaystyle n > \frac{M}{s-1}$.\\

Let now\, $|r| < 1$\, and $\varepsilon$ be an arbitrarily small positive number.\, Then\, $\displaystyle|r| = \frac{1}{s}$\, with\,$s > 1$.\, By the above remark,
$$|r^n| = |r|^n = \frac{1}{s^n} < \frac{1}{n(s-1)} < \varepsilon$$
when\, $\displaystyle n > \frac{1}{(s-1)\varepsilon}$.\, Hence, 
$$\lim_{n\to\infty}r^n =0,$$
which easily implies (1) for any real number $a$.
%%%%%
%%%%%
\end{document}
