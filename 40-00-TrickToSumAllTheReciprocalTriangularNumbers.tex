\documentclass[12pt]{article}
\usepackage{pmmeta}
\pmcanonicalname{TrickToSumAllTheReciprocalTriangularNumbers}
\pmcreated{2013-03-22 18:58:21}
\pmmodified{2013-03-22 18:58:21}
\pmowner{juanman}{12619}
\pmmodifier{juanman}{12619}
\pmtitle{trick to sum all the reciprocal triangular numbers}
\pmrecord{6}{41834}
\pmprivacy{1}
\pmauthor{juanman}{12619}
\pmtype{Result}
\pmcomment{trigger rebuild}
\pmclassification{msc}{40-00}
\pmclassification{msc}{11A99}
%\pmkeywords{sum of reciprocals}
\pmrelated{TriangularNumbers}

% this is the default PlanetMath preamble.  as your knowledge
% of TeX increases, you will probably want to edit this, but
% it should be fine as is for beginners.

% almost certainly you want these
\usepackage{amssymb}
\usepackage{amsmath}
\usepackage{amsfonts}

% used for TeXing text within eps files
%\usepackage{psfrag}
% need this for including graphics (\includegraphics)
%\usepackage{graphicx}
% for neatly defining theorems and propositions
%\usepackage{amsthm}
% making logically defined graphics
%%%\usepackage{xypic}

% there are many more packages, add them here as you need them

% define commands here

\begin{document}
The following trick to sum all the reciprocals of the triangular numbers is funny:

\begin{eqnarray*}
\sigma&=&1 + \frac{1}{3} + \frac{1}{6} + \frac{1}{10} + \frac{1}{15} + \frac{1}{21} + 
  \frac{1}{28} + \frac{1}{36} + \frac{1}{45} + \frac{1}{55} + \frac{1}{66} + 
  \frac{1}{78} +...\\
      &=&1 + \left(\frac{1}{3} + \frac{1}{6}\right)+\left( \frac{1}{10} + \frac{1}{15}\right) + 
\left(\frac{1}{21} + \frac{1}{28}\right) + \left(\frac{1}{36} + \frac{1}{45}\right) +\left( \frac{1}{55} + 
\frac{1}{66}\right) +...\\
&=&1 + \frac{1}{2} + \frac{1}{2}\cdot\frac{1}{3} + \frac{1}{2}\cdot\frac{1}{6} + \frac{1}{2}\cdot\frac{1}{10} 
+ \frac{1}{2}\cdot\frac{1}{15} +... \\
&=&1 + \frac{1}{2}\left( 1 + \frac{1}{3} + \frac{1}{6} + \frac{1}{10} + \frac{1}{15} +...\right)
\end{eqnarray*}
which implies $\sigma=1+\sigma/2$ and hence 
$$1 + \frac{1}{3} + \frac{1}{6} + \frac{1}{10} + \frac{1}{15} + \frac{1}{21} + 
  \frac{1}{28} + \frac{1}{36} + \frac{1}{45} + \frac{1}{55} + \frac{1}{66} + 
  \frac{1}{78} +...=2$$



%%%%%
%%%%%
\end{document}
