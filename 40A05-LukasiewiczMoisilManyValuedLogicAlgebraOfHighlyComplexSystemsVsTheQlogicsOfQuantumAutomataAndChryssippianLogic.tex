\documentclass[12pt]{article}
\usepackage{pmmeta}
\pmcanonicalname{LukasiewiczMoisilManyValuedLogicAlgebraOfHighlyComplexSystemsVsTheQlogicsOfQuantumAutomataAndChryssippianLogic}
\pmcreated{2013-03-11 19:54:39}
\pmmodified{2013-03-11 19:54:39}
\pmowner{bci1}{20947}
\pmmodifier{}{0}
\pmtitle{Lukasiewicz-Moisil Many-Valued Logic Algebra of Highly-Complex Systems vs the Q-logics of Quantum  Automata and Chryssippian Logic}
\pmrecord{1}{50172}
\pmprivacy{1}
\pmauthor{bci1}{0}
\pmtype{Definition}

%none for now
\begin{document}
\documentclass[12pt]{amsart}

\usepackage{amsmath, amssymb, amsfonts, amsthm, amscd, latexsym,color,enumerate}
\usepackage{xypic}
\usepackage[mathscr]{eucal}

% this is the default PlanetMath preamble. as your %of TeX increases, you will probably want to edit this, 

\usepackage{amsmath, amssymb, amsfonts, amsthm, amscd, latexsym,color,enumerate}
\usepackage{xypic}
\xyoption{curve}
\usepackage[mathscr]{eucal}
% define commands here
\theoremstyle{plain}
\newtheorem{lemma}{Lemma}[section]
\newtheorem{proposition}{Proposition}[section]
\newtheorem{theorem}{Theorem}[section]
\newtheorem{corollary}{Corollary}[section]
\theoremstyle{definition}
\newtheorem{definition}{Definition}[section]
\newtheorem{example}{Example}[section]
%\theoremstyle{remark}
\newtheorem{remark}{Remark}[section]
\newtheorem*{notation}{Notation}
\newtheorem*{claim}{Claim}
\renewcommand{\thefootnote}{\ensuremath{\fnsymbol{footnote}}}

\numberwithin{equation}{section}


\newcommand{\Ce}{\mathcal C}
\newcommand{\D}{\mathcal D}
\newcommand{\E}{\mathcal E}
\newcommand{\F}{\mathcal F}
\newcommand{\G}{\mathcal G}
\newcommand{\Q}{\mathcal Q}
\newcommand{\R}{\mathcal R}
\newcommand{\cS}{\mathcal S}
\newcommand{\cU}{\mathcal U}
\newcommand{\W}{\mathcal W}
\newcommand{\bA}{\mathbb{A}}
\newcommand{\bB}{\mathbb{B}}
\newcommand{\bC}{\mathbb{C}}
\newcommand{\bD}{\mathbb{D}}
\newcommand{\bE}{\mathbb{E}}
\newcommand{\bF}{\mathbb{F}}
\newcommand{\bG}{\mathbb{G}}
\newcommand{\bK}{\mathbb{K}}
\newcommand{\bM}{\mathbb{M}}
\newcommand{\bN}{\mathbb{N}}
\newcommand{\bO}{\mathbb{O}}
\newcommand{\bP}{\mathbb{P}}
\newcommand{\bR}{\mathbb{R}}
\newcommand{\bV}{\mathbb{V}}
\newcommand{\bZ}{\mathbb{Z}}
\newcommand{\bfE}{\mathbf{E}}
\newcommand{\bfX}{\mathbf{X}}
\newcommand{\bfY}{\mathbf{Y}}
\newcommand{\bfZ}{\mathbf{Z}}
\renewcommand{\O}{\Omega}
\renewcommand{\o}{\omega}
\newcommand{\vp}{\varphi}
\newcommand{\vep}{\varepsilon}
\newcommand{\diag}{{\rm diag}}
\newcommand{\grp}{{\mathbb G}}
\newcommand{\dgrp}{{\mathbb D}}
\newcommand{\desp}{{\mathbb D^{\rm{es}}}}
\newcommand{\Geod}{{\rm Geod}}
\newcommand{\geod}{{\rm geod}}
\newcommand{\hgr}{{\mathbb H}}
\newcommand{\mgr}{{\mathbb M}}
\newcommand{\ob}{{\rm Ob}}
\newcommand{\obg}{{\rm Ob(\mathbb G)}}
\newcommand{\obgp}{{\rm Ob(\mathbb G')}}
\newcommand{\obh}{{\rm Ob(\mathbb H)}}
\newcommand{\Osmooth}{{\Omega^{\infty}(X,*)}}
\newcommand{\ghomotop}{{\rho_2^{\square}}}
\newcommand{\gcalp}{{\mathbb G(\mathcal P)}}
\newcommand{\rf}{{R_{\mathcal F}}}
\newcommand{\glob}{{\rm glob}}
\newcommand{\loc}{{\rm loc}}
\newcommand{\TOP}{{\rm TOP}}
\newcommand{\wti}{\widetilde}
\newcommand{\what}{\widehat}
\renewcommand{\a}{\alpha}
\newcommand{\be}{\beta}
\newcommand{\ga}{\gamma}
\newcommand{\Ga}{\Gamma}
\newcommand{\de}{\delta}
\newcommand{\del}{\partial}
\newcommand{\ka}{\kappa}
\newcommand{\si}{\sigma}
\newcommand{\ta}{\tau}
\newcommand{\lra}{{\longrightarrow}}
\newcommand{\ra}{{\rightarrow}}
\newcommand{\rat}{{\rightarrowtail}}
\newcommand{\oset}[1]{\overset {#1}{\ra}}
\newcommand{\osetl}[1]{\overset {#1}{\lra}}
\newcommand{\hr}{{\hookrightarrow}}
\theoremstyle{plain}
\renewcommand{\thefootnote}{\ensuremath{\fnsymbol{footnote}}}
\numberwithin{equation}{section}

\theoremstyle{definition}
\theoremstyle{plain}
\renewcommand{\thefootnote}{\ensuremath{\fnsymbol{footnote}}}
\numberwithin{equation}{section}
\newcommand{\Ad}{{\rm Ad}}
\newcommand{\Aut}{{\rm Aut}}
\newcommand{\Cl}{{\rm Cl}}
\newcommand{\Co}{{\rm Co}}
\newcommand{\DES}{{\rm DES}}
\newcommand{\Diff}{{\rm Diff}}
\newcommand{\Dom}{{\rm Dom}}
\newcommand{\Hol}{{\rm Hol}}
\newcommand{\Mon}{{\rm Mon}}
\newcommand{\Hom}{{\rm Hom}}
\newcommand{\Ker}{{\rm Ker}}
\newcommand{\Ind}{{\rm Ind}}
\newcommand{\IM}{{\rm Im}}
\newcommand{\Is}{{\rm Is}}
\newcommand{\ID}{{\rm id}}
\newcommand{\GL}{{\rm GL}}
\newcommand{\Iso}{{\rm Iso}}
\newcommand{\Sem}{{\rm Sem}}
\newcommand{\St}{{\rm St}}
\newcommand{\Sym}{{\rm Sym}}
\newcommand{\SU}{{\rm SU}}
\newcommand{\Tor}{{\rm Tor}}
\newcommand{\U}{{\rm U}}

\newcommand{\A}{\mathcal A}
\renewcommand{\O}{\Omega}
\renewcommand{\o}{\omega}

\renewcommand{\leq}{{\leqslant}}
\renewcommand{\geq}{{\geqslant}}

\begin{document}


\title[\L{}ukasiewicz-Moisil Many-Valued Logic Algebra of Highly-Complex Systems \emph{vs}  the Q-logics of Quantum  Automata and Chryssippian Logic]{\L{}ukasiewicz-Moisil Many-Valued Logic Algebra of Highly-Complex Systems \emph{vs}  the Q-logics of Quantum  Automata and Chryssippian Logic}


\date{Oct 19th, 2010}

\author[I. C. Baianu, G. Georgescu and J. F. Glazebrook]
{I. C. Baianu, G. Georgescu and J. F. Glazebrook}

\address{University of Illinois at Urbana--Champaign\\
FSHN and NPRE Departments\\AFC--NMR and NIR
Microspectroscopy Facility
\\ Urbana IL 61801 USA}

\email[I. C. Baianu]{ibaianu@illinois.edu}

\address{Univerity of Bucharest, Romania}

\email[G. Georgescu]{georgescu@funinf.cs.unibuc.ro}


\address{Department of Mathematics and Computer Science\\
 Eastern Illinois University\\ 600 Lincoln Ave.
 \\Charleston IL 61920--3099 USA}

\email[J. F. Glazebrook]{jfglazebrook@eiu.edu}


\textbf{Keywords:}\emph{LM--algebraic logic, LM--logic algebras, algebraic category of LM-logic algebras, fundamental theorems of LM-logic algebra, centered n--\L{}ukasiewicz algebras, categories of n--\L{}ukasiewicz logic algebras, categories of Boolean logic algebras, Adjointness theorem, many-valued logics of highly complex systems and Categorical Ontology, non-linear genetic networks, nonlinear dynamics, Epigenomics, Cellular Interactomics, Immunology and hormonal regulatory systems, category theory, functors and natural transformations, toposes and Heyting logic algebras, quantum automata categories, limits and colimits, quantum automata homomorphisms, Abelian category, bicomplete categories, Quantum Relational Biology, generalised metabolic-replication $(M,R)$--systems, complex bionetworks, quantum logic, non-commutative lattices, Hilbert space, quantum computers, computability of complex biological systems, Cartesian closed categories, extended quantum topos, category of groupoids, dynamic realisations of generalised $(M,R)$-systems (GMRs), category of GMRs, compact subsystems of GMRs, direct sums, fundamental groupoid of a dynamic state space, Hopf algebra, quantum groups, bialgebra, quantum groupoids, weak Hopf algebra, Yang-Baxter equations, Homotopy theory, homotopy category, n-categories, Higher Dimensional Algebra.}

\bigbreak
2000 \textit{Mathematics Subject Classification}: OG320, OG330, 03G30 (Algebraic logic::  Categorical logic, topoi), 03G20 (Algebraic logic::  Lukasiewicz algebras), 03G12 (Quantum logic), 93A30 (Systems theory; control :: General :: Mathematical modeling ), 18A15 (Category theory; General theory of categories and functors : Foundations, relations to logic), 18A40 (Category theory: General theory of categories and functors:: Adjoint functors ), 93B15 (Systems theory; control:: Realizations from input-output data), 92D10 (Genetics and population dynamics::Genetics), 18B40 (Category theory: Special categories :: Groupoids, semigroupoids, semigroups, groups), 18G55 (Category theory;Homotopical algebra),  55U40 (Algebraic topology: Applied category theory:: Foundations of homotopy theory),  55U35 (Algebraic topology:: Applied category theory :: Abstract and axiomatic homotopy theory).


\section{Introduction}
\label{1.}
% Always give a unique label
% and use \ref{<label>} for cross-references
% and \cite{<label>} for bibliographic references
% use \sectionmark{}
% to alter or adjust the section heading in the running head

Long before Boole, Chrysippus of Soli  (Ancient Greek: Chrysippos ho Soleus; c. 279--206 BC) -- who  was a Greek Stoic philosopher-- introduced the idea
of a  two-value logic, now called Boolean. It is the operational logic of everyday computing and Hilbert's theory of Logical Predicates or Symbolic Logic.
The first system of multiple-valued logic was introduced by Jan  \L{}ukasiewicz in 1920, and  independently, by  E. Post who also introduced in 1921 a different multiple-valued logic that carries his name. Subsequently, in 1930's  \L{}ukasiewicz and Tarski studied a logic with truth values in the continuous interval $[0,1]$ of real numbers. However, an \emph{algebraic} formulation of many-valued (MV)  logic was missing until its introduction by Acad. Grigore C.  Moisil who defined the 3-valued and  4-valued  \L{}ukasiewicz algebras in 1940 [31], and the $n$-valued
 \L{}ukasiewicz algebras ($n>2$) in 1942. Boolean algebras, that are defined as algebraic models of classical, or chryssipian-Boolean logic, are then only particular cases of the new structures. of MV-logic. In the description of a logical system, the implication was traditionally the principal connector. The n-valent system of  \L{}ukasiewicz had as truth values the elements of a certain set  $L_n$ and was built around a new concept of implication, on which are based the definitions of the other connectors. On the other hand for Moisil and modern mathematical logicians, the basic structure is that of a \emph{lattice}, to which he added a negation (thus generating the so-called "De Morgan algebra") together with certain unary operations (that were called by Moisil \emph{``chryssipian endomorphisms"}), that  represent the ``nuancing", or \emph{nuances} of MV-logic.  Therefore, the fundamental concept of Moisil logic is that of nuancing.The \L{}ukasiewicz implication has then taken secondary place, and-- in the case of an arbitrary valence-- was completely left out.  lt is  henceforth appropriate to refer to such $MV$-algebras as \emph{\L{}ukasiewicz-Moisil algebras} or \textbf{$LM$- logic algebras}.  The logic corresponding to $n$-valued \L{}ukasiewicz-Moisil algebras was published by Moisil in 1964 [35].

A characterization of the category of \textbf{$LM$- logic algebras} was published by Georgescu and Vraciu in 1970 [15].  A more detailed account of $LM$--logic algebras than the early work reported in [15] was recently published in refs.[1] and [16], with the latter paper also including a comparison among several MV-logics such as the Post logic, as well as the relevance of MV-logics to Complex Systems Biology (CSB), highly-complex systems, fuzzy structures, the MV-logic foundations of Probability Theory and Biostatistics.   

Interestingly, Acad. Moisil also considered the applications of LM-logic algebras to switching circuits, new designs of automata and computers based on LM-logic algebras [32]-[33], [37]-[39];  his early results in the latter fields are also pertinent to this article. An alternative approach is provided by the formalism of variable categories [40] that would allow in principle the construction of artificial intelligence systems with varying transition functions which would exhibit adaptive, learning behaviors. 
 We are taking further these previous results and ideas through a comparison of LM-logic algebra categories-- and their appications to representations of functional genomes and genetic networks--with the operational Q--logic algebras of  categories of both quantum automata and quantum supercomputers that may result in a deeper understanding of highly-complex systems and also in improved design, biomimetic strategies for artificial intelligent meta--systems. 

\section{Algebraic Logic, Operational and \L{}ukasiewicz Quantum Logic}
\label{2.}

As pointed out by Birkhoff and von Neumann as early as 1936, a logical foundation of quantum mechanics consistent with quantum algebra is essential for both the completeness and mathematical validity of the theory. The development of Quantum Mechanics from its very beginnings both inspired and required the consideration of specialized logics that are designed to be compatible with a new theory of measurements applicable to  microphysical systems. Such a specialized logic was initially formulated by Birkhoff and von Neumann in 1936, and called `Quantum Logic' or Q-logic  (QL). However, in recent QL research several distinct approaches were developed involving several types of non-distributive lattice, and their corresponding algebras, for $n$--valued quantum logics. Thus, modifications of the \L{}ukasiewicz, and indeed of  \L{}ukasiewicz-Moisil (LM) logic algebras, that were introduced in the context of algebraic categories [14] by Georgescu and Popescu [15] (also recently reviewed and expanded by Georgescu [16]) can provide an appropriate framework for representing quantum systems; alternatively, in their unmodified form, such LM-logic algebras were found to be instrumental in describing, or formally representing, the activities of complex networks in categories of  such LM-logic algebras [6]. 

There is, nevertheless, a serious problem remaining in the recent Q-logic literature which is caused by the logical inconsistency between any quantum algebra and the Heyting logic algebra that has been seriously suggested as a candidate for quantum logic. Furthermore, quantum algebra and topological approaches that are ultimately based on \emph{set-theoretical} concepts and differentiable spaces (manifolds) also encounter serious problems of internal inconsistency. Since it has been shown that standard set theory-- which is subject to the axiom of choice-- relies on Boolean logic, there appears to exist a basic logical inconsistency between the quantum logic--which is not Boolean--and the Boolean logic underlying all differentiable manifold approaches that rely on continuous spaces of points, or certain specialized sets of elements. A possible solution to such inconsistencies is the definition of a generalized Topos concept, and more specifically, of an Extended Quantum Topos (EQT) concept which is consistent with both Q-logic and Quantum logic algebras, thus being potentially suitable for the development of a framework that may unify quantum field theories with ultra-complex system modeling and Complex Systems Biology (CSB).

%\lhead{}
%\chead{Baianu et al.:  LM \& MV Logic Algebras, Complex Biosystems \& Q-logics}
%\rhead{}

\section{Lattices and Von Neumann-Birkhoff (VNB) Quantum Logic: Definition and Some Logical Properties}
\label{3.}


We commence here by giving the \emph{set-based definition of a lattice}. An \emph{s--lattice} $\mathbf{L}$, or a `set-based' lattice, is defined as a \emph{partially ordered set} that has all
binary products (defined by the $s$--lattice operation `` $\bigwedge$") and coproducts (defined by the $s$--lattice operation ``$ \bigvee$ "), with the "partial ordering" between two
elements X and Y belonging to the $s$--lattice being written as ``$X \preceq Y$". The partial order defined by $\preceq$ holds in \textbf{L }as $X \preceq  Y$ if and only if
 $X = X \bigwedge Y $ (or equivalently, $Y = X \bigvee Y $ Eq.(3.1).

 A \emph{lattice} can also be defined as a \emph{category} (see for example, ref.[9]) subject to all ETAC axioms, (but not subject, in general, to the Axiom of Choice usually encountered with sets relying on (distributive) Boolean Logic), that has all binary products and all binary coproducts, as well as the following 'partial ordering' properties:



%%Use the \LaTeX\ automatism for your citations
%%\cite{monograph}.


As an example, let us consider the logical structure formed with all truth 'nuances' or assertions of the type $<<$ \emph{system} $A$ is excitable to the $i$-th level and \emph{system} $B$ is excitable to the $j$-th level $>>$ that defines a special type of lattice which is subject to the axioms introduced by Georgescu and Vraciu [15], thus becoming a \emph{$n$-valued \L{}ukasiewicz-Moisil, or LM-Algebra}. Further algebraic and logic details are provided in refs.[16] and [9]. Such a logical structure is usually associated with the functioning of a neural network, either natural in an organism or artificial as in Artificial Intelligence (AI)  systems. However, it can also be found \emph{ in vivo} in functional genomes of various organisms.

In order to have the $n$-valued \L{}ukasiewicz-Moisil logic (LML) algebra represent correctly the basic behavior of quantum systems [3], [17],[24],[44] --which is usually observed through measurements that involve a quantum system interactions with a macroscopic measuring instrument-- several of these axioms have to be significantly changed so that the resulting lattice becomes non-distributive and also (possibly) non-associative. Several encouraging results in this direction were recently obtained by Dala Chiara and coworkers. With an appropriately defined quantum logic of events one can proceed to define Hilbert, or `nuclear'/Frechet, spaces in order to be able to utilize the `standard' procedures of quantum theories [17], [24].
\bigbreak
 On the other hand, for classical systems, modeling with the unmodified \L{}ukasiewicz logic algebra can also include both stochastic and fuzzy behaviors. For examples of such models the reader is referred to a previous report [6] where the activities of complex genetic networks are considered from a classical standpoint. One can also define as in [8] the `centers' of certain types of \L{}ukasiewicz $n$-logic algebras; then one has the following important theorem for such centered \L{}ukasiewicz $n$-logic algebras which actually defines an equivalence relation.


\begin{theorem} {\rm{\textbf{The Logic Adjointness Theorem} (Georgescu and Vraciu (1970) in ref.[15].}}
There exists an Adjointness between the Category of Centered \L ukasiewicz $n$-Logic Algebras, \textbf{CLuk--$n$}, and the Category
of Boolean Logic Algebras (\textbf{Bl}).
\end{theorem}

\begin{remark}
The logic adjointness relation between \textbf{CLuk--$n$} and \textbf{Bl} is naturally defined by the left- and -right adjoint
functors between these two categories of logic algebras.
\end{remark}

\begin{remark}
The natural equivalence logic classes defined by the
adjointness relationships in the above Adjointness Theorem define
a fundamental, \emph{`logical groupoid'} structure.
\end{remark}

\begin{remark}
In order to adapt the standard \L{}ukasiewicz-Moisil logic algebra to
the appropriate Quantum \L{}ukasiewicz-Moisil logic algebra,
\textbf{$LQL$}, a few axioms of LM-algebra need modifications, such as : $N(N(X)) = Y \neq  X  $ (instead of the restrictive identity $N(N(X)) = X$,  whenever the context, `reference frame for the measurements', or  `measurement preparation' interaction conditions
for quantum systems are incompatible with the standard `negation'
operation $N$ of the \L{}ukasiewicz-Moisil logic algebra; the latter remains however valid for classical or semi--classical systems, such as various complex networks with  $n$-states (cf.[6]-- [7]). Further algebraic and conceptual details were provided in a rigorous review by George Georgescu [16], and also in  [6] as well as in recently published reports [10], [41].
\end{remark}

\section{ Quantum Automata and Quantum Computation}
\label{4.}


Quantum computation and quantum `machines' (or nanobots) were much publicized in the early 1980's by Richard Feynman (Nobel Laureate in Physics for his approach to Quantum Electrodynamics, or QED), and subsequently a very large number of papers were published on this topic by a rapidly growing number of quantum theoreticians and some applied mathematicians. Two such specific definitions of QAs are briefly considered next.

\emph{Quantum automata} were defined in refs. [2] and [3] as \emph{generalized, probabilistic automata with quantum state spaces}. Their next-state functions operate
 through transitions between quantum states defined by the quantum equations of motions in the Schr\"{o}dinger representation, with both initial and 
boundary conditions in space-time. Such quantum automata are here renamed as \emph{S-quantum automata}.


\textbf{Definition 1.}
One obtains a simple, formal definition of \emph{S-quantum automaton} by considering instead of the transition function of a classical sequential machine, the (quantum) transitions in a finite quantum system with definite probabilities determined by quantum dynamics. The \emph{quantum state space} of a \emph{quantum automaton} is thus defined as a quantum groupoid over a bundle of Hilbert spaces, or over rigged Hilbert spaces.

Formally, whereas a sequential machine, or state machine with state space S, input set I and output set O, is defined as a quintuple: $(S, I, O, \delta : S \times S \rightarrow S, \lambda: S \times I\rightarrow O)$, an \emph{S- quantum automaton} is defined by a triple $(\emph{H}, \Delta: \emph{H} \rightarrow \emph{H}, \mu)$, where \emph{H} is either a Hilbert space or a rigged Hilbert space of quantum states and operators acting on \emph{H}, and $\mu$ is a measure related to the quantum logic, LM, and (quantum) transition probabilities of this quantum system.
\bigbreak

Two new theorems are also noted in this context (albeit stated here without proof):

\begin{theorem} {\rm{\textbf{Bicompleteness Theorem.} }}
The category of S-quantum automata and S-quantum automata homomorphisms has both limits and colimits.
\end{theorem}

\begin{theorem} {\rm{\textbf{Classical Embedding Theorem.} }}
The category of  classical, finite automata is a subcategory of the  category of S-quantum automata.
\end{theorem}

Therefore, both categories of S-quantum automata and classical automata (sequential machines) are \emph{bicomplete}  as  \textbf{Theorem 3}  states that the standard automata category, $C_{SA}$,  is a subcategory of the S-quantum automata category, $C_{QA}$, or in shorthand notation: $C_{SA} \prec C_{QA}$.

\begin{remark}
Quantum computation becomes possible only when macroscopic blocks of quantum states can be controlled \emph{via} quantum preparation and subsequent, classical observation. Obstructions to actually building, or constructing quantum computers are known to exist in dimensions greater than $2$ as a result of the standard \\
 \textbf{K-S} theorem. Subsequent definitions of quantum computers reflect attempts to either avoid or surmount such difficulties often without seeking solutions through quantum operator algebras and their representations related to extended quantum symmetries which define fundamental invariants that are key to actual constructions of this type of quantum computers.
\end{remark}

Defining a quantum automaton as an object of a Quantum Algebraic Topology (QAT)  theory requires the concept of \emph{quantum groupoid} (or of a weak Hopf algebra) which is defined as follows.
\bigbreak
\textbf{Definition 2} \emph{Quantum groupoids}, $Q_{Gd}$' s, are currently defined either as quantized, locally compact groupoids endowed with a left Haar measure system, $(Gd,\mu)$, or as \emph{weak Hopf algebras} (WHA). This concept can also be considered as an extension of the notion of quantum group, which is sometimes represented by a Hopf algebra, $\mathsf{H}$. 

\bigbreak
The concept of Hopf algebra, or `quantum group', will be introduced next in three steps.

\textbf{Definition 3.}
Firstly, an \emph{unital associative algebra} consists of a linear space $A$ together with two linear maps

$$m  : A \otimes A \rightarrow A ~ ~ {\rm{(multiplication)}}, ~
\eta : \mathbf{C} \rightarrow A ~ ~   {\rm{(unity)}},$$
%\label{(2.21)}
satisfying the conditions:
$$m(m \otimes {\mathbf 1})  = m ({\mathbf 1} \otimes m),$$
and
$$ ~ m({\mathbf 1} \otimes \eta)  = m (\eta \otimes {\mathbf 1}) = \ID.$$

Next let us consider `reversing the arrows', and take an
algebra $A$ equipped with a linear homorphisms 
$\Delta : A \rightarrow A \otimes A$, satisfying, for $a,b \in A$:

$$ \Delta(ab) = \Delta(a) \Delta(b)$$,

$$\qquad (\Delta \otimes \ID) \Delta  = (\ID \otimes \Delta) \Delta.$$

We call $\Delta$ a \emph{comultiplication}, which is said to be \emph{coassociative}.
There is also a counterpart to $\eta$, the \emph{counity} map
$\epsilon : A \rightarrow \textbf{C}$ satisfying

\begin{equation}
(\ID \otimes \epsilon) \circ \Delta = (\epsilon \otimes \ID) \circ \Delta
= \ID. \end{equation}

\textbf{Definition 4.}
A \emph{bialgebra} $(A, m, \Delta, \eta, \epsilon)$ is defined as a linear space $A$ with maps $m$, $\Delta$, $\eta$, $\epsilon$ satisfying the above properties.

Now, in order to recover anything resembling a group structure, one must append such a bialgebra with an \emph{antihomomorphism}
$S : A \rightarrow A$,  satisfying
\begin{equation}
 S(ab) = S(b) S(a),
\end{equation} for $a,b \in A$.  This map is defined implicitly via the property:

\begin{equation}
m(S \otimes
\ID \circ \Delta)= m(\ID \otimes S) \circ \Delta = \eta \circ \epsilon
\end{equation}

We call $S$ the \emph{antipode map}.

\textbf{Definition 5.}
 A \emph{Hopf algebra} is then defined as a bialgebra $(A,m, \eta, \Delta, \epsilon)$ equipped with an antipode
map $S$.

Commutative and non-commutative Hopf algebras form the backbone of quantum groups [17],[23], [24], and are thus essential to the generalizations of
symmetry. Indeed, in most respects a quantum group is identifiable with a Hopf algebra. When such algebras are actually
associated with proper groups of matrices there is considerable scope for their representations on both finite
and infinite dimensional Hilbert spaces.

\bigbreak

Alternatively,  as defined in refs.[17],[23] quantum groupoids can be regarded simply as \emph{weak Hopf algebras}. Algebroid symmetries, on the other hand, figure prominently both in the theory of dynamical deformations of quantum `groups' (e.g., Hopf algebras) and the quantum Yang--Baxter equations.
\bigbreak
\textbf{Definition 6.} In order to define a \emph{weak Hopf algebra}, one can relax certain axioms of a Hopf algebra as follows:
\begin{itemize}\itemsep=0pt
\item[(1)] The comultiplication is not necessarily unit-preserving.
\item[(2)] The counit $\epsilon$ is not necessarily a homomorphism of algebras.
\item[(3)] The axioms for the antipode map $S : A \rightarrow A$ with respect to the
counit are as follows. For all $h \in H$,

\begin{equation} m(\ID \otimes S) \Delta (h)  = (\epsilon \otimes
\ID)(\Delta (1) (h \otimes 1)),\end{equation}
\begin{equation}
 m(S \otimes \ID) \Delta (h)  = (\ID \otimes \epsilon)((1 \otimes h) \Delta(1)),
\end{equation}

\begin{equation}
 \qquad S(h)  = S(h_{(1)})
S(h_{(2)})  S(h_{(3)}) \end{equation}
\end{itemize}

\bigbreak
Several mathematicians substitute the term \emph{quantum groupoid} for a weak Hopf algebra, although this algebra in
itself is not a proper groupoid, but it may have a component \emph{group} algebra as in certain examples of the quantum double; nevertheless, weak Hopf algebras generalise Hopf algebras  that, with additional properties, were previously introduced as` quantum groups'  by mathematical physicists.


Note, however, that the requirement of \emph{local compactness} for quantum groupoids, as well as that of the existence of a left Haar measure system,  is not generally considered for quantum groups. \emph{Quantum groupoid representations} can thus define extended quantum symmetries beyond the `Standard Model' (SUSY) in Mathematical Physics or Noncommutative Geometry.
\bigbreak

\textbf{Definition 7.}
An \emph{ algebraic quantum automaton}, or \emph{$A$-quantum automaton} can now be defined as a quantum algebraic topology object-- the triplet

 \begin{equation}Q_A = (G_d,H-R_{G_d}, \emph{Aut}(G)), \end{equation}
where $G_d$ is a locally compact \emph{quantum groupoid}, $H-R_{G_d}$ are the unitary representations of $G_d$ on rigged Hilbert spaces $R_{G_d}$ of quantum states and quantum operators on $H$, and $ \emph{Aut}(G_d)$ is the transformation, or automorphism, groupoid of quantum transitions that represents all flip-flop quantum transitions of one qubit each between the permitted quantum states of the quantum automaton.

\begin{remark}
Other definitions of quantum automata and quantum computations have also been reported that are closely related to recent experimental attempts at constructing quantum computing devices. One can consider next the \emph{category of quantum automata}.
\end{remark}

\textbf{Definition 8.}
The \emph{category of algebraic quantum automata} $C_{QA}$
is defined as an {\em algebraic category whose  objects are $A$-quantum automata defined by triples} $(H, \Delta: H \rightarrow H, \mu)$ (where $H$ is either a Hilbert space or a rigged Hilbert space of quantum states and operators acting on $H$, and $\mu$ is a measure related to the quantum logic, $LM$, and (quantum) transition probabilities of this quantum system, and whose morphisms are defined between such triples by homomorphisms of Hilbert spaces, $\Delta: H \rightarrow H$, naturally compatible with the operators $\Delta$, and by homomorphisms between the associated Haar measure systems.

\bigbreak
An alternative definition is also possible based on {\em Quantum Algebraic Topology}.
\bigbreak
\textbf{Definition 9.}  A \emph{quantum algebraic topology} definition of the {\em category of algebraic quantum automata} $C_{QA}$  involves the objects specified above in \textbf{Definition 4} as $A$-quantum automaton triples $(Q_A)$, and quantum automata homomorphisms defined between such triples; these $Q_A$ morphisms are defined by groupoid homomorphisms $h: Gd \rightarrow Gd ^*$ and $\alpha: \emph{Aut}(Gd) \rightarrow  \emph{Aut}(Gd ^*)$, together with unitarity preserving mappings $u$ between unitary representations of $Gd$ on rigged Hilbert spaces (or Hilbert space bundles).

\begin{theorem} {\rm{\textbf{Quantum-Algebraic Bicompleteness Conjecture.} }}  
\emph{The category of $A$-quantum automata and $A$-quantum automata homomorphisms has both limits and colimits.}
\end{theorem}

With these definitions we can now turn to the question of how one can apply quantum automata to modelling problems of highly complex systems and Complex Systems Biology.


\section{Quantum Automata Applications to Modelling Complex Systems}
\label{5.}

One finds that the quantum automata category has a faithful representation in the category of generalised  $(M,R)$ -systems (GMRs) which are open, dynamic bio-networks [6] with defined biological relations that represent physiological functions of primordial(s), single cells and higher organisms. A new \emph{category of quantum computers}, $C_{QC}$, can also be defined in terms of \emph{reversible} quantum automata with quantum state spaces represented by topological groupoids that admit a local characterization through unique `quantum' \emph{Lie algebroids}. On the other hand, the category of $n$-\L ukasiewicz algebras has a subcategory of \emph{centered} $n$- \L{}ukasiewicz algebras [15] (which can be employed to design and construct subcategories of quantum automata based on $n$-{}\L ukasiewicz diagrams of existing VLSI. Furthermore, as shown in ref. [15] the category of centered $n$-{}\L ukasiewicz algebras and the category of Boolean algebras are naturally equivalent.

Variable machines with a varying transition function were previously discussed informally by Norbert Wiener as possible models for complex biological systems although how this might be achieved in \textit{Biocybernetics} has not been specifcally, or mathematically, presented by Wiener.
Therefore, let us consider the formal definitions of simple  $(M,R)$-systems and their generalisations.   The simplest $MR$-system is a relational model of the primordial organism which is defined by the following \emph{categorical sequence (or diagram) of sets and set-theoretical mappings}:
$f: A \rightarrow B, ~\phi: B \rightarrow \Hom_{MR}(A,B)$, where $A$ is the set of inputs to the
$MR$-system, $B$ is the set of its outputs, and $\phi$ is the `repair map', or $R$-component, of the $MR$-system which associates to a certain product, or output $b$, the `metabolic' component (such as an enzyme, E, for example)
represented by the set-theoretical mapping $f$. Then, $\Hom_{MR}(A,B)$ is defined as the set of all such metabolic (set-theoretical) mappings (occasionally written incorrectly by some authors as $\left\{f\right\}$).

\textbf{Definition 10.}  A \emph{general $(M,R)$-system} was defined by Rosen (1958a,b) as the network or graph of the metabolic and repair components of the type specified specified above in the definition of a simple  $(M,R)$ -system; such components are networked in a complex, abstract `organism' defined by all the abstract relations and connecting maps between the sets specifying all the metabolic and repair components of such a general, abstract model of the biological organism. The mappings between $(M,R)$-systems are defined as the the metabolic and repair set-theoretical mappings, such as $f$ and $\phi$ (as specified in the definition of a simple  $(M,R)$ -system); moreover, there is also a finite number of sets (just like A and B, respectively, the input and output sets, that are present in the definition of a simple  $(M,R)$ -system): $A_i, B_i$, where $f_i \in \Hom_{MR_i}(A_i,B_i)$ and
$\phi \in \Hom_{MR_i}[B, \Hom_{MR_i}(A_i,B_i)]$, with $i \in I$, and $I$ being a finite index set, or directed set, for a a finite number $n$ of distinct metabolic and repair components pairs 
 $(f_i,\phi_i)$. Alternatively, one may think of a general $MR$-system as consisting of a finite number $N$ of inter--connected metabolic-repair,$MR_i$, modules, each one such $MR_i$ module  having the input sets $A_i$ and output sets $B_i$, with $i=1, 2, ..., n$ being finite integers. To sum up:
a \emph{general MR-system} can be defined as a \emph{family of $n$ inter--connected quartets}:
$\left\{(A_i, B_i, f_i, \phi_i)\right\}_{i \in I}$, where $I$ is an index set of finite integers $i=1, 2, ..., n$.

\bigbreak

With these concepts available we can now turn to defining the category of  $(M,R)$--systems, $\textbf{C}_{MR}$.
\bigbreak

\textbf{Definition 11.}
A \emph{category $\textbf{C}_{MR}$  of $(M,R)$-system quartet modules}, \\
 $\left\{(A_i, B_i, f_i, \phi_i)\right\}_{i \in I}$, with I being an index set of integers $i=1,2,..., n$,
is a small category of sets with set-theoretical mappings defined by the $(M,R)$-morphisms between the quarted modules
 $\left\{(A_i, B_i, f_i, \phi_i)\right\}_{i \in I}$, and also with repair components defined as
$\phi_i \in \Hom_{MR_i}[B, \Hom_{MR_i}(A_i,B_i)]$, where the $(M,R)$-morphism composition is defined by the usual composition of functions between sets.

\bigbreak

With a few, additional notational changes it can be shown that the category of $(M,R)$-systems is a subcategory of the category of automata (or sequential machines), $\mathcal{S}_{[M,A]}$. Similar conclusions were also reached independently in ref. [22].  Moreover, one has the following important property of the category  $\textbf{C}_{MR}$ of
simple  $(M,R)$-systems.

\begin{theorem} {\rm{\textbf{ $(M,R)$-Systems Category Theorem (Baianu, 1973 in ref. [4].} }}
The category $\textbf{C}_{MR}$ of simple  $(M,R)$-systems and their homomorphisms is Cartesian closed.
\end{theorem}

\begin{remark}
Thus, the category $\textbf{C}_{MR}$ of simple  $(M,R)$-systems belongs to the important family of categories that are Cartesian closed, which also includes the category of sequential machines/classic automata and the category of groupoids; therefore, one would be able to develop a homotopy theory of dynamic realisations of $(M,R)$-systems [20] based on dynamic realisations in a homotopy category, in a manner broadly similar to the development of the current Homotopy Theory for groupoids [12]-[13], including concepts such as the fundamental groupoid of a dynamic state space generated by any dynamic realisation of a simple $(M,R)$- system. Such dynamic realisations of  $(M,R)$-systems can thus lead to higher homotopy and a Higher Dimensional Algebra of extended, or generalised $(M,R)$- systems that are endowed with dynamic, topological structures. 
\end{remark}

\bigbreak

On the other hand,  \emph{generalised}  $(M,R)$-systems, or GMRs, can be constructed functorially by employing the Yoneda Lemma, as shown in refs.[4]-[5]; GMRs are no longer restricted to sets, and can be also endowed with \emph{structure}, such as those possesed by quantum groupoids or quantum automata.  It follows then immediately that, unless the structure of GMRs is restricted to that of quantum groupoids or quantum automata, the categories of quantum automata or quantum groupoids can be either isomorphic  or equivalent  only to a subcategory of the GMR category, $C_{GMR}$, and not to the category of all GMRs.  A `no-go' conjecture is then here proposed:


\begin{theorem} {\rm{\textbf{No-Go Conjecture for Recursive Computation of Generalised (\textbf{M,R})--Systems.} }}  
 \emph{The high level of complexity of generalised (\textbf{M,R})--Systems [5], [20] that represent functional (living) organisms in non-commutative modelling encoding diagrams [6],[10] prevents their complete computability \emph{via} recursive programming functions or algorithms by either standard or quantum automata which require commutative encoding computation diagrams [9]-- [10],[14]}.
\end{theorem}

 The concepts of quantum automata and quantum computation were initially studied, and are also currently further investigated, in the contexts of quantum genetics, genetic networks with nonlinear dynamics, and bioinformatics. In a previous publication [2]-- after introducing the formal concept of quantum automaton--the possible implications of this concept for correctly modeling genetic and metabolic activities in functional (living) cells and organisms were also considered. This was followed by a formal report on quantum and abstract, symbolic computation based on the theory of categories, functors and natural transformations [3]. The notions of topological semigroup, quantum automaton, or quantum computer, were then suggested with a view to their potential applications to the analogous simulation of biological systems, and especially genetic activities and nonlinear dynamics in genetic networks [6]--[7].

Further, detailed studies of nonlinear dynamics in genetic networks were carried out in categories of $n$-valued, \L{}ukasiewicz Logic Algebras that showed significant dissimilarities [6] from the widespread Boolean models of human neural networks that may have begun with the early publication of [18]. Molecular models in terms of categories, functors and natural transformations were then formulated for uni-molecular chemical transformations, multi-molecular chemical and biochemical transformations [7]. Previous applications of computer modeling, classical automata theory, and relational biology to molecular biology, oncogenesis and medicine were extensively reviewed in a monograph [7], and several important conclusions were reached regarding both the potential and the limitations of computation-assisted modeling of biological systems, especially those concerned with very complex organisms such as \emph{Homo sapiens sapiens} [7]-- [10]. Computer modeling and recursive computation models are thus often restricted only to \emph{compact} subsystems (see also the following \textbf{Compactness Lemma 1})  of complex living organisms that are represented by GMRs [4]- [5],[7].

Novel approaches to solving the realisation problems of Relational Biology models in Complex Systems Biology were introduced in terms of natural transformations between functors of such molecular categories. Several applications of such natural transformations of functors were then presented to protein biosynthesis, embryogenesis and nuclear transplant experiments. Other possible realisations in Molecular Biology and Relational Biology of organisms were then suggested in terms of quantum automata models in Quantum Genetics and Cellular Interactomics. Future developments of this novel approach are likely to also include: fuzzy relations in Biology and Epigenomics, Relational Biology modeling of complex immunological and hormonal regulatory systems, $n$-categories and \emph{generalised $LM$}--Topoi of \L{}ukasiewicz Logic Algebras and intuitionistic logic (Heyting) algebras for modeling nonlinear dynamics and cognitive processes in complex neural networks that are present in the human brain, as well as stochastic modeling of genetic networks in \L{}ukasiewicz Logic Algebras (LLAs).

\bigbreak
A special case of \emph{compact} subsystems of a GMR representing some component of an organism, such as the skeleton, may however escape the interdiction imposed by the `no-go' \textbf{Conjecture 2}. Such compact objects of a GMR--in the sense of category theory-- form either an additive  or an Abelian category $\mathcal{A}$  to which the following \emph{compactness lemma} applies.

\begin{lemma}
An object $X$ in an Abelian category  $\mathcal{A}$ with arbitrary direct sums (also called \emph{coproducts}) is compact if and only if the functor $\hom_{\mathcal{A}}(X,-)$ commutes with arbitrary direct sums, that is, if
\begin{equation}
\hom_{\mathcal{A}}(X,\bigoplus_{\alpha \in S} Y_{\alpha}) =
\bigoplus_{\alpha \in S} \hom_{\mathcal{A}}(X,Y_{\alpha}).
\end{equation}
\end{lemma}
(\textbf{Compactness Lemma}  from ref. [21]).
\bigbreak

\section{Conclusions}
\label{6.}

Non-distributive varieties of many-valued, LM-logic algebras that are also noncommutative open new possibilities for formal treatments of both complex quantum systems and highly complex biological networks, such as genetic nets, metabolic-replication systems (see for example refs. [19]--[20] and [22]), the interactome [6] and neural networks [7]. This novel approach that involves both Algebraic Logic and Category Theory, provides an important framework for understanding the complexity inherent in intelligent systems and their flexible, adaptive behaviors. A consequence of the Logical Adjointness Theorem-- which defines categorically the natural equivalence between the category of centered LM-logic algebras and that of Boolean logic algebras-- is that one may be able to define Artificial Intelligence analogs of neural networks based on centered LM-logic algebras. In this process, higher dimensional algebra (HDA; [12]-[13]) and categorical models of human brain dynamics (refs. [8]--[11]) were predicted to play a central role. These new approaches are also relevant for resolving the tug-of-war between nature-vs.-nurture theories of human development and the `natural' emergence through co-evolution of intelligence in the first \emph{H. sapiens sapiens} societies.


%\begin{table}
%\centering \caption{Please write your table caption here}
%\label{tab:1}       % Give a unique label
%
% For LaTeX tables use
%
%\begin{tabular}{lll}
%\hline\noalign{\smallskip}
%first & second & third  \\
%\noalign{\smallskip}\hline\noalign{\smallskip}
%number & number & number \\
%number & number & number \\
%\noalign{\smallskip}\hline
%\end{tabular}
%\end{table}

\newpage

\textbf{References}
\bigbreak
\bibliographystyle{References}
\bibliography{References}


[1]
Georgescu G, Iorgulescu A, Rudeanu S (2006) International Journal of Computers, Communications and Control, vol.\textbf{1} (1): 81--99

[2]
Baianu I C (1971a)  Organismic Supercategories and Qualitative Dynamics of Systems. \emph{Bull. Math.Biophysics}., \textbf{33}:339-353

[3]
Baianu I C (1971b) Categories, Functors and Quantum Algebraic Computations. In: Suppes P (ed) \emph{Proceed. Fourth Intl. Congress LMPS}, September 1-4, 1971, University of Bucharest


[4]
Baianu I C  (1973) Some Algebraic Properties of (M,R)-Systems in Categories. \emph{Bull. Math. Biophysics} \textbf{35}: 213-218


[5]
Baianu I C  and Mircea M. Marinescu (1974) On a Functorial Construction of Generalized  $\textbf{(M,R)}$- Systems. \emph{Rev.Roum.Math Pur.et Appl.} \textbf{19}:389--392.

[6]
Baianu I C (1977) A Logical Model of Genetic Activities in \L{}ukasiewicz Algebras: The Non-linear Theory. \emph{Bulletin of Mathematical Biology} \textbf{39}:249--258 

[7]
Baianu I.C (1987) Computer Models and Automata Theory in Biology and Medicine(A Review). In:M. Witten (ed)  \emph{Mathematical Models in Medicine.} vol.\textbf{7}:.1513--1577. Pergamon Press, New York

[8]
Baianu I C, Brown R, and  Glazebrook J F (2007a) Categorical ontology of complex spacetime structures: the emergence of life and human consciousness, \emph{Axiomathes} \textbf{17}: 223--352.

[9]
Baianu I.C, Brown R, and Glazebrook J F  (2007b) A conceptual construction of complexity levels theory in spacetime categorical ontology: non-abelian algebraic topology, many-valued logics and dynamic systems. \emph{Axiomathes} \textbf{17}:409--493.

[10]
Baianu I.C, Georgescu G,  Glazebrook J F, and Brown R (2010)  \L{}ukasiewicz-Moisil many-valued logic algebra of highly-complex systems. Broad Research in Artificial Intelligence and Neuroscience (BRAIN). In: Iantovics B,   Radoiu D, Maruteri M, and Dehmer, M (eds) Special Issue on Complexity in Sciences and Artificial Intelligence.  (ISSN: 2067-3957)  \textbf{1}:1--11
 
[11]
Brown R, and Porter T (2003) Category theory and higher dimensional algebra: potential descriptive tools in neuroscience. In: Singh N (ed) Proceedings of the International Conference on Theoretical Neurobiology (February 2003). Conference Proceedings vol.\textbf{1}:80-92. National Brain Research Centre, Delhi 

[12]
Brown R (2004) Crossed complexes and homotopy groupoids as non commutative tools for higher dimensional local-to-global problems. In: \emph{Proceedings of the Fields Institute Workshop on Categorical Structures for Descent and Galois Theory, Hopf Algebras and Semiabelian Categories}, September 23-28, 2004, \emph{Fields Institute Communications}  \textbf{43}:101--130.

%\bibitem{BHKP}

[13]
Brown R, Hardie K A,  Kamps K H, and Porter T (2002) A homotopy double groupoid of a Hausdorff space. \emph{Theory and Applications of Categories} \textbf{10}:71-93.

[14]
Georgescu G, and Popescu D (1968) On Algebraic Categories. \emph{Revue Roumaine de Mathematiques Pures et Appliqu\'ees} \textbf{13}:337--342.

[15]
Georgescu G, and Vraciu C (1970)  On the Characterization of \L{}ukasiewicz Algebras. \emph{J. Algebra}, 16 (4):486--495.

[16]
Georgescu G (2006) N-valued Logics and \L ukasiewicz--Moisil Algebras. \emph{Axiomathes} \textbf{16} (1--2): 123--136.

[17]
Landsman N P (1998) \emph{Mathematical topics between classical and quantum mechanics}. Springer Verlag, New York.

[18]
McCullough E, and Pitts M (1945) {\em Bull. Math. Biophys} \textbf{ 7}:112--145.

[19]
Rosen R (1958) The Representation of Biological Systems from the Standpoint of the Theory of Categories. \emph{Bull. Math. Biophys}., \textbf{20}, 317-341.

[20]
Rosen R (1973) On the Dynamical realization of (M,R)-Systems. \emph{Bull. Math. Biology.} \textbf{35}:1--10.

%%bibitem{Popescu1}
[21] 
Popescu, N (1973) \emph{Abelian Categories with Applications to Rings and Modules.}  Academic Press, New York and London, 2nd edn. 1975. \emph{(English translation by I.C. Baianu)}.

[22]
Warner M (1982) Representations of (M,R)-Systems by Categories of Automata. \emph{Bull. Math. Biol.} \textbf{44}: 661-668.

[23]
Baianu I C,  Glazebrook J F, and Brown R.2009. Algebraic Topology Foundations of Supersymmetry and Symmetry Breaking in Quantum Field Theory and Quantum Gravity: A Review. \emph{Symmetry, Integrability and Geometry: Methods and Applications} (SIGMA) \textbf{5}: 051, 70 pages. arXiv:0904.3644. doi:10.3842/SIGMA.2009.051

[24]
Alfsen E M, and Schultz F W (2003) \emph{Geometry of State Spaces of Operator Algebras.} Birkh\"auser, Boston--Basel--Berlin

[25]
G M, Roscoe A W,  and  Wachter R F ( eds) (1991) \emph{Topology and Category Theory in Computer Science}. Oxford University Press, USA, 498 pp. ISBN-10: 0198537603.

[26]
Moggi E, and Rosolini G (eds) (1997) \emph{Category Theory and Computer Science}: 7th International Conference, CTCS'97, Santa Margherita Ligure, Italy, September 4-6, 1997, \emph{Proceedings (Lecture Notes in Computer Science, 1290) .} ISBN-10: 354063455X ;  see also the proceedings of the  5th and 6th Conferences in the same series.

[27]
Adian S, and Nerode, A (eds) (1997) \emph{Logical Foundations of Computer Science.}, In: Proceedings, vol. \textbf{1233}. IX.Berlin, Springer -Verlag, 431 pages.

[28]
Neumann J (2002) Learning the Systematic Transformation of Holographic Reduced Representations.\emph {Cognitive Systems Research }\textbf {3}: 227-235.

[29]
Paine J (2010)  What might Category Theory do for Artificial Intelligence and Cognitive Science? April 15, popx\@j-paine.org.\\
 http://www.drdobbs.com/blog/archives/2010/04/whatmightcate.html\\;jsessionid=IY4ULLJKFMHNXQE1GHRSKHWATMY32JVN

[30]
Lawvere W F (1994) Tools for the Advancement of Objective Logic: Closed Categories and Toposes. In: Macnamara J, and. Reyes G E (eds) \emph{The Logical Foundations of Cognition.} Oxford University Press, Oxford, UK

[31]
 Moisil  Gr C (1940)  Recherches sur les logiques non-chrysippiennes. \emph{Ann Sci Univ Jassy} \textbf{26}: 431-466

[32]
 Moisil  Gr C (1959)  Utilization of three-valued logics to the theory of switching circuits. VI. Polarized relays with unstable neutral. VII. Operation of ordinary relays under low self-maintaining current. VIII. -two terminals with contacts and resistances. IX. -two-terminals with contacts, valves and resistances. X. Physical interpretation of the characteristic function of a multiterminal (Romanian), \emph{Comunic. Acad. R.P. Romane} \textbf{9}: 411--413, 531--532, 533--535, 665--666, and 667--669

[33]
Moisil  Gr C  (1960) Sur les id\'eaux des alg\`ebres \L{u}kasiewicziennes trivalentes. \emph{Analele Univ. C.I. Parhon, Seria Acta Logica} \textbf{3}:  83--95

[34]
Moisil  Gr C (1964) Sur les logiques de \L{u}kasiewicz \`a un nombre fini de valeurs. \emph{Rev. Roum. Math. Pures Appl.} \textbf{9}: 905-920

[35] 
Moisil  Gr C (1965) \emph {Old and New Essays on Non-Classical Logics}. Editura Stiintifica, Bucuresti

[36] 
Moisil  Gr C (1967) \emph{``Th\'eorie structurelle des automats finis."}. Gauthiers-Villars, Paris

[37]
Moisil  Gr C (1968) \L{u}kasiewicz algebras. \emph{Computing Center: University of Bucharest}: (preprint--unpublished), pp.311-324.

[38]
Moisil  Gr C (1969) \emph{The Algebraic Theory of Switching Circuits} (orig. in Romanian). Editura Tehnica Bucuresti. (English translation in1969),
Pergamon Press, Oxford and Editura Tehnica, Bucuresti

[39]
 Moisil  Gr C (1972) \emph{Essais sur les logiques non-chrysippiennnes.} Editura Academiei R.S. Romania,  Bucuresti

[40]
Janelidze G, Schumacher D and R. Street R (1991) Galois theory in variable categories. \emph{Applied Categorical Structures}, vol \textbf{1}, No 1: 103-110, DOI: 10.1007/BF00872989.

[41] 
Mac Lane S, and Moerdijk, I (1992) Sheaves in geometry and logic: A first introduction to topos theory, in Prologue, on p.1.  Springer, Berlin

[42] 
Eytan M (1981) \emph{Fuzzy sets and Systems.}  Elsevier Publs, London and New York

[43]
Stout  L N (1984) Topoi and categories of fuzzy sets.  \emph{Fuzzy Sets and Systems}, vol \textbf{12}, Issue 2, pp 169--184,
Elsevier Publs,  London and New York

[44]
Baianu I C (2004) Quantum Nano-Automata (QNA): Microphysical Measurements with Microphysical QNA Instruments. CERN Preprint \emph{EXT-2004-125}.

[45]
Baianu I C, Glazebrook J F, and Georgescu G (2004) Categories of Quantum Automata and N-Valued Lukasiewicz Algebras in Relation to Dynamic Bionetworks,
$(M,R)$-Systems and Their Higher Dimensional Algebra.  \emph{Abstract and Preprint of Report}:\\
http://www.medicalupapers.com/quantum+automata+math+categories+baianu/


\end{document}


%%%%%
\end{document}
