\documentclass[12pt]{article}
\usepackage{pmmeta}
\pmcanonicalname{UncountableSumsOfPositiveNumbers}
\pmcreated{2013-03-22 15:44:47}
\pmmodified{2013-03-22 15:44:47}
\pmowner{rspuzio}{6075}
\pmmodifier{rspuzio}{6075}
\pmtitle{uncountable sums of positive numbers}
\pmrecord{5}{37698}
\pmprivacy{1}
\pmauthor{rspuzio}{6075}
\pmtype{Definition}
\pmcomment{trigger rebuild}
\pmclassification{msc}{40-00}
\pmrelated{SupportOfIntegrableFunctionWithRespectToCountingMeasureIsCountable}

\endmetadata

% this is the default PlanetMath preamble.  as your knowledge
% of TeX increases, you will probably want to edit this, but
% it should be fine as is for beginners.

% almost certainly you want these
\usepackage{amssymb}
\usepackage{amsmath}
\usepackage{amsfonts}

% used for TeXing text within eps files
%\usepackage{psfrag}
% need this for including graphics (\includegraphics)
%\usepackage{graphicx}
% for neatly defining theorems and propositions
%\usepackage{amsthm}
% making logically defined graphics
%%%\usepackage{xypic}

% there are many more packages, add them here as you need them

% define commands here
\begin{document}
The notion of sum of a series can be generalized to sums of nonnegative
real numbers over arbitrary index sets.

let $I$ be a set and let $c$ be a mapping from $I$ to the nonnegative
real numbers.  Then we may define the sum as follows:
 \[\sum_{i \in I} c_i = \sup_{s \subset I \atop \#s < \infty} \sum_{i
 \in s} c_i\]
In words, we are taking the supremum over all sums over finite subsets
of the index set.  This agrees with the usual notion of sum when our
set is countably infinite, but generalizes this notion to uncountable
index sets.

An important fact about this generalization is that the sum can only
be finite if the number of elements $i \in I$ such that $c_i > 0$ is
countable.  To demonstrate this fact, define the sets $s_n$ (where n
is a nonnegative integer) as follows:
 \[ s_0 = \{ i \in I \mid c_i \ge 1 \}\]
when $n>0$,
 \[ s_n = \{ i \in I \mid 1/n > c_i \ge 1/(n+1) \}\]
If any of these sets is infinite, then the sum will diverge so, for
the sum to be finite, all these sets must be finite.  However, if
these sets are all finite, then their union is countable.  In other
words, the number of indices for which $c_i > 0$ will be countable.

This notion finds use in places such as non-separable Hilbert spaces.
For instance, given a vector in such a space and a complete
orthonormal set, one can express the norm of the vector as the sum of
the squares of its components using this definition even when the
orthonormal set is uncountably infinite.

This discussion can also be phrased in terms of Lesbegue integration 
with respect to counting measure.  For this point of view, please see the entry
support of integrable function with respect to counting measure is countable.
%%%%%
%%%%%
\end{document}
