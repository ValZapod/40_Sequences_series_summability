\documentclass[12pt]{article}
\usepackage{pmmeta}
\pmcanonicalname{DerivationOfFourierCoefficients1}
\pmcreated{2013-03-11 19:30:41}
\pmmodified{2013-03-11 19:30:41}
\pmowner{swapnizzle}{13346}
\pmmodifier{}{0}
\pmtitle{Derivation of Fourier Coefficients}
\pmrecord{1}{50099}
\pmprivacy{1}
\pmauthor{swapnizzle}{0}
\pmtype{Definition}

%none for now
\begin{document}
\documentclass[11pt]{article}
\usepackage{amssymb}
\usepackage{amsmath}
\usepackage{amsthm}
\usepackage{amsfonts}
\usepackage{array}
\usepackage{txfonts}
\usepackage[mathcal]{eucal}
\usepackage{xy}
\textheight 9in
\textwidth 7in
\oddsidemargin 0in
\evensidemargin 0in
\topmargin 0in
\headheight 0in
\headsep 0in
\title{Derivation of Fourier Coefficients}
\author{Swapnil Sunil Jain}
\date{December 28, 2006}
\begin{document}
\maketitle

As you know, any periodic function $f(t)$ can be written as a Fourier series like the following
\begin{eqnarray}
f(t) &=& c_0 + \sum_{n=1}^{\infty} a_n\cos(\omega_n t) + b_n\sin(\omega_n t)
\end{eqnarray}
where $\omega_n = n\omega_0$ and $\omega_0 = \frac{2\pi}{T}$  

In the process to find an explicit expression for the coefficients $c_0, a_n, b_n$ in terms of $f(t)$, we write (1) in a slightly different way as the following
\begin{align}
f(t) =& \; c_0 + a_1\cos(\omega_1 t) + a_2\cos(\omega_2 t) + ... + a_k\cos(\omega_k t) + ...\nonumber \\
& \; + b_1\sin(\omega_1 t) + b_2\sin(\omega_2 t) + ... + b_k\sin(\omega_k t) + ...
\end{align}
where $k$ is a positive integer.


In order to derive the coefficient $c_0$, we take the integral of both sides of (2) over one period. 
\begin{align*}
\int_{\tau} f(t) dt =& \; \int_{\tau} c_0 dt + \int_{\tau} a_1\cos(\omega_1 t) dt + \int_{\tau} a_2\cos(\omega_2 t) dt + ... + \int_{\tau} a_k\cos(\omega_k t) dt + ...\\
& \; + \int_{\tau} b_1\sin(\omega_1 t) dt + \int_{\tau} b_2\sin(\omega_2 t) dt + ... + \int_{\tau} b_k\sin(\omega_k t) dt + ...
\end{align*}
where $\tau = [t_0, t_0 + T]$. After evaluating the above equation, all the integrals on the right side with a sine or a cosine term drop out (since the integral of a sine or cosine over one period is zero) and we get
\begin{align*}
\int_{\tau} f(t) dt =& \; c_0 \int_{\tau} dt \\
\Rightarrow  \int_{\tau} f(t) dt =& \; c_0 (T) \\
\Rightarrow c_0 =& \; \frac{1}{T} \int_{\tau} f(t)dt = \frac{1}{T} \int_{t_0}^{t_0 + T} f(t)dt \\
\end{align*}

Now, in order to find $a_k$, we multiply both sides of (2) by $\cos(\omega_k t)$ and we arrive at 
\begin{align*}
f(t)\cos(\omega_k t) =& \; c_0\cos(\omega_k t) + a_1\cos(\omega_1 t)\cos(\omega_k t) + a_2\cos(\omega_2 t)\cos(\omega_k t) + ... + a_k{\cos}^2(\omega_k t) + ...\\
& \; + b_1\sin(\omega_1 t)\cos(\omega_k t) + b_2\sin(\omega_2 t)\cos(\omega_k t) + ... + b_k\sin(\omega_k t)\cos(\omega_k t) + ...
\end{align*}
Then we take the integral of both sides of the above equation over one period and we get
\begin{align*}
\int_{\tau} f(t)\cos(\omega_k t)dt =& \; \int_{\tau} c_0\cos(\omega_k t)dt + \int_{\tau} a_1\cos(\omega_1 t)\cos(\omega_k t)dt + \int_{\tau} a_2\cos(\omega_2 t)\cos(\omega_k t)dt + ... \\
& \; + \int_{\tau} a_k{\cos}^2(\omega_k t)dt + ... + \int_{\tau} b_1\sin(\omega_1 t)\cos(\omega_k t)dt + \int_{\tau} b_2\sin(\omega_2 t)\cos(\omega_k t)dt + ... \\
& \; + \int_{\tau} b_k\sin(\omega_k t)\cos(\omega_k t)dt + ... 
\end{align*}
By using orthogonality relationships or by literally evaluating the above integrals, we get the following
\begin{align*}
\int_{\tau} f(t)\cos(\omega_k t)dt =& \; \int_{\tau} a_k {\cos}^2(\omega_k t)dt \\
\Rightarrow \int_{\tau} f(t)\cos(\omega_k t)dt =& \; a_k (\frac{T}{2}) \\
\Rightarrow a_k =& \; \frac{2}{T}\int_{\tau} f(t)\cos(\omega_k t)dt  = \frac{2}{T}\int_{t_0}^{t_0 + T} f(t)\cos(\omega_k t)dt \\
\end{align*}

Now, the process of finding $b_k$ is similar. We multiply both sides of (2) by $\sin(\omega_k t)$ and we get 
\begin{align*}
f(t)\sin(\omega_k t) =& \; c_0\cos(\omega_k t) + a_1\cos(\omega_1 t)\sin(\omega_k t) + a_2\cos(\omega_2 t)\sin(\omega_k t) + ... + a_k\cos(\omega_k t)\sin(\omega_k t) + ...\\
& \; + b_1\sin(\omega_1 t)\sin(\omega_k t) + b_2\sin(\omega_2 t)\sin(\omega_k t) + ... + b_k{\sin}^2(\omega_k t) + ...
\end{align*}
Then we take the integral of both sides of the above equation over one period and we arrive at
\begin{align*}
\int_{\tau} f(t)\sin(\omega_k t)dt =& \; \int_{\tau} c_0\cos(\omega_k t)dt + \int_{\tau} a_1\cos(\omega_1 t)\cos(\omega_k t)dt + \int_{\tau} a_2\cos(\omega_2 t)\cos(\omega_k t)dt + ... \\
& \; + \int_{\tau} a_k\cos(\omega_k t)\sin(\omega_k t)dt + ... + \int_{\tau} b_1\sin(\omega_1 t)\cos(\omega_k t)dt + \int_{\tau} b_2\sin(\omega_2 t)\cos(\omega_k t)dt + ... \\
& \; + \int_{\tau} b_k{\sin}^2(\omega_k t)dt + ...
\end{align*}
By using orthogonality relationships or by literally evaluating the above integrals, we get the following
\begin{align*}
\int_{\tau} f(t)\sin(\omega_k t)dt =& \; \int_{\tau} b_k {\sin}^2(\omega_k t)dt \\
\Rightarrow \int_{\tau} f(t)\sin(\omega_k t)dt =& \; b_k (\frac{T}{2}) \\
\Rightarrow b_k =& \; \frac{2}{T}\int_{\tau} f(t)\sin(\omega_k t)dt = \frac{2}{T}\int_{t_0}^{t_0 + T} f(t)\sin(\omega_k t)dt \\
\end{align*}
 
\end{document}
%%%%%
\end{document}
