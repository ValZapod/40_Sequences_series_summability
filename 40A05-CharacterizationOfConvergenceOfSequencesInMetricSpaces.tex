\documentclass[12pt]{article}
\usepackage{pmmeta}
\pmcanonicalname{CharacterizationOfConvergenceOfSequencesInMetricSpaces}
\pmcreated{2013-03-22 15:06:10}
\pmmodified{2013-03-22 15:06:10}
\pmowner{gumau}{3545}
\pmmodifier{gumau}{3545}
\pmtitle{characterization of convergence of sequences in metric spaces}
\pmrecord{4}{36832}
\pmprivacy{1}
\pmauthor{gumau}{3545}
\pmtype{Theorem}
\pmcomment{trigger rebuild}
\pmclassification{msc}{40A05}
\pmclassification{msc}{54E35}

\endmetadata

% this is the default PlanetMath preamble.  as your knowledge
% of TeX increases, you will probably want to edit this, but
% it should be fine as is for beginners.

% almost certainly you want these
\usepackage{amssymb}
\usepackage{amsmath}
\usepackage{amsfonts}

% used for TeXing text within eps files
%\usepackage{psfrag}
% need this for including graphics (\includegraphics)
%\usepackage{graphicx}
% for neatly defining theorems and propositions
%\usepackage{amsthm}
% making logically defined graphics
%%%\usepackage{xypic}

% there are many more packages, add them here as you need them

% define commands here
\begin{document}
Let $(M, d)$ be a metric space, and let $ \mathbb{N} = P_{1} \bigcup \cdots \bigcup P_{k}$ be a partition of the set of natural numbers such that $P_{i}$ is infinite for every $i$, that is, there is a bijection $f_{i} \colon \mathbb{N} \to P_{i}$. Then, given a sequence $(x_n)_{n \in \mathbb{N}}$, it converges to $x \in M$ if and only if the subsequence $$(x_{f_i(n)})_n$$ converges to $x$ for every $i = 1, \cdots, k$.


\textbf{Examples}

If you have a sequence $(x_n)_n$ and a natural number $k$, and you know that it converges to $x$ for every corresponding subsequence over the classes of remainders modulo $k$, then it converges to $x$.
%%%%%
%%%%%
\end{document}
