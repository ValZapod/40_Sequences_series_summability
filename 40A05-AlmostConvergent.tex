\documentclass[12pt]{article}
\usepackage{pmmeta}
\pmcanonicalname{AlmostConvergent}
\pmcreated{2013-03-22 15:29:51}
\pmmodified{2013-03-22 15:29:51}
\pmowner{kompik}{10588}
\pmmodifier{kompik}{10588}
\pmtitle{almost convergent}
\pmrecord{12}{37356}
\pmprivacy{1}
\pmauthor{kompik}{10588}
\pmtype{Definition}
\pmcomment{trigger rebuild}
\pmclassification{msc}{40A05}
\pmclassification{msc}{40C99}
\pmrelated{Banachlimit}
\pmdefines{almost convergent}

\endmetadata

% this is the default PlanetMath preamble.  as your knowledge
% of TeX increases, you will probably want to edit this, but
% it should be fine as is for beginners.

% almost certainly you want these
\usepackage{amssymb}
\usepackage{amsmath}
\usepackage{amsfonts}

% used for TeXing text within eps files
%\usepackage{psfrag}
% need this for including graphics (\includegraphics)
%\usepackage{graphicx}
% for neatly defining theorems and propositions
%\usepackage{amsthm}
% making logically defined graphics
%%%\usepackage{xypic}

% there are many more packages, add them here as you need them

% define commands here
\begin{document}
A real sequence $(x_n)$ is said to be \textbf{almost convergent} to $L$ if each Banach limit assigns
the same value $L$ to the sequence $(x_n)$.

Lorentz \cite{lorentz} proved that $(x_n)$ is almost convergent to $L$ if and only if
$$\lim\limits_{p\to\infty} \frac{x_{n}+\ldots+x_{n+p-1}}p=L$$
uniformly in $n$.

The above limit can be rewritten in detail as
$$(\forall \varepsilon>0) (\exists p_0) (\forall p>p_0) (\forall n) \left|\frac{x_{n}+\ldots+x_{n+p-1}}p-L\right|<\varepsilon.$$

Almost convergence is studied in summability theory. It is an example of a summability method
which cannot be represented as a matrix method.

\begin{thebibliography}{9}
\bibitem{benkal}
G.~Bennett and N.J. Kalton: 
{Consistency theorems for almost convergence.}
\emph{Trans. Amer. Math. Soc.}, 198:23--43, 1974.

\bibitem{boos}
J.~Boos:
\emph{Classical and modern methods in summability}.
Oxford University Press, New York, 2000.

\bibitem{congre}
Jeff Connor and K.-G. Grosse-Erdmann:
{Sequential definitions of continuity for real functions.}
\emph{Rocky Mt. J. Math.}, 33(1):93--121, 2003.

\bibitem{lorentz}
G.~G. Lorentz:
A contribution to the theory of divergent sequences.
\emph{Acta Math.}, 80:167--190, 1948.
\end{thebibliography}
%%%%%
%%%%%
\end{document}
