\documentclass[12pt]{article}
\usepackage{pmmeta}
\pmcanonicalname{ProofOfLeibnizsTheoremusingDirichletsConvergenceTest}
\pmcreated{2013-03-22 13:22:17}
\pmmodified{2013-03-22 13:22:17}
\pmowner{mathcam}{2727}
\pmmodifier{mathcam}{2727}
\pmtitle{proof of Leibniz's theorem (using Dirichlet's convergence test)}
\pmrecord{7}{33900}
\pmprivacy{1}
\pmauthor{mathcam}{2727}
\pmtype{Proof}
\pmcomment{trigger rebuild}
\pmclassification{msc}{40A05}
\pmrelated{AlternatingSeries}

\endmetadata

% this is the default PlanetMath preamble.  as your knowledge
% of TeX increases, you will probably want to edit this, but
% it should be fine as is for beginners.

% almost certainly you want these
\usepackage{amssymb}
\usepackage{amsmath}
\usepackage{amsfonts}

% used for TeXing text within eps files
%\usepackage{psfrag}
% need this for including graphics (\includegraphics)
%\usepackage{graphicx}
% for neatly defining theorems and propositions
%\usepackage{amsthm}
% making logically defined graphics
%%%\usepackage{xypic}

% there are many more packages, add them here as you need them

% define commands here
\begin{document}
\newcommand{\sN}[0]{\mathbb{N}}
\emph{Proof.} 
Let us define the sequence $\alpha_n =(-1)^n$ for
$n\in \sN=\{0,1,2,\ldots\}.$ Then
 \[\sum_{i=0}^n \alpha_i =\left\{
 \begin{array}{ll}
 1 &\mbox{for even} \,n, \\
 0 & \mbox{for odd}\, n,
 \end{array}\right.\]
so the sequence $\sum_{i=0}^n \alpha_i$ is bounded.
By assumption $\{a_n\}_{n=1}^\infty$ is a bounded decreasing 
sequence with limit $0$.
For $n\in \sN$ we set $b_{n}:=a_{n+1}$. 
Using Dirichlet's convergence test, it follows that the series
$\sum_{i=0}^{\infty} \alpha_i b_i$ converges. Since 
$$\sum_{i=0}^{\infty} \alpha_i b_i =\sum_{n=1}^{\infty} (-1)^{n+1} a_n,$$
the claim follows.
$\Box$
%%%%%
%%%%%
\end{document}
