\documentclass[12pt]{article}
\usepackage{pmmeta}
\pmcanonicalname{ThePlanetMathFAQ1}
\pmcreated{2013-05-17 12:22:55}
\pmmodified{2013-05-17 12:22:55}
\pmowner{mathwizard}{128}
\pmmodifier{unlord}{1}
\pmtitle{The PlanetMath FAQ}
\pmrecord{4}{50035}
\pmprivacy{1}
\pmauthor{mathwizard}{1}
\pmtype{Definition}

%none for now
\begin{document}
\documentclass[11pt]{article}
\usepackage{html}

% \Xy-pic logo (copied from xy.doc)
\newcommand*{\Xy}{{\kern-.1em X\kern-.3em\lower.4ex\hbox{Y\kern-.15em}}}
\begin{htmlonly}
\renewcommand*{\Xy}{Xy}
\end{htmlonly}

\pagestyle{empty}

\begin{document}
\tableofcontents

\section{General Questions}
% 
\subsection{What is PlanetMath?}
PlanetMath is a free, collaborative, online mathematics encyclopedia. The stress is on peer review, rigour, openness, pedagogy, real-time content, interlinked content, and community-drivenness.

\subsection{Is PlanetMath anything like Wikipedia (or wiki software in general)?}
Yes. PlanetMath contains a set of interlinked concepts which derive meaning from both their content and their connections.  The interlinking is exposed as hyperlinks.  The ``meaning'' of a node is not just its textual content, but also its position in the network.  

But PlanetMath is different from wikis in some important regards.  First and foremost, the default model of content authorship is that of object ``ownership'', where a single person acts as the gatekeeper to the object's content.  However, you can also make your articles world-editable, or editable only by a specific set of co-authors.

\subsection{Is PlanetMath competing with MathWorld?}
PlanetMath was started in 2001, when MathWorld was taken offline in the course of legal proceedings against its author.   At that time, we assumed that there would be no MathWorld anymore.  The goal was then not to compete with MathWorld, but to make up for its absence with something that wasn't susceptible to the same pitfalls. 

In order to generate a lot of content quickly and efficiently, it was obvious that collaboration was needed, so this made PlanetMath a horse of a different colour right from the start.  Now that MathWorld is back, you have at least two choices for finding mathematical reference material online.   We believe PlanetMath is more attractive to contributors because of our open licensing policy.

\subsection{How do PlanetMath's goals differ from MathWorld's?}
Both PlanetMath and MathWorld have as a goal to be a comprehensive online encyclopedia of mathematics.  PlanetMath is built for collaborative authoring and peer-review.  It is a ``bazaar'' instead of a ``cathedral'' (if you're into \htmladdnormallink{Raymondite}{http://catb.org/~esr/writings/cathedral-bazaar/} terminology).  The guiding philosophy of the site's design is that the community can ``police itself.''

Another important goal of PlanetMath is to be immune from the courtroom, and to provide agreeable intellectual property rights to contributors.  This is achieved through the Creative Commons Attribution-ShareAlike license, which allows contributors to retain rights to their contributions, and allows PlanetMath.org (and others) to retain rights to a copy.

This approach shifts emphasis to be on sharing knowledge.

\subsection{Is there a downloadable version of the PlanetMath encyclopedia?}
There should be.  Historically, a daily snapshot of the website, containing both LaTeX, HTML, and PDF has been available.   Right now the snapshot-generating system is broken.

The content of the site has also been made available as a large PDF file suitable for printing (if you have a lot of paper).  However, this hasn't been attempted in a while.  You can still find copies of the Free Encyclopedia of Mathematics online, however, and new versions will be created when we have time.

\subsection{Do I need to know \TeX/\LaTeX{} to write a PlanetMath entry?}
It's strongly recommended. However, \LaTeX{} is not hard to learn, and we have 
provided an easy way to view the source code of any existing PlanetMath entry
to faciliate learning how to write mathematics in \TeX .  A PlanetMath-specific \LaTeX{} guide can can be found as a ``Site Doc''.

There are also quite a few good online references and tutorials for TeX and LaTeX:
\begin{itemize}
\item \htmladdnormallink{Getting Started With \LaTeX}{http://www.maths.tcd.ie/~dwilkins/LaTeXPrimer/}
\item \htmladdnormallink{Hypertext Help with \LaTeX}{http://www.giss.nasa.gov/latex/}
\item \htmladdnormallink{The Harvard Guide to \TeX}{http://abel.math.harvard.edu/computing/latex/manual/texman.html}
\item \htmladdnormallink{A Guide to \LaTeX}{http://www.astro.rug.nl/~kuijken/latex.html}
\item \htmladdnormallink{The Not So Short Introduction to \LaTeX2e}{http://www.ctan.org/tex-archive/info/lshort/english/lshort.pdf}
\end{itemize}

\subsection{Who can contribute?}
Anyone can make an account and immediately start creating entries.  Of
course, if you can't take constructive criticism, you might want to
reconsider.

\subsection{Who reviews content, and how?}
Anyone who wants to. The chief means of peer review is via the corrections system. Anyone can submit corrections (of type addendum or erratum) to any entry. Points are given for accepted corrections, providing a means of crediting the reviewing party's contribution.

It is up to the author to determine of a correction is worth accepting. However, corrections are always available for viewing. If a correction is either wrongly accepted or rejected, others may notice this and file further corrections.

\subsection{Who owns contributed material?}
The authors of content retain all rights they posses by law.  By joining PlanetMath, they also agree to license to PlanetMath all content they submit to the site.  Under this arrangement, authorship recognition must be preserved, but aside from that, anybody can make digital copies, print versions, or hard copy duplicates.  Anybody can set up their own web site with content from PlanetMath.  Anybody can sell compilations of that content.  For more details, see the Creative Commons Attribution-ShareAlike license.

\subsection{How can I help the project -- financially or otherwise?}
Please see our page on supporting PlanetMath.

\section{Usage Questions}
\subsection{How can I contribute?}
\begin{enumerate}
\item Make an account.
\item Browse around, get a feel for the site.
\item Read the New User \htmladdnormallink{documentation}{http://aux.planetmath.org/doc/}.
\item Log in, click on ``Encyclopedia'' under ``add'', in your user box.
\item Write your entry. Have a \LaTeX{} guide handy, if you don't already know \LaTeX{}.
\item Preview, submit. 
\end{enumerate}

\subsection{How do I know what content has already been added?}
You don't really need to, other than to avoid directly duplicating a concept someone else has already written an entry for.  However, this can be avoided by simply using the search engine. In addition, the system will provide a warning at entry-adding time if the title of your entry is suspiciously similar to the title of an existing entry.

Aside from this, many people are worried about how they can possibly hyperlink their entries to others without knowing which things have been defined in PlanetMath already. Luckily you don't have to know this at all -- PlanetMath is designed to do the reference linking between entries automatically, and instantly.

This works both ways: not only will your entries immediately link to existing PlanetMath entries, but when new entries are added for concepts mentioned in your entry, your entry will actually be updated to link to them. The end result is that each person can be ignorant of the contents of the actual corpus (up to intent to add new entries.)

For more information about the automatic reference linking, see the expository document on \htmladdnormallink{Automatic Reference Linking}{http://planetmath.org/planetmathautomaticreferencelinking1} and the \htmladdnormallink{User Linking Controls}{http://planetmath.org/controllinglinking1} guide.

The reason this system works is that the particular content of PlanetMath is not novel -- it's all concepts that have already been thought of, that are well-known, and that are known by consistent handles. In general semantic net frameworks where new knowledge is formulated, there is less of a need for automatic linking, and this is less possible since concepts don't have well-known handles. Hence this isn't a big priority for ``brain-storming'' type Wiki-systems.

\subsection{How do I view the \TeX{} source of an entry?}
Use the Source tab that comes with every article.

\subsection{How do I include diagrams in my entry?}
Note that this question is \emph{not} the same as ``How do I include pictures in my entry?'' For our purposes, diagrams are logical figures, and are (or should be) scale and resolution-invariant.

This means that for diagrams, it is preferable to utilize a logical, description-based format, instead of a raster image (array of values representing pixels.)

For PlanetMath, the most useful of these formats are \Xy-pic and EPS (encapsulated postscript).

\Xy-pic is a language for describing figures directly within your TeX markup. This means you don't need to include separate files for images. \Xy-pic excels at simple geometric diagrams and array-based diagrams with arrows and lines of various styles between elements. You can see an example of \Xy-pic usage (and the source) on PlanetMath \htmladdnormallink{here}{http://planetmath.org/?op=getobj;from=objects;id=2865}.

EPS is a variant of the postscript language which is understood by \LaTeX. EPS files are separate from \TeX{} source and are included via the \texttt{\textbackslash{}includegraphics} directive. EPS is almost never generated by hand. Under unix, a good program for generating EPS files is \htmladdnormallink{xfig}{http://www.xfig.org/}. In windows, try \htmladdnormallink{Mayura Draw}{http://mayura.com} (both programs are free). \htmladdnormallink{Here}{http://planetmath.org/encyclopedia/BridgesOfKoenigsberg.html} and \htmladdnormallink{here}{http://planetmath.org/?op=getobj;from=objects;id=1452} are examples of figures generated by the EPS method in xfig and Mayura draw, respectively.

You can upload graphics files to the gallery, and then include them by writing \verb|\includegraphics{FILENAME}|.

\subsection{I am having trouble reading an entry because the \TeX{} is not displaying correctly.  What do I do?}
Post a comment and we'll look into it.

\end{document}
%%%%%

\end{document}
