\documentclass[12pt]{article}
\usepackage{pmmeta}
\pmcanonicalname{ZeroSequence}
\pmcreated{2015-07-10 21:03:45}
\pmmodified{2015-07-10 21:03:45}
\pmowner{pahio}{2872}
\pmmodifier{pahio}{2872}
\pmtitle{zero sequence}
\pmrecord{11}{41934}
\pmprivacy{1}
\pmauthor{pahio}{2872}
\pmtype{Definition}
\pmcomment{trigger rebuild}
\pmclassification{msc}{40A05}
\pmsynonym{null sequence}{ZeroSequence}

% this is the default PlanetMath preamble.  as your knowledge
% of TeX increases, you will probably want to edit this, but
% it should be fine as is for beginners.

% almost certainly you want these
\usepackage{amssymb}
\usepackage{amsmath}
\usepackage{amsfonts}

% used for TeXing text within eps files
%\usepackage{psfrag}
% need this for including graphics (\includegraphics)
%\usepackage{graphicx}
% for neatly defining theorems and propositions
 \usepackage{amsthm}
% making logically defined graphics
%%%\usepackage{xypic}

% there are many more packages, add them here as you need them

% define commands here

\theoremstyle{definition}
\newtheorem*{thmplain}{Theorem}

\begin{document}
Let a field $k$ be equipped with a rank one valuation $|.|$.\, A sequence 
\begin{align}
\langle a_1,\,a_2,\,\ldots \rangle
\end{align}
of elements of $k$ is called a \emph{zero sequence} or a \emph{null sequence}, if\, $\displaystyle\lim_{n\to\infty}a_n = 0$\,
in the metric induced by $|.|$.\\

If $k$ together with the metric induced by its valuation $|.|$ is a 
complete ultrametric field, it's clear that its sequence
(1) has a limit (in $k$) as soon as the sequence
$$\langle a_2\!-a_1,\, a_3\!-\!a_2,\,a_4\!-\!a_3,\,\ldots \rangle$$
is a zero sequence.\\

If $k$ is not complete with respect to its valuation $|.|$, its 
\PMlinkname{completion}{Completion} can be made as follows.\, The 
Cauchy sequences (1) form an integral domain $D$ when the 
operations ``$+$'' and ``$\cdot$'' are defined componentwise.\, The 
subset $P$ of $D$ formed by the zero sequences is a 
maximal ideal, whence the quotient ring $D/P$ is a field 
$K$.\, Moreover, $k$ may be isomorphically embedded into $K$ and 
the valuation $|.|$ may be uniquely extended to a valuation of 
$K$.\, The field $K$ then is complete with respect to $|.|$ and $k$ 
is dense in $K$.

%%%%%
%%%%%
\end{document}
