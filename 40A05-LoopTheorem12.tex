\documentclass[12pt]{article}
\usepackage{pmmeta}
\pmcanonicalname{LoopTheorem12}
\pmcreated{2013-03-11 19:24:08}
\pmmodified{2013-03-11 19:24:08}
\pmowner{juanman}{12619}
\pmmodifier{}{0}
\pmtitle{Loop theorem}
\pmrecord{1}{50065}
\pmprivacy{1}
\pmauthor{juanman}{0}
\pmtype{Definition}

\endmetadata

%none for now
\begin{document}
\documentclass[12pt]{article}
\usepackage[pctex32]{graphics}
\setlength{\parindent}{0.20in}
\setlength{\parskip}{12pt}


%\renewcommand{\baselinestretch}{1.25} 

\textwidth=17.0truecm
%\textheight=22.0truecm
%\hoffset=-1.3cm
%\voffset=-2.0cm


\newcommand{\modu}[1] {\left|\begin{array}{c} #1 \end{array}\right| }
\newcommand{\paren}[1]{\left(\begin{array}{c} #1 \end{array}\right) }
\newcommand{\llav}[1] {\left\{\begin{array}{c} #1 \end{array}\right\} }
\newcommand{\corch}[1]{\left[\begin{array}{c} #1 \end{array}\right] }

\DeclareSymbolFont{AMSb}{U}{msb}{m}{n}
\DeclareSymbolFontAlphabet{\Bbb}{AMSb}
%\setcounter{page}{0}

\begin{document}
{THE LOOP THEOREM}

MSC 57M35

In the topology of 3-manifolds, {\bf the loop theorem} is generalization of an ansatz discovered by Max Dehn (The Dehn's Lemma), 
who saw that {\em if a continuous map from a 2-disk to a 3-manifold whose restriction to the boundary's disk has no singularities, 
then it exists another embedding whose restriction to the boundary's disk is equal to the boundary's restriction original map}.

The following statement called the Loop Theorem is a version from J. Stalling, but written in W. Jaco's book.

{\em Let $M$ be a three-manifold and let $S$
 be a connected surface in $\partial M $. Let $N\subset \pi_(M)$ be a normal subgroup.
Let  $f \colon D^2\to M $
be a {\bf continuous map} such that $f(\partial D^2)\subset S$
and $[f|\partial D^2]\notin N$.\\
Then there exists an {\bf embedding} 
$g\colon D^2\to M $ such that 
$g(\partial D^2)\subset S$
and 
$[g|\partial D^2]\notin N$},

The proof is a clever construction due to C. Papakyriakopoulos about a sequence (a tower) of covering spaces.
Maybe the best detailed redaction is due to A. Hatcher.
But in general, accordingly to  Jaco's opinion, {\it ''... for anyone unfamiliar with the techniques of 3-manifold-topology and are here to gain a working knowledge  for the study of problems in this area..., there is no better place to start.''}



==References==

W. Jaco, {\it Lectures on 3-manifolds topology}, A.M.S. regional conference series in Math 43.

J. Hempel, {\it 3-manifolds}, Princeton University Press 1976.


A. Hatcher, {\it Notes on 3-manifolds},

\end{document}
%%%%%
\end{document}
