\documentclass[12pt]{article}
\usepackage{pmmeta}
\pmcanonicalname{PlanetMathCommunityGuidelines1}
\pmcreated{2013-03-11 19:35:49}
\pmmodified{2013-03-11 19:35:49}
\pmowner{rspuzio}{6075}
\pmmodifier{}{0}
\pmtitle{PlanetMath Community Guidelines}
\pmrecord{1}{50137}
\pmprivacy{1}
\pmauthor{rspuzio}{0}
\pmtype{Definition}

%none for now
\begin{document}
\documentclass{article}
\usepackage{hyperref}
\usepackage{ulem}
\begin{document}

\title{PlanetMath Community Guidelines}

\date{\today}

\maketitle

\tableofcontents

This guide spells out some of the ground rules on how users should
interact with each other on PlanetMath, whether in the PM Forums, or via PM Encyclopedia contributions, and points to some of the possible consequences if these rules are broken. In a nutshell, the ground rules ask that the users be respectful and considerate of one another. More specifically, the ground rules contain the following:

\section{With regard to posting on PlanetMath}
\begin{enumerate}
\item Try to stay on the subject of mathematics, or things related to PlanetMath.
\item When referring to other individuals (persons or entities), be respectful and complimentary.
\item If you don't have anything good to say, sometimes the best course of action is to say nothing.
\item If you do have disagreements and want to discuss it in public, do it in a constructive manner. Try not to make judgment or use sarcasm in any negative way. Do not make unfounded accusatory comments. Avoid starting a flame war.
\end{enumerate}

\section{With regard to a filing a correction notice to a PlanetMath entry}
\begin{enumerate}
\item If you have an issue with an entry, file a correction notice first. Explain why you have an issue, and point out any possible course of action for a solution if you know one. Be polite and to the point.
\item If the correction is not done to your satisfaction, you may consider filing another correction notice.
\item Try to resolve the issue peacefully with the author of the entry. You may contact the Content Committee if you are not able to resolve the issue.
\item Or, you may consult the rest of the PlanetMath users for additional input. Do so in a constructive manner.
\end{enumerate}

\section{With regard to writing and maintaining a PlanetMath entry}
\begin{enumerate}
\item If you are knowledgeable about some math concept and do not find it on PlanetMath, or if you know an alternative way of describing a concept (or proving a proposition) that already appears on PlanetMath, feel free to contribute an article/entry on it.
\item On the other hand, if you are not knowledgeable about the concept, and still like to see it on PlanetMath, file an entry request so others may contribute.
\item Also, consider collaborating with others if the task of writing an entry alone is too great. Make the entry world-editable if necessary.
%\item Please do not copy material from other sources, even if such an action is legal; if such practice is discovered, PlanetMath administration may take action to either confiscate or remove the offending entries, and/or issue warnings to offending authors.
\item If you would like to fill a pending request or adopt an orphaned/abandoned entry and feel that you do not have adequate knowledge regarding the subject matter, please consult the PM public or the requester for more information first, or
\item Better yet, be considerate and leave the requests and orphaned/abandoned entries alone so others who are more knowledgeable can take care of the request/orphaned entry.
\item Consider giving up an entry to others either by transferring, orphaning/abandoning, or turning it world-editable, if the task of owning an entry is too great, or there is no time at the moment to take care of filed corrections.
\item Please take care of outstanding correction notices in a timely manner. The PM system will issue nagging email if an outstanding correction notice has not been taken care of for an extended period of time, and will orphan the entry that the correction notice is attached to if no action is taken.
\item Please do not keep entries with outstanding correction notices by transferring them back and forth between friends. If such a practice is discovered, PM administration will confiscate the entries, and issue warnings to offenders.
\end{enumerate}

\section{With regard to rating of entries}
\begin{enumerate}
\item Please be as objective as you can, including not rating your own entries, and not rating an entry that you are not really familiar with.
\item Abuse of the entry rating system may be subject to rating reset (meaning erasure of all previous ratings of the entry in question), as well as possibly points being deducted from offending raters. Abusive behavior includes (but is not limited to) consistently giving a high rating to an entry simply because the author is your friend, giving a low rating to an entry simply because you don't like the author, or any of the behavior stated in item 4.1.
\end{enumerate}

\section{With regard to diversity and disagreement}
\begin{enumerate}
\item The PlanetMath community is a diverse collection of individuals who
share a common interest in mathematics and make use of common
resources to promote this interest.
\item Because of the differences of backgrounds, interests, and abilities
between members of the community, there are bound to arise differences
of opinion. Handled appropriately, these differences can lead to
growth and be a source of strength and vitality for the community.
\item While it is perfectly legitimate to disagree with someone else, this
should be done in a productive manner. Merely stating disagreement is
not likely to accomplish much --- explain the reasons why you do not
agree and be specific as to the points of disagreement. This way
leads to dialog which can resolve the disagreement rather than
fights which typically only entrench both sides in the conviction that
each side is right and the other side is wrong.
\item Do not take differences of opinion personally.
\item One side convincing the other side that it was wrong is not the only
way of resolving a disagreement. It is also possible that each
disputant will come to understand the other's point of view and both
will together come out with a new proposal which combines the good
points of both proposals but is not open to the objections which were
raised.
\item Be tolerant of others who may have different tastes, styles, and
interests. Do not object to something someone else is doing unless it
is causing harm. However, feel free to give and receive criticism as
long as it is done in a courteous, constructive fashion.
\end{enumerate}

\section{With regard to changes and improvements}
\begin{enumerate}
\item If there is something you would like to see happen at PlanetMath, work
towards making it happen. If it is a small thing and you have the
necessary skills, consider doing it yourself as time permits. If it
is a large project or you do not have all the necessary skills, do as
much as you can --- putting effort forth shows that you are serious
about making your proposal happen and therefore may serve to
interest others in participating.
\item While you can certainly try to interest others in what you are doing
and can expect that others should respect your choice of priorities,
no one else has an obligation to collaborate with you since
participation in PlanetMath is voluntary. In order to interest
others in helping you, you might want to consider offering to help
others and finding common ground with what others are doing.
\item If you would like to do something which is going to have at best
minimal impact on how others use the website, feel free to proceed.
\item If you would like to do something which has the potential to
significantly affect how others use the site, contact the affected
parties to discuss your plans and be prepared to modify your plans to
accommodate their needs and involve them in the planning process.
\item If you have an objection to something someone is doing or planning
to do, please raise that objection in a timely fashion. It is
disrespectful to wait until someone has already invested significant
effort towards a project to make known objections which would
have that person making significant changes to plans or undoing work
already done.
\end{enumerate}

\end{document}
%%%%%
\end{document}
