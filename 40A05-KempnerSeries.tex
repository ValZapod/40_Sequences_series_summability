\documentclass[12pt]{article}
\usepackage{pmmeta}
\pmcanonicalname{KempnerSeries}
\pmcreated{2013-03-22 19:12:59}
\pmmodified{2013-03-22 19:12:59}
\pmowner{pahio}{2872}
\pmmodifier{pahio}{2872}
\pmtitle{Kempner series}
\pmrecord{8}{42136}
\pmprivacy{1}
\pmauthor{pahio}{2872}
\pmtype{Result}
\pmcomment{trigger rebuild}
\pmclassification{msc}{40A05}
\pmsynonym{depleted harmonic series}{KempnerSeries}

\endmetadata

% this is the default PlanetMath preamble.  as your knowledge
% of TeX increases, you will probably want to edit this, but
% it should be fine as is for beginners.

% almost certainly you want these
\usepackage{amssymb}
\usepackage{amsmath}
\usepackage{amsfonts}

% used for TeXing text within eps files
%\usepackage{psfrag}
% need this for including graphics (\includegraphics)
%\usepackage{graphicx}
% for neatly defining theorems and propositions
 \usepackage{amsthm}
% making logically defined graphics
%%%\usepackage{xypic}

% there are many more packages, add them here as you need them

% define commands here

\theoremstyle{definition}
\newtheorem*{thmplain}{Theorem}

\begin{document}
\PMlinkescapeword{contain}

The harmonic series
\begin{align}
\sum_{n=1}^\infty\frac{1}{n} \;=\; \frac{1}{1}+\frac{1}{2}+\frac{1}{3}+\frac{1}{4}+\ldots
\end{align}
is divergent.\, The situation is different when one omits from this series all terms whose denominators contain in the \PMlinkname{decimal system}{PositionalNumberSystems} some digits 9.\, Kempner proved 1914 very simply that such a ``\emph{depleted harmonic series}'' is convergent and that its sum is less than 90.\, This series is called \emph{Kempner series}.\, A better value of the sum with 15 decimals is 22.920676619264150.

The digit 9 here has no special status; one has \PMlinkescapetext{similar results} for other digits 
$0,\,1,\,2,\,\ldots,\,8$ and \PMlinkescapetext{even} for digit strings, as ``716''.\, E.g., we show the convergence of the partial series
\begin{align*}
K_0\;
\begin{cases}
=&\!\!\frac{1}{1}+\frac{1}{2}+\frac{1}{3}+\frac{1}{4}+\frac{1}{5}+\frac{1}{6}+\frac{1}{7}+\frac{1}{8}+\frac{1}{9}\\
+&\!\!\frac{1}{11}+\frac{1}{12}+\ldots+\frac{1}{19}+\frac{1}{21}+\ldots\ldots+\frac{1}{99}\\
+&\!\!\frac{1}{111}+\frac{1}{112}+\ldots+\frac{1}{119}+\frac{1}{121}+\ldots\ldots+\frac{1}{999}\\
+&.\;.\;.\;.\;.\;.
\end{cases}
\end{align*}
of (1) where the denominators contain no 0's.\, Every digit in the denominators has nine possibilities.\, For this series we thus get the estimation
$$K_0 \;<\; 9\!\cdot\!\frac{1}{1}+9\!\cdot\!9\cdot\!\frac{1}{10}+9\!\cdot\!9\!\cdot\!9\!\cdot\frac{1}{100}+\ldots 
\;=\; 9+9\!\cdot\!\frac{9}{10}+9\!\cdot\!\left(\frac{9}{10}\right)^{\!2}+\ldots
\;=\; \frac{9}{1\!-\!\frac{9}{10}} \;=\; 90$$
(a sum of convergent geometric series).

For determining more accurately the sums of depleted harmonic series (cf. \PMlinkexternal{depleted uranium}{http://en.wikipedia.org/wiki/Depleted_uranium}), see the article \PMlinkexternal{Summing the curious series of Kempner and Irwin}{http://arxiv.org/ftp/arxiv/papers/0806/0806.4410.pdf} of Baillie.

%%%%%
%%%%%
\end{document}
