\documentclass[12pt]{article}
\usepackage{pmmeta}
\pmcanonicalname{MonotonicallyNondecreasing}
\pmcreated{2013-03-22 12:22:38}
\pmmodified{2013-03-22 12:22:38}
\pmowner{akrowne}{2}
\pmmodifier{akrowne}{2}
\pmtitle{monotonically nondecreasing}
\pmrecord{8}{32136}
\pmprivacy{1}
\pmauthor{akrowne}{2}
\pmtype{Definition}
\pmcomment{trigger rebuild}
\pmclassification{msc}{40-00}
\pmsynonym{monotone nondecreasing}{MonotonicallyNondecreasing}
\pmrelated{MonotonicallyNonincreasing}

\endmetadata

\usepackage{amssymb}
\usepackage{amsmath}
\usepackage{amsfonts}

%\usepackage{psfrag}
%\usepackage{graphicx}
%%%\usepackage{xypic}
\begin{document}
A sequence $(s_n)$ (with real elements)  is called \emph{monotonically nondecreasing} if 

$$ s_m \ge s_n \;\forall\; m > n $$

Similarly, a real function $f(x)$ is called monotonically nondecreasing if 

$$ f(x) \ge f(y) \;\forall\; x > y $$

Compare this to monotonically increasing.

\textbf{Conflict note.}  In other contexts, such as \cite{NIST}, this is called \emph{monotonically increasing} (despite the fact that the sequence could be ``flat.''  In such a context, our definition of ``monotonically increasing'' is called \emph{strictly increasing}.

\begin{thebibliography}{3}
\bibitem{NIST} ``\PMlinkexternal{monotonically increasing}{http://www.nist.gov/dads/HTML/monotoncincr.html},'' from the NIST Dictionary of Algorithms and Data Structures, Paul E. Black, ed.
\end{thebibliography}
%%%%%
%%%%%
\end{document}
