\documentclass[12pt]{article}
\usepackage{pmmeta}
\pmcanonicalname{ProofOfAbelsLimitTheorem}
\pmcreated{2013-03-22 14:09:40}
\pmmodified{2013-03-22 14:09:40}
\pmowner{mathwizard}{128}
\pmmodifier{mathwizard}{128}
\pmtitle{proof of Abel's limit theorem}
\pmrecord{5}{35583}
\pmprivacy{1}
\pmauthor{mathwizard}{128}
\pmtype{Proof}
\pmcomment{trigger rebuild}
\pmclassification{msc}{40A30}
\pmrelated{ProofOfAbelsConvergenceTheorem}

% this is the default PlanetMath preamble.  as your knowledge
% of TeX increases, you will probably want to edit this, but
% it should be fine as is for beginners.

% almost certainly you want these
\usepackage{amssymb}
\usepackage{amsmath}
\usepackage{amsfonts}

% used for TeXing text within eps files
%\usepackage{psfrag}
% need this for including graphics (\includegraphics)
%\usepackage{graphicx}
% for neatly defining theorems and propositions
%\usepackage{amsthm}
% making logically defined graphics
%%%\usepackage{xypic}

% there are many more packages, add them here as you need them

% define commands here
\begin{document}
Without loss of generality we may assume $r=1$, because otherwise we can set $a^\prime_n:=a_r^n$, so that $\sum a^\prime_nx^n$ has radius $1$ and $\sum a^\prime$ is convergent if and only if $\sum a_nr^n$ is.
We now have to show that the function $f(x)$ generated by $\sum a_nx^n$ (with $r=1$)is continuous from below at $x=1$ if it is defined there.
Let $s:=\sum a_n$. We have to show that 
$$\lim_{x\to1^-}f(x)=s.$$
If $|x|<1$ we have:
\begin{align*}
s-f(x)&=\sum_{n=0}^\infty a_n-\sum_{n=0}^\infty a_nx^n\\
&=\sum_{n=0}^\infty(1-x^n)a_n\\
&=(1-x)\sum_{n=1}^\infty(x^{n-1}+x^{n-2}+\dots+x+1)a_n\\
&=(1-x)\sum_{n=0}^\infty(s-s_n)x^n
\end{align*}
with $s_n:=\sum_{i=0}^na_i$. Now, since $s-s_n\to0$ as $n\to\infty$ we can choose an $N$ for every $\varepsilon>0$ such that $|s-s_n|<\frac{\varepsilon}{2}$ for all $m>N$. So for every $0<x<1$ we have:
\begin{align*}
|s-f(x)|&<(1-x)\sum_{n=0}^m|r_n|x^n+\frac{\varepsilon}{2}(1-x)\sum_{n=m+1}^\infty x^n\\
&<(1-x)\sum_{n=0}^m|r_n|+\frac{\varepsilon}{2}.
\end{align*}
This is smaller than $\varepsilon$ for all $x<1$ sufficiently close to $1$, which proves 
$$\lim_{x\to r^-}\sum a_nx^n=\sum a_nr^n=\sum\lim_{x\to r^-}a_nx^n.$$
%%%%%
%%%%%
\end{document}
