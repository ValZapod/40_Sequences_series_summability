\documentclass[12pt]{article}
\usepackage{pmmeta}
\pmcanonicalname{ProofThatEIsNotANaturalNumber}
\pmcreated{2013-03-22 15:39:52}
\pmmodified{2013-03-22 15:39:52}
\pmowner{CWoo}{3771}
\pmmodifier{CWoo}{3771}
\pmtitle{proof that e is not a natural number}
\pmrecord{10}{37598}
\pmprivacy{1}
\pmauthor{CWoo}{3771}
\pmtype{Proof}
\pmcomment{trigger rebuild}
\pmclassification{msc}{40A25}
\pmclassification{msc}{40A05}
\pmclassification{msc}{11J72}
\pmrelated{EIsTranscendental}

\endmetadata

\usepackage{amssymb,amscd}
\usepackage{amsmath}
\usepackage{amsfonts}

% used for TeXing text within eps files
%\usepackage{psfrag}
% need this for including graphics (\includegraphics)
%\usepackage{graphicx}
% for neatly defining theorems and propositions
\usepackage{amsthm}
% making logically defined graphics
%%%\usepackage{xypic}

% define commands here
\begin{document}
Here, we are going to show that the natural log base $e$ is not a natural number by showing a sharper result: that $e$ is between $2$ and $3$.

\textbf{Proposition.} $2<e<3$.
\begin{proof}  There are several infinite series representations of $e$.  In this proof, we will use the most common one, the Taylor expansion of $e$:
\begin{eqnarray}
\sum_{i=0}^{\infty}\frac{1}{i!}=\frac{1}{0!}+\frac{1}{1!}+\frac{1}{2!}+\cdots+\frac{1}{n!}+\cdots.
\end{eqnarray}

We chop up the Taylor expansion of $e$ into two parts: the first part $a$ consists of the sum of the first two terms, and the second part $b$ consists of the sum of the rest, or $e-a$.  The proof of the proposition now lies in the estimation of $a$ and $b$.

\textbf{Step 1: e$>$2.}  First, $a=\frac{1}{0!}+\frac{1}{1!}=1+1=2$.  Next, $b>0$, being a sum of the terms in (1), all of which are positive (note also that $b$ must be bounded because (1) is a convergent series).  Therefore, $e=a+b=2+b>2+0=2$.

\textbf{Step 2: e$<$3.}  This step is the same as showing that $b=e-a=e-2<3-2=1$.  With this in mind, let us compare term by term of the series (2) representing $b$ and another series (3):
\begin{eqnarray}
\frac{1}{2!}+\frac{1}{3!}+\cdots+\frac{1}{n!}+\cdots
\end{eqnarray}
and 
\begin{eqnarray}
\frac{1}{2^{2-1}}+\frac{1}{2^{3-1}}+\cdots+\frac{1}{2^{n-1}}+\cdots.
\end{eqnarray}
It is well-known that the second series (a geometric series) sums to 1.  Because both series are convergent, the term-by-term comparisons make sense.  Except for the first term, where $\frac{1}{2!}=\frac{1}{2}=\frac{1}{2^{2-1}}$, we have $\frac{1}{n!}<\frac{1}{2^{n-1}}$ for all other terms.  The inequality $\frac{1}{n!}<\frac{1}{2^{n-1}}$, for $n$ a positive number can be translated into the basic inequality $n!>2^{n-1}$, the proof of which, based on mathematical induction, can be found \PMlinkname{here}{AnExampleOfMathematicalInduction}.  

Because the term comparisons show 
\begin{itemize}
\item that the terms from (2) $\le$ the corresponding terms from (3), and 
\item that at least one term from (2) $<$ than the corresponding term from (3),
\end{itemize}
we conclude that (2) $<$ (3), or that $b<1$.  This concludes the proof.
\end{proof}
%%%%%
%%%%%
\end{document}
