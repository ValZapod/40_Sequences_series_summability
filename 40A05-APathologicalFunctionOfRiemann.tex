\documentclass[12pt]{article}
\usepackage{pmmeta}
\pmcanonicalname{APathologicalFunctionOfRiemann}
\pmcreated{2013-03-22 18:34:17}
\pmmodified{2013-03-22 18:34:17}
\pmowner{pahio}{2872}
\pmmodifier{pahio}{2872}
\pmtitle{a pathological function of Riemann}
\pmrecord{9}{41295}
\pmprivacy{1}
\pmauthor{pahio}{2872}
\pmtype{Example}
\pmcomment{trigger rebuild}
\pmclassification{msc}{40A05}
\pmclassification{msc}{26A15}
\pmclassification{msc}{26A03}
\pmsynonym{example of semicontinuous function}{APathologicalFunctionOfRiemann}
%\pmkeywords{jump discontinuity}
\pmrelated{DirichletsFunction}
\pmrelated{ValueOfTheRiemannZetaFunctionAtS2}

\endmetadata

% this is the default PlanetMath preamble.  as your knowledge
% of TeX increases, you will probably want to edit this, but
% it should be fine as is for beginners.

% almost certainly you want these
\usepackage{amssymb}
\usepackage{amsmath}
\usepackage{amsfonts}

% used for TeXing text within eps files
%\usepackage{psfrag}
% need this for including graphics (\includegraphics)
%\usepackage{graphicx}
% for neatly defining theorems and propositions
 \usepackage{amsthm}
% making logically defined graphics
%%%\usepackage{xypic}

% there are many more packages, add them here as you need them

% define commands here

\theoremstyle{definition}
\newtheorem*{thmplain}{Theorem}

\begin{document}
\PMlinkescapeword{term} \PMlinkescapeword{terms}

The periodic mantissa function \;$t \mapsto t\!-\!\lfloor{t}\rfloor$\; has at each integer value of $t$ a jump (saltus) equal to $-1$, being in these points continuous from the right but not from the left.\, For every real value $t$, one has
\begin{align}
0 \leqq t\!-\!\lfloor{t}\rfloor < 1.
\end{align}
Let us consider the series
\begin{align}
\sum_{n=1}^\infty\frac{nx\!-\!\lfloor{nx}\rfloor}{n^2}
\end{align}
due to Riemann.\, Since by (1), all values of\, $x \in \mathbb{R}$\, and\, $n \in \mathbb{Z}_+$\, satisfy
\begin{align}
0 \;\leqq\; \frac{nx\!-\!\lfloor{nx}\rfloor}{n^2} \;<\; \frac{1}{n^2},
\end{align}
the series is, by Weierstrass' M-test, uniformly convergent on the whole $\mathbb{R}$ (see also the p-test).\, We denote by $S(x)$ the sum function of (2).

The $n^\mathrm{th}$ term of the series (2) defines a periodic function
\begin{align}
x \mapsto \frac{nx\!-\!\lfloor{nx}\rfloor}{n^2}
\end{align}
with the \PMlinkname{period}{PeriodicFunctions} $\frac{1}{n}$ and having especially for\, $0 \leqq x < \frac{1}{n}$\, the value $\frac{x}{n}$.\, The only points of discontinuity of this function are
\begin{align}
x \;=\; \frac{m}{n} \qquad (m = 0,\,\pm1,\,\pm2,\,\ldots),
\end{align}
where it vanishes and where it is continuous from the right but not from the left; in the point (5) this function apparently has the jump\, $\displaystyle-\frac{1}{n^2}$.

The theorem of the entry one-sided continuity by series \, implies that the sum function $S(x)$ is continuous in every irrational point $x$, because the series (2) is uniformly convergent for every $x$ and its terms are continuous for irrational points $x$.

Since the terms (4) of (2) are continuous from the right in the rational points (5), the same theorem implies that 
$S(x)$ is in these points continuous from the right.\, It can be shown that $S(x)$ is in these points discontinuous from the left having the jump equal to\, $\displaystyle-\frac{\pi^2}{6n^2}$.

\begin{thebibliography}{8}
\bibitem{lindelof}{\sc E. Lindel\"of}: {\em Differentiali- ja integralilasku
ja sen sovellutukset III.2}.\, Mercatorin Kirjapaino Osakeyhti\"o, Helsinki (1940).
\end{thebibliography}
%%%%%
%%%%%
\end{document}
