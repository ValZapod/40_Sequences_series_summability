\documentclass[12pt]{article}
\usepackage{pmmeta}
\pmcanonicalname{LimitOfRealNumberSequence}
\pmcreated{2015-01-30 17:53:27}
\pmmodified{2015-01-30 17:53:27}
\pmowner{pahio}{2872}
\pmmodifier{pahio}{2872}
\pmtitle{limit of real number sequence}
\pmrecord{17}{41263}
\pmprivacy{1}
\pmauthor{pahio}{2872}
\pmtype{Definition}
\pmcomment{trigger rebuild}
\pmclassification{msc}{40A05}
\pmsynonym{limit of sequence of real numbers}{LimitOfRealNumberSequence}
%\pmkeywords{real number sequence}
\pmrelated{GeometricSequence}
\pmrelated{BriggsianLogarithms}
\pmrelated{InfiniteProductOfSums1a_i}
\pmdefines{limit}

% this is the default PlanetMath preamble.  as your knowledge
% of TeX increases, you will probably want to edit this, but
% it should be fine as is for beginners.

% almost certainly you want these
\usepackage{amssymb}
\usepackage{amsmath}
\usepackage{amsfonts}

% used for TeXing text within eps files
%\usepackage{psfrag}
% need this for including graphics (\includegraphics)
%\usepackage{graphicx}
% for neatly defining theorems and propositions
 \usepackage{amsthm}
% making logically defined graphics
%%%\usepackage{xypic}

% there are many more packages, add them here as you need them

% define commands here

\theoremstyle{definition}
\newtheorem*{thmplain}{Theorem}

\begin{document}
An endless real number sequence
\begin{align}
a_1,\,a_2,\,a_3,\,\ldots
\end{align}
has the real number $A$ as its {\em limit}, if the distance 
between $A$ and $a_n$ can be made smaller than an arbitrarily 
small positive number $\varepsilon$ by chosing the 
\PMlinkescapetext{ordinal number} $n$ of $a_n$ sufficiently great, i.e. greater than a number $N$ (the \PMlinkescapetext{size} of which depends on the value of $\varepsilon$); accordingly
$$|A-a_n| < \varepsilon \quad \mbox{when} \quad n > N.$$
Then we may denote
\begin{align}
\lim_{n\to\infty}a_n \;=\; A
\end{align}
or equivalently
\begin{align}
a_n \to A \quad \mbox{as} \quad n \to \infty.
\end{align}

\textbf{Remark 1.}\, One should not think, that\, $a_n = A$\, when\, $n = \infty$.\, The symbol ``$\infty$'' \PMlinkescapetext{represents} no number, one cannot set it for the value of $n$.\, It's only a question of allowing $n$ to exceed any necessary 
value.\\

\textbf{Example 1.}\, Using the notation (2) we can write a result
             $$\lim_{n\to\infty}\frac{2n}{n\!+\!1} \;=\; 2.$$
It's a question of that the real number sequence
     $$\frac{2}{2},\;\frac{4}{3},\;\frac{6}{4},\;\ldots$$
has the limit value 2 (e.g. the nine hundred ninety-ninth member\, $\frac{1998}{1000} = 1.998$\, is already ``almost'' 2!).\, For justificating the result, let $\varepsilon$ be an arbitrary positive number, as small as you want.\, Then
\begin{align}
\left|2-\frac{2n}{n\!+\!1}\right| = \left|\frac{2n\!+\!2}{n\!+\!1}-\frac{2n}{n\!+\!1}\right| 
= \left|\frac{2}{n\!+\!1}\right| = \frac{2}{n\!+\!1} < \varepsilon,
\end{align}
when $n$ is chosen so big that
\begin{align}
n > \frac{2}{\varepsilon}\!-\!1.
\end{align}
The condition (5) is obtained from (4) by solving this inequality for $n$.\, In this case, we have\, 
$N = \frac{2}{\varepsilon}\!-\!1$.\\

\textbf{Example 2.}\, The so-called decimal expansions, i.e. endless decimal numbers, such as
\begin{align}
3.14159265\ldots \,=\, \pi, \quad 0.636363\ldots, \quad 0.99999\ldots,
\end{align}
are, as a matter of fact, limits of certain real number sequences.\, E.g. the last of these is related to the sequence
\begin{align}
0.9,\;0.99,\;0.999,\;\ldots
\end{align}
which may be also written as
$$1\!-\!\frac{1}{10},\; 1\!-\!\frac{1}{10^2},\; 1\!-\!\frac{1}{10^3},\;\ldots $$
The limit of (7) is 1.\, Actually, if\, $\varepsilon > 0$,\, the distance between 1 and the $n^\mathrm{th}$ member of (7) is
$$\left|1-\left(1\!-\!\frac{1}{10^n}\right)\right| = \frac{1}{10^n} < \varepsilon,$$
when\, $10^n > \frac{1}{\varepsilon}$,\, i.e. when\, $n > -\log_{10}\varepsilon = N$.

The endless decimal notations (6) and others are, in fact, limit notations --- no finite amount of decimals in them suffices to give their exact values.\\

\textbf{Remark 2.}\, In both of the above examples, no of the sequence members was equal to the limit, but it does not need always to be so; thus for example
$$\lim_{n\to\infty}\frac{1\!+\!(-1)^n}{2n} = 0$$
and every other member of the sequence in question is 0.\\

\subsection*{Infinite limits of real number sequences}

There are sequences that have no limit at all, for example\, $1,\,-1,\,1,\,-1,\,1,\,-1,\,\ldots$.\, Some real number sequences (1) have the property, that the member $a_n$ may exeed every beforehand given real number $M$ if one takes 
$n$ greater than some value $N$ (which depends on $M$):
$$a_n > M \quad \mbox{when} \quad n > N.$$
Then we write
$$\lim_{n\to\infty}a_n \;=\; \infty.$$
Similarly, the sequence (1) may be such that for each positive $M$ there is $N$ such that
$$a_n < -M \quad \mbox{when} \quad n > N,$$
and then we write
$$\lim_{n\to\infty}a_n \;=\; -\infty.$$
E.g.
$$\lim_{n\to\infty}n^2 \;=\; \infty, \quad \lim_{n\to\infty}(1\!-\!n) \;=\; -\infty.$$




%%%%%
%%%%%
\end{document}
