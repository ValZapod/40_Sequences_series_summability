\documentclass[12pt]{article}
\usepackage{pmmeta}
\pmcanonicalname{ProofOfAbelsConvergenceTheorem}
\pmcreated{2013-03-22 13:07:39}
\pmmodified{2013-03-22 13:07:39}
\pmowner{rmilson}{146}
\pmmodifier{rmilson}{146}
\pmtitle{proof of Abel's convergence theorem}
\pmrecord{9}{33562}
\pmprivacy{1}
\pmauthor{rmilson}{146}
\pmtype{Proof}
\pmcomment{trigger rebuild}
\pmclassification{msc}{40G10}
\pmrelated{ProofOfAbelsLimitTheorem}

\endmetadata

\usepackage{amsmath}
\usepackage{amsfonts}
\usepackage{amssymb}
\newcommand{\reals}{\mathbb{R}}
\newcommand{\natnums}{\mathbb{N}}
\newcommand{\cnums}{\mathbb{C}}
\newcommand{\znums}{\mathbb{Z}}
\newcommand{\lp}{\left(}
\newcommand{\rp}{\right)}
\newcommand{\lb}{\left[}
\newcommand{\rb}{\right]}
\newcommand{\supth}{^{\text{th}}}
\newtheorem{proposition}{Proposition}
\newtheorem{definition}[proposition]{Definition}

\newtheorem{theorem}[proposition]{Theorem}
\begin{document}
\PMlinkescapeword{term}
Suppose that
$$\sum_{n=0}^\infty a_n=L$$
is a convergent series, and set
$$f(r) = \sum_{n=0}^\infty a_n r^n.$$
Convergence of the first series implies that $a_n\rightarrow 0$, and
hence $f(r)$ converges for $|r|<1$.  We will show that
$f(r)\rightarrow L$ as $r\rightarrow 1^-$.

Let $$s_N=a_0+\cdots+a_N,\quad N\in\natnums,$$
denote the corresponding partial sums.  Our proof relies on the
following identity
\begin{equation}
  \label{eq:ident}
  f(r)=\sum_n a_n r^n = (1-r) \sum_n s_n r^n.  
\end{equation}
The above identity obviously works at the level of formal power
series.  Indeed,
$$
\begin{array}{crcrcrc}
   &a_0 &+& (a_1+a_0)\, r &+& (a_2+a_1+a_0)\,r^2 &+\, \cdots \\
- \,( &    && a_0\, r &+& (a_1+a_0)\,r^2 &+\,\cdots )\\
=   &a_0 &+& a_1\, r &+& a_2\, r^2 &+\, \cdots
\end{array}
$$
Since the partial sums $s_n$ converge to $L$, they are bounded, and
hence $\sum_n s_n r^n$ converges for $|r|<1$.  Hence for $|r|<1$, identity
\eqref{eq:ident} is also a genuine functional equality.

Let $\epsilon>0$ be given. Choose an $N$ sufficiently large so that
all partial sums, $s_n$ with $n>N$, satisfy $|s_n-L|\le\epsilon$. Then, for all
$r$ such that $0<r<1$, one obtains
$$\left|\sum_{n=N+1}^\infty (s_n - L) r^n\right| \le
\epsilon\,\frac{r^{N+1}}{1-r}\,.$$
Note that
$$f(r) - L = (1-r)\sum_{n=0}^N (s_n - L) r^n
+ (1-r) \sum_{n=N+1}^\infty (s_n - L) r^n.$$
As $r\rightarrow 1^-$, the first term tends to $0$. The absolute value of the
second term is estimated by $\epsilon\, r^{N+1}\le \epsilon$. Hence,
$$\limsup_{r\rightarrow 1^-} |f(r) - L| \le \epsilon.$$ Since $\epsilon>0$ was
arbitrary, it follows that $f(r)\rightarrow L$ as $r\rightarrow 1^-$.
QED

%%%%%
%%%%%
\end{document}
