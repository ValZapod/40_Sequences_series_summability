\documentclass[12pt]{article}
\usepackage{pmmeta}
\pmcanonicalname{ProofOfBaronisTheorem}
\pmcreated{2013-03-22 13:32:33}
\pmmodified{2013-03-22 13:32:33}
\pmowner{mathwizard}{128}
\pmmodifier{mathwizard}{128}
\pmtitle{Proof of Baroni's theorem}
\pmrecord{5}{34142}
\pmprivacy{1}
\pmauthor{mathwizard}{128}
\pmtype{Proof}
\pmcomment{trigger rebuild}
\pmclassification{msc}{40A05}
%\pmkeywords{sequence}
%\pmkeywords{convergence}
%\pmkeywords{limit}

\endmetadata

% this is the default PlanetMath preamble.  as your knowledge
% of TeX increases, you will probably want to edit this, but
% it should be fine as is for beginners.

% almost certainly you want these
\usepackage{amssymb}
\usepackage{amsmath}
\usepackage{amsfonts}

% used for TeXing text within eps files
%\usepackage{psfrag}
% need this for including graphics (\includegraphics)
%\usepackage{graphicx}
% for neatly defining theorems and propositions
%\usepackage{amsthm}
% making logically defined graphics
%%%\usepackage{xypic} 

% there are many more packages, add them here as you need them

% define commands here
\begin{document}
Let $m=\inf A'$ and $M=\sup A'$ . If $m=M$ we are done since the sequence is
convergent and $A'$ is the degenerate interval composed of the point $l \in
\mathbb{\overline{R}}$ , where $\displaystyle l=\lim_{n\rightarrow \infty} x_n$.

Now , assume that $m<M$ . For every $\lambda \in (m,M)$ , we will construct
inductively two subsequences $x_{k_n}$ and $x_{l_n}$ such that $\displaystyle
\lim_{n \rightarrow \infty} x_{k_n} = \lim_{n\rightarrow \infty} x_{l_n} = \lambda$
and $\displaystyle x_{k_n} < \lambda < x_{l_n}$

From the definition of $M$ there is an $N_1 \in \mathbb{N}$ such that :
$$ \lambda < x_{N_1} < M $$

Consider the set of all such values $N_1$ . It is bounded from below (because it
consists only of natural numbers and has at least one element) and thus it has a
smallest element . Let $n_1$ be the smallest such element and from its definition we
have $x_{n_1-1} \leq \lambda < x_{n_1}$ . So , choose $k_1=n_1 - 1$ , $l_1=n_1$ .
Now, there is an $N_2 > k_1$ such that :
$$ \lambda < x_{N_2} < M$$
Consider the set of all such values $N_2$ . It is bounded from below and it has a
smallest element $n_2$ . Choose $k_2 = n_2-1$ and $l_2=n_2$ . Now , proceed by
induction to construct the sequences $k_n$ and $l_n$ in the same fashion . Since
$l_n-k_n=1$ we have :
$$ \lim_{n\rightarrow \infty} x_{k_n} = \lim_{n\rightarrow \infty} x_{l_n}$$ and
thus they are both equal to $\lambda$.
%%%%%
%%%%%
\end{document}
