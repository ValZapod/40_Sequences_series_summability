\documentclass[12pt]{article}
\usepackage{pmmeta}
\pmcanonicalname{ProofOfDivergenceOfHarmonicSeriesbySplittingOddAndEvenTerms}
\pmcreated{2013-03-22 17:38:26}
\pmmodified{2013-03-22 17:38:26}
\pmowner{rspuzio}{6075}
\pmmodifier{rspuzio}{6075}
\pmtitle{proof of divergence of harmonic series (by splitting odd and even terms)}
\pmrecord{4}{40062}
\pmprivacy{1}
\pmauthor{rspuzio}{6075}
\pmtype{Definition}
\pmcomment{trigger rebuild}
\pmclassification{msc}{40A05}

\endmetadata

% this is the default PlanetMath preamble.  as your knowledge
% of TeX increases, you will probably want to edit this, but
% it should be fine as is for beginners.

% almost certainly you want these
\usepackage{amssymb}
\usepackage{amsmath}
\usepackage{amsfonts}

% used for TeXing text within eps files
%\usepackage{psfrag}
% need this for including graphics (\includegraphics)
%\usepackage{graphicx}
% for neatly defining theorems and propositions
%\usepackage{amsthm}
% making logically defined graphics
%%%\usepackage{xypic}

% there are many more packages, add them here as you need them

% define commands here

\begin{document}
Suppose that the series $\sum_{n=1}^\infty 1/n$ converged.  Since all the terms are positive,
we could regroup them as we please, in particular, split the series into two series, that of
even terms and that of odd terms:
\[
 \sum_{n=1}^{\infty} {1 \over n} =
 \sum_{n=1}^{\infty} {1 \over 2n} +
 \sum_{n=1}^{\infty} {1 \over 2n-1}
\]
Since $\sum_{n=1}^\infty 1/n = 2 \sum_{n=1}^\infty 1/(2n)$, we would conclude that
\[
 \sum_{n=1}^{\infty} {1 \over 2n} =
 \sum_{n=1}^{\infty} {1 \over 2n-1} .
\]
But $2n-1 < 2n$, hence $1/(2n) < 1/(2n-1)$, so we would also have
\[
 \sum_{n=1}^{\infty} {1 \over 2n} <
 \sum_{n=1}^{\infty} {1 \over 2n-1} ,
\]
which contradicts the previous conclusion.  Thus, the assumption that the
series converged is untenable.
%%%%%
%%%%%
\end{document}
