\documentclass[12pt]{article}
\usepackage{pmmeta}
\pmcanonicalname{DirichletsConvergenceTest}
\pmcreated{2013-03-22 13:19:53}
\pmmodified{2013-03-22 13:19:53}
\pmowner{lieven}{1075}
\pmmodifier{lieven}{1075}
\pmtitle{Dirichlet's convergence test}
\pmrecord{5}{33844}
\pmprivacy{1}
\pmauthor{lieven}{1075}
\pmtype{Theorem}
\pmcomment{trigger rebuild}
\pmclassification{msc}{40A05}

% this is the default PlanetMath preamble.  as your knowledge
% of TeX increases, you will probably want to edit this, but
% it should be fine as is for beginners.

% almost certainly you want these
\usepackage{amssymb}
\usepackage{amsmath}
\usepackage{amsfonts}

% used for TeXing text within eps files
%\usepackage{psfrag}
% need this for including graphics (\includegraphics)
%\usepackage{graphicx}
% for neatly defining theorems and propositions
%\usepackage{amsthm}
% making logically defined graphics
%%%\usepackage{xypic}

% there are many more packages, add them here as you need them

% define commands here
\begin{document}
Theorem. Let $\{a_n\}$ and $\{b_n\}$ be sequences of real numbers such that $\{\sum_{i=0}^n a_i\}$ is bounded and $\{b_n\}$ decreases with $0$ as limit.
Then $\sum_{n=0}^\infty a_nb_n$ converges.

Proof. Let $A_n:=\sum_{i=0}^n a_n$ and let $M$ be an upper bound for $\{|A_n|\}$. By Abel's lemma, 

\begin{eqnarray*}
\sum_{i=m}^n a_ib_i &=& \sum_{i=0}^n a_ib_i - \sum_{i=0}^{m-1} a_ib_i\\
                    &=& \sum_{i=0}^{n-1} A_i(b_i-b_{i+1}) - \sum_{i=0}^{m-2} A_i(b_i-b_{i+1}) +A_nb_n - A_{m-1}b_{m-1}\\
&=&\sum_{i=m-1}^{n-1}A_i(b_i-b_{i+1}) +A_nb_n -A_{m-1}b_{m-1}\\
|\sum_{i=m}^{n} a_ib_i|&\leq& \sum_{i=m-1}^{n-1}|A_i(b_i-b_{i+1})| + |A_nb_n| + |A_{m-1}b_{m-1}|\\ 
&\leq& M \sum_{i=m-1}^{n-1}(b_i-b_{i+1}) + |A_nb_n| + |A_{m-1}b_{m-1}|\\
\end{eqnarray*}

Since $\{b_n\}$ converges to $0$, there is an $N(\epsilon)$ such that both $\sum_{i=m-1}^{n-1}(b_i-b_{i+1})<\frac{\epsilon}{3M}$ and $b_i<\frac{\epsilon}{3M}$ for $m,n>N(\epsilon)$. Then, for $m,n>N(\epsilon)$, $|\sum_{i=m}^n a_ib_i|<\epsilon$ and $\sum a_nb_n$ converges.
%%%%%
%%%%%
\end{document}
