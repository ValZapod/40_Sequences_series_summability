\documentclass[12pt]{article}
\usepackage{pmmeta}
\pmcanonicalname{AbsoluteConvergenceOfDoubleSeries}
\pmcreated{2013-03-22 18:46:45}
\pmmodified{2013-03-22 18:46:45}
\pmowner{pahio}{2872}
\pmmodifier{pahio}{2872}
\pmtitle{absolute convergence of double series}
\pmrecord{7}{41573}
\pmprivacy{1}
\pmauthor{pahio}{2872}
\pmtype{Definition}
\pmcomment{trigger rebuild}
\pmclassification{msc}{40A05}
\pmrelated{DoubleSeries}
\pmrelated{DiagonalSumming}
\pmdefines{row series}
\pmdefines{column series}
\pmdefines{diagonal series}

\endmetadata

% this is the default PlanetMath preamble.  as your knowledge
% of TeX increases, you will probably want to edit this, but
% it should be fine as is for beginners.

% almost certainly you want these
\usepackage{amssymb}
\usepackage{amsmath}
\usepackage{amsfonts}

% used for TeXing text within eps files
%\usepackage{psfrag}
% need this for including graphics (\includegraphics)
%\usepackage{graphicx}
% for neatly defining theorems and propositions
 \usepackage{amsthm}
% making logically defined graphics
%%%\usepackage{xypic}

% there are many more packages, add them here as you need them

% define commands here

\theoremstyle{definition}
\newtheorem*{thmplain}{Theorem}

\begin{document}
Let us consider the double series
\begin{align}
\sum_{i,j=1}^\infty u_{ij}
\end{align}
of real or complex numbers $u_{ij}$.\, Denote the {\em row series} $u_{k1}\!+\!u_{k2}\!+\ldots$ by $R_k$, the {\em column series} $u_{1k}\!+\!u_{2k}\!+\ldots$ by $C_k$ and the {\em diagonal series} $u_{11}\!+\!u_{12}\!+\!u_{21}\!+\!u_{13}\!+\!u_{22}\!+\!u_{31}\!+\ldots$
by DS.\, Then one has the\\


\textbf{Theorem.}\, All row series, all column series and the diagonal series converge absolutely and
\[
\sum_{k=1}^\infty R_k \;=\; \sum_{k=1}^\infty C_k \;=\; \mbox{DS},
\]
if one of the following conditions is true:
\begin{itemize}
\item The diagonal series converges absolutely. 
\item There exists a positive number $M$ such that every finite sum of the numbers $|u_{ij}|$ is $\leqq M$.
\item The row series $R_k$ converge absolutely and the series $W_1\!+\!W_2\!+\ldots$ with \PMlinkescapetext{terms} 
\[
\sum_{j=1}^\infty|u_{kj}| = W_k
\]
is convergent.\, An analogical condition may be formulated for the column series $C_k$.
\end{itemize}


\textbf{Example.}\, Does the double series
\[
\sum_{m=2}^\infty\sum_{n=3}^\infty n^{-m}
\]
converge?\, If yes, determine its sum.

The column series $\displaystyle\sum_{m=2}^\infty\left(\frac{1}{n}\right)^m$ have positive terms and are absolutely converging geometric series having the sum
\[
\frac{(1/n)^2}{1-1/n} \;=\; \frac{1}{n(n\!-\!1)} \;=\; \frac{1}{n\!-\!1}-\frac{1}{n} \;=\; W_n.
\]
The series $W_3\!+\!W_4\!+\ldots$ is convergent, since its partial sum is a telescoping sum
\[
\sum_{n=3}^N W_n \;=\; \sum_{n=3}^N\left(\frac{1}{n\!-\!1}-\frac{1}{n}\right) \;=\;
\left(\frac{1}{2}-\frac{1}{3}\right)+\left(\frac{1}{3}-\frac{1}{4}\right)+\left(\frac{1}{4}-\frac{1}{5}\right)
+\ldots+\left(\frac{1}{N\!-\!1}-\frac{1}{N}\right)
\]
equalling simply $\frac{1}{2}\!-\!\frac{1}{N}$ and having the limit $\frac{1}{2}$ as\, $N \to \infty$.\, Consequently, the given double series converges and its sum is $\frac{1}{2}$.



%%%%%
%%%%%
\end{document}
