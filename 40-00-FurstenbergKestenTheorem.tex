\documentclass[12pt]{article}
\usepackage{pmmeta}
\pmcanonicalname{FurstenbergKestenTheorem}
\pmcreated{2014-03-19 22:14:18}
\pmmodified{2014-03-19 22:14:18}
\pmowner{Filipe}{28191}
\pmmodifier{Filipe}{28191}
\pmtitle{Furstenberg-Kesten theorem}
\pmrecord{3}{88075}
\pmprivacy{1}
\pmauthor{Filipe}{28191}
\pmtype{Theorem}
\pmrelated{Oseledet's decomposition}
\pmrelated{multiplicative cocycle}

\endmetadata

% this is the default PlanetMath preamble.  as your knowledge
% of TeX increases, you will probably want to edit this, but
% it should be fine as is for beginners.

% almost certainly you want these
\usepackage{amssymb}
\usepackage{amsmath}
\usepackage{amsfonts}

% need this for including graphics (\includegraphics)
\usepackage{graphicx}
% for neatly defining theorems and propositions
\usepackage{amsthm}

% making logically defined graphics
%\usepackage{xypic}
% used for TeXing text within eps files
%\usepackage{psfrag}

% there are many more packages, add them here as you need them

% define commands here

\begin{document}
Consider $\mu$ a probability measure, and $f:M\rightarrow M$ a measure preserving dynamical system. Consider $A:M\rightarrow GL(d,\textbf{R})$, a measurable transformation, where GL(d,\textbf{R}) is the space of invertible square matrices of size $d$.
Consider the multiplicative cocycle $(\phi^n(x))_n$ defined by the transformation $A$.

If $\log^+||A||$ is integrable, where $\log^+||A||=\max\{ \log ||A||,0\}$, then:
$$\lambda_{\max}(x)=\lim_n \frac{1}{n} \log ||\phi^n(x)||$$
exists almost everywhere, and $\lambda^+_{\max}$ is integrable and
$$\int \lambda_{\max} d\mu = \lim_n \frac{1}{n} \int \log ||\phi^n|| d\mu = \inf_n \frac{1}{n} \int \log ||\phi^n||d\mu$$

If $\log^+||A^{-1}||$ is integrable, then:
$$\lambda_{\min}(x)=\lim_n -\frac{1}{n} \log ||\phi^{-n}(x)||$$
exists almost everywhere, and $\lambda^+_{\min}$ is integrable and
$$\int \lambda_{\min} d\mu = \lim_n -\frac{1}{n} \int \log ||\phi^{-n}|| d\mu = \sup_n -\frac{1}{n} \int \log ||\phi^{-n}||d\mu$$

Furthermore, both $\lambda_{\min}$ and $\lambda_{\max}$ are invariant for the tranformation $f$, that is, $\lambda_{\min}\circ f(x)=\lambda_{\min}(x)$ and $\lambda_{\max}\circ f(x)=\lambda_{\max}(x)$, for $\mu$ almost everywhere.

This theorem is a direct consequence of Kingman's subadditive ergodic theorem, by observing that both
$$\log ||\phi^n(x)||$$ and $$\log ||\phi^{-n}(x)||$$ are subadditive sequences.

The results in this theorem are strongly improved by Oseledet's multiplicative ergodic theorem, or Oseledet's decomposition.
\end{document}
