\documentclass[12pt]{article}
\usepackage{pmmeta}
\pmcanonicalname{ProofOfRatioTest}
\pmcreated{2013-03-22 12:24:46}
\pmmodified{2013-03-22 12:24:46}
\pmowner{vitriol}{148}
\pmmodifier{vitriol}{148}
\pmtitle{proof of ratio test}
\pmrecord{6}{32296}
\pmprivacy{1}
\pmauthor{vitriol}{148}
\pmtype{Proof}
\pmcomment{trigger rebuild}
\pmclassification{msc}{40A05}
\pmclassification{msc}{26A06}

% this is the default PlanetMath preamble.  as your knowledge
% of TeX increases, you will probably want to edit this, but
% it should be fine as is for beginners.

% almost certainly you want these
\usepackage{amssymb}
\usepackage{amsmath}
\usepackage{amsfonts}

% used for TeXing text within eps files
%\usepackage{psfrag}
% need this for including graphics (\includegraphics)
%\usepackage{graphicx}
% for neatly defining theorems and propositions
%\usepackage{amsthm}
% making logically defined graphics
%%%\usepackage{xypic}

% there are many more packages, add them here as you need them

% define commands here
\begin{document}
Assume $k < 1$. By definition $\exists N$ such that \newline $n > N \implies |\frac{a_{n+1}}{a_n} - k| < \frac{1-k}{2} \implies |\frac{a_{n+1}}{a_n}| < \frac{1+k}{2} < 1$


i.e. eventually the series $|a_n|$ becomes less than a convergent geometric series, therefore a shifted subsequence of $|a_n|$ converges by the comparison test. Note that a general sequence $b_n$ converges iff a shifted subsequence of $b_n$ converges. Therefore, by the absolute convergence theorem, the series $a_n$ converges.

Similarly for $k > 1$ a shifted subsequence of $|a_n|$ becomes greater than a geometric series tending to $\infty$, and so also tends to $\infty$. Therefore $a_n$ diverges.
%%%%%
%%%%%
\end{document}
