\documentclass[12pt]{article}
\usepackage{pmmeta}
\pmcanonicalname{PrimeTheoremOfAConvergentSequenceA}
\pmcreated{2013-03-22 14:49:45}
\pmmodified{2013-03-22 14:49:45}
\pmowner{georgiosl}{7242}
\pmmodifier{georgiosl}{7242}
\pmtitle{prime theorem of a convergent sequence, a}
\pmrecord{24}{36494}
\pmprivacy{1}
\pmauthor{georgiosl}{7242}
\pmtype{Theorem}
\pmcomment{trigger rebuild}
\pmclassification{msc}{40-00}
\pmrelated{ArithmeticMean}
\pmrelated{GeometricMean}

\endmetadata

% this is the default PlanetMath preamble.  as your knowledge
% of TeX increases, you will probably want to edit this, but
% it should be fine as is for beginners.

% almost certainly you want these
\usepackage{amssymb}
\usepackage{amsmath}
\usepackage{amsfonts}

% used for TeXing text within eps files
%\usepackage{psfrag}
% need this for including graphics (\includegraphics)
%\usepackage{graphicx}
% for neatly defining theorems and propositions
\usepackage{amsthm}
% making logically defined graphics
%%%\usepackage{xypic}

% there are many more packages, add them here as you need them

% define commands here
\newtheorem*{theorem*}{Theorem}
\def\N{\mathbb{N}}
\def\R{\mathbb{R}}
\begin{document}
\PMlinkescapeword{time}

\begin{theorem*}
Suppose $(a_n)$ is a positive real sequence that converges to $L$.  Then 
the sequence of arithmetic means $(b_n)=(n^{-1}\sum_{k=1}^na_k)$ and the sequence of geometric means $(c_n)=(\sqrt[n]{a_1\cdots a_n})$ also converge to $L$.
\end{theorem*}

\begin{proof}
We first show that $(b_n)$ converges to $L$.  Let $\varepsilon>0$.  Select a positive integer $N_0$ such that $n\ge N_0$ implies $|a_n-L|<\varepsilon/2$.  Since $(a_n)$ converges to a finite value, there is a finite $M$ such that 
$|a_n-L|<M$ for all $n$.  Thus we can select a positive integer $N\ge N_0$ for which $(N_0-1)M/N<\varepsilon/2$.  

By the triangle inequality,
\begin{align*}
|b_n-L|
&\le\frac{1}{n}\sum_{k=1}^n|a_k-L| \\
&<\frac{(N_0-1)M}{n}+\frac{(n-N_0+1)\varepsilon}{2n} \\
&<\varepsilon/2+\varepsilon/2.
\end{align*}
Hence $(b_n)$ converges to $L$.

To show that $(c_n)$ converges to $L$, we first define the sequence $(d_n)$ 
by $d_n=c_n^n=a_1\cdots a_n$.  Since $d_n$ is a positive real sequence, we have that
\[
\liminf \frac{d_{n+1}}{d_n} \le
\liminf \sqrt[n]{d_n}       \le
\limsup \sqrt[n]{d_n}       \le
\limsup \frac{d_{n+1}}{d_n},
\]
a proof of which can be found in~\cite{Ru}.  But $d_{n+1}/d_n=a_{n+1}$, which by assumption converges to $L$.  Hence $\sqrt[n]{d_n}=c_n$ must also converge to $L$.
\end{proof}

\begin{thebibliography}{1}
\bibitem{Ru}
Rudin, W., \emph{Principles of Mathematical Analysis}, 3rd ed., McGraw-Hill, New York, 1976.
\end{thebibliography}
%%%%%
%%%%%
\end{document}
