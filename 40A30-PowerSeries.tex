\documentclass[12pt]{article}
\usepackage{pmmeta}
\pmcanonicalname{PowerSeries}
\pmcreated{2013-03-22 12:32:55}
\pmmodified{2013-03-22 12:32:55}
\pmowner{azdbacks4234}{14155}
\pmmodifier{azdbacks4234}{14155}
\pmtitle{power series}
\pmrecord{23}{32793}
\pmprivacy{1}
\pmauthor{azdbacks4234}{14155}
\pmtype{Definition}
\pmcomment{trigger rebuild}
\pmclassification{msc}{40A30}
\pmclassification{msc}{30B10}
\pmrelated{TaylorSeries}
\pmrelated{FormalPowerSeries}
\pmrelated{TermwiseDifferentiation}
\pmrelated{AbelsLimitTheorem}
\pmdefines{constant term}

% this is the default PlanetMath preamble.  as your knowledge
% of TeX increases, you will probably want to edit this, but
% it should be fine as is for beginners.

% almost certainly you want these
\usepackage{amssymb}
\usepackage{amsmath}
\usepackage{amsfonts}

% used for TeXing text within eps files
%\usepackage{psfrag}
% need this for including graphics (\includegraphics)
%\usepackage{graphicx}
% for neatly defining theorems and propositions
%\usepackage{amsthm}
% making logically defined graphics
%%%\usepackage{xypic}

% there are many more packages, add them here as you need them

% define commands here
\begin{document}
A \emph{power series} is a series of the form
$$\sum_{k=0}^{\infty}a_k(x-x_0)^k,$$
with $a_k,x_0\in\mathbb{R}$ or $\in\mathbb{C}$. The $a_k$ are called the coefficients and $x_0$ the \PMlinkescapeword{center}center of the power series. $a_0$ is called the \emph{constant term}.

Where it converges the power series defines a function, which can thus be represented by a power series. This is what power series are usually used for.
Every power series is convergent at least at $x=x_0$ where it converges to $a_0$. In addition it is absolutely and uniformly convergent in the region $\{x\mid |x-x_0|<r\}$, with
$$r=\liminf_{k\to\infty}\frac{1}{\sqrt[k]{|a_k|}}$$
It is divergent for every $x$ with $|x-x_0| > r$. For $|x-x_0|= r$ no general predictions can be made. If $r=\infty$, the power series converges absolutely and uniformly for every real or complex $x.$ The real number $r$ is called the \textbf{radius of convergence} of the power series.

Examples of power series are:
\begin{itemize}
\item Taylor series, for example:
$$e^x=\sum_{k=0}^{\infty}\frac{x^k}{k!}.$$
\item The geometric series:
$$\frac{1}{1-x}=\sum_{k=0}^{\infty}x^k,$$
with $|x|<1$.
\end{itemize}

Power series have some important \PMlinkescapetext{properties}:
\begin{itemize}
\item If a power series converges for a $z_0\in\mathbb{C}$ then it also converges for all $z\in\mathbb{C}$ with $|z-x_0|<|z_0-x_0|$.
\item Also, if a power series diverges for some $z_0\in\mathbb{C}$ then it diverges for all $z\in\mathbb{C}$ with $|z-x_0|>|z_0-x_0|$.
\item For $|x-x_0|<r$ Power series can be added by adding coefficients and multiplied in the obvious way:
$$\sum_{k=0}^\infty a_k(x-x_o)^k\cdot\sum_{l=0}^\infty b_j(x-x_0)^j = a_0b_0+(a_0b_1+a_1b_0)(x-x_0)+(a_0b_2+a_1b_1+a_2b_0)(x-x_0)^2\ldots.$$
\item (Uniqueness) If two power series are equal and their \PMlinkescapetext{centers} are the same, then their coefficients must be equal.
\item Power series can be termwise differentiated and integrated. These operations keep the radius of convergence.
\end{itemize}
%%%%%
%%%%%
\end{document}
