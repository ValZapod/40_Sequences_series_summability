\documentclass[12pt]{article}
\usepackage{pmmeta}
\pmcanonicalname{ProofOfStolzCesaroTheorem}
\pmcreated{2013-03-22 13:17:45}
\pmmodified{2013-03-22 13:17:45}
\pmowner{slash}{33}
\pmmodifier{slash}{33}
\pmtitle{Proof of Stolz-Cesaro theorem}
\pmrecord{4}{33795}
\pmprivacy{1}
\pmauthor{slash}{33}
\pmtype{Proof}
\pmcomment{trigger rebuild}
\pmclassification{msc}{40A05}
%\pmkeywords{sequence}
%\pmkeywords{limit}
%\pmkeywords{convergence}

% this is the default PlanetMath preamble.  as your knowledge
% of TeX increases, you will probably want to edit this, but
% it should be fine as is for beginners.

% almost certainly you want these
\usepackage{amssymb}
\usepackage{amsmath}
\usepackage{amsfonts}

% used for TeXing text within eps files
%\usepackage{psfrag}
% need this for including graphics (\includegraphics)
%\usepackage{graphicx}
% for neatly defining theorems and propositions
%\usepackage{amsthm}
% making logically defined graphics
%%%\usepackage{xypic} 

% there are many more packages, add them here as you need them

% define commands here
\begin{document}
From the definition of convergence , for every $\epsilon > 0$ there is $N(\epsilon) \in \mathbb{N}$ such that $(\forall) n \geq N(\epsilon)$ , we have :
$$ l-\epsilon < \frac{a_{n+1}-a_n}{b_{n+1}-b_n} < l + \epsilon $$
Because $b_n$ is strictly increasing we can multiply the last equation with $b_{n+1}-b_n$ to get :
$$ (l-\epsilon)(b_{n+1}-b_n) < a_{n+1}-a_n < (l+\epsilon)(b_{n+1}-b_n) $$
Let $k>N(\epsilon)$ be a natural number . Summing the last relation we get :
$$ (l-\epsilon)\sum_{i=N(\epsilon)}^{k}(b_{i+1}-b_i) < \sum_{i=N(\epsilon)}^{k}(a_{n+1}-a_n) < (l+\epsilon)\sum_{i=N(\epsilon)}^{k}(b_{i+1}-b_i) \Rightarrow $$
$$ (l-\epsilon)(b_{k+1}-b_{N(\epsilon)}) < a_{k+1} - a_{N(\epsilon)} < (l+\epsilon)(b_{k+1}-b_{N(\epsilon)})$$
Divide the last relation by $b_{k+1}>0$ to get :
$$ (l-\epsilon)(1 - \frac{b_{N(\epsilon)}}{b_{k+1}}) < \frac{a_{k+1}}{b_{k+1}} - \frac{a_{N(\epsilon)}}{b_{k+1}}<(l+\epsilon)(1 - \frac{b_{N(\epsilon)}}{b_{k+1}}) \Leftrightarrow$$
$$ (l-\epsilon)(1 - \frac{b_{N(\epsilon)}}{b_{k+1}}) + \frac{a_{N(\epsilon)}}{b_{k+1}}< \frac{a_{k+1}}{b_{k+1}}<(l+\epsilon)(1 - \frac{b_{N(\epsilon)}}{b_{k+1}}) + \frac{a_{N(\epsilon)}}{b_{k+1}} $$
This means that there is some $K$ such that for $k \geq K$ we have :
$$(l-\epsilon)<\frac{a_{k+1}}{b_{k+1}}< (l+\epsilon)$$
(since the other terms who were left out converge to 0)

This obviously means that :
$$ \lim_{n \rightarrow \infty} \frac{a_n}{b_n}=l$$
and we are done .
%%%%%
%%%%%
\end{document}
