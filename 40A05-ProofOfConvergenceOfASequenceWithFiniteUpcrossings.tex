\documentclass[12pt]{article}
\usepackage{pmmeta}
\pmcanonicalname{ProofOfConvergenceOfASequenceWithFiniteUpcrossings}
\pmcreated{2013-03-22 18:49:39}
\pmmodified{2013-03-22 18:49:39}
\pmowner{gel}{22282}
\pmmodifier{gel}{22282}
\pmtitle{proof of convergence of a sequence with finite upcrossings}
\pmrecord{4}{41631}
\pmprivacy{1}
\pmauthor{gel}{22282}
\pmtype{Proof}
\pmcomment{trigger rebuild}
\pmclassification{msc}{40A05}
\pmclassification{msc}{60G17}
%\pmkeywords{upcrossings}
%\pmkeywords{supremum limit}
%\pmkeywords{infimum limit}

\endmetadata

% almost certainly you want these
\usepackage{amssymb}
\usepackage{amsmath}
\usepackage{amsfonts}

% used for TeXing text within eps files
%\usepackage{psfrag}
% need this for including graphics (\includegraphics)
%\usepackage{graphicx}
% for neatly defining theorems and propositions
\usepackage{amsthm}
% making logically defined graphics
%%%\usepackage{xypic}

% there are many more packages, add them here as you need them

% define commands here
\newtheorem*{theorem*}{Theorem}
\newtheorem*{lemma*}{Lemma}
\newtheorem*{corollary*}{Corollary}
\newtheorem*{definition*}{Definition}
\newtheorem{theorem}{Theorem}
\newtheorem{lemma}{Lemma}
\newtheorem{corollary}{Corollary}
\newtheorem{definition}{Definition}

\begin{document}
\PMlinkescapeword{limit}
\PMlinkescapeword{number}
\PMlinkescapeword{finite}
\PMlinkescapeword{conversely}
\PMlinkescapeword{integer}
\PMlinkescapeword{infinite}
\PMlinkescapeword{interval}

We show that a sequence $x_1,x_2,\ldots$ of real numbers converges to a limit in the extended real numbers if and only if the number of upcrossings $U[a,b]$ is finite for all $a<b$.

Denoting the infimum limit and supremum limit by
\begin{equation*}
l = \liminf_{n\rightarrow\infty}x_n,\ u=\limsup_{n\rightarrow\infty}x_n,
\end{equation*}
then $l\le u$ and the sequence converges to a limit if and only if $l=u$.

We first show that if the sequence converges then $U[a,b]$ is finite for $a<b$. If $l>a$ then there is an $N$ such that $x_n>a$ for all $n\ge N$. So, all upcrossings of $[a,b]$ must start before time $N$, and we may conclude that $U[a,b]\le N$ is finite. On the other hand, if $l\le a$ then $u=l<b$ and we can infer that $x_n<b$ for all $n\ge N$ and some $N$. Again, this gives $U[a,b]\le N$.

Conversely, suppose that the sequence does not converge, so that $u>l$. Then choose $a<b$ in the interval $(l,u)$. For any integer $n$, there is then an $m>n$ such that $x_m>b$ and an $m>n$ with $x_m<a$. This allows us to define infinite sequences $s_k,t_k$ by $t_0=0$ and
\begin{align*}
&s_k=\inf\left\{m\ge t_{k-1}\colon X_{m}<a\right\},\\
&t_k=\inf\left\{m\ge s_{k}\colon X_{m}>b\right\},
\end{align*}
for $k\ge 1$. Clearly, $s_1<t_1<s_2<\cdots$ and $x_{s_k}<a<b<x_{t_k}$ for all $k\ge 1$, so $U[a,b]=\infty$.

%%%%%
%%%%%
\end{document}
