\documentclass[12pt]{article}
\usepackage{pmmeta}
\pmcanonicalname{ProofThatEulersConstantExists}
\pmcreated{2013-03-22 16:34:48}
\pmmodified{2013-03-22 16:34:48}
\pmowner{rm50}{10146}
\pmmodifier{rm50}{10146}
\pmtitle{proof that Euler's constant exists}
\pmrecord{6}{38771}
\pmprivacy{1}
\pmauthor{rm50}{10146}
\pmtype{Proof}
\pmcomment{trigger rebuild}
\pmclassification{msc}{40A25}

% this is the default PlanetMath preamble.  as your knowledge
% of TeX increases, you will probably want to edit this, but
% it should be fine as is for beginners.

% almost certainly you want these
\usepackage{amssymb}
\usepackage{amsmath}
\usepackage{amsfonts}

% used for TeXing text within eps files
%\usepackage{psfrag}
% need this for including graphics (\includegraphics)
%\usepackage{graphicx}
% for neatly defining theorems and propositions
%\usepackage{amsthm}
% making logically defined graphics
%%%\usepackage{xypic}

% there are many more packages, add them here as you need them

% define commands here
\newtheorem{thm}{Theorem}

\begin{document}
\begin{thm} The limit
\[\gamma = \lim_{n\to\infty}\left(\sum_{k=1}^n \frac{1}{k} - \ln n\right)\]
exists.
\end{thm}

\textbf{Proof. } Let
\[C_n=\frac{1}{1}+\frac{1}{2}+\cdots+\frac{1}{n}-\ln n\]
and
\[D_n=C_n-\frac{1}{n}\]
Then
\[C_{n+1}-C_n=\frac{1}{n+1}-\ln\left(1+\frac{1}{n}\right)\]
and
\[D_{n+1}-D_n=\frac{1}{n}-\ln\left(1+\frac{1}{n}\right)\]
Now, by considering the Taylor series for $\ln(1+x)$, we see that
\[\frac{1}{n+1}<\ln\left(1+\frac{1}{n}\right)<\frac{1}{n}\]
and so
\[C_{n+1}-C_n < 0 < D_{n+1}-D_n\]
Thus, the $C_n$ decrease monotonically, while the $D_n$ increase monotonically, since the differences are negative (positive for $D_n$). Further, $D_n<C_n$ and thus $D_1=0$ is a lower bound for $C_n$. Thus the $C_n$ are monotonically decreasing and bounded below, so they must converge.


\begin{thebibliography}{10}
\bibitem{bib:artin}
E.~Artin, \emph{The Gamma Function}, Holt, Rinehart, Winston 1964.
\end{thebibliography}
%%%%%
%%%%%
\end{document}
