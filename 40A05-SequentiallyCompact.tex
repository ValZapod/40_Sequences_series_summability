\documentclass[12pt]{article}
\usepackage{pmmeta}
\pmcanonicalname{SequentiallyCompact}
\pmcreated{2013-03-22 12:50:05}
\pmmodified{2013-03-22 12:50:05}
\pmowner{mps}{409}
\pmmodifier{mps}{409}
\pmtitle{sequentially compact}
\pmrecord{12}{33159}
\pmprivacy{1}
\pmauthor{mps}{409}
\pmtype{Definition}
\pmcomment{trigger rebuild}
\pmclassification{msc}{40A05}
\pmclassification{msc}{54D30}
\pmsynonym{sequential compactness}{SequentiallyCompact}
%\pmkeywords{topology}
%\pmkeywords{sequence}
%\pmkeywords{convergence}
\pmrelated{Compact}
\pmrelated{LimitPointCompact}
\pmrelated{BolzanoWeierstrassTheorem}
\pmrelated{Net}

\endmetadata

% this is the default PlanetMath preamble.  as your knowledge
% of TeX increases, you will probably want to edit this, but
% it should be fine as is for beginners.

% almost certainly you want these
\usepackage{amssymb}
\usepackage{amsmath}
\usepackage{amsfonts}

% used for TeXing text within eps files
%\usepackage{psfrag}
% need this for including graphics (\includegraphics)
%\usepackage{graphicx}
% for neatly defining theorems and propositions
%\usepackage{amsthm}
% making logically defined graphics
%%%\usepackage{xypic}

% there are many more packages, add them here as you need them

% define commands here
\begin{document}
A topological space $X$ is \emph{sequentially compact} if every sequence in $X$ has a convergent subsequence.

Every sequentially compact space is countably compact.  Conversely, every first countable countably compact space is sequentially compact.  The ordinal space $W(2\omega_1)$ is sequentially compact but not first countable, since $\omega_1$ has not countable local basis.

Next, compactness and sequential compactness are not compatible.  In other words, neither one implies the other.  Here's an example of a compact space that is not sequentially compact.  Let $X=I^I$, where $I$ is the closed unit interval (with the usual topology), and $X$ is equipped with the product topology.  Then $X$ is compact (since $I$ is, together with Tychonoff theorem).  However, $X$ is not sequentially compact.  To see this, let $f_n:I\to I$ be the function such that for any $r\in I$, $f(r)$ is the $n$-th digit of $r$ in its binary expansion.  But the sequence $f_1,\ldots, f_n,\ldots$ has no convergent subsequences: if $f_{n_1},\ldots, f_{n_k},\ldots$ is a subsequence, let $r\in I$ such that its binary expansion has its $k$-th digit $0$ iff $k$ is odd, and $1$ otherwise.  Then $f_{n_1}(r),\ldots, f_{n_k}(r), \ldots$ is the sequence $0,1,0,1,\ldots$, and is clearly not convergent.  The ordinal space $\Omega_0:=W(\omega_1)$ is an example of a sequentially compact space that is not compact, since the cover $\lbrace W(\alpha)\mid \alpha\in \Omega_0\rbrace$ has no finite subcover.

When $X$ is a metric space, the following are equivalent:
\begin{itemize}
\item
$X$ is sequentially compact.
\item
$X$ is limit point compact.
\item
$X$ is compact.
\item
$X$ is totally bounded and complete.
\end{itemize}
%%%%%
%%%%%
\end{document}
