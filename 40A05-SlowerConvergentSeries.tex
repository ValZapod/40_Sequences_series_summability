\documentclass[12pt]{article}
\usepackage{pmmeta}
\pmcanonicalname{SlowerConvergentSeries}
\pmcreated{2013-03-22 15:08:24}
\pmmodified{2013-03-22 15:08:24}
\pmowner{pahio}{2872}
\pmmodifier{pahio}{2872}
\pmtitle{slower convergent series}
\pmrecord{11}{36884}
\pmprivacy{1}
\pmauthor{pahio}{2872}
\pmtype{Theorem}
\pmcomment{trigger rebuild}
\pmclassification{msc}{40A05}
\pmrelated{SlowerDivergentSeries}
\pmrelated{NonExistenceOfUniversalSeriesConvergenceCriterion}

\endmetadata

% this is the default PlanetMath preamble.  as your knowledge
% of TeX increases, you will probably want to edit this, but
% it should be fine as is for beginners.

% almost certainly you want these
\usepackage{amssymb}
\usepackage{amsmath}
\usepackage{amsfonts}

% used for TeXing text within eps files
%\usepackage{psfrag}
% need this for including graphics (\includegraphics)
%\usepackage{graphicx}
% for neatly defining theorems and propositions
 \usepackage{amsthm}
% making logically defined graphics
%%%\usepackage{xypic}

% there are many more packages, add them here as you need them

% define commands here

\theoremstyle{definition}
\newtheorem*{thmplain}{Theorem}
\begin{document}
\begin{thmplain}
\, If 
\begin{align}
a_1\!+\!a_2\!+\!a_3\!+\cdots
\end{align}
is a converging series with positive \PMlinkescapetext{terms}, then one can always form another converging series
$$g_1\!+\!g_2\!+\!g_3\!+\cdots$$
such that 
\begin{align}
\lim_{n\to\infty}\frac{g_n}{a_n} = \infty
\end{align}
\end{thmplain}

{\em Proof.}\, Let $S$ be the sum of (1),\, $S_n = a_1\!+\!a_2\!+\cdots+\!a_n$\, the $n^\mathrm{th}$ partial sum of (1) and\, $R_{n+1} = S\!-\!S_n = a_{n+1}\!+\!a_{n+2}\!+\cdots$\, the corresponding remainder term.\, Then we have 
$$a_n = R_n\!-\!R_{n+1} = (\sqrt{R_n}\!+\!\sqrt{R_{n+1}})(\sqrt{R_n}\!-\!\sqrt{R_{n+1}}).$$
We set
$$g_n := \frac{a_n}{\sqrt{R_n}\!+\!\sqrt{R_{n+1}}} 
= \sqrt{R_n}\!-\!\sqrt{R_{n+1}}
\quad \forall n = 1,\,2,\,3,\,\ldots$$
Then the series\, $g_1\!+\!g_2\!+\!g_3\!+\cdots$\, fulfils the requirements in the theorem.\, Its \PMlinkescapetext{terms} $g_n$ are positive.\, Further, it converges because its $n^\mathrm{th}$ partial sum is equal to 
$\sqrt{R_1}\!-\!\sqrt{R_{n+1}}$ which tends to the limit\, 
$\sqrt{R_1} = \sqrt{S}$\, as\, $n\to\infty$\, since\, $R_{n+1}\to 0$;\, this implies also (2).
%%%%%
%%%%%
\end{document}
