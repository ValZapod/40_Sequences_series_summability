\documentclass[12pt]{article}
\usepackage{pmmeta}
\pmcanonicalname{UniformConvergence}
\pmcreated{2013-03-22 13:13:49}
\pmmodified{2013-03-22 13:13:49}
\pmowner{Koro}{127}
\pmmodifier{Koro}{127}
\pmtitle{uniform convergence}
\pmrecord{14}{33700}
\pmprivacy{1}
\pmauthor{Koro}{127}
\pmtype{Definition}
\pmcomment{trigger rebuild}
\pmclassification{msc}{40A30}
\pmrelated{CompactOpenTopology}
\pmrelated{ConvergesUniformly}
\pmdefines{uniformly convergent}

% this is the default PlanetMath preamble.  as your knowledge
% of TeX increases, you will probably want to edit this, but
% it should be fine as is for beginners.

% almost certainly you want these
\usepackage{amssymb}
\usepackage{amsmath}
\usepackage{amsfonts}

% used for TeXing text within eps files
%\usepackage{psfrag}
% need this for including graphics (\includegraphics)
%\usepackage{graphicx}
% for neatly defining theorems and propositions
%\usepackage{amsthm}
% making logically defined graphics
%%%\usepackage{xypic}

% there are many more packages, add them here as you need them

% define commands here
\begin{document}
Let $X$ be any set, and let $(Y,d)$ be a metric space. 
A sequence $f_1,f_2,\dots$ of functions mapping $X$ to $Y$ is said to be 
\emph{uniformly convergent} to another function $f$ if, for each $\varepsilon>0$, there exists $N$ such that, for all $x$ and all $n>N$, we have $d(f_n(x),f(x))<\varepsilon$. 
This is denoted by $f_n\xrightarrow[]{u} f$, or ``$f_n\rightarrow f$ uniformly'' or, less frequently, by $f_n\rightrightarrows f$.
%%%%%
%%%%%
\end{document}
