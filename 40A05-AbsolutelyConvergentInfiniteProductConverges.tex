\documentclass[12pt]{article}
\usepackage{pmmeta}
\pmcanonicalname{AbsolutelyConvergentInfiniteProductConverges}
\pmcreated{2013-03-22 18:41:15}
\pmmodified{2013-03-22 18:41:15}
\pmowner{pahio}{2872}
\pmmodifier{pahio}{2872}
\pmtitle{absolutely convergent infinite product converges}
\pmrecord{6}{41439}
\pmprivacy{1}
\pmauthor{pahio}{2872}
\pmtype{Theorem}
\pmcomment{trigger rebuild}
\pmclassification{msc}{40A05}
\pmclassification{msc}{30E20}
\pmsynonym{convergence of absolutely convergent infinite product}{AbsolutelyConvergentInfiniteProductConverges}
%\pmkeywords{infinite product}
%\pmkeywords{complex numbers}
\pmrelated{AbsoluteConvergenceImpliesConvergenceForAnInfiniteProduct}
\pmrelated{AbsoluteConvergenceOfInfiniteProductAndSeries}

\endmetadata

% this is the default PlanetMath preamble.  as your knowledge
% of TeX increases, you will probably want to edit this, but
% it should be fine as is for beginners.

% almost certainly you want these
\usepackage{amssymb}
\usepackage{amsmath}
\usepackage{amsfonts}

% used for TeXing text within eps files
%\usepackage{psfrag}
% need this for including graphics (\includegraphics)
%\usepackage{graphicx}
% for neatly defining theorems and propositions
 \usepackage{amsthm}
% making logically defined graphics
%%%\usepackage{xypic}

% there are many more packages, add them here as you need them

% define commands here

\theoremstyle{definition}
\newtheorem*{thmplain}{Theorem}

\begin{document}
\textbf{Theorem.}\; An \PMlinkname{absolutely convergent}{AbsoluteConvergenceOfInfiniteProduct} infinite product
\begin{align}
\prod_{\nu=1}^\infty(1\!+\!c_\nu) \;=\; (1\!+\!c_1)(1\!+\!c_2)(1\!+\!c_3)\cdots
\end{align}
of complex numbers is convergent.\\

{\em Proof.}\, We thus assume the convergence of the \PMlinkname{product}{Product}
\begin{align}
\prod_{\nu=1}^\infty(1\!+\!|c_\nu|) \;=\; (1\!+\!|c_1|)(1\!+\!|c_2|)(1\!+\!|c_3|)\cdots
\end{align}
Let $\varepsilon$ be an arbitrary positive number.\, By the general convergence condition of infinite product, we have
$$|(1\!+\!|c_{n+1}|)(1\!+\!|c_{n+2}|)\cdots(1\!+\!|c_{n+p}|)-1| < \varepsilon 
\quad \forall\; p \in \mathbb{Z}_+$$
when\, $n \geqq$ certain $n_\varepsilon$.\, Then we see that
\begin{align*}
|(1\!+\!c_{n+1})(1\!+\!c_{n+2})\cdots(1\!+\!c_{n+p})-1| 
& =    |1+\sum_{\nu=n+1}^{n+p}c_\nu+\sum_{\mu,\,\nu}c_\mu c_\nu+\ldots+c_{n+1}c_{n+2}\cdots c_{n+p}-1|\\
& \leqq 1+\sum_{\nu=n+1}^{n+p}|c_\nu|+\sum_{\mu,\,\nu}|c_\mu||c_\nu|+\ldots+|c_{n+1}||c_{n+2}|\cdots|c_{n+p}|-1\\
& = |(1\!+\!|c_{n+1}|)(1\!+\!|c_{n+2}|)\cdots(1\!+\!|c_{n+p}|)-1| < \varepsilon 
\qquad \forall\; p \in \mathbb{Z}_+
\end{align*}
as soon as\, $n \geqq n_\varepsilon$.\; I.e., the infinite product (1) converges, by the same convergence condition.

%%%%%
%%%%%
\end{document}
