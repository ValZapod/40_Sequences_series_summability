\documentclass[12pt]{article}
\usepackage{pmmeta}
\pmcanonicalname{Monotonic}
\pmcreated{2013-03-22 12:22:43}
\pmmodified{2013-03-22 12:22:43}
\pmowner{akrowne}{2}
\pmmodifier{akrowne}{2}
\pmtitle{monotonic}
\pmrecord{5}{32141}
\pmprivacy{1}
\pmauthor{akrowne}{2}
\pmtype{Definition}
\pmcomment{trigger rebuild}
\pmclassification{msc}{40-00}
\pmsynonym{monotonically}{Monotonic}
\pmsynonym{monotone}{Monotonic}

\usepackage{amssymb}
\usepackage{amsmath}
\usepackage{amsfonts}

%\usepackage{psfrag}
%\usepackage{graphicx}
%%%\usepackage{xypic}
\begin{document}
A sequence or function is said to be \emph{monotonic} if it is 

\begin{itemize}
\item monotonically increasing
\item monotonically decreasing
\item monotonically nondecreasing
\item monotonically nonincreasing
\end{itemize}

Intuitively, this means that a monotone sequence can be thought of a ``staircase'' going either only up, or only down, with the stairs any height and any depth.  The same goes for step functions.  Smooth (analytic) monotone functions are like ``slopes,'' either up or down at potentially varying steepnesess.
%%%%%
%%%%%
\end{document}
