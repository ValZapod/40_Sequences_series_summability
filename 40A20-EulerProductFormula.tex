\documentclass[12pt]{article}
\usepackage{pmmeta}
\pmcanonicalname{EulerProductFormula}
\pmcreated{2013-03-22 18:39:38}
\pmmodified{2013-03-22 18:39:38}
\pmowner{pahio}{2872}
\pmmodifier{pahio}{2872}
\pmtitle{Euler product formula}
\pmrecord{6}{41404}
\pmprivacy{1}
\pmauthor{pahio}{2872}
\pmtype{Theorem}
\pmcomment{trigger rebuild}
\pmclassification{msc}{40A20}
\pmclassification{msc}{11M06}
\pmclassification{msc}{11A51}
\pmclassification{msc}{11A41}
\pmrelated{RiemannZetaFunction}
\pmrelated{EulerProduct}

\endmetadata

% this is the default PlanetMath preamble.  as your knowledge
% of TeX increases, you will probably want to edit this, but
% it should be fine as is for beginners.

% almost certainly you want these
\usepackage{amssymb}
\usepackage{amsmath}
\usepackage{amsfonts}

% used for TeXing text within eps files
%\usepackage{psfrag}
% need this for including graphics (\includegraphics)
%\usepackage{graphicx}
% for neatly defining theorems and propositions
 \usepackage{amsthm}
% making logically defined graphics
%%%\usepackage{xypic}

% there are many more packages, add them here as you need them

% define commands here

\theoremstyle{definition}
\newtheorem*{thmplain}{Theorem}

\begin{document}
\PMlinkescapeword{terms}

\textbf{Theorem (Euler).}\, If\, $s > 1$,\, the infinite product
\begin{align}
\prod_{p}\frac{1}{1-\frac{1}{p^s}}
\end{align}
where $p$ runs the positive rational primes, converges to the sum of the over-harmonic series
\begin{align}
\sum_{n=1}^\infty\frac{1}{n^s} \,=\, \zeta(s).
\end{align}


{\em Proof.}\, Denote the sequence of prime numbers by $p_1 < p_2 < p_3 <\,\ldots$\, For any\, $s > 0$,\, we can form convergent geometric series
$$\frac{1}{1-\frac{1}{p_1^s}} \,=\, 1+\frac{1}{p_1^s}+\frac{1}{p_1^{2s}}+\ldots 
\,=\, \sum_{\nu_1=0}^\infty\frac{1}{p_1^{\nu_1s}},$$
$$\frac{1}{1-\frac{1}{p_2^s}} \,=\, 1+\frac{1}{p_2^s}+\frac{1}{p_2^{2s}}+\ldots
\,=\, \sum_{\nu_2=0}^\infty\frac{1}{p_2^{\nu_2s}}.$$
Since these series are absolutely convergent, their product (see multiplication of series) may be written as
$$\frac{1}{1-\frac{1}{p_1^s}}\cdot\frac{1}{1-\frac{1}{p_2^s}} 
\;=\; \sum_{\nu_1,\nu_2=0}^\infty\frac{1}{p_1^{\nu_1s}}\cdot\frac{1}{p_2^{\nu_2s}} 
\;=\; \sum_{\nu_1,\nu_2=0}^\infty\frac{1}{\left(p_1^{\nu_1}p_2^{\nu_2}\right)^s}$$
where $\nu_1$ and $\nu_2$ independently on each other run all nonnegative integers.\, This equation can be generalised by induction to
\begin{align}
\prod_{\nu=1}^k \frac{1}{1-\frac{1}{p_\nu^s}} \;=\; 
\sum_{\nu_1,\nu_2,\ldots,\nu_k=0}^\infty\frac{1}{\left(p_1^{\nu_1}p_2^{\nu_2}\cdots p_k^{\nu_k}\right)^s}
\end{align}
for\, $s > 0$\, and for arbitrarily great $k$; the exponents \,$\nu_1,\,\nu_2,\,\ldots,\,\nu_k$\, run independently all nonnegative integers.

Because the prime factorization of positive integers is \PMlinkname{unique}{FundamentalTheoremOfArithmetic}, we can rewrite (3) as
\begin{align}
\prod_{\nu=1}^k \frac{1}{1-\frac{1}{p_\nu^s}} \;=\; \sum_{(n)}\frac{1}{n^s},
\end{align}
where $n$ runs all positive integers not containing greater prime factors than $p_k$.\, Then the inequality
\begin{align}
\sum_{n=1}^{p_k}\frac{1}{n^s} < \prod_{\nu=1}^k \frac{1}{1-\frac{1}{p_\nu^s}},
\end{align}
holds for every $k$, since all the terms \,$1,\,\frac{1}{p_1^s},\,\ldots,\,\frac{1}{p_k^s}$\, are in the series of the right hand side of (4).\, On the other hand, this series contains only a part of the terms of (2).\, Thus, for\, $s > 1$,\, the product (3) is less than the sum $\zeta(s)$ of the series (2), and consequently
\begin{align}
\sum_{n=1}^{p_k}\frac{1}{n^s} < \prod_{\nu=1}^k \frac{1}{1-\frac{1}{p_\nu^s}} < \zeta(s).
\end{align}
Letting\, $k \to \infty$,\, we have\, $p_k \to \infty$,\, and the sum on the left hand side of (6) tends to the limit $\zeta(s)$, therefore also tends the product (3).\, Hence we get the limit equation
\begin{align}
\prod_{p}\frac{1}{1-\frac{1}{p^s}} \;=\; \zeta(s) \qquad (s > 1).
\end{align}

\begin{thebibliography}{8}
\bibitem{lindelof}{\sc E. Lindel\"of}: {\em Differentiali- ja integralilasku
ja sen sovellutukset III.2}.\, Mercatorin Kirjapaino Osakeyhti\"o, Helsinki (1940).
\end{thebibliography}

%%%%%
%%%%%
\end{document}
