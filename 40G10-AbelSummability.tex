\documentclass[12pt]{article}
\usepackage{pmmeta}
\pmcanonicalname{AbelSummability}
\pmcreated{2013-03-22 13:07:03}
\pmmodified{2013-03-22 13:07:03}
\pmowner{rmilson}{146}
\pmmodifier{rmilson}{146}
\pmtitle{Abel summability}
\pmrecord{7}{33549}
\pmprivacy{1}
\pmauthor{rmilson}{146}
\pmtype{Definition}
\pmcomment{trigger rebuild}
\pmclassification{msc}{40G10}
\pmrelated{CesaroSummability}
\pmrelated{AbelsLimitTheorem}
\pmdefines{Abelian theorem}
\pmdefines{Tauberian theorem}

\usepackage{amsmath}
\usepackage{amsfonts}
\usepackage{amssymb}
\newcommand{\reals}{\mathbb{R}}
\newcommand{\natnums}{\mathbb{N}}
\newcommand{\cnums}{\mathbb{C}}
\newcommand{\znums}{\mathbb{Z}}
\newcommand{\lp}{\left(}
\newcommand{\rp}{\right)}
\newcommand{\lb}{\left[}
\newcommand{\rb}{\right]}
\newcommand{\supth}{^{\text{th}}}
\newtheorem{proposition}{Proposition}
\newtheorem{definition}[proposition]{Definition}

\newtheorem{theorem}[proposition]{Theorem}
\begin{document}
\PMlinkexternal{Abel}{http://www-groups.dcs.st-and.ac.uk/~history/Mathematicians/Abel.html}
summability is a generalized convergence criterion for power series.
It extends the usual definition of the sum of a series, and gives a
way of summing up certain divergent series.  Let us start with a
series $\sum_{n=0}^\infty a_n$, convergent or not, and use that series
to define a power series
$$f(r) = \sum_{n=0}^\infty a_n r^n.$$
Note that for $|r|<1$ the
summability of $f(r)$ is easier to achieve than the summability of the
original series.  Starting with this observation we say that the
series $\sum a_n$ is \emph{Abel summable} if the defining series
for $f(r)$ is convergent for all $|r|<1$, and if $f(r)$ converges to
some limit $L$ as $r\rightarrow 1^-$.  If this is so, we shall say
that $\sum a_n$ Abel converges to $L$.

Of course it is important to ask whether an ordinary convergent series
is also Abel summable, and whether it converges to the same limit?
This is true, and the result is known as Abel's limit theorem,
or simply as Abel's theorem.
\begin{theorem}[Abel]
  Let $\sum_{n=0}^\infty a_n$ be a series; let 
  $$s_N=a_0+\cdots+a_N,\quad N\in\natnums,$$
  denote the corresponding
  partial sums; and let $f(r)$ be the corresponding power series
  defined as above.  If $\sum a_n$ is convergent, in the
  usual sense that the $s_N$ converge to some limit $L$ as
  $N\rightarrow\infty$, then the series is also Abel summable and
  $f(r)\rightarrow L$ as $r\rightarrow 1^-$.
\end{theorem}

The standard example of a divergent series that is nonetheless Abel
summable is the alternating series
$$\sum_{n=0}^\infty (-1)^n.$$
The corresponding power series is
$$\frac{1}{1+r}=\sum_{n=0}^\infty (-1)^n r^n.$$
Since 
$$\frac{1}{1+r}\rightarrow \frac{1}{2}\quad \text{as}\quad
r\rightarrow 1^-,$$
this otherwise divergent series Abel converges to $\frac{1}{2}$.

Abel's theorem is the prototype for a number of other theorems about
convergence, which are collectively known in analysis as Abelian
theorems.  An important class of associated results are the so-called
Tauberian theorems.  These describe various convergence criteria, and
sometimes provide partial converses for the various Abelian theorems.

The general converse to Abel's theorem is false, as the example above
illustrates\footnote{We want the converse to be false; the whole idea
  is to describe a method of summing certain divergent series!}.
However, in the 1890's
\PMlinkexternal{Tauber}{http://www-groups.dcs.st-and.ac.uk/~history/Mathematicians/Tauber.html}
proved the following partial converse.
\begin{theorem}[Tauber]
Suppose that $\sum a_n$ is an Abel summable series and that $n a_n
\rightarrow 0$ as $n\rightarrow \infty$.  Then, $\sum_n a_n$ is
convergent in the ordinary sense as well.  
\end{theorem}
The proof of the above theorem is not hard, but the same cannot be
said of the more general Tauberian theorems.  The more famous of these
are due to Hardy, Hardy-Littlewood, Weiner, and Ikehara.  In all
cases, the conclusion is that a certain series or a certain integral
is convergent.  However, the proofs are lengthy and require
sophisticated techniques.  Ikehara's theorem is especially noteworthy
because it is used to prove the prime number theorem.
%%%%%
%%%%%
\end{document}
